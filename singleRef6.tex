\subsubsection{Sufficient remappings for condition (c)}
\label{sec:necessary-c}

The following example shows that we need to consider a wider range of
remappings when condition~(c) is relevant.
%
\begin{example}
Let 
\begin{eqnarray*}
pre & = & (fixed; Th(T_0,N_0,N_1), Nd_A(N_1,N_2)), \\
v & = & (fixed; Th'(T_0,N_0,N_1), Nd_B(N_1,N_2)).
\end{eqnarray*}
%
Consider a result-defining function~$\pi_{rd} = \set{N_0 \mapsto N_0}$ (this
would be necessary if $N_0$ is a parameter of $post.\fixed$, say).  Consider a
consistent extension~$\pi$.  Both $pre.\princ$ and $v.\princ$ have a missing
reference to a component with identity~$N_0$.  We consider how to instantiate
this component.

Suppose it is an invariant of the system that all views involving the $Th$
component and the missing component are equivalent to $(fixed;
Th(T_0,N_0,N_1), Nd_C(N_0,N_1,n))$ for some~$n$ (so the two components agree
on the parameter denoted~$N_1$).  This $Nd_C$ component has a reference to the
secondary component of~$pre$, so condition~(c) would also require a view with
those two components; suppose it is an invariant that the only such view is
equivalent to $(fixed; Nd_C(N_0,N_1,N_2), Nd_A(N_1,N_2))$, i.e.~instantiating
$n$ with~$N_2$ (so the two components agree on the parameter denoted~$N_2$).
This means that we need to instantiate the missing component with
%
\begin{eqnarray*}
c & = & Nd_C(N_0,N_1,N_2).
\end{eqnarray*}

Condition~(c) then also requires a view equivalent to
\[\mit
(fixed; Th'(\pi(T_0),N_0,\pi(N_1)), Nd_C(N_0,N_1,N_2)).
\]
We need to consider separately the cases $\pi(N_1) = N_2$ or $\pi(N_1) \ne
N_2$, since the resulting views are not equivalent.  (Note we cannot have
$\pi(N_1) = N_1$, since that would require a unification; likewise, we cannot
have $\pi(T_0) = T_0$.)

In the case that $\pi(N_1) = N_2$, the component~$c$ has a reference to the
secondary component of~$\pi(v)$.  Condition~(c) would also require a view
equivalent to
\[\mit
(fixed; Nd_C(N_0,N_1,N_2), Nd_B(N_2, \pi(N_2)).
\]
In this case, we need to consider separately the cases  $\pi(N_2) = N_1$ and
$\pi(N_2) \ne N_1$, since the resulting views are not equivalent.

In summary, we need to consider remappings equivalent to each of the following
extensions of~$\pi_{rd}$:
\[
\begin{array}{c}
\set{T_0 \mapsto T_1, N_0 \mapsto N_0, N_1 \mapsto N_2, N_2 \mapsto N_1}, \\
\set{T_0 \mapsto T_1, N_0 \mapsto N_0, N_1 \mapsto N_2, N_2 \mapsto N_3}, \\
\set{T_0 \mapsto T_1, N_0 \mapsto N_0, N_1 \mapsto N_3, N_2 \mapsto N_4}.
\end{array}
\]
\end{example}

%%%%%

%Define $\pi_{rd}$ creates a missing-common-component.

The following definition describes when a result-defining remapping means that
there is necessarily a common missing component.
%
\begin{definition}
We say that $\pi_{rd}$ \emph{creates a common missing component} with
identity~$id$ if: $pre$ has a missing component with identity~$id$; $v$ has a
missing component with some identity~$x$; and $(x \mapsto id) \in \pi_{rd}$.
\end{definition}

Our approach is as follows.  If a result-defining remapping does create a
common missing component, we simply consider all consistent extensions,
mapping parameters of~$v$ either to a parameter of~$pre$, or to the next fresh
parameter.  Clearly if any consistent extension satisfies condition~(c), then
one of these renamings does.

Note, though, that we only do this when necessary: when the minimal
result-defining remapping creates a common missing component; or our treatment
of condition~(b) produces a remapping that does so.

\framebox{Performance}

\begin{conjecture}
I think it would be enough to:
\begin{itemize}
\item Map parameters of $v.\princ$ to match parameters in $pre$ or fresh
  values, in all possible ways;

\item Map identities of secondary components of $v$ to match parameters in
  $pre$ or fresh values, in all possible ways;

\item If the identity of a secondary component of $v$ is mapped to a parameter
of~$v$, then map its remaining parameters to match parameters in $pre$
or fresh values, in all possible ways.
\end{itemize}
%
This would gain because it omits the case where a secondary component~$c_2$
of~$v$ has its identity mapped to a fresh value, but another parameter~$x$
mapped to a parameter of~$v$.  In this case, we could also map $x$ to a fresh
value, because then $c_2$ isn't involved in any required views (this needs to
hold for all occurrences of~$x$).
\end{conjecture}

%%%%%%%%%%%%%%%%%%%%%%%%%%%%%%%%%%%%%%%%%%%%%%%%%%%%%%%

\subsubsection{Calculating sufficient remappings}
\label{sec:necessary-algorithm}

We now define our technique for extending a minimal result-defining
remapping~$\pi_{rd}$.  If any extension of~$\pi_{rd}$ satisfies conditions~(b)
and~(c), then our technique produces such an extension.  

Our technique calculates $makeExtensions(\pi_{rd}, \set{})$, where
$makeExtensions$ is given in Figure~\ref{fig:makeExtensions}.  The subsidiary
parameter $doneB$ stores those pairs $(c,c')$, with $c$ a component of~$pre$
and $c'$ a component of~$v$, for which we have already considered
condition~(b) (the proof of Lemma~\ref{lem:makeExtensions-doneB} makes this
more formal).  The function produces a set of result-consistent extensions
that are completing extensions with respect to all pairs $(c,c') \in doneB$;
we say that they \emph{respect~$doneB$}.

\begin{definition}
We say that a total (over the parameters of~$v$) extension~$\pi_{te}$ of a
remapping function~$\pi$ \emph{respects $doneB$} if
%
\begin{enumerate}
\item it is result-consistent, i.e.~$\pi_{te}$ adds no extra result-relevant
parameters to the range; and
\item for each $(c,c') \in doneB$,\, $\pi_{te}$ is a completing extension with
respect to~$c$ and~$c'$, i.e.~it does not map any additional parameter of~$c'$
to a parameter of~$c$.
\end{enumerate}
\end{definition}

%%%%%

\begin{figure}[th]
\[
\begin{align}
makeExtensions(\pi: Remapping, doneB: \power(\S \cross \S): 
   \power Remapping =  \\
\quad\begin{align}
  \sm{if}(\mbox{$\pi$ creates a common missing component})
     \; \sm{return } allExtensions(\pi, doneB) \\
  \sm{else}  \\
  \quad\begin{align}
    \sm{let } linksB = \mbox{all pairs $(c,c')$ such that $\pi$ creates a
     cross reference between $c$ and $c'$} \\
    \sm{if}(linksB \subseteq doneB) \; \sm{return } \set{extendWithFresh(\pi)} \\
    \sm{else}  \\
    \quad\begin{align}
      \sm{pick } (c,c') \in linksB - doneB \\
      \sm{let } \Pi = 
        \begin{align}
        \mbox{all cross-reference-view-defining extensions of
           $\pi$ for $c$ and $c'$} \\
        \mbox{that respect $doneB$}
        \end{align} \\
      %\sm{val } \Pi = crossRefViewDefiners(\pi, c, c') \\
      \sm{return } \Union_{\pi' \in \Pi}
        makeExtensions(\pi', doneB \union \set{(c,c')})
       \end{align}
  \end{align} \\         
  \end{align} \\
\end{align}
\]
\caption{The $makeExtensions$ function.}
\label{fig:makeExtensions}
\end{figure}

%%%%%



The $makeExtensions$ function uses two subsidiary functions:
%
\begin{itemize}
\item 
$allExtensions(\pi, doneB)$ returns all consistent extensions of~$\pi$,
mapping each undefined parameter to either a parameter of~$pre$ or the next
fresh parameter, and that respect~$doneB$.

\item $extendWithFresh(\pi)$ extends $\pi$, mapping each undefined parameter
  to the next fresh parameter (the result necessarily respects~$doneB$).
\end{itemize}
 
%%%%%

The following lemma captures that when $makeExtensions(\pi,doneB)$ is called,
we have already considered condition~(b) for the elements of $doneB$. Either
all relevant extensions of~$\pi$ satisfy condition~(b) for every element of
$doneB$, or none does.  That means that we do not need to consider elements of
$doneB$ further when picking extensions of~$\pi$ (other than ensuring that
they continue to respect~$doneB$).
%
\begin{lemma}
\label{lem:makeExtensions-doneB}
Consider a recursive call of $makeExtensions(\pi,doneB)$.  Suppose $\pi_{te}$
is a total extension of~$\pi$ that respects~$doneB$, and satisfies
condition~(b) for every $(c,c') \in doneB$.  Then \emph{every} total extension
of~$\pi$ that respects~$doneB$ satisfies condition~(b) for every $(c,c') \in
doneB$.
\end{lemma}

%% .  Then for each $(c,c') \in
%% doneB$, if $\pi_{te}$ satisfies condition~(b) for~$c$ and~$c'$, then so
%% does \emph{every} total extension of~$\pi$ that respects~$doneB$.
%% % (in particular, $extendWithFresh(\pi)$).

%%%%%

%\framebox{re-express?}  

\begin{proof}
We prove the result by induction on the depth of the recursion. 
%
The result holds for the initial call, with $doneB = \set{}$, vacuously. 

Consider a call $makeExtensions(\pi,doneB)$ and an immediate recursive call
$make\-Extensions(\pi', doneB \union \set{(c,c')})$.  Suppose $\pi_{te}$ is a
total extension of~$\pi'$ that respects $doneB \union \set{(c,c')}$ and
satisfies condition~(b) for each element of $doneB \union \set{(c,c')}$.
Consider some other total extension~$\pi_{te}'$ that respects
$doneB \union \set{(c,c')}$.  We show that $\pi_{te}'$ satisfies condition~(b)
for all elements of $doneB \union \set{(c,c')}$, via a case analysis.
%
\begin{itemize}
\item Note that both $\pi_{te}$ and~$\pi_{te}'$ are also extensions of~$\pi$
that respect~$doneB$; and $\pi_{te}$ satisfies condition~(b) for all elements
of~$doneB$.  Hence by the inductive hypothesis $\pi_{te}'$ satisfies
condition~(b) for all elements of~$doneB$.

\item Suppose $\pi_{te}$ satisfies condition~(b) for $c$ and~$c'$.  Then by
Lemma~\ref{lem:cross-reference-view-defining}, so does every extension
of~$\pi'$ that respects $\set{(c,c')}$.  But that includes~$\pi_{te}'$.
\end{itemize}
\end{proof}


%% .  We show
%% that for each $(c_1,c_1') \in doneB \union \set{(c,c')}$, if $\pi_{te}$
%% satisfies condition~(b) for~$c_1$ and~$c_1'$, then so does \emph{every} total
%% extension of~$\pi'$ that respects~$doneB \union \set{(c,c')}$.
%% %
%% \begin{itemize}
%% \item 
%% Note that $\pi_{te}$ is also an extension of~$\pi$ that respects~$doneB$.
%% Hence by the inductive hypothesis, for each $(c_1,c_1') \in doneB$, if
%% $\pi_{te}$ satisfies condition~(b) for~$c_1$ and~$c_1'$, then so does every
%% total extension of~$\pi$ that respects~$doneB$.  But that includes every
%% total extension of~$\pi'$ that respects $doneB \union \set{(c,c')}$.

%% \item Suppose $\pi_{te}$ satisfies condition~(b) for $c$ and~$c'$.  Then by
%% Lemma~\ref{lem:cross-reference-view-defining}, so does every extension
%% of~$\pi'$ that respects $\set{(c,c')}$.  But that includes every extension
%% of~$\pi'$ that respects $doneB \union \set{(c,c')}$.
%% \end{itemize}
%% \end{proof}

%%%%%


Main result: 

\begin{prop}
\label{prop:makeExtensions}
Let $\pi_{rd}$ be a minimal result-defining remapping $\pi_{rd}$.  If any
result-consistent extension of $\pi_{rd}$ satisfies conditions~(b) and~(c),
then $makeExtensions(\pi_{rd}, \set{})$ produces such an extension.
\end{prop}
%
\begin{proof}
We show that for each call $makeExtensions(\pi,doneB)$, if any extension
of~$\pi$ that respects $doneB$ satisfies conditions~(b) and~(c), then this
call returns such an extension.  This implies the result of the proposition
when $doneB = \set{}$. 

Note that the usage of $doneB$ ensures termination ($doneB$ increases in size
at each recursive call, and is bounded).  This allows us to assume (by
induction) that the result holds for all recursive calls.

So consider a particular call $makeExtensions(\pi,doneB)$.  Let $\pi_{te}$ be
a total extension of~$\pi$ that respects $doneB$ and that satisfies
conditions~(b) and~(c).  

%% By Lemma~\ref{lem:makeExtensions-doneB}, for each
%% $(c,c') \in doneB$, every extension of $\pi$ that respects~$doneB$ satisfies
%% condition~(b) for~$c$ and~$c'$.

We perform a case analysis over the branch selected.
%
\begin{itemize}
\item Suppose $\pi$ creates a common missing component.  In this case, the use
of $allExtensions$ ensures that we return a remapping equivalent to $\pi_{te}$
(perhaps using different fresh values from~$\pi_{te}$), and that remapping
satisfies conditions~(b) and~(c). 

\item Suppose $linksB \subseteq doneB$.  Consider $\hat\pi =
  extendWithFresh(\pi)$.  This respects $doneB$, so by
  Lemma~\ref{lem:makeExtensions-doneB}, it satisfies condition~(b) for each
  $(c,c') \in doneB$, i.e.~it satisfies condition~(b) overall.
%
Further, $\hat\pi$ satisfies condition~(c) vacuously: $\pi$ does not create a
common missing component, and so neither does~$\hat\pi$.

\item Otherwise consider the pair $(c,c') \in linksB - doneB$ picked by the
function.  By Lemma~\ref{lem:cross-reference-view-defining-partition},
$\pi_{te}$ is a completing extension of some $\pi'$ that is a
cross-reference-defining extension of~$\pi$ for~$c$ and~$c'$.  Also $\pi'$
respects $doneB$ (since $\pi_{te}$ does), so $\pi' \in \Pi$.  Consider the
recursive call $makeExtensions(\pi', doneB \union \set{(c,c')})$.  Note that
$\pi_{te}$ respects $doneB \union \set{(c,c')}$.  Hence by the inductive
hypothesis, this recursive call returns an extension~$\hat\pi$ of~$\pi$ that
respects $doneB$ and satisfies conditions~(b) and~(c).  This $\hat\pi$ is in
turn returned by $makeExtensions(\pi,doneB)$, as required.
\end{itemize}
\end{proof}

%% We show that if any extension of $\pi$ satisfies
%% conditions~(b) and~(c), then this call to $makeExtensions$ produces such an
%% extension.

%% Note that the usage of $doneB$ ensures termination.  This allows us, below, to
%% assume that each recursive call satisfies the conditions of the proposition.
%% (Formally it's by induction on the number of pairs of components $(c,c')$ not
%% in $doneB$.)

%% In the case that $\pi_{rd}$ creates a missing common component, this holds
%% trivially, because we return a representative of every consistent extension.

%% In the case $linksB \subseteq doneB$, it follows from the above lemma.

%% Otherwise .........

%%%%%%%%%%%%%%%%%%%%%%%%%%%%%%%%%%%%%%%%%%%%%%%%%%%%%%%

\subsubsection{Unsure below here}

\framebox{***} Folowing looks wrong; see Example 35

\begin{lemma}
\label{lem:necessary-renamings}
Consider a (primary- or secondary-)result-defining mapping~$\pi_{rd}$, and an
extension~$\pi$ defined over all the parameters of~$v$.  Consider some set $X$
of linkage parameters for~$\pi$.  Let $\pi'$ be obtained from $\pi$ by
replacing each element of~$X$ in the range of~$\pi$ by a distinct fresh
variable.  Suppose that
%
\begin{enumerate}
\item No member of~$X$  is in the range of~$\pi_{rd}$;

\item If $\pi'$ produces a linkage that for condition~(b) creates a linkage
  involving component $c \in pre.\cpts$, then $\pi$ and~$\pi'$ range-agree on
  all parameters of~$c$.
\end{enumerate}
%
Then the views required to satisfy the linkages for $\pi'$ are a subset of
those for~$\pi$, and the resulting views are equivalent.
\end{lemma}

%%%%%

Example~\ref{example:singleRef:unnecessary-linkage} corresponds to applying
this lemma with $X = \set{N_1}$. 

%%%%%

\begin{proof}
That $\pi$ and~$\pi'$ produce the same resulting views follows from the fact
that they are both extensions of the same result-defining mapping~$\pi_{rd}$. 

Every linkage for $\pi'$ is also a linkage for~$\pi$ (excluding those on
parameters from~$X$).  Hence the views required for~$\pi'$ are a subset of
those required for~$\pi$. 
\end{proof}

\framebox{Not true}  first example in this subsection is a counterexample. 

%%%%%%%%%%%%%%%%%%%%%%%%%%%%%%%%%%%%%%%%%%%%%%%%%%%%%%%


