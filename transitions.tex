
\section{Calculating abstract transitions}
\label{sec:transitions}

In this section we describe how to calculate an upper bound to $aPost(V) =
\alpha(post(\gamma(V)))$.  In this section, we will be dealing with
potentially infinite sets of views.  But in Section~\ref{sec:symmetry}, we
will perform a form of symmetry reduction, so as to consider only a single
view from each symmetry equivalence class.  Our goal in this section,
therefore, is to restrict the set of views we need to consider to a finite
number of equivalence classes.

In order to motivate our approach, we present example abstract transitions
from the queue example, and how they can be extracted from
corresponding concrete transitions of components.  We use these examples to
define how we calculate (an over-approximation of) the abstract post-image of
a particular set of views.  We then prove that our approach is correct.

\subsection{Active-process transitions}

We start by considering abstract transitions triggered by an active process
(either an active principal or an active fixed process) within the view.  We
call these \emph{active-process transitions}.

Informally, our approach will be as follows.  We will start with a view $v \in
V$, and consider transitions involving either an active principal or active
fixed process synchronising on the transition.  Often, other fixed processes
or secondary components in~$v$ will synchronise on the transition (i.e.~have
the relevant event in their alphabet).  We call such a transition a \emph{view
  transition}, i.e.~a transition formed by considering a single view in
isolation.
%
\begin{definition}
We define a \emph{view transition} of view~$v$ to be a transition $v \trans[e]
v'$ such that
%
\begin{itemize}
\item Either: (1) $v$ has an active principal with~$e$ in its alphabet; (2)~$v$
  has an active fixed process with $e$ in its alphabet, and the principal
  of~$v$ is a passive component with $e$ in its alphabet; (3) only fixed
  processes have $e$ in their alphabet; or (4) $e = \tau$.
%% $v$ has either an active principal or an active fixed process with $e$
%%   in its alphabet.  \framebox{OR} only fixed process.
%, and either the
%  principal or no component has $e$ in its alphabet.

\item If $e \ne \tau$ then $v'$ is the same as~$v$ except replacing every
  (fixed or component) process~$p$ that has $e$ in its alphabet with a process
  $p'$ such that \( p \trans[e] p' \).  This synchronises all relevant
  processes on the transition.

  If $e = \tau$ then $v'$ is the same as~$v$ except a single process~$p$,
  either a fixed process or the principal, is replaced  with a process
  $p'$ such that \( p \trans[\tau] p' \). 

  Note that in each case $v'$ might not be a view if the principal component
  gains or loses a reference as a result of the transition; we discuss this
  further below.
\end{itemize}
\end{definition}

\begin{impNote}
\texttt{system.transitions} generates a representation of the view
  transitions, together with the identity of another process that synchronises
  on the transition (where applicable).  We might be more restrictive with
  active-fixed-process transitions. 
\end{impNote}

For example, consider a transition of the queue where a thread~$t$
enqueueing~$x$ performs a $lock$ event.  This is a synchronisation between the
thread and lock processes, and can be captured by a view transition of the
form
%
\begin{equation}
(fixed(\_, n_h, n_l); Enq^x_1(t))  \trans(3)[lock.t]
  (fixed(t, n_h, n_l); Enq^x_2(t)) .
\label{trans:lock}
\end{equation}
%
%for $t \in ThreadID$.  
%
In this particular case, the view transition is also an abstract transition.
Each of the pre- and post-states in this transition is a view, with $t$ as the
principal component.  Further, the pre-view contains all components with
$lock.t$ in their alphabet (just the principal), so in a concrete state, no
other process would synchronise on the transition.

% \framebox{IMPROVE:} maybe consider the $setHead$ transition instead, here. 

As an example of a transition with an active fixed process, the construction
of the initial dummy header node is described by the following view
transition.
\[
\begin{align}
(\Lock, \Head(Null), \Last(Null), \Con_0; \InitNode(n)) 
  \trans(9)[initNode.C.n.v_0] \\
\qquad (\Lock, \Head(Null), \Last(Null), \Con_1(n); \Node_{v_0}(n, Null)).
\end{align}
\]
Here the only component that synchronises on the transition is the principal.
This view transition is again an abstract transition.

In some other cases, we will need to expand the view to include additional
relevant components: either other components that synchronise on the
transition, or to which the principal component obtains a reference.  Recall
(clause~\ref{assump:max-one-extra-component} of Assumption~\ref{assump:2})
that we assume that there is at most one such new component (for simplicity).
%% When we expand a view in this way, we do so only in a way that is compatible
%% with the set $V$ of views (in a sense that we define below).
%% The above two abstract transitions can be found by looking just at the view
%% transitions of the pre-view.  However, this isn't always the case.
Consider a transition where a dequeueing thread reads the (non-null) $next$
reference from the head node.  This can be partially described by the view
transition
\[
\begin{align}
(fixed(t, n_h, n_l); Deq_3(t, n_h), \Node_x(n_h, n))
  \trans(7)[getNext.t.n_h.n] \\
\qquad (fixed(t, n_h, n_l); Deq_4(t, n), \Node_x(n_h, n)).
\end{align}
\]
%% \begin{eqnarray*}
%% (\Lock'(t), Top(n); Push_3^x(t)) & \trans{getTop.t.n} &
%%   (\Lock'(t), Top(n); Push_4^x(t, n)).
%% \end{eqnarray*}
However, the post-state is not a view, as it does not contain the node~$n$
referenced by~$t$, and additionally it does include the node~$n_h$ which is no
longer referenced by~$t$.  Considering just the former point (for the moment),
we can consider the above view transition to be a transition with a ``hole'';
we call it a \emph{transition template}, and denote it as
\[
\begin{align}
(fixed(t, n_h, n_l); Deq_3(t, n_h), \Node_x(n_h, n), \bighole{n})
  \trans(7)[getNext.t.n_h.n] \\
\qquad (fixed(t, n_h, n_l); Deq_4(t, n), \Node_x(n_h, n), \bighole{n}).
\end{align}
\]
Each ``$\hole{n}$'' represents a hole to be filled by a component with
identity~$n$.  For example, we could fill the hole to produce a transition
\begin{equation}
\begin{align}
(fixed(t, n_h, n_l); Deq_3(t, n_h), \Node_x(n_h, n), \Node_y(n, n'))
  \trans(7)[getNext.t.n_h.n] \\
\qquad (fixed(t, n_h, n_l);   Deq_4(t, n), \Node_x(n_h, n), \Node_y(n, n')).
\label{trans:getNext}
\end{align}
\end{equation}
%
We call the above an \emph{extended transition}, and we call the pre- and
post-states \emph{extended views}: we have extended the views in the
transition template by adding the component for~$n$.
%% (Recall that we assume that it is necessary to add at most one such node,
%% i.e.~the principal component obtains at most one new reference in each
%% transition.)

The above transition also illustrates that the post-state might be strictly
larger than the corresponding view of the principal component: the thread
loses the reference to~$n_h$ in the transition.  It is straightforward to
remove components that are no longer referenced in order to obtain the
relevant view.  We write $viewOf(v)$ for the view extracted from~$v$. 

We can generate an extended transition by considering a transition template;
then extending the pre-state to add a consistent state of a component to which
the principal component acquires a reference; and adding the corresponding
state of each such component to the post state (here the added component does
not synchronise on the transition so remains in the same state).  We write $v
\uplus c$ for the result of extending view~$v$ with component state~$c$:
\begin{eqnarray*}
v \uplus c & \defs & (v.\fixed, v.\cpts \union \set{c}).
\end{eqnarray*}
%
We denote a generic transition template as $pre \uplus \hole{id} \trans[e]
post \uplus \hole{id}$. 

%% \begin{impNote}
%% The implementation represents a transition that requires an
%%   additional component by a \texttt{TransitionTemplate}.
%% \end{impNote}

The following definitions describe what we mean by \emph{consistent}: that the
extended view is included in some system state in~$\gamma(V)$.
%
\begin{definition}
Let $V$ be a set of views.  A component state~$c$ is \emph{consistent} with a
view~$v \in V$ (with $c.\id$ disjoint from the identities in~$v$) if there
exists a state $s \in \gamma(V)$ such that $v \uplus c \sqle s$.
\end{definition}
%
The above condition is rather expensive to check, so we weaken it slightly.
We first introduce a property, \emph{accordance}, which is a necessary
condition for two views to correspond to the same system state: they agree on
all common processes.
%
\begin{definition}
Two views $v$ and~$v'$ are \emph{accordant} if (1)~$v.\fixed = v'.\fixed$, and
(2)~if $v$ and~$v'$ both contain a component with a particular identity~$id$,
then those components are equal.  In this case we write $v \uplus v'$ for
$(v.\fixed, v.\cpts \union v'.\cpts)$.  
\end{definition}


\begin{definition}
\label{def:compatible}
Let $V$ be a set of views.  A component state~$c$ is \emph{compatible} with a
view $v \in V$ if:
%
\begin{enumerate}
\item\label{item:compatible-1} $V$ contains a view $v_c$ such that
$v_c.\princ = c$,
%$c$ is a component of~$v_c$, 
and $v$ and~$v_c$ are accordant.

%% (a)~$v_c.\fixed = v.\fixed$, (b)~$v_c.\princ = c$, and (c)~if $c$ has a
%% reference to a component of~$v$, then $v$ and~$v_c$ agree on that
%% component.

\item\label{item:compatible-2} If any component $c'$ of~$v$ has a reference
  to~$c$ then $V$ contains a view $v'$ such that (a)~$v'$ and~$v$ are
  accordant,
% $v'.\fixed = v.\fixed$,
  (b)~$v'.\princ = c'$, and (c)~$v'$ contains $c$.
\end{enumerate}
\end{definition}
%
Informally, the condition says that we can find individual views that
correspond to references within $v \uplus c$. 

We prove as Lemma~\ref{lem:consistent-implies-compatible} that consistency
implies compatibility.  The implementation creates extended transitions by
extending a view transition with a compatible state, thus (slightly)
over-approximating the transitions.  Our experience is that this approximation
does not lead to false positives.

\begin{improve}
In clause~(2), we could require accordance, requiring $v$ and $v'$ to agree on
any other components.  It's not clear if this is best. 
\end{improve}

In transition~(\ref{trans:getNext}), the state $\Node_y(n, n')$ is compatible
with the pre-view, assuming the set~$V$ of views contains: (1)~a view of the
form $v_c = (fixed(t, n_h, n_l);\linebreak[1] \Node_y(n, n'), \linebreak[1]
\Node_z(n', n''))$ for some~$z$ and~$n''$, to satisfy
clause~\ref{item:compatible-1}; and (2) a view of the form $v' = (fixed(t,
n_h, n_l); \Node_x(n_h, n), \linebreak[1] \Node_y(n, n'))$ to satisfy
clause~\ref{item:compatible-2}, i.e.~showing that the states of the two nodes
are compatible with one another.

\begin{impNote}
This is checked in
  \texttt{Extendability.\linebreak[1]compatible\-With}, with
  \texttt{Extendability.isExtendable} checking the latter clause.
\end{impNote}

In the transitions we have considered so far, the pre-view contains every
component that synchronises on the transition.  When this isn't the case, we
need to extend the pre-view to add each compatible state of such a component.
For example, $initNode$ transitions correspond to transition templates such as
\[
\begin{alignc}
(fixed(t, n_h, n_l);   Enq_3^x(t, n_l), \Node_y(n_l, n'), \bighole{n})
    \trans(7)[initNode_x.t.n.Null] \\
\qquad (fixed(t, n_h, n_l);
   Enq_4^x(t, n_l, n), \Node_y(n_l, n'), \bighole{n}).
\end{alignc}
\]
Here the hole needs to be instantiated with a component with identity~$n$
(different from any component in the view, so $n \ne n_l$),
which synchronises on the transition.  This must necessarily be a transition
from the $\InitNode$ state:
%
\begin{equation}
\begin{alignc}
(fixed(t, n_h, n_l);   Enq_3^x(t, n_l), \Node_y(n_l, n'), \InitNode(n))
    \trans(7)[initNode_x.t.n.Null] \\
\qquad (fixed(t, n_h, n_l);
   Enq_4^x(t, n_l, n), \Node_y(n_l, n'), \Node_x(n, Null)).
\end{alignc}
\label{trans:initNode}
\end{equation}

The following definition summarises the construction of this subsection.
%
\begin{definition}
\label{def:active-process-transition}
Given a set $V$ of views, for each $v \in V$, we consider each view transition
$v \trans[e] v'$ of~$v$.
%
We consider each identity~$id$, not matching the identity of a component
of~$v$ and such that
%
\begin{itemize}
\item $e$ is in the alphabet of~$id$; or

\item $v.princ$ acquires a reference to~$id$ in the transition.
\end{itemize}
%
By assumption, there is at most one such~$id$.

If there is no such $id$, then the view transition $v \trans[e] v'$ contains
all relevant components; we say that it is \emph{closed}.  Note that in this
case, every parameter in $v'$ is also in~$v$, by Assumption~\ref{assump:2}.
We extract from~$v'$ the view~$v'' = viewOf(v')$ of the principal component,
to create an abstract transition $v \trans[e] v''$.

If there is such an $id$, we build the transition template $v \uplus \hole{id}
\trans[e] v' \uplus \hole{id}$.  We then find each component state~$c$ from a
view of~$V$ such that $c$ has identity~$id$ and is compatible with~$v$.  We
form all pre-states $pre = v \uplus c$ by extending $v$ with each such
component state~$c$.  We form corresponding post-states $post$ by extending
$v'$ with the corresponding post-states: if $e$ is in the alphabet of~$id$
then we build $post = v' \uplus c'$ for each state~$c'$ such that $c \trans[e]
c'$; or otherwise $post = v' \uplus c$.  Thus we form extended transitions
$pre \trans[e] post$ including all the relevant component states.  Finally, we
extract from $post$ the view~$v'' = viewOf(post)$ of the principal component,
to create an abstract transition $v \trans[e] v''$.

We define an \emph{active-process transition} to be an abstract transition
built in either of these two ways.
\end{definition}

%%%%%%%%%%

In order to prove a result about such transitions, we first compare the
properties of compatibility and consistency.
%
\begin{lemma}
\label{lem:consistent-implies-compatible}
Let $V$ be a set of views.
If a component state~$c$ is consistent with a view~$v \in V$, then
$c$ is compatible with~$v$. 
\end{lemma}
%
\begin{proof}
Suppose $c$ is consistent with~$v$.  Then there exists $s \in \gamma(V)$ such
that $v \uplus c \sqle s$.  Taking $v_c = view(s, c)$ satisfies
clause~\ref{item:compatible-1} of Definition~\ref{def:compatible}: note that
$v_c \in \V$ and $v_c \sqle s \in \gamma(V)$ so $v_c \in V$, by definition
of~$\gamma$.  For each component $c'$ of~$v$ that has a reference to~$c$,
taking $v' = view(s,c')$ satisfies clause~\ref{item:compatible-2}: again $v'
\in V$.
\end{proof}

%%%%%%%%%%

\begin{lemma}
\label{lem:active-process-transitions}
Let $V$ be a set of views.  Consider an abstract transition
\[
\gamma(V) \ni s \trans[e] s' \sqge_{\V} v'
\]
such that one of the following holds for the concrete transition $s \trans[e]
s'$:
\begin{enumerate}
\item The principal of~$v'$ is the active component of the transition;

\item The transition has an active fixed process, and involves one passive
  component which is the principal of~$v'$;

\item The transition involves only the fixed processes.
\end{enumerate}
%
(Note that the first and third cases include the possibility that $e = \tau$.)
Then $v'$ is generated by an active-process transition.
\end{lemma}

%%%%%

\begin{proof}
Let $pId = v'.\princ.\id$ be the identity of the principal of~$v'$, and
suppose that component has states $princ$ and $princ'$ in~$s$ and~$s'$,
respectively (so $v' = view(s', princ')$). Let $v = view(s, princ)$.  Note
that $v \in \V$ and $v \sqle s \in \gamma(V)$, so $v \in V$.  We consider the
three cases of the lemma in order.
%
\begin{enumerate}
\item
Suppose another component is necessary for the transition, as in
Definition~\ref{def:active-process-transition}, i.e.~either another component
synchronises on the transition, or the principal gains a reference; and
suppose that component has state~$c$ in~$s$.  Then $c$ is consistent
with~$v$, since $v \uplus c \sqle s \in \gamma(V)$.  Hence $c$ is compatible
with~$v$, by Lemma~\ref{lem:consistent-implies-compatible}. 
%
Let $pre = v \uplus c$, or let $pre = v$ if no additional component is
necessary; and let $post$ be the corresponding states in~$s'$.  Then the
technique of Definition~\ref{def:active-process-transition} builds the
transition \( pre \trans[e] post \).  Finally, the principal's view of $post$
is extracted; this equals~$v'$ by construction, as required. 

\item Then no additional component is necessary, and the technique of
  Definition~\ref{def:active-process-transition} builds the transition $v
  \trans[a] v'$.

\item This is identical to the previous case.
\end{enumerate}
\end{proof}


%%%%%%%%%%%%%%%%%%%%%%%%%%%%%%%%%%%%%%%%%%%%%%%%%%%%%%%

As an aside, the converse of Lemma~\ref{lem:consistent-implies-compatible}
doesn't hold, informally because the views required for the different
constraints of Definition~\ref{def:compatible} might come from different
system states.  More concretely, consider a system containing just the
following system states.
%
\begin{eqnarray*}
s_0 & = &
   (fixed; Th(t,n_1,n_2), N_A(n_1,n_3), N_B(n_2), N_X(n_3), N_D(n_4,n_3)), \\
s_1 & = &
  (fixed; Th(t,n_1,n_2), N_X(n_1), N_B(n_2), N_C(n_3,n_2), N_D(n_4,n_1)),\\
s_2 & = & 
  (fixed; Th(t,n_1,n_2), N_A(n_1,n_3), N_X(n_2), N_C(n_3,n_2), N_D(n_4,n_2)).
\end{eqnarray*}
%
Let $V = \alpha(\set{s_0,s_1,s_2})$ be all views of these states.  Let
\[
v = (fixed; Th(t,n_1,n_2), N_A(n_1, n_3), N_B(n_2)).
\]
This is a view of the system, resulting from~$s_0$.  Let 
\[
c = N_C(n_3,n_2).
\]  
Then  $c$ is compatible with~$v$:
%
clause~\ref{item:compatible-1} of Definition~\ref{def:compatible} is satisfied
by $(fixed; N_C(n_3,n_2), N_B(n_2))$, which is a view resulting
from~$s_1$;
%
and clause~\ref{item:compatible-2} is satisfied for $c' = N_A(n_1,n_3)$ by
$(fixed; N_A(n_1, n_3), N_C(n_3,n_2))$, which is a view resulting
from~$s_2$. 

However $c$ is not consistent with~$v$.  Suppose we have \( v \uplus
c \sqle s \) for some $s \in \gamma(V)$.  Then $s$ is necessarily of the form
\[
\begin{align}
(fixed; Th(t,n_1,n_2), N_A(n_1, n_3), N_B(n_2), N_C(n_3,n_2), c_4) \\
\qquad \mbox{where $c_4 = N_D(n_4,x)$ for some $x$.}
\end{align}
\]
But no value of~$x$ works: in each case, we have that $view(s, c_4) \nin
\gamma(V)$, since in no system state does $n_4$ have a reference to a
component in state~$N_A$, $N_B$ or~$N_C$.
