\section{Views capturing a single reference}
\label{sec:singleRef}

In this section we consider the variation on our basic scheme, where each view
contains the state of only one secondary component referenced by the principal
(or none, if the principal references no component other than itself).  We
call these \emph{restricted views}, and use the term \emph{full views} for the
earlier version where we recorded all components referenced by the principal. 

We start by formalising restricted views.  Consider a system state~$s$, and a
particular component $princ \in s.\cpts$.
\begin{itemize}
\item 
Suppose $princ$ has no reference to another component.  Then we write $view(s,
princ)$ for the view of the system state from~$princ$, i.e.~the fixed processes
and $princ$ itself.
\begin{eqnarray*}
view(s, princ) & = &  (s.\fixed; princ).
\end{eqnarray*}
This is the same as in the case of full views.

\item
Now suppose $princ$ has $k > 1$ parameters (recall that its first parameter is
its own identity).  Suppose $1 < i \le k$, and the $i$th parameter, denoted
$princ.\params(i)$, is not a distinguished value.  Then we write $view(s, princ,
i)$ for the view containing $princ$ and the state of the component referenced by
its $i$th parameter.
\[
\begin{align}
view(s, princ, i)  =  (s.\fixed; princ, c)\\ 
\mbox{where $c \in s.\cpt$ is such that $c.\id = princ.\params(i)$}.
\end{align}
\]
\end{itemize}
%
We then define the set of views~$\V$ to be all such values:
%
\begin{eqnarray*}
\V & = & 
  \begin{align}
  \set{ view(s, princ) \| s \in \S, princ \in s.\cpts, \\
    \qquad   \mbox{$princ$ has no reference to another component}} \union\null
  \\
  \set{ view(s, princ, i) \| s \in \S, princ \in s.\cpts, 
    1 < i \le length(princ.\params), \\
    \qquad \mbox{$princ.\params(i)$ is not distinguished} }.
  \end{align}
\end{eqnarray*}

When the principal of a restricted view~$v$ has a reference~$id$ to a
component, but no  component with identity~$id$ is present in~$v$, then we
say that component is \emph{missing}.

\framebox{Road map.}  What aspects stay the same?  

%%%%%%%%%%%%%%%%%%%%%%%%%%%%%%%%%%%%%%%%%%%%%%%%%%%%%%%

\section{Building abstract transitions}

We now consider how to adapt the constructions from the previous section so as
to build abstract transitions from such views.  Some of the changes we outline
are necessary for correctness; others are necessary to avoid false positives.

\framebox{Road map}

%% Note that if a transition represents a synchronisation between the principal
%% and a secondary component, we generate that only from the view that includes
%% that secondary component. (?)


We adapt the definition of a view transition to specify that if the principal
synchronises with one of the components that it references, then that
secondary component must be included in the view transition.  The definition
of an extended transition is then as earlier. 
%
\begin{definition}
\label{def:active-process-transition-singleRef}
We define a \emph{view transition} of view~$v$ to be a transition $v \trans[e]
v'$ such that
%
\begin{itemize}
\item $v$ has either an active principal or an active fixed process with $e$
  in its alphabet;

\item If $v.\princ$ has a reference to a component with $e$ in its alphabet,
  then $v$ contains such a component;

\item $v'$ is the same as~$v$ except replacing every (fixed or component)
  process~$p$ that has $e$ in its alphabet with a process $p'$ such that \( p
  \trans[e] p' \).
\end{itemize}

Given a view transition $v \trans[e] v'$, we define the corresponding
\emph{transition templates} and \emph{extended transitions} as in
Definition~\ref{def:active-process-transition}.  Finally, we extract views of
the principal component.
\end{definition}

For example, consider the following view transition using full
views.
%
\begin{equation}
\begin{align}
\label{trans:setNext}
(fixed; Th(t, n, n'), Nd_x(n, null), Nd_y(n', n'')) 
  \trans(6)[setNext.t.n.n'] \\
\qquad (fixed'; Th'(t, n, n'), Nd_x(n, n'), Nd_y(n', n'')).
\end{align}
\end{equation}
%
When we use restricted views, we would include the following view transition,
which contains the node~$n$ that synchronises on the transition. 
%
\begin{equation}
\label{trans:setNext-singleRef}
\begin{align}
(fixed; Th(t, n, n'), Nd_x(n, null)) 
  \trans(6)[setNext.t.n.n'] 
 (fixed'; Th'(t, n, n'), Nd_x(n, n')).
\end{align}
\end{equation}
%
However we do not consider a view transition from
$(fixed; Th(t, n, n'), Nd_y(n', n''))$,
because that view does not contain a component for~$n$, which synchronises on
the transition.  Instead, such transitions are induced: see
Section~\ref{sec:singleRef-induced}. 

\begin{impNote}
\texttt{system.transitions} suppresses transitions that involve a
synchronisation with a missing component.
\end{impNote}

%%%%%%%%%%%%%%%%%%%%%%%%%%%%%%%%%%%%%%%%%%%%%%%%%%%%%%%
