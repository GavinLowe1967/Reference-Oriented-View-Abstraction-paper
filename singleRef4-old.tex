
%%%%%

More generally, if there are two renamings $\pi$ and~$\pi'$, where the two
induced views are equivalent, and the linkages produced by $\pi$ are a subset
of those produced by~$\pi'$, then there is no need to consider~$\pi'$.  In
this case, we say that the additional linkages produced by~$\pi'$ are
\emph{incidental}.  We aim to identify sufficient conditions for linkages
being incidental.  We consider more examples.

%%%%%

\begin{example}\label{example:30}
Now consider the normal-form transition
\begin{eqnarray*}
(fixed(N_0); Th(T_0,N_0), Nd_x(N_0,N_1)) & \trans{} &
  (fixed'(N_0); Th'(T_0,N_0), Nd_x(N_0,N_1))
\end{eqnarray*}
acting on 
\[
(fixed(N_0); Nd_y(N_1,N_2), Nd_z(N_2,N_0)).
\]
Again no unification is possible.  However, any unification function~$\pi$
necessarily contains $N_0 \mapsto N_0$, so creates a linkage via~$N_0$ that
would require a view equivalent to $(fixed(N_0); Nd_z(\pi(N_2),N_0),
Nd_x(N_0,N_1))$.  This linkage is not incidental because it uses the
parameter~$N_0$ in the fixed processes.
\end{example}

%%%%%

Similarly, a linkage involving a parameter that appears in $post.\fixed$ might
not be incidental.  
%

\framebox{***} Any unification that creates a cross reference must necessarily
be within $pre$ or~$v$.

%%%%%

\begin{example}
Consider the transition
\begin{eqnarray*}
(fixed; Th(T_0,N_1,N_2), Nd_x(N_1,N_3)) & \trans{} &
  (fixed; Th'(T_0,N_1), Nd_y(N_1,N_3,N_2))
\end{eqnarray*}
acting on
\[
(fixed; Nd_x(N_0,N_1), Nd_z(N_1,N_2) .
\]

Consider the mapping $\pi = \set{N_0 \mapsto N_1, N_1 \mapsto N_3, N_2 \mapsto
  N_2}$.  This unifies the two $Nd_x$ components.  It also creates a linkage
via $N_2$ which requires (for clause~(c) of
Definition~\ref{def:induced-transition-singleRef}), for some component~$c$
with identity~$N_2$, that the current set of representative views include
views equivalent to each of $(fixed; Th(T_0,N_1,N_2), c)$ and $(fixed;
Nd_y(N_3,N_2), c)$.  If such views do exist, this induces a transition to
\[
(fixed; Nd_y(N_1,N_3,N_2), Nd_z(N_3,N_2)) \equiv 
  (fixed; Nd_y(N_0,N_1,N_2), Nd_z(N_1,N_2)).
\]

Alternatively, consider the mapping $\pi' = \set{N_0 \mapsto N_1, N_1 \mapsto
  N_3,\linebreak[1] {N_2 \mapsto N_4}}$, which again unifies the $Nd_x$
components, but does not create a linkage via~$N_2$.  This induces a
transition to
\[
(fixed; Nd_y(N_1,N_3,N_2), Nd_z(N_3,N_4)) \equiv
  (fixed; Nd_y(N_0,N_1,N_2), Nd_z(N_1,N_3)).
\]
This is not the same as the view produced using~$\pi$.  Hence the linkage
via~$N_2$ produced by~$\pi$ is not incidental.  The reason for this is that
$N_2$ appears in two components of the induced view: one taken from $post$;
and one taken from~$v$.
\end{example}

%% How do we induce the transition to 
%% \[
%% (fixed; Nd_y(N_1,N_3,N_2), Nd_?(N_2,?)) ?
%% \]


%%%%%


\begin{definition}
\label{def:incidental-linkage}
Consider a transition $pre \trans{} post$ and an accordant view~$v$, with
neither necessarily in normal form.  Define a linkage using parameter~$x$
between a component of~$v$ and a component of~$pre$ to be \emph{significant}
if one of the following holds.
%
\begin{enumerate}
\item \label{item:incidental-linkage-fixed}
$x$ is a parameter of $pre.\fixed = v.\fixed$;

\item \label{item:incidental-linkage-common-component}
$x$ is a parameter of a component that appears in both $pre$ and $v$;

\item \label{item:incidental-linkage-post-fixed} $x$ is a parameter of
  $post.\fixed$;

\item \label{item:incidental-linkage-post-common-component} $x$ is a parameter
  of a component~$c'$ of~$post$ such that there is a component~$c$ in~$v$ with
  the same identity as~$c'$ ($c$ is necessarily also in~$pre$);

\item \label{item:incidental-linkage-acquired-reference} the principal of~$v$
  is also a component of~$pre$ and changes state in the transition to acquire
  a reference to a component~$c$ of~$post$, and $x$ is a component of~$c$;

\item $x$ is a linkage between component~$c$ of~$v$ and component~$c'$
  of~$pre$, and these components have a significant linkage via some other
  parameter~$y$ (this step is applied inductively).
\end{enumerate}
%
We say that a linkage is \emph{incidental} if it is not significant.
\end{definition}

\framebox{Just primary below?}

I'm not sure about \ref{item:incidental-linkage-post-common-component} and
\ref{item:incidental-linkage-acquired-reference}.  It might be when the
linkage variable appears in two components of the induced view, one taken
from~$post$ and one from~$v$.

%%%%%

\begin{lemma}
Consider a transition $pre \trans{} post$ acting on a view~$v_1$ to induce a
transition to~$v_1'$.  Let $\pi$ be a remapping that maps the parameters of
some incidental linkages to distinct fresh parameters (not in $pre$, $post$
or~$v_1$), but is the identity on all other parameters (including those of
significant linkages).  Let $v_2 = \pi(v_1)$.  Then the transition acting
on~$v_2$ induces a transition to $v_2' = \pi(v_1')$; note that $v_2' \equiv
v_1'$.

\framebox{***} I think I want to say something about the relative linkages.
\end{lemma}

Note that the lemma implies that the transition from~$v_2$ subsumes that
from~$v_1$, and involves fewer linkages. 

%%%%%

\begin{proof}
Our proof is in two parts: we show that \emph{if} the conditions of
Definition~\ref{def:induced-transition-singleRef} are satisfied for~$v_2$,
then the induced transition produces $v_2' = \pi(v_1')$; and we show that those
conditions are implied by the fact that they are satisfied for~$v_1$.

By condition~\ref{item:incidental-linkage-fixed} of
Definition~\ref{def:incidental-linkage}, there are no incidental linkages
involving parameters of $v_1.\fixed$.  Hence $\pi$ is the identity on the
parameters of $v_1.\fixed$, so $v_2.\fixed = v_1.\fixed = pre.\fixed$.

By condition~\ref{item:incidental-linkage-common-component}, there are no
incidental linkages involving any component~$c$ of~$v_1$ that also appears
in~$pre$.  Hence $\pi$ is the identity on the parameters of~$c$, so $c$ also
appears in~$v_2$.  Recall, that when $\pi$ is not the identity mapping, it
maps parameters to fresh parameters: it cannot remap a component identity
in~$v_1$ to match an identity in~$pre$.  Hence the common components between
$v_1$ and $pre$ are the same as between $v_2$ and~$pre$; so $v_2$ and $pre$
are accordant.

We now show that $\pi$ is the identity on all parameters of processes
of~$v_1'$ taken from $post$; and hence ---if the relevant side conditions are
satisfied, and so a transition from $v_2$ is indeed induced--- that $v_2'$
agrees with $v_1'$ on all such processes.
%
\begin{itemize}
\item This is true of the fixed processes, by
  condition~\ref{item:incidental-linkage-post-fixed}.

\item Suppose~$c'$ is a component of~$post$ for which there is a component~$c$
  with the same identity in~$v_1$ (necessarily, the same component~$c$ is in
  $pre$; and so it is also in~$v_2$ by the above).  Then $\pi$ is the identity
  on the parameters of~$c'$ by
  condition~\ref{item:incidental-linkage-post-common-component}.

\item
Suppose the principal~$p$ of~$v_1$ is also a component of~$pre$, changes state
in the transition, and acquires a reference to a component~$c$ of~$post$; then
this is true of~$c$, by
condition~\ref{item:incidental-linkage-acquired-reference}.
\end{itemize}



\end{proof}

