\section{Implementation}

The implementation uses the following types.
\begin{itemize}
\item $Transition = (Concretization, EventInt, Concretization)$.  The tuple
  $(pre, e, post)$ represents the extended transition $pre \trans{e} post$.
  The pre-state extends a view to include all other components synchronising
  on the transition, and all components to which a process gains a reference.
  The post-state gives the same components in their post-transition states. 

\item $TransitionTemplate = (Concretization, Concretization,\linebreak[1]
  Process\-Identity,\linebreak[1] EventInt, Boolean)$.  A transition template
  can be thought of as a transition with a ``hole'' for a component with a
  given identity to be slotted into.  The tuple $(pre, post, id, e, include)$
  represents an extended transition $pre \union \set{st} \trans{e} post \union
  \set{st'}$ for every state $st$ and $st'$ such that (1) $st$ and $st'$ have
  identity $id$; (2)~$st$ is compatible with $pre$; (3) if $include$ then $st
  \trans{e} st'$, otherwise $st = st'$.
\end{itemize}


The implementation includes the following state.
\begin{itemize}
\item $sysAbsViews: ViewSet$: all views found.

\item $transitions: TransitionSet$: the extended transitions found so far.

\item $transitionTemplates: TransitionTemplateSet$: the transition templates
  found so far. 

\item $nextNewViews: ArrayBuffer[ComponentView]$: new views found on the
  current ply, to be considered on the next ply.

\item $newTransitions: ArrayBuffer[Transition]$: new transitions found on the
  current ply.

\item $newTransitionTemplates: ArrayBuffer[TransitionTemplate]$: new
  transition templates found on the current ply.
\end{itemize}

In the first half of each ply, new views are added to $nextNewViews$; in the
secondhalf, they are added to $sysAbsViews$.  Likewise transitions and
transition templates are initially added to $newTransitions$ and
$new\-Transition\-Templates$, respectively.  This ensures that the sets are not
updated while we iterate over them.  On the next iteration, the program
iterates over the new views.

Figure~\ref{fig:impl} shows the relationship between some of the functions in
the implementation. 

\begin{figure}
\begin{center}
\small
\begin{tikzpicture}[>=angle 90,yscale = 1.5]
\draw (0,0) node (process) {$process$};
\draw (process) ++ (-4,-1) node (processTransition) {$processTransition$};
%
\path[draw,->] (process) -- (processTransition);
\draw (process) ++ (0,-1) node (effectPrevTT)
   {$\begin{array}{c} effectOfPrevious\mbox{-} \\ 
       TransitionTemplates\end{array}$}; 
\path[draw,->] (process) -- (effectPrevTT);
\draw (process) ++ (4,-1) node (effectPrevTrans) 
  {$\begin{array}{c} effectOfPrevious\mbox{-} \\ Transitions\end{array}$}; 
\path[draw,->] (process) -- (effectPrevTrans);
%
\draw (processTransition) ++ (1,-1) node (instantiateTT) 
  {$\begin{array}{c} instantiate\mbox{-}\\ TransitionTemplate \end{array}$};
\path[draw,->] (processTransition) -- (instantiateTT);
%
\draw (instantiateTT) ++ (0,-1) node (instantiateTTBy)
  {$\begin{array}{c} instantiate\mbox{-}\\ TransitionTemplateBy \end{array}$};
\path[draw,->] (instantiateTT) -- (instantiateTTBy);
\path[draw,<-] (instantiateTTBy.north)++(1.5,-0.3) -- (effectPrevTT);
%
\draw (instantiateTTBy) ++ (-1,-1) node (addTransition) {$addTransition$};
\path[draw,->] (instantiateTTBy) -- (addTransition);
\draw (addTransition)++(-1,0) node (addTransitionL) {};
\path[draw,->] (processTransition.south)++(-1,0) -- (addTransitionL);
%
\draw (addTransition) ++ (4,0) node (effectOnOthers) {$effectOnOthers$};
\path[draw,->] (addTransition) -- (effectOnOthers);
%
\draw (effectOnOthers) ++ (4,0) node (effectOn) {$effectOn$}; 
\path[draw,->] (effectOnOthers) -- (effectOn);
\path[draw,->] (effectPrevTrans) -- (effectOn);
\end{tikzpicture}
\end{center}
\caption{Relationship of functions}
\label{fig:impl}
\end{figure}

The function $process(v)$ acts on a single view~$v$ found on the previous ply.
It calculates all the transitions from~$v$, together with the identities of
other components relevant to the view (if any).  It then calls
$processTransition$ on each such transition.  It also calls
$effectOfPreviousTransitions$ and $effect\-Of\-Previous\-TransitionTemplates$
on the view.

The function $processTransition(pre, e, post, ids)$ corresponds to processing
either (1)~the transition $pre \trans{e} post$, if $ids$ is empty, or
(2)~transition template $(pre, post, id, e, true)$ if $ids = \set{id}$,
i.e.~transitions with an extra component with identity~$id$ that synchronises
on~$e$.  It calculates whether the principal component gains a reference to a
component other than those in~$pre$ and~$id$.  If not and $ids$ is empty, the
transition is complete: $addTransition$ is called on it.  Otherwise, it is
added to $newTransitionTemplates$, and
$instantiate\-TransitionTemplate$ is called.

$addTransition(pre,e,post)$ adds a transition $pre \trans{e} post$ to
$new\-Transitions$, and calls $effectOnOthers$.

$effectOnOthers(pre,e,post)$ considers the effect of a new transition $pre
\trans{e} post$ on views found on previous plys.  For each such view that
matches $pre$'s fixed processes, it calls $effectOn$.

$effectOn(pre, e, post, v)$ considers the effect of a transition $pre
\trans{e} post$ on another view~$v$.  It forms all ways of combining $pre$ and
$v$, calculates the corresponding post-view for $v.\princ$, and adds it to
$nextNewViews$.

$instantiateTransitionTemplate(pre, post, id, e, inc)$ considers how to
instantiate the transition template $(pre, post,\linebreak[1] id,\linebreak[1]
e, inc)$.  For each view in $sysAbsViews$ that matches $pre$'s fixed
processes, it calls $instantiate\-Transition\-TemplateBy$.

$instantiateTransitionTemplateBy(pre, post, id, e, inc, v)$ considers how
to instantiate the transition template $(pre, post,\linebreak[1]
id,\linebreak[1] e, inc)$, so that a component of $v$ forms the extra
component with identity~$id$.  For each resulting transition, it calls
$add\-Transition$.

$effectOfPreviousTransitions(v)$ considers the effect of all previous
transitions on a view~$v$ found on the previous ply.  For each
transition in $transitions$ whose initial servers matches $v$'s servers, it
calls $effectOn$.

$effectOfPreviousTransitionTemplates(v)$ considers using a view~$v$ found on
the previous ply to instantiate the missing component in a transition
template.  For each transition template in $transitionTemplates$ whose initial
servers matches $v$'s servers, it calls $instantiateTransitionTemplateBy$.

We briefly discuss the combination of views and either transitions or
transition templates: it is important that we consider each such combination,
just once, regardless of the plys each are found or created on.  Consider a
view~$v$ found on ply~$n$, and expanded on ply~$n+1$.
%
\begin{itemize}
\item On ply~$n+1$, $v$ will be considered (in $effectOfPreviousTransitions$)
  in combination with each transition created up to ply~$n$.

\item For each transition created on ply~$n+1$ or later, its effect on~$v$
  will be considered (in $effectOnOthers$) on the ply in which the transition
  is created. 
 %% A transition from~$v$ will be considered in ply~$n+1$, and its effect
 %%  considered (in $effectOnOthers$) on all views~$v'$ found up to ply~$n$.
\end{itemize}
%
Thus the combination of a view and a transition will be treated: (1)~by the
first bullet if the transition is created on a ply no later than the view is
found; and (2)~by the second bullet if the transition is created on a ply
later than the view is found.  

Likewise
%
\begin{itemize}
\item On ply $n+1$, $v$ will be considered (in
  $effect\-Of\-Previous\-Transition\-Templates$) to create the missing
  component in each transition template created up to ply~$n$.

\item For each transition template created on ply~$n+1$ or later, its
  combination with~$v$ will be considered (in
  $instantiate\-Transition\-Template$) on the ply in which the transition
  template is created.
%% \item A transition template from~$v$ will be created in ply~$n+1$, and all
%%   ways of creating the missing component from views found up to ply~$n$ will
%%   be considered (in $instantiateTransitionTemplate$).
\end{itemize}
%
Thus the combination of a view and a transition template will be treated:
(1)~by the first bullet if the transition template is created on a ply no
later than the view is found; and (2)~by the second bullet if the transition
template is created on a ply later than the view is found.

%%%%%%%%%%%%%%%%%%%%%%%%%%%%%%%%%%%%%%%%%%%%%%%%%%%%%%%

\subsection{instantiateTransitiontemplateBy}

$instantiateTransitionTemplateBy(pre, post, id, e, inc, v)$ considers how
to instantiate the transition template $(pre, post,\linebreak[1]
id,\linebreak[1] e, inc)$, so that a component of $v$ forms the extra
component with identity~$id$.  For each resulting transition, it calls
$add\-Transition$.


The function $consistentStates(pre, id, e, inc, v)$ considers how to rename
$v$ to potentially produce the extra component of the transition template
$(pre, post,\linebreak[1] id,\linebreak[1] e, inc)$.  More precisely, it finds
all ways of renaming~$v$ to obtain a view~$v'$ so that (1) a component of~$v'$
has identity~$id$, (2)~$v'$ and~$pre$ agree on common components, and (3)~if
$inc$ then $v'$ can perform~$e$.  For each, it returns the renamed component
with identity id, and the next states. 

For each such extra component~$st$,\, $isExtendable(pre, st)$ tests whether
$st$ is consistent with $pre$, in the sense that for each component $c$ of
$pre \union st$, the current set of views contains a view with $c$ as
principal component and agreeing on common processes (modulo renaming).

$compatibleWith(servers, components, st)$ tests whether there is an existing
view with $st$ as principal component that agrees with $servers$ and
$components$ on common processes. 

$containsReferencingView(pre, st, j)$ assumes that $pre.components(j)$ has a
reference to $st$.  If tests whether there is a view with $pre.components(j)$
as principal component, containing $st$, and otherwise compatible with~$pre$
(modulo renaming).

*** If the views that ensure consistency are found later, do we get all
instantiations?  I think so.  The last relevant view will include the state we
want. 
