\subsection{Systems}

Each system consists of a fixed process and some number of components.  We
assume a \emph{single} fixed process here, for simplicity: a system with
multiple fixed processes can be modelled by considering the parallel
composition of those processes as a single process.

\begin{definition}
A collection of systems is defined by
%
\begin{itemize}
\item A set $T$ of identities, partitioned into different subtypes;
%%  let $\hat{T}$ be the non-distinguished values of~$T$ (i.e., real
%% identities);

\item A symmetric parameterised state machine $M_f = (Q_f, \Sigma_f,
  \delta_f)$ for the fixed process;

\item For each $id \in T$ that is not a distinguished value, a symmetric
  parameterised state machine $M_{id} = (Q_{id}, \Sigma_{id}, \delta_{id})$
  for the component with identity~$id$; we require that for every $q \in
  Q_{id}$,\, $q.\id = id$ (i.e.~$id$ is the identity in all states);

\item A set $Init \subseteq \S$ of initial system states, where $\S$ is the
  set of all system states, defined below. 
\end{itemize}

We require that the state machines for components of the same family are
symmetric to one another: for every remapping~$\pi$,\,
\[
\set{ (cs(\vec{x}), cs(\pi(\vec{x}))) \| 
  cs(\vec{x}) \in \textstyle\Union_{id \in \hat{T}} Q_{id}}
\]
is a $\pi$-bisimulation.

Let $\set{id_1,\ldots,id_n}$ be a subset of the non-distinguished
values of~$T$.  A corresponding state of the system comprises a state~$q_f$ of
the fixed process, and, for each~$i$, a state~$q_i$ for the component with
identity~$id_i$, i.e.~a pair $(q_f, \set{q_1,\ldots,q_n})$.  For ease of
notation, we often denote this state by \( (q_f; q_1,\ldots,q_n) \). We write
$\S$ for the set of all system states.  Given a system state~$s$, we write
$s.\fixed$ and $s.\cpt$ for the fixed processes and components, respectively.

We now define the transitions of the systems.  Let $\Sigma = \Sigma_f \union
\Union_i \Sigma_{id_i}$ be all visible events of processes (fixed or
components).  For $a \in \Sigma$, a transition synchronises all processes that
have $a$ in their alphabet.  Processes can perform internal transitions
(event~$\tau$) without synchronising.  (We write subscripts on the transition
arrows to indicate the relevant state machine.)
\[
\begin{array}{c}
\begin{infrule}
\If a \in \Sigma_f \ \THEN q_f \trans(1)[a]_f q'_f \ \ELSE q'_f = q_f \\
\forall i \in 1 \upto n \spot 
  \If a \in \Sigma_{id_i} \ \THEN q_{i} \trans(1)[a]_i q'_{i} 
  \ \ELSE q'_{i} = q_{i} \\
a \in \Sigma
\derive
(q_f; q_{1}, \ldots, q_{n}) \trans(1)[a] (q_f'; q_{1}', \ldots, q_{n}')
\end{infrule} 
%
\bigskip \\
%
\begin{infrule}
q_f \trans(1)[\tau]_f q'_{f}
\derive
(q_f; q_{1}, \ldots q_{n}) \trans(1)[\tau] (q_f'; q_{1}, \ldots q_{n})
\end{infrule}
%
\qquad
%
\begin{infrule}
q_{i} \trans(1)[\tau]_i q'_{i} 
\derive
(q_f; q_{1}, \ldots, q_i, \ldots q_{n}) \trans(1)[\tau]
  (q_f; q_{1}, \ldots, q_i', \ldots q_{n})
\end{infrule}
\end{array}
\]
%% The initial state is $(q^0_{f}; q^0_{id_1}, \ldots, q^0_{id_n})$.
\end{definition}

%% Each system can be defined by the set of identities of the components.  Each
%% component is started in the initial state for the corresponding state
%% machine. 

%% %%  We say that $s$ \emph{references} $id'$ if $id'$ is a parameter of~$s$
%% %% other than its own identity.

%% \begin{definition} 
%% A \emph{system state} is a pair $(fixed, cpts)$ where $fixed$ is the state of
%% the fixed process, and $cpts$ is a set of states of components.  Given a
%% system state~$s$, we write $s.\fixed$ and $s.\cpt$ for the fixed processes or
%% components, respectively.  We write $\S$ for the set of all system states.
%% \end{definition}

Note that each initial system state defines the identities of the components
in the system; and these identities are preserved by transitions.  Thus by
varying the initial state, we consider a collection of systems.  In the queue
example, each initial state $s_0$ could be defined by a set $NodeId$ of node
identities and a set $ThreadId$ of thread identities, so 
\[ s_0 =
  \begin{align}
  ( \Lock, \Head(Null), \Last(Null), \Con_0; \\
  \quad  \set{InitNode(n) \| n \in NodeID} \union 
    \set{Thread(t) \| t \in ThreadId} )
  \end{align}
\]
(where the first term represents the parallel composition of the
fixed processes). 

Given a collection of systems, for $S \subseteq \S$, we define 
\[
post(S) = \set{ s' \| 
  \exists s \in S, a \in \Sigma \union \set{\tau} \spot s \trans[a] s'},
\]
i.e.~the post-image of $S$ under transitions.  Let $post^*$ be the reflexive
transitive closure of~$post$,\, so $post^*(S) = \Union_{i \ge 0}
post^i(S)$.  Let $\R$ be the set of all states reachable from an initial
state by zero or more transitions, $\R = post^*(Init)$.

Recall that our models use a distinguished event $error$.  Our correctness
condition, then, is that no reachable state has a transition on~$error$.


%%%%%%%%%%%%%%%%%%%%%%%%%%%%%%%%%%%%%%%%%%%%%%%%%%%%%%%

\subsection{Assumptions}

We now outline some assumptions we make about the model.  These assumptions
make the subsequent theory and implementation simpler, without notably
reducing the range of systems that can be analysed.  

We assume that components are partitioned into \emph{active} and
\emph{passive} components.  Informally, active components control which events
occur, and passive components react; in our examples, threads are active, and
nodes are passive.  Likewise, we assume that fixed processes are partitioned
into active and passive processes; in our earlier examples, the constructor is
active, but all other fixed processes are passive.  The analyst is responsible
for classifying processes as active or passive. 

\begin{assumption}
We assume that each transition involves either:
%
\begin{enumerate}
\item[1a.] precisely one active component or active fixed process;
\item[1b.] at most one passive component;
\item[1c.] possibly other passive fixed processes;
\end{enumerate}
%
or
\begin{enumerate}
\item[2.] only fixed processes, and no components.
\end{enumerate}
%
More formally, each visible event~$a$ is in alphabets of processes as above. 
\end{assumption}
%
These assumptions hold of our earlier examples, and of other examples we have
considered.  They seem like reasonable assumptions, given our informal
definition of an active process, and that in most cases, an active process
interacts with one component at a time.  The first point helps to make the
implementation more efficient.  The second point simplifies the theory and
implementation.  The third point, allowing arbitrary many fixed processes to
synchronise on an event, is more liberal: it is useful in some examples (for
example, allowing the watchdog to see a synchronisation that also involves
other fixed processes); and it does not add extra difficulties.  The second
disjunct is useful for the distinguished event~$error$: this event is
typically performed by a fixed watchdog processes, which is passive, so such
transitions do not fit into the first disjunct.


\begin{improve}
check this in implementation.
\end{improve}

A component may acquire a reference to another component in a transition.  In
the queue example, a thread acquires a reference to a node when it initialises
that node, reads the |head| or |last| variable, or reads the |next| field of a
node; a node acquires a reference to another node when a thread sets its
|next| reference.  We also allow the possibility that a component~$c$
synchronises on a transition even though the active component holds no
reference to~$c$ either before or after the transition.  The following
assumption places limits on these. 
%
\begin{assumption}
\label{assump:2}
We assume, in each transition
\begin{enumerate}
\item\label{assump:max-one-extra-component} There is at most one component to
  which the active component doesn't initially hold a reference, and that
  either synchronises on the transition, or to which the active component
  gains a reference.

\item\label{assump:secondary-cpts-new-refs} A passive component can gain a
  reference only to a component to which the active process initially holds a
  reference.
\end{enumerate}
\end{assumption}

%%   We assume (for simplicity) that a component can acquire at
%% most one such new reference in a transition.   

%% For a transition with an active component~$princ$, assume that at most one
%% component~$c$ to which $princ$ does not initially have a reference, and such
%% that either $c$ synchronises on the transition or $princ$ acquires a
%% reference.
%% %
%% Also assume no other process acquires a reference to a component to which
%% $princ$ does not initially hold a reference.  

%% For a transition with an active server, ?? in view transitions, take the
%% principal to be the synchronising passive component if any, or arbitrary
%% otherwise.  Assume that process acquires at more one reference from the
%% transition ??? 
%% List of assumptions:
%
%% \begin{assumption}\label{assump}
%% \begin{enumerate}
%% \item Each concrete transition involves exactly one active process (either a
%% component or fixed process).

%% \item Each transition involves at most one passive component.

%% \item Each transition can involve any number of passive fixed processes.

%% \item\label{assump:max-one-extra-component}
%% For each transition there is at most one component to which the
%% active component doesn't initially hold a reference, and that either
%% synchronises on the transition, or to which the active component gains a
%% reference.

%% \item\label{assump:secondary-cpts-new-refs}
%% In each transition, a passive component can gain a reference only to a
%% component to which the active process initially holds a reference.
%% \end{enumerate}
%% \end{assumption}


