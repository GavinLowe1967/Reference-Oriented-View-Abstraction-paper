\subsection{Systems}

We now describe how systems, and their semantics, are defined. 
Each system consists of the fixed processes and some number of components. 
%%  We assume a \emph{single} fixed process in this section, for simplicity: a
%% system with multiple fixed processes can be modelled by considering the
%% parallel composition of those processes as a single process.
%
\begin{definition}
\label{def:system}
A collection of systems is defined by
%
\begin{itemize}
\item A set $\T$ of identities, partitioned into different subtypes, and
  including some distinguished values;
%%  let $\hat{T}$ be the non-distinguished values of~$T$ (i.e., real
%% identities);

\item A symmetric parameterised state machine $M_f = (Q_f, \Sigma_f, q_f^0,
  \delta_f)$ for the fixed processes.  (We assume a single state machine for
  the combined fixed processes, for simplicity; in practice, it is constructed
  from state machines for the individual fixed processes.)

\item For each $id \in \hat\T$ (i.e.~excluding distinguished values), a
  symmetric state machine $M_{id} = (Q_{id}, \Sigma_{id}, q_{id}^0,
  \delta_{id})$ for the component with identity~$id$.  We require that for
  every $q \in Q_{id}$,\, $q.\id = id$ (i.e.~$id$ is the identity in all
  states).

%% \item A set $Init \subseteq \S$ of initial system states, where $\S$ is the
%%   set of all system states, defined below. 
\end{itemize}

We require that the state machines for components of the same family are
symmetric to one another: for every remapping~$\pi$, the relation
\[
\set{ (cs(\vec{x}), cs(\pi(\vec{x}))) \| 
  cs(\vec{x}) \in \textstyle\Union_{id \in \hat{\T}} Q_{id}}
\]
is a $\pi$-bisimulation; and also this relation includes $(q^0_{id},
q^0_{id'})$ whenever $id$ and~$id'$ are from the same family (so components
from the same family start in the same control state).

Each particular system is defined by a set $T \subseteq \T$ of identities,
including each of the distinguished values.  For instance, in the queue
example we might have
%
\begin{eqnarray*}
T & = & \set{T_0, T_1, C, N_0, N_1, N_2, Null}. \\
%\hat{T} & = & \set{T_0, T_1, N_0, N_1, N_2}.
\end{eqnarray*}
%
The idea is that $\hat{T}$ represents the identities of components actually
present in the system.

Suppose $\hat{T} = \set{id_1,\ldots,id_n}$.  A corresponding state of the
system comprises a state~$q_f$ of the fixed processes, and, for each~$i$, a
state~$q_i$ for the component with identity~$id_i$.  Thus a system state is a
pair $(q_f, \set{q_1,\ldots,q_n})$; for ease of notation, we often denote this
state by $(q_f; q_1,\ldots,q_n)$.  Given a system state~$s$, we write
$s.\fixed$ and $s.\cpt$ for the fixed processes and components, respectively.
The initial state of the system is $(q^0_{f}; q^0_{id_1}, \ldots,
q^0_{id_n})$.  We write $\S$ for the set of all system states, and $Init$ for
the set of all initial states, as $T$ ranges over subsets of~$\T$.

%% Note that each initial system state defines the identities of the components
%% in the system; and these identities are preserved by transitions.  Thus by
%% varying the initial state, we consider a collection of systems.  In the queue
%% example, each initial state $s_0$ could be defined by a set $NodeId$ of node
%% identities and a set $ThreadId$ of thread identities, so 
%% \[ s_0 =
%%   \begin{align}
%%   ( \Lock, \Head(Null), \Last(Null), \Con_0; \\
%%   \quad  \set{InitNode(n) \| n \in NodeID} \union 
%%     \set{Thread(t) \| t \in ThreadId} )
%%   \end{align}
%% \]
%% (where the first term represents the parallel composition of the
%% fixed processes). 

We now define the transitions of the systems.  Fix a set $\hat{T} =
\set{id_1,\ldots,id_n} \subseteq \hat\T$ of component identities.  We build
transitions with events from the set $\Sigma \union \set\tau$, where
\begin{eqnarray*}
\Sigma & = & 
 (\Sigma_f \union \textstyle\Union_{id \in \hat{T}} \Sigma_{id}) -
  (\textstyle\Union_{id \in \hat{\T} - \hat{T}} \Sigma_{id}) .
\end{eqnarray*}
Note that we exclude events using identities outside $\hat{T}$, since these
represent components that are not in the system under consideration.  For $a
\in \Sigma$, a transition labelled with~$a$ synchronises all processes that
have $a$ in their alphabet, as captured by the following rule (we write
subscripts on the transition arrows to indicate the relevant state machine):
\[
% \begin{array}{c}
\begin{infrule}
\If a \in \Sigma_f \ \THEN q_f \trans(1)[a]_f q'_f \ \ELSE q'_f = q_f \\
\forall i \in 1 \upto n \spot 
  \If a \in \Sigma_{id_i} \ \THEN q_{i} \trans(1)[a]_i q'_{i} 
  \ \ELSE q'_{i} = q_{i} \\
a \in \Sigma
\derive
(q_f; q_{1}, \ldots, q_{n}) \trans(1)[a] (q_f'; q_{1}', \ldots, q_{n}')
\end{infrule} 
\]
Processes can perform internal transitions (event~$\tau$) without
synchronising:
\[
%\bigskip \\
\begin{infrule}
q_f \trans(1)[\tau]_f q'_{f}
\derive
(q_f; q_{1}, \ldots q_{n}) \trans(1)[\tau] (q_f'; q_{1}, \ldots q_{n})
\end{infrule}
%
\qquad
%
\begin{infrule}
q_{i} \trans(1)[\tau]_i q'_{i} 
\derive
(q_f; q_{1}, \ldots, q_i, \ldots q_{n}) \trans(1)[\tau]
  (q_f; q_{1}, \ldots, q_i', \ldots q_{n})
\end{infrule}
%\end{array}
\]
%% The initial state is $(q^0_{f}; q^0_{id_1}, \ldots, q^0_{id_n})$.
\end{definition}

For $S \subseteq \S$, we define 
\[
post(S) = \set{ s' \| 
  \exists s \in S, a \in \Sigma \union \set{\tau} \spot s \trans[a] s'},
\]
i.e.~the post-image of $S$ under transitions.  Let $post^*$ be the reflexive
transitive closure of~$post$,\, so $post^*(S) = \Union_{i \ge 0}
post^i(S)$.  Let $\R$ be the set of all states reachable from an initial
state by zero or more transitions, $\R = post^*(Init)$.

Recall that our models use a distinguished event $error$.  Our correctness
condition, then, is that no reachable state has a transition on~$error$.

%% \framebox{Do we need the following?} -- I think not

%% Consider the system defined by a particular set $\hat T$ of identities.  We now
%% make some assumptions that ensure that these are the only non-distinguished
%% identities that appear in reachable states or transitions.  Informally, these
%% say that new identities do not appear from nowhere.
%% %
%% \begin{enumerate}
%% \item The initial state $q_f^0$ of the fixed processes contains no identity.
%%   For each $id \in \hat{T}$,  the initial state $q_{id}^0$ for the component
%%   with identity~$id$ contains a single parameter, namely~$id$.

%% \item For every system transition, every identity in either the event or the
%%   post-state of one of the synchronising processes is also in the pre-state of
%%   one of the synchronising processes.
%% \end{enumerate}
%% %
%% For instance, consider a transition with event $getNext.t.n.n'$ from the queue
%% example, which is a synchronisation between thread~$t$ and node~$n$.  The
%% identity~$t$ is in the pre-state of~$t$; the identity~$n$ is in the pre-state
%% of both components; the identity~$n'$ is in the pre-state of~$t$.  The
%% post-state of~$t$ acquires a reference to~$n'$, but this is in the pre-state
%% of~$n$.

%% Given this assumption, an easy induction shows that every identity in every
%% reachable state is from~$\hat{T}$. 

%%%%%%%%%%%%%%%%%%%%%%%%%%%%%%%%%%%%%%%%%%%%%%%%%%%%%%%

\subsection{Assumptions}

We now outline some assumptions we make about the model.  These assumptions
make the subsequent theory and implementation simpler, without notably
reducing the range of systems that can be analysed.  

We assume that components are partitioned into \emph{active} and
\emph{passive} components.  Informally, active components control which events
occur, and passive components react; in our examples, threads are active, and
nodes are passive.  Likewise, we assume that fixed processes are partitioned
into active and passive processes; in our earlier examples, the constructor is
active, but all other fixed processes are passive.  The analyst is responsible
for classifying processes as active or passive. 

\begin{assumption}
We assume that each transition on a visible event~$a$ involves either:
\begin{enumerate}
\item (a)~precisely one active component or active fixed process; (b)~at most
  one passive component; and (c)~possibly other passive fixed processes; or

\item only fixed processes, and no components.
\end{enumerate}
%% %
%% \begin{enumerate}
%% \item[1a.] precisely one active component or active fixed process;
%% \item[1b.] at most one passive component;
%% \item[1c.] possibly other passive fixed processes;
%% \end{enumerate}
%% %
%% or
%% \begin{enumerate}
%% \item[2.] only fixed processes, and no components.
%% \end{enumerate}
%
More formally, the event~$a$ is in alphabets of processes as above. 
\end{assumption}
%
These assumptions hold of our earlier examples, and of other examples we have
considered.  They seem like reasonable assumptions, given our informal
definition of an active process, and that in most cases, an active process
interacts with one component at a time.  Item~1(a) helps to make the
implementation more efficient.  Item~1(b) simplifies the theory and
implementation.  Item~1(c), allowing arbitrary many fixed processes to
synchronise on an event, is more liberal: it is useful in some examples (for
example, allowing the watchdog to see a synchronisation that also involves
other fixed processes); and it does not add extra difficulties.  Item~2 is
useful for the distinguished event~$error$: this event is typically performed
by a fixed watchdog processes, which is passive, so such transitions do not
fit into the first disjunct.


\begin{improve}
check this in implementation.
\end{improve}

A component may acquire a reference to another component in a transition.  In
the queue example, a thread acquires a reference to a node when it initialises
that node, reads the |head| or |last| variable, or reads the |next| field of a
node; a node acquires a reference to another node when a thread sets its
|next| reference.  We also allow the possibility that a component~$c$
synchronises on a transition even though the active component holds no
reference to~$c$ either before or after the transition.  The following
assumption places limits on these. 
%
\begin{assumption}
\label{assump:2}
We assume, in each transition
\begin{enumerate}
\item\label{assump:max-one-extra-component} There is at most one component to
  which the active component doesn't initially hold a reference, and that
  either synchronises on the transition, or to which the active component
  gains a reference.

\item\label{assump:extra-event-field} In the case that there is an additional
  component as in the previous item, then every identity in the event is
  either a reference initially held by the active component, or the identity
  of that additional component.

\item\label{assump:secondary-cpts-new-refs} A passive process can gain a
  reference only to a component to which the active process initially holds a
  reference.
\end{enumerate}
\end{assumption}

