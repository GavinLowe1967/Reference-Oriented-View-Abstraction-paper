\subsection{Significant and incidental linkages}

\framebox{Introduce} term \emph{linkage} earlier.

Having to deal with linkages when considering induced transitions adds
considerable complexity to the checking procedure.  In this section we show
that it is sound not to consider certain combinations of transitions and views
when applying symmetry reduction: we show that in such cases equivalent views
can be produced by other induced transitions with fewer linkages.  

We start by considering when two different extensions of the same partial
unification function produce equivalent primary induced views.  The following
examples help to illustrate. 

\begin{example}\label{example:29}
Consider the normal-form transition
\begin{eqnarray*}
(fixed(N_0); Th(T_0,N_1), Nd_x(N_1,N_2)) & \trans{} &
  (fixed'(N_0); Th'(T_0,N_1), Nd_x(N_1,N_2))
\end{eqnarray*}
acting on 
\[
(fixed(N_0); Nd_y(N_1,N_2), Nd_z(N_2,N_3)).
\]
No unification is possible.  If we consider the mapping $\pi_1 = \set{N_0
  \mapsto N_0, \linebreak[1] {N_1 \mapsto N_3,} \linebreak[1] N_2 \mapsto N_4,
  N_3 \mapsto N_5}$ (mapping all parameters other than those in the fixed
processes to fresh values), then this induces a transition to
\[
(fixed'(N_0); Nd_y(N_3,N_4), Nd_z(N_4,N_5)) \equiv 
  (fixed'(N_0); Nd_y(N_1,N_2), Nd_z(N_2,N_3)).
\] 

Consider instead the mapping $\pi_2 = \set{N_0 \mapsto N_0, N_1 \mapsto N_3,
  N_2 \mapsto N_4,\linebreak[1] {N_3 \mapsto N_1}}$.  This mapping produces a
linkage that would require a view equivalent to $v_\cross = (fixed(N_0);
Nd_z(N_4,N_1), Nd_x(N_1,N_2))$.  If such a view exists, the induced transition
produces
\[
(fixed'(N_0); Nd_y(N_3,N_4), Nd_z(N_4,N_1)) \equiv 
  (fixed'(N_0); Nd_y(N_1,N_2), Nd_z(N_2,N_3)).
\]
The two mappings produce equivalent views.  Thus it is enough to consider one
of them.  We consider only the mapping~$\pi_1$, as it subsumes~$\pi_2$, and is
simpler to deal with as it does not require the additional view equivalent
to~$v_\cross$.

The same is true of all remappings with $N_1$ and/or $N_2$ in the range (there
are seven in total).
\end{example}

%%%%%

\begin{example}\label{example:singleRef-remapping-post-servers}
Consider the transition
% 
\begin{eqnarray*}
(fixed(N_0); Th(T_0,N_1), Nd_x(N_1,N_2)) & \trans{} &
  (fixed'(N_1); Th'(T_0,N_1), Nd_x(N_1,N_2))
\end{eqnarray*}
acting on
\[
v = (fixed(N_0); Nd_y(N_1,N_2), Nd_z(N_2,N_3)).
\]
Again no unification is possible.  The unification function $\pi_1 = \set{N_0
  \mapsto N_0, \linebreak[1] {N_1 \mapsto N_3}, \linebreak[1] N_2 \mapsto N_4,
  \linebreak[1] {N_3 \mapsto N_5}}$ induces a transition to
\[
(fixed'(N_1); Nd_y(N_3,N_4), Nd_z(N_4,N_5)) \equiv
(fixed'(N_0); Nd_y(N_1,N_2), Nd_z(N_2,N_3)).
\]

Alternatively, the unification function $\pi_2 = \set{N_0 \mapsto N_0, N_1
  \mapsto N_3, N_2 \mapsto N_4, \linebreak[1] {N_3 \mapsto N_1}}$  creates
a linkage via~$N_1$, requiring a view equivalent to $(fixed(N_0);
Nd_z(N_4,N_1), Nd_x(N_1,N_2))$.  If such a view does exist, the induced
transition produces
\[
(fixed'(N_1); Nd_y(N_3,N_4), Nd_z(N_4,N_1)) \equiv
(fixed'(N_0); Nd_y(N_1,N_2), Nd_z(N_2,N_0)).
\]
This is not equivalent to the view produced previously.  Thus it is necessary
to consider both unification functions.

The critical point is that $\pi_2$ maps~$N_3$ to a value in $post.\fixed$.
\end{example}

%% the parameter~$N_1$ of the linkage appears twice in
%% the induced view (prior to reduction to normal form): once in the fixed
%% processes and once in the $Nd_z$ component taken from $\pi'(v)$.  It is
%% necessary for the remapping function to include $N_3 \mapsto N_1$ in order to
%% create this coincidence, and that creates the linkage. 

%%%%%


We say that two remapping functions~$\pi_1$ and~$\pi_2$ \emph{range-agree}
on~$y$ if they map the same variables to~$y$:
\[
\forall x \spot \pi_1(x) = y \iff \pi_2(x) = y.
\]
Necessarily there is at most one~$x$ that each remapping function maps
to~$y$.

\begin{lemma}
\label{lem:singleRef:remappings-equivalent}
Consider a transition $pre \trans{} post$ and a view~$v$, both in normal form.
Consider a partial unification function~$\pi_0$, and two consistent
extensions~$\pi_1$ and~$\pi_2$.  Let $v_1 = \pi_1(v)$ and $v_2 = \pi_2(v)$.
Suppose that $\pi_1$ and $\pi_2$ range-agree on:
%
\begin{enumerate}
\item\label{item:singleRef:remappings-equivalent:post-servers} All parameters
  of $post.\fixed$;

\item For each component~$c$ of~$v$ that is unified with a component of~$pre$,
  all new parameters gained by~$c$ in the transition;

\item If the principal of~$v$ is unified with a component of~$pre$, and
  acquires a reference to a component~$c'$ of~$post$, then all parameters
  of~$c'$. 
\end{enumerate}
%
Then for each view~$v_1'$ produced by a primary induced transition from~$v_1$,
there is a view~$v_2'$ produced by a primary induced transition from~$v_2$,
such that $v_1' \equiv v_2'$; and vice versa.
\end{lemma}

%%%%%

\framebox{**} Discussion of examples.
Example~\ref{example:singleRef-remapping-post-servers} illustrates
item~\ref{item:singleRef:remappings-equivalent:post-servers}.  Examples to
illustrate other aspects. 

%%%%%

\begin{proof}
Consider a particular view~$v_1'$ produced by a primary induced transition
from~$v_1$.  Let $v_2'$ be the corresponding view produced by a primary
induced transition from~$v_2$, i.e.~if the secondary component of~$v_1'$ (if
any) corresponds to the $k$th reference of~$v_1'.\princ$, then the secondary
component of~$v_2'$ corresponds to the $k$th reference of~$v_2'.\princ$.

Note that if the principals of $v_1$ and~$v_2$ are unified with a component
of~$pre$ (necessarily the same component in each case), the principals of the
induced views are the corresponding components of~$post$ (so equal).
Concerning secondary components, note that one of the following holds:
%
\begin{itemize}
\item The secondary components of~$v_1'$ and~$v_2'$ are the same component
  of~$post$; 

\item The secondary components of~$v_1'$ and~$v_2'$ are, respectively, the
  secondary components of~$v_1$ and~$v_2$, which were not unified with a
  component of~$pre$; or

\item Neither has a secondary component.
\end{itemize}

This implies that $v_1'$ and~$v_2'$ agree on all parameters of~$post.\fixed$,
and all parameters of components that were taken from~$post$, by the three
conditions of the lemma.  All other parameters are only in components of~$v_1$
or~$v_2$, respectively, that did not unify with any component and so did not
change state in the transition; the parameters will be $\pi_1(x)$
and~$\pi_2(x)$, respectively, where $x$ is the corresponding parameter of~$v$.
Thus $v_1' \equiv v_2'$.
\end{proof}

Optimisation: I think we don't need to agree on parameters of the secondary
component if that isn't included in the subsequent view.  That means qualify
condition~(2) of the lemma with: ``if $c$ is a secondary component of~$v$,
then the principal of~$v$ does not lose the reference to~$c$.''  This is an
issue only if both components of~$v$ change state, which is quite rare.

%%%%%

%% \framebox{Examples} Example~\ref{example:29} satisfies conditions of
%% Lemma~\ref{lem:singleRef:remappings-equivalent}.  Example~\ref{example:30}
%% doesn't, via parameter of $post.\fixed$. 

\begin{definition}
Consider a partial unification function~$\pi_u$.  We call a partial
extension~$\pi_{rd}$ an \emph{result-defining extension} if all values in the
range of~$\pi_{rd}$ that are not in the range of~$\pi_u$ correspond to one of the
three cases of Lemma~\ref{lem:singleRef:remappings-equivalent}.
\end{definition}

For instance, in Example \ref{example:29}, the extension $\set{N_0
  \mapsto N_0, \linebreak[1] {N_3 \mapsto N_1}}$ of $\set{N_0 \mapsto N_0}$ is
result-defining since $N_1$ corresponds to the first case of
Lemma~\ref{lem:singleRef:remappings-equivalent}. 

Note that all extensions of a result-defining remapping produce the same
resulting view (up to equivalence).

Out approach will be as follows: (1) consider all unification
functions~$\pi_u$; (2) for each such unification function, produce all
result-defining extensions~$\pi_{rd}$; (3) extend each such result-defining
extension.  However, it turns out that we might have to consider more than one
extension of each~$\pi_{rd}$: it might be that only certain extensions satisfy
conditions~(b) and~(c) of Definition~\ref{def:induced-transition-singleRef}.
The following example illustrates this.

%%   ,  decide which components to unify, and produce minimal unifying
%% map; (2) decide which parameters to map to parameters under the three
%% conditions of the above lemma; (3) where there is a linkage involving
%% component~$c$ of~$v$ and component~$c'$ of~$pre$, consider all combinations of
%% mapping a parameter of~$c$ to a parameter of~$v'$ (including no such); (4) map
%% all other parameters to fresh parameters. 

%%%%%


\begin{example}
Consider the effect of the transition
\begin{eqnarray*}
(fixed(N_0); Th(T_0,N_1), Nd(N_1,N_0)) & \trans{} &
 (fixed'(N_0); Th'(T_0), Nd(N_1,N_0))
\end{eqnarray*}
on
\begin{eqnarray*}
v & = & (fixed(N_0); Th''(T_0,N_0), Nd'(N_0,N_1)).
\end{eqnarray*}
%
No unification of components is possible.  The only result-defining extension
is $\pi_{rd} = \set{{N_0 \mapsto N_0}}$.  This necessarily creates a linkage
via~$N_0$.  

For an extension~$\pi$ of~$\pi_{rd}$, condition~(b) of
Definition~\ref{def:induced-transition-singleRef}, would require the current
set of representative views to contain
%
\begin{eqnarray*}
v_\cross & = & (fixed(N_0); Nd(N_1,N_0), Nd'(N_0,\pi(N_1)))
\end{eqnarray*}
in order to induce a transition to
\[
\begin{align}
(fixed'(N_0); Th''(T_1,N_0),  Nd'(N_0,N_2)) \equiv \\
\qquad  (fixed'(N_0); Th''(T_0,N_0,N_1), Nd'(N_0,N_1)).
\end{align}
\]
Depending upon the value of $\pi(N_1)$, values for~$v_\cross$ fall into two
equivalence classes, with representative members
\[
\begin{align}
(fixed(N_0); Nd(N_1,N_0), Nd'(N_0,N_1)), \\
(fixed(N_0); Nd(N_1,N_0), Nd'(N_0,N_2)).
\end{align}
\]
It is necessary to consider both these possibilities: it is possible that the
current set of representative views contains only one of them.

Note that the former case (with $\pi(N_1) = N_1$) creates two new linkages
via~$N_1$, which for condition~(b) would require
\[
\begin{array}{c}
(fixed(N_0); Nd'(N_0,N_1), Nd(N_1,N_0)), \\
(fixed(N_0);Th''(T_0,N_0,N_1),Nd(N_1,N_0)).
\end{array}
\]
\end{example}

%%%%%

%% \begin{example}
%% Consider the effect of the transition
%% \begin{eqnarray*}
%% (fixed(N_0); Th(T_0,N_1), Nd(N_1,N_0)) & \trans{} &
%%  (fixed'(N_0); Th'(T_0), Nd(N_1,N_0))
%% \end{eqnarray*}
%% on
%% \begin{eqnarray*}
%% v & = & (fixed(N_0); Th''(T_0,N_0,N_1), Nd'(N_0,N_1)).
%% \end{eqnarray*}
%% %
%% As in the previous example, the only result-defining extension $\set{{N_0
%%     \mapsto N_0}}$ creates a linkage via~$N_0$.  

%% Mapping $N_1$ to~$N_1$ requires $(fixed(N_0); Nd(N_1,N_0), Nd'(N_0,N_1))$.
%% But also creates two new linkages via~$N_1$ which requires
%% \[
%% (fixed(N_0); Th''(T_0,N_0,N_1), Nd(N_1,N_0))
%% (fixed(N_0);
%% \end{example}


%% However, suppose the current representative views do not contain~$v_\cross$,
%% but do contain the two views
%% \begin{eqnarray*}
%% v_\cross' & = & (fixed(N_0); Nd(N_1,N_0), Nd'(N_0,N_1)) \\
%% v_\cross'' & = & (fixed(N_0); Nd'(N_0,N_1), Nd(N_1,N_0)).
%% \end{eqnarray*}
%% %
%% With the additional clause~5, we also consider the renaming $\pi'' =
%% \set{{T_0 \mapsto T_1}, \linebreak[1] N_0 \mapsto N_0,\linebreak[1] {N_1
%%     \mapsto N_1}}$, because the component $c = Nd(N_1,N_0)$ contains a
%% reference to the component $Nd'(N_0,N_1)$ of~$\pi''(v)$.  Condition~(b) is now
%% satisfied for $\pi''(v)$, because of the presence of~$v_\cross'$
%% and~$v_\cross''$.  This then induces a transition to
%% \[
%% \begin{align}
%% (fixed'(N_0); Th''(T_1,N_0), Nd'(N_0,N_1)) \equiv \\
%% \qquad (fixed'(N_0); Th''(T_0,N_0), Nd'(N_0,N_1)),
%% \end{align}
%% \]
%% as one would expect.

Given a result-defining extension mapping~$\pi$, we identify the linkages
defined by~$\pi$.  Consider a linkage involving component~$c$ of~$v$ and
component~$c'$ of~$pre$.  We extend~$\pi$ to the remaining parameters of~$c$,
mapping each either to a parameter of~$c'$ or to a fresh variable.  The former
choice has the possibility of creating another linkage involving different
components; we repeat the process with those.  Once all such linkages have
been considered, we map the remaining parameters to distinct fresh
parameters. 

Note that the above process creates linkages only where they are, in a sense,
required.  We will show below that any other remapping is subsumed by one of
the remappings we do consider.  The following example illustrates the idea. 
%%%%%%
\begin{example}
Consider again (from Example~\ref{example:29}) the normal-form transition
\begin{eqnarray*}
(fixed(N_0); Th(T_0,N_1), Nd_x(N_1,N_2)) & \trans{} &
  (fixed'(N_0); Th'(T_0,N_1), Nd_x(N_1,N_2))
\end{eqnarray*}
acting on 
\[
(fixed(N_0); Nd_y(N_1,N_2), Nd_z(N_2,N_3)).
\]
This has result-defining mapping~$\pi_{rd} = \set{N_0 \mapsto N_0}$.  Consider
the extension $\pi = \set{N_0 \mapsto N_0, N_1 \mapsto N_3, N_2 \mapsto
  N_4,\linebreak[1] {N_3 \mapsto N_1}}$.  This creates a linkage on~$N_1$.
However, this linkage is unnecessary: replacing $N_1$ in the range of~$\pi$
with a fresh parameter $N_5$ creates the remapping $\pi' = \set{N_0 \mapsto
  N_0, N_1 \mapsto N_3, N_2 \mapsto N_4,\linebreak[1] {N_3 \mapsto N_5}}$,
which would lead to the same final induced view (up to equivalence), but
requires a subset of the linking views.
\end{example}

%% A linkage on parameter~$x$ is \emph{necessary} for a remapping~$\pi$ if either:
%% %
%% \begin{itemize}
%% \item It is defined by~$\pi$, i.e., the linking parameter~$x$ is in the range
%%   of~$\pi$;

%% \item There is a necessary linkage on some parameter~$y$, and extending the
%%   remapping, as above, to create a remapping~$\pi'$ produces a new linkage

%%  For some necessary linkage, extending the mapping as above to create
%%   remapping~$\pi'$, then $\pi'$  creates a new
%%   linkage.
%% \end{itemize}

\begin{lemma}
Consider a result-defining mapping~$\pi_{rd}$, and an extension~$\pi$ defined
over all the parameters of~$v$.  Consider some set $X$ of linkage parameters
for~$\pi$, none of which is in the range of~$\pi_{rd}$.  Let $\pi'$ be
obtained from $\pi$ by replacing each element of~$X$ in the range of~$\pi$ by
a distinct fresh variable.  Then the views required to satisfy the linkages
for $\pi'$ are a subset of those for~$\pi$, and the resulting views are
equivalent.
\end{lemma}

%%%%%

\begin{proof}
That $\pi$ and~$\pi'$ produce the same resulting views follows from the fact
that they are both extensions of the same result-defining mapping~$\pi_{rd}$. 

Every linkage for $\pi'$ is also a linkage for~$\pi$ (excluding those on
parameters from~$X$).  Hence the views required for~$\pi'$ are a subset of
those required for~$\pi$. 
\end{proof}


%%%%%%%%%%%%%%%%%%%%%%%%%%%%%%%%%%%%%%%%%%%%%%%%%%%%%%%


%%%%%

More generally, if there are two renamings $\pi$ and~$\pi'$, where the two
induced views are equivalent, and the linkages produced by $\pi$ are a subset
of those produced by~$\pi'$, then there is no need to consider~$\pi'$.  In
this case, we say that the additional linkages produced by~$\pi'$ are
\emph{incidental}.  We aim to identify sufficient conditions for linkages
being incidental.  We consider more examples.

%%%%%

\begin{example}\label{example:30}
Now consider the normal-form transition
\begin{eqnarray*}
(fixed(N_0); Th(T_0,N_0), Nd_x(N_0,N_1)) & \trans{} &
  (fixed'(N_0); Th'(T_0,N_0), Nd_x(N_0,N_1))
\end{eqnarray*}
acting on 
\[
(fixed(N_0); Nd_y(N_1,N_2), Nd_z(N_2,N_0)).
\]
Again no unification is possible.  However, any unification function~$\pi$
necessarily contains $N_0 \mapsto N_0$, so creates a linkage via~$N_0$ that
would require a view equivalent to $(fixed(N_0); Nd_z(\pi(N_2),N_0),
Nd_x(N_0,N_1))$.  This linkage is not incidental because it uses the
parameter~$N_0$ in the fixed processes.
\end{example}

%%%%%

Similarly, a linkage involving a parameter that appears in $post.\fixed$ might
not be incidental.  
%

\framebox{***} Any unification that creates a cross reference must necessarily
be within $pre$ or~$v$.

%%%%%

\begin{example}
Consider the transition
\begin{eqnarray*}
(fixed; Th(T_0,N_1,N_2), Nd_x(N_1,N_3)) & \trans{} &
  (fixed; Th'(T_0,N_1), Nd_y(N_1,N_3,N_2))
\end{eqnarray*}
acting on
\[
(fixed; Nd_x(N_0,N_1), Nd_z(N_1,N_2) .
\]

Consider the mapping $\pi = \set{N_0 \mapsto N_1, N_1 \mapsto N_3, N_2 \mapsto
  N_2}$.  This unifies the two $Nd_x$ components.  It also creates a linkage
via $N_2$ which requires (for clause~(c) of
Definition~\ref{def:induced-transition-singleRef}), for some component~$c$
with identity~$N_2$, that the current set of representative views include
views equivalent to each of $(fixed; Th(T_0,N_1,N_2), c)$ and $(fixed;
Nd_y(N_3,N_2), c)$.  If such views do exist, this induces a transition to
\[
(fixed; Nd_y(N_1,N_3,N_2), Nd_z(N_3,N_2)) \equiv 
  (fixed; Nd_y(N_0,N_1,N_2), Nd_z(N_1,N_2)).
\]

Alternatively, consider the mapping $\pi' = \set{N_0 \mapsto N_1, N_1 \mapsto
  N_3,\linebreak[1] {N_2 \mapsto N_4}}$, which again unifies the $Nd_x$
components, but does not create a linkage via~$N_2$.  This induces a
transition to
\[
(fixed; Nd_y(N_1,N_3,N_2), Nd_z(N_3,N_4)) \equiv
  (fixed; Nd_y(N_0,N_1,N_2), Nd_z(N_1,N_3)).
\]
This is not the same as the view produced using~$\pi$.  Hence the linkage
via~$N_2$ produced by~$\pi$ is not incidental.  The reason for this is that
$N_2$ appears in two components of the induced view: one taken from $post$;
and one taken from~$v$.
\end{example}

%% How do we induce the transition to 
%% \[
%% (fixed; Nd_y(N_1,N_3,N_2), Nd_?(N_2,?)) ?
%% \]


%%%%%


\begin{definition}
\label{def:incidental-linkage}
Consider a transition $pre \trans{} post$ and an accordant view~$v$, with
neither necessarily in normal form.  Define a linkage using parameter~$x$
between a component of~$v$ and a component of~$pre$ to be \emph{significant}
if one of the following holds.
%
\begin{enumerate}
\item \label{item:incidental-linkage-fixed}
$x$ is a parameter of $pre.\fixed = v.\fixed$;

\item \label{item:incidental-linkage-common-component}
$x$ is a parameter of a component that appears in both $pre$ and $v$;

\item \label{item:incidental-linkage-post-fixed} $x$ is a parameter of
  $post.\fixed$;

\item \label{item:incidental-linkage-post-common-component} $x$ is a parameter
  of a component~$c'$ of~$post$ such that there is a component~$c$ in~$v$ with
  the same identity as~$c'$ ($c$ is necessarily also in~$pre$);

\item \label{item:incidental-linkage-acquired-reference} the principal of~$v$
  is also a component of~$pre$ and changes state in the transition to acquire
  a reference to a component~$c$ of~$post$, and $x$ is a component of~$c$;

\item $x$ is a linkage between component~$c$ of~$v$ and component~$c'$
  of~$pre$, and these components have a significant linkage via some other
  parameter~$y$ (this step is applied inductively).
\end{enumerate}
%
We say that a linkage is \emph{incidental} if it is not significant.
\end{definition}

\framebox{Just primary below?}

I'm not sure about \ref{item:incidental-linkage-post-common-component} and
\ref{item:incidental-linkage-acquired-reference}.  It might be when the
linkage variable appears in two components of the induced view, one taken
from~$post$ and one from~$v$.

%%%%%

\begin{lemma}
Consider a transition $pre \trans{} post$ acting on a view~$v_1$ to induce a
transition to~$v_1'$.  Let $\pi$ be a remapping that maps the parameters of
some incidental linkages to distinct fresh parameters (not in $pre$, $post$
or~$v_1$), but is the identity on all other parameters (including those of
significant linkages).  Let $v_2 = \pi(v_1)$.  Then the transition acting
on~$v_2$ induces a transition to $v_2' = \pi(v_1')$; note that $v_2' \equiv
v_1'$.

\framebox{***} I think I want to say something about the relative linkages.
\end{lemma}

Note that the lemma implies that the transition from~$v_2$ subsumes that
from~$v_1$, and involves fewer linkages. 

%%%%%

\begin{proof}
Our proof is in two parts: we show that \emph{if} the conditions of
Definition~\ref{def:induced-transition-singleRef} are satisfied for~$v_2$,
then the induced transition produces $v_2' = \pi(v_1')$; and we show that those
conditions are implied by the fact that they are satisfied for~$v_1$.

By condition~\ref{item:incidental-linkage-fixed} of
Definition~\ref{def:incidental-linkage}, there are no incidental linkages
involving parameters of $v_1.\fixed$.  Hence $\pi$ is the identity on the
parameters of $v_1.\fixed$, so $v_2.\fixed = v_1.\fixed = pre.\fixed$.

By condition~\ref{item:incidental-linkage-common-component}, there are no
incidental linkages involving any component~$c$ of~$v_1$ that also appears
in~$pre$.  Hence $\pi$ is the identity on the parameters of~$c$, so $c$ also
appears in~$v_2$.  Recall, that when $\pi$ is not the identity mapping, it
maps parameters to fresh parameters: it cannot remap a component identity
in~$v_1$ to match an identity in~$pre$.  Hence the common components between
$v_1$ and $pre$ are the same as between $v_2$ and~$pre$; so $v_2$ and $pre$
are accordant.

We now show that $\pi$ is the identity on all parameters of processes
of~$v_1'$ taken from $post$; and hence ---if the relevant side conditions are
satisfied, and so a transition from $v_2$ is indeed induced--- that $v_2'$
agrees with $v_1'$ on all such processes.
%
\begin{itemize}
\item This is true of the fixed processes, by
  condition~\ref{item:incidental-linkage-post-fixed}.

\item Suppose~$c'$ is a component of~$post$ for which there is a component~$c$
  with the same identity in~$v_1$ (necessarily, the same component~$c$ is in
  $pre$; and so it is also in~$v_2$ by the above).  Then $\pi$ is the identity
  on the parameters of~$c'$ by
  condition~\ref{item:incidental-linkage-post-common-component}.

\item
Suppose the principal~$p$ of~$v_1$ is also a component of~$pre$, changes state
in the transition, and acquires a reference to a component~$c$ of~$post$; then
this is true of~$c$, by
condition~\ref{item:incidental-linkage-acquired-reference}.
\end{itemize}



\end{proof}

