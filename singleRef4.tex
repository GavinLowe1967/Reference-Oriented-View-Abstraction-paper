\subsection{Identifying sufficient remappings}

\framebox{Introduce} term \emph{linkage} earlier.

\framebox{*** combine this with the next subsection}  Arguably some of this
should be covered when considering full views. 

In this section we consider the problem of calculating all induced transitions
produced by a transition $pre \trans{} post$ acting on view~$v$, given that
both are represented in normal form.  We want to produce all resulting views
corresponding to the transition acting on a remapping of~$v$, up to
equivalence.  Our aim, then, is to identify sufficient remappings to guarantee
this.  But, in the interests of efficiency, we want to avoid considering too
many remappings.  The necessity of considering linkages complicates the
process over that for full views.

%% Having to deal with linkages when considering induced transitions adds
%% considerable cost to the checking procedure.  In this section we show that it
%% is sound to omit certain renamings of a view~$v$ when considering the effects
%% of a transition on it: we show that in such cases equivalent views can be
%% produced by induced transitions from a different renaming.  

There are two facets to consider.
%
\begin{enumerate}
\item Two different remappings of the view~$v$ may lead to equivalent views
  being produced by the induced transition (on the assumption that
  conditions~(b) and~(c) are satisfied in each case).

\item Tho different remappings of~$v$ that lead to equivalent views may
  require different views to satisfy conditions~(b) and~(c).  When these
  required views are unrelated, we need to consider both remappings.  However,
  in some cases, the required views for one remapping may be a proper subset
  of those for the other remapping (up to equivalence); in this case it is
  sound, and more efficient, to consider just the former.  
\end{enumerate}

Our approach is as follows.  We build up each remapping function in stages; in
early stages, the remapping will be a partial function, that is gradually
extended to give values for all parameters of the view~$v$. 
\begin{itemize}
\item We start with a partial unification function~$\pi_u$. 

\item We then extend $\pi_u$ in all ways that will lead to distinct resulting
  views (cf.~item~1 above).  For each distinct resulting view (up to
  equivalence), we define a single extension that will produce it.  We call
  each such extension a \emph{result-defining remapping}.  Each is a partial
  function, extending $\pi_u$ over those parameters that are necessary to
  define the result.  
  We describe our approach in Sections~\ref{sec:primary-result-defining}
  and~\ref{sec:secondary-result-defining} for primary- and secondary-induced
  transitions, respectively.

\item We then consider conditions~(b) and~(c) of
  Definition~\ref{def:induced-transition-singleRef}.  For each result-defining
  remapping, we consider different extensions that will correspond to
  different required views, but omitting a remappings that requires a
  proper superset of those for another (cf.~item~2 above).  
  We consider conditions~(b) and~(c) in Sections~\ref{sec:necessary-b}
  and~\ref{sec:necessary-c}, respectively, proving appropriate results.  We
  then present and prove our algorithm for calculating sufficient
  remappings in Section~\ref{sec:necessary-algorithm}.

\end{itemize}

We fix a transition $pre \trans{} post$ and a view~$v$ for the remainder of
this section. 


%%%%%%%%%%%%%%%%%%%%%%%%%%%%%%%%%%%%%%%%%%%%%%%%%%%%%%%

\subsubsection{Remappings for equivalent primary induced transitions}
\label{sec:primary-result-defining}

We start by considering when two different extensions of the same partial
unification function produce equivalent primary induced views.  The following
examples help to illustrate. 

\begin{example}\label{example:29}
Consider the normal-form transition
\begin{eqnarray*}
(fixed(N_0); Th(T_0,N_1), Nd_A(N_1,N_2)) & \trans{} &
  (fixed'(N_0); Th'(T_0,N_1), Nd_A(N_1,N_2))
\end{eqnarray*}
acting on 
\[
(fixed(N_0); Nd_B(N_1,N_2), Nd_C(N_2,N_3)).
\]
No unification is possible, so the only unification function is $\pi_{u} =
\set{N_0 \mapsto N_0}$.  If we consider the mapping $\pi_1 = \set{N_0 \mapsto
  N_0, \linebreak[1] {N_1 \mapsto N_3,} \linebreak[1] {N_2 \mapsto N_4},
  \linebreak[1] N_3 \mapsto N_5}$ (mapping all parameters other than those in
the fixed processes to fresh values), then this induces a transition to
\[
(fixed'(N_0); Nd_B(N_3,N_4), Nd_C(N_4,N_5)) \equiv 
  (fixed'(N_0); Nd_B(N_1,N_2), Nd_C(N_2,N_3)).
\] 

Consider instead the mapping $\pi_2 = \set{N_0 \mapsto N_0, N_1 \mapsto N_3,
  N_2 \mapsto N_4,\linebreak[1] {N_3 \mapsto N_1}}$.  This mapping produces a
linkage that would require a view equivalent to $v_\cross = (fixed(N_0);
Nd_C(N_4,N_1), Nd_A(N_1,N_2))$.  If such a view exists, the induced transition
produces
\[
(fixed'(N_0); Nd_B(N_3,N_4), Nd_C(N_4,N_1)) \equiv 
  (fixed'(N_0); Nd_B(N_1,N_2), Nd_C(N_2,N_3)).
\]
The two mappings produce equivalent views.  Thus it is enough to consider one
of them.  We consider only the mapping~$\pi_1$, as it subsumes~$\pi_2$, and is
simpler to deal with as it does not require the additional view equivalent
to~$v_\cross$.

The same is true of all other remappings with $N_1$ and/or $N_2$ in the range
(there are seven in total).
\end{example}

%%%%%

The following example shows how different remappings can produce
non-equivalent views.
% 
\begin{example}\label{example:singleRef-remapping-post-servers}
Consider the transition
% 
\begin{eqnarray*}
(fixed(N_0); Th(T_0,N_1), Nd_A(N_1,N_2)) & \trans{} &
  (fixed'(N_1); Th'(T_0,N_1), Nd_A(N_1,N_2))
\end{eqnarray*}
acting on
\[
v = (fixed(N_0); Nd_B(N_1,N_2), Nd_C(N_2,N_3)).
\]
Again no unification is possible.  The remapping function $\pi_1 = \set{N_0
  \mapsto N_0, \linebreak[1] {N_1 \mapsto N_3}, \linebreak[1] {N_2 \mapsto
    N_4}, \linebreak[1] {N_3 \mapsto N_5}}$ induces a transition to
\[
(fixed'(N_1); Nd_B(N_3,N_4), Nd_C(N_4,N_5)) \equiv
(fixed'(N_0); Nd_B(N_1,N_2), Nd_C(N_2,N_3)).
\]

Alternatively, the remapping function $\pi_2 = \set{{N_0 \mapsto N_0}, {N_1
    \mapsto N_3}, \linebreak[1] {N_2 \mapsto N_4}, \linebreak[1] {N_3 \mapsto
    N_1}}$ creates a linkage via~$N_1$, requiring a view equivalent to
$(fixed(N_0); Nd_C(N_4,N_1), Nd_A(N_1,N_2))$.  If such a view does exist, the
induced transition produces
\[
(fixed'(N_1); Nd_B(N_3,N_4), Nd_C(N_4,N_1)) \equiv
(fixed'(N_0); Nd_B(N_1,N_2), Nd_C(N_2,N_0)).
\]
This is not equivalent to the view produced previously.  Thus it is necessary
to consider both unification functions.

The critical point is that $\pi_2$ maps~$N_3$ to a value in $post.\fixed$,
which leads to a different induced view to that for $\pi_1$.

Note that neither~$N_1$ nor~$N_2$ can be mapped to~$N_1$, since this would
require a unification of components. 
\end{example}

%%%%%

We now give sufficient conditions for two induced transitions to produce
equivalent views, on the assumption that conditions~(b) and~(c) of
Definition~\ref{def:induced-transition-singleRef} are satisfied. 
We say that two remapping functions~$\pi_1$ and~$\pi_2$ \emph{range-agree}
on~$y$ if they map the same variables to~$y$:
\[
\forall x \spot \pi_1(x) = y \iff \pi_2(x) = y.
\]
Necessarily there is at most one~$x$ that the remapping functions map
to~$y$.

\begin{definition}
Consider a transition $pre \trans{} post$ and a view~$v$, both in normal form.
Consider a partial unification function~$\pi_u$.  We define the
\emph{result-relevant parameters} to be:
%
\begin{enumerate}
\item\label{item:singleRef:remappings-equivalent:post-servers} All parameters
  of $post.\fixed$;

\item For each component~$c$ of~$v$ that is unified with a component of~$pre$,
  all new parameters gained by~$c$ in the transition;

\item If the principal of~$v$ is unified with a component of~$pre$, and
  acquires a reference to a component~$c'$ of~$post$, then all parameters
  of~$c'$. 
\end{enumerate}
\end{definition}

\begin{lemma}
\label{lem:singleRef:remappings-equivalent-primary}
Consider a transition $pre \trans{} post$ and a view~$v$, both in normal form.
Consider a partial unification function~$\pi_u$, and two consistent
extensions~$\pi_1$ and~$\pi_2$, and suppose that $\pi_1$ and $\pi_2$
range-agree on all result-relevant parameters.  Let $v_1 = \pi_1(v)$ and $v_2
= \pi_2(v)$.  Suppose, further, that conditions~(b) and~(c) of
Definition~\ref{def:induced-transition-singleRef} are satisfied for the
induced transitions from~$v_1$ and~$v_2$.  Then for each view~$v_1'$ produced
by a primary induced transition from~$v_1$, there is a view~$v_2'$ produced by
a primary induced transition from~$v_2$, such that $v_1' \equiv v_2'$; and
vice versa.
\end{lemma}

%%%%%

\framebox{**} Discussion of examples.
Example~\ref{example:singleRef-remapping-post-servers} illustrates
item~\ref{item:singleRef:remappings-equivalent:post-servers}.  Examples to
illustrate other aspects. 

%%%%%

\begin{proof}
Consider a particular view~$v_1'$ produced by a primary induced transition
from~$v_1$.  Let $v_2'$ be the corresponding view produced by a primary
induced transition from~$v_2$, i.e.~if the secondary component of~$v_1'$ (if
any) corresponds to the $k$th reference of~$v_1'.\princ$, then the secondary
component of~$v_2'$ corresponds to the $k$th reference of~$v_2'.\princ$.

Note that if the principals of $v_1$ and~$v_2$ are unified with a component
of~$pre$ (necessarily the same component in each case), the principals of the
induced views are the corresponding components of~$post$ (so equal).
Otherwise, the principals of the induced views are the principals of~$v_1$
and~$v_2$, respectively. 

Concerning secondary components, note that one of the following holds:
%
\begin{itemize}
\item The secondary components of~$v_1'$ and~$v_2'$ are the same component
  of~$post$; 

\item The secondary components of~$v_1'$ and~$v_2'$ are, respectively, the
  secondary components of~$v_1$ and~$v_2$, which were not unified with a
  component of~$pre$; or

\item Neither has a secondary component.
\end{itemize}

This implies that $v_1'$ and~$v_2'$ agree on all parameters of~$post.\fixed$,
and all parameters of components that were taken from~$post$, by the three
cases in the definition of result-relevant parameters.  All other parameters
are only in components taken from~$v_1$ or~$v_2$, respectively, that did not
unify with any component and so did not change state in the transition; the
parameters will be $\pi_1(x)$ and~$\pi_2(x)$, respectively, where $x$ is the
corresponding parameter of~$v$.  Thus $v_1' \equiv v_2'$.
\end{proof}

Optimisation: I think we don't need to agree on parameters of the secondary
component if that isn't included in the subsequent view.  That means qualify
condition~(2) of the lemma with: ``if $c$ is a secondary component of~$v$,
then the principal of~$v$ does not lose the reference to~$c$.''  This is an
issue only if both components of~$v$ change state, which is quite rare.

%%%%%

\begin{definition}
Consider a partial unification function~$\pi_u$.  We call a partial consistent
extension~$\pi_{rd}$ a \emph{minimal primary-result-defining remapping} if all
values in the range of~$\pi_{rd}$ that are not in the range of~$\pi_u$ are
result-relevant parameters.

We say a consistent extension~$\pi$ of~$\pi_{rd}$ is a
\emph{result-consistent} extension if it adds no new result-relevant
parameters to the range.  We will also sometimes say that $\pi$ is a (not
necessarily minimal) primary-result-defining remapping.
\end{definition}

%%%%%

\begin{example}
Consider again the processes of
Example~\ref{example:singleRef-remapping-post-servers}.  The unification
function $\pi_u = \set{N_0 \mapsto N_0}$ has minimal primary-result-defining
remappings $\pi_u$ itself and $\pi_u \union \set{N_2 \mapsto N_1}$.  Each
result-consistent extension of one of these will map no other parameter
to~$N_1$. 
\end{example}


%% For instance, in Example \ref{example:29}, the extension $\set{N_0
%%   \mapsto N_0, \linebreak[1] {N_3 \mapsto N_1}}$ of $\set{N_0 \mapsto N_0}$ is
%% result-defining since $N_1$ corresponds to the first case of
%% Lemma~\ref{lem:singleRef:remappings-equivalent-primary}. 

The following corollary shows that all result-consistent extensions of a
primary-result-defining remapping produce the same resulting view (up to
equivalence); it follows immediately from
Lemma~\ref{lem:singleRef:remappings-equivalent-primary}.
%
\begin{corollary}
Let $\pi_{rd}$ be a (not necessarily minimal) primary-result-defining
remapping.  Let $\pi_1$ and~$\pi_2$ be two result-consistent extensions
of~$\pi_{rd}$.  Let $v_1 = \pi_1(v)$ and $v_2 = \pi_2(v)$.  Suppose
conditions~(b) and~(c) of Definition~\ref{def:induced-transition-singleRef}
are satisfied for~$v_1$ and~$v_2$.  Then for each view~$v_1'$ produced by a
primary induced transition from~$v_1$, there is a view~$v_2'$ produced by a
primary induced transition from~$v_2$, such that $v_1' \equiv v_2'$; and vice
versa.
\end{corollary}

\framebox{***} does the implementation consider only result-consistent
extensions? 

%%%%%%%%%%%%%%%%%%%%%%%%%%%%%%%%%%%%%%%%%%%%%%%%%%%%%%%


\subsubsection{Remappings for equivalent secondary induced transitions}
\label{sec:secondary-result-defining}

We now consider secondary induced transitions. 
% 
\begin{definition}
Let $\pi_u$ be a unification function.  Suppose $x$ is a parameter of some
secondary component~$sc$ of $post$,  of the same type as
$v.\princ.\id$, and such that $sc$ acquires that reference~$x$ in the
transition.  Consider $\pi_{rfd} = \pi_u \union \set{v.\princ.\id \mapsto x}$, and
suppose this is a consistent extension: so either
\begin{itemize}
\item $(v.\princ.\id \mapsto x) \in \pi_u$; or
\item $v.\princ.\id \nin \dom \pi_u$,\, $x \nin \ran \pi_u$, and $x$ is not
  the identity of a component in~$pre$ and $post$.
\end{itemize}
%
Then we say that $\pi_{rfd}$ is a \emph{secondary-reference-defining}
function.  Note that it creates a reference from~$sc$ to the renamed principal
of~$v$.
\end{definition}

%%%%%
%% Let $\pi_{rd}$ be a consistent extension of~$\pi_1$, adding mappings for
%% parameters of~$v.\princ$ (if necessary).  Then we say that $\pi_{rd}$ is a
%% \emph{secondary-result-defining function}.  This remapping corresponds to a
%% secondary induced transition that produces $(post.\fixed; \set{sc,
%%   \pi_{rd}(v.\princ)})$.

\begin{definition}
Consider a transition $pre \trans{} post$ and a view~$v$, both in normal form.
Consider a secondary-reference-defining function~$\pi_{rfd}$ corresponding to
secondary component~$sc$ acquiring a reference to~$v.\princ$.  We define the
\emph{result-relevant parameters} to be all parameters of $post.\fixed$
and~$sc$.
\end{definition}

%%%%%

\begin{lemma}
\label{lem:singleRef:remappings-equivalent-secondary}
Consider a transition $pre \trans{} post$ and a view~$v$, both in normal form.
Consider a secondary-reference-defining function~$\pi_{rfd}$ corresponding to
secondary component~$sc$ acquiring a reference to~$v.\princ$.  Consider two
consistent extensions~$\pi_1$ and~$\pi_2$ such that $\pi_1$ and~$\pi_2$
restricted to the parameters of~$v.\princ$ range-agree on all result-relevant
parameters.  
Let $v_1 = \pi_1(v)$ and $v_2 =
\pi_2(v)$. 
%
Suppose, further, that conditions~(b) and~(c) of
Definition~\ref{def:induced-transition-singleRef} are satisfied for~$v_1$
and~$v_2$.  Let $v_1'$ and $v_2'$ be the views produced for the secondary
induced transitions (for~$sc$) from~$v_1$ and~$v_2$, respectively.  Then $v_1'
\equiv v_2'$.
\end{lemma}

\begin{proof}
The proof is similar to that of
Lemma~\ref{lem:singleRef:remappings-equivalent-primary}.  The two induced
views are
\begin{eqnarray*}
v_1' & = & (post.\fixed, \set{sc, \pi_1(v.\princ)}), \\
v_2' & = & (post.\fixed, \set{sc, \pi_2(v.\princ)}).
\end{eqnarray*}
%
By construction, $\pi_1(v.\princ)$ and $\pi_2(v.\princ)$ agree on all
parameters that appear elsewhere in these views.  Elsewhere, they have
corresponding parameters $\pi_1(x)$ and~$\pi_2(x)$ where $x$ is the
corresponding parameter of~$v$.  Hence $v_1' \equiv v_2'$.
\end{proof}

\framebox{Example}

\begin{definition}
Consider a secondary-reference-defining function $\pi_{rfd}$ corresponding to
secondary component $sc$ of the transition obtaining a reference
to~$v.\princ$.  We call a partial consistent extension~$\pi_{rd}$ a
\emph{minimal secondary-result-defining remapping} if all maplets $x \mapsto y$
in~$\pi_{rd}$ but not in~$\pi_{rfd}$ are such that $x$ is a parameter of
$v.\princ$, and $y$ is a result-relevant parameter.

We say that a consistent extension~$\pi$ or~$\pi_{rd}$ is a
\emph{result-consistent} extension if it adds no new 
%
maplet $x \mapsto y$ such that $x$ is a parameter of
$v.\princ$, and $y$ is a result-relevant parameter.
%
\framebox{Check above}.
%% result-relevant
%% parameters to the range.  We will sometimes say that $\pi$ is a (not
%% necessarily minimal) secondary-result-defining remapping.
\end{definition}


The following corollary shows that all result-consistent extensions of a
secondary-result-defining remapping produce the same resulting view (up to
equivalence); it follows immediately from
Lemma~\ref{lem:singleRef:remappings-equivalent-secondary}.
%
\begin{corollary}
Let $\pi_{rd}$ be a (not necessarily minimal) secondary-result-defining
remapping (for a particular secondary component).  Let $\pi_1$ and~$\pi_2$ be
two result-consistent extensions of~$\pi_{rd}$.  Let $v_1 = \pi_1(v)$ and $v_2
= \pi_2(v)$.  Suppose conditions~(b) and~(c) of
Definition~\ref{def:induced-transition-singleRef} are satisfied.  Let $v_1'$
and $v_2'$ be the views produced for the secondary induced transitions
from~$v_1$ and~$v_2$, respectively.  Then $v_1' \equiv v_2'$.
\end{corollary}
