\subsection{Instantiating transition templates}

In this section we describe how we build instantiations of transition
templates, given that just normal forms of templates and views are stored.

The following lemma shows that equivalent transition templates are
instantiated to equivalent transitions, and so produce equivalent views.
%
\begin{lemma}
\label{lem:equiv-transition-templates}
Consider a transition template 
\[
v \uplus \hole{id} \trans[e] v' \uplus \hole{id} \,.
\]
%% where $id$ is distinct from any parameter in~$v$.  Let $id'$ be another
%% identity distinct from any parameter in~$v$.  Let $\pi$ be the remapping such
%% that $\pi(id) = id'$,\, $\pi(id') = id$, and otherwise $\pi$~is the identity
%% function.  
Let $\pi$ be a remapping, and let $id' = \pi(id)$.
Consider the transition template
\[
\pi(v) \uplus \hole{id'} \trans[\pi(e)] \pi(v') \uplus \hole{id'} \,.
\]
Then the two transition templates have instantiations that produce equivalent
views. 
\end{lemma}

%%%%%

\begin{proof}
Consider an instantiation $v \uplus c \trans[e] v' \uplus c'$ of the former
template, so $c$ and~$c'$ have identity~$id$.  Then by
Lemma~\ref{lem:system-pi-bisimilar}
\[
\pi(v \uplus c) = \pi(v) \uplus \pi(c) \trans[\pi(e)] 
  \pi(v' \uplus c') = \pi(v') \uplus \pi(c') .
\]
This is an instantiation of the latter template.
\end{proof}

%%%%%

The algorithm of Section~\ref{ssec:algorithm-2} instantiates transition
templates in two ways:
%
\begin{itemize}
\item When it encounters a new transition template, it tries to instantiate it
  based on each previous view;

\item When it encounters a new view, it tries to use it to instantiate each
  previous transition template.
\end{itemize}
%
To combine this with symmetry reduction, we need to solve the following
problem: given a normal-form transition template~$temp_n$ and a normal form
view~$v_n$, for every template $temp$ equivalent to~$temp_n$, and for every
view~$v_c$ equivalent to~$v_n$, produce a transition equivalent to the
instantiation of~$temp$ with a component~$c$ of~$v_c$.  However,
Lemma~\ref{lem:equiv-transition-templates} tells us that each instantiation of
$temp$ is equivalent to an instantiation of~$temp_n$; so it is enough to
consider instantiations of~$temp_n$. 

So consider a normal-form transition template
\begin{eqnarray*}
temp_n & \defs &  v \uplus \hole{id} \trans[e] v' \uplus \hole{id}
\end{eqnarray*}
and a particular normal-form view~$v_n$.  
%
Abstractly, we need to consider each component~$c_n$ of~$v_n$, and each $v_c
\equiv v_n$, say with $v_c = \pi(v_n)$; then let $c = \pi(c_n)$ be the
corresponding component of~$v_c$, and, if $c.\id = id$, check the conditions
of compatibility (Definition~\ref{def:compatible}).  However, many choices
of~$\pi$ and~$v_c$ (for a given~$v_n$ and~$c_n$) will produce equivalent
instantiations of the template: it is enough to consider explicitly just one
from each equivalence class.

Note that we require $v.\fixed = v_c.\fixed = \pi(v_n.\fixed)$, for accordance
of~$v$ and~$v_c$.  The following lemma follows from the way we have defined
normal forms.
%
\begin{lemma}
\label{lem:remapping-id-on-fixed-params}
If $v_1$ and~$v_2$ are both in normal form, and $\pi(v_1)$ and~$v_2$ are
accordant, then $v_1.\fixed = v_2.\fixed$ and $\pi$ is the identity over the
parameters of~$v_1.\fixed$.
\end{lemma}
%
Hence we require that $v.\fixed = v_n.\fixed$ and that $\pi$ is the identity
over the parameters of those fixed processes.  Further, we require $c.\id =
\pi(c_n.\id) = id$.  (These constraints may be inconsistent, in which case no
renaming of this~$c_n$ will work.)

However, other parameters of~$c$ can be mapped by~$\pi$ in several
ways: they might be mapped to other identities appearing in the transition
template; or they might be mapped to a different identity.  We use the term
\emph{fresh} for identities that do not appear in the transition template.  We
show that two remappings that map a particular parameter of~$c$ to different
fresh identities, but are otherwise equal, create equivalent instantiations of
the template.  This means that it is enough to consider one such fresh
identity for each parameter of~$c$. 

%%%%%

%% \begin{lemma}
%% Suppose $v_1$ and $v_2$ are views or extended views in normal form.  Suppose
%% $\pi$ is a remapping such that~$v_1$ and $\pi(v_2)$ are accordant.  Then there
%% exists a remapping~$\pi'$ such that (1)~$v_1$ and $\pi'(v_2)$ are accordant;
%% (2)~$\pi(v_2)$ and~$\pi'(v_2)$ are equivalent; and (3)~$\pi'$ maps  


%%%%%

\begin{lemma}
Let $v \uplus \hole{id} \trans[e] v' \uplus \hole{id}$ be a normal-form
transition template, and let $c_n$ be a component state.  Let $\pi_1$
and~$\pi_2$ be remappings that are the identity over parameters
of~$v.\fixed$,\, $\pi_1(c_n.\id) = \pi_2(c_n.\id) = id$, and such that for
every other parameter~$x$ of~$c_n$, either (1)~$\pi_1(x) = \pi_2(x)$, or
(2)~$\pi_1(x)$ and~$\pi_2(x)$ are fresh identities.  Then instantiations of
the template using $\pi_1(c_n)$ and~$\pi_2(c_n)$ gives equivalent transitions.
\end{lemma}

%%%%%

\begin{proof}
Let $c'$ be such that $\pi_1(c_n) \trans[e] c'$ if $e$ is in the alphabet of
the component; and let $c' = \pi_1(c_n)$ otherwise.  Consider the instantiation
\[
v \uplus \pi_1(c_n) \trans[e] v' \uplus c'.
\]
Let $\pi'$ be such that $\pi_2(x) = \pi'(\pi_1(x))$ for all parameters~$x$: so
$\pi'$ maps each fresh parameter in $\pi_1(c_n)$ to the corresponding fresh
parameter in~$\pi_2(c_n)$, and is the identity over all parameters of the
template. 

Note that $\pi_1(c_n)$ and~$\pi_2(c_n)$ are components of the same state
machine (since they have the same identity~$id$).  Hence if $e$ is in the
alphabet of this state machine, so $\pi_1(c_n) \trans[e] c'$, then \(
\pi_2(c_n) = \pi'(\pi_1(c_n)) \trans[\pi'(e)] \pi'(c') \).  However,
clause~\ref{assump:extra-event-field} of Assumption~\ref{assump:2} implies in
this case that every identity in~$e$ is from the template; hence $\pi'(e) =
e$.  If $e$ is not in the alphabet, then $\pi_2(c_n) = \pi'(\pi_1(c_n)) =
\pi'(c')$.

In each case, we also have the instantiation
\[
\pi'(v \uplus \pi_1(c_n)) =
v \uplus \pi_2(c_n) \trans[e] v' \uplus \pi'(c') = \pi'(v' \uplus c').
\]
This is equivalent to the previous instantiation. 
\end{proof}

%%%%%

Hence it is enough to map each parameter of~$c_n$ that is not a parameter of
the fixed processes or~$c_n.\id$ to either (1)~a parameter in the template (while
keeping the remapping injective); or~(2)~the next fresh parameter.

The following example illustrates the technique.  (No transition template in
our running example illustrates the different aspects, so we assume different
state machines here.)
%
\begin{example}
\label{example:template-1}
Consider the normal-form transition template
\[
\begin{align}
(fixed(T_0,N_0); Th(T_1,N_1), \Node_x(N_1,N_2)) \uplus \bighole{N_2}
  \trans(7)[getNext.T_1.N_1.N_2] \\
\qquad
(fixed(T_0,N_0); Th'(T_1,N_1,N_2), \Node_x(N_1,N_2)) \uplus \bighole{N_2},
\end{align}
\]
where the transition is a synchronisation between~$T_1$ and~$N_1$ (but
not~$N_2$). 
Suppose we are considering instantiating the template using the principal
component of the normal-form view
\[
v_n = (fixed(T_0,N_0); \Node_y(N_1,N_2), \Node_z(N_2,N_3)).
\]
It is worth noting that the identifiers $N_1$, $N_2$ and~$N_3$ that appear in
both te template and~$v_n$ might represent different parameters, since each
substate is just a representative of its equivalence class.  The fact that the
same identifiers appear in each substate is just an artefact of the way we
produce normal forms.

Let $\pi$ be the remapping applied to this component, as described above.
Then:
%
\begin{itemize}
\item  $\pi(T_0) = T_0$ and $\pi(N_0) = N_0$, since $\pi$ needs to be the
identity on the parameters of the fixed processes;

\item $\pi(N_1) = N_2$ to match the identity of the missing component of the
  template;

\item We can map $N_2$ to either $N_1$ (matching a parameter of the template)
  or~$N_3$ (the next fresh identity).
\end{itemize}
%
This gives us the following candidate states for~$c = \pi(\Node_y(N_1,N_2))$:
\[
\Node_y(N_2,N_1) \qquad\mbox{and}\qquad \Node_y(N_2,N_3).
\]

If the conditions for compatibility are satisfied, this will give us the
instantiations
\[
\begin{align}
(fixed(T_0,N_0); Th(T_1,N_1), \Node_x(N_1,N_2), \Node_y(N_2,N_1))
  \trans(7)[getNext.T_1.N_1.N_2] \\
\qquad (fixed(T_0,N_0); Th'(T_1,N_1,N_2), \Node_x(N_1,N_2), \Node_y(N_2,N_1)),
\\
(fixed(T_0,N_0); Th(T_1,N_1), \Node_x(N_1,N_2), \Node_y(N_2,N_3))
  \trans(7)[getNext.T_1.N_1.N_2] \\
\qquad (fixed(T_0,N_0); Th'(T_1,N_1,N_2), \Node_x(N_1,N_2), \Node_y(N_2,N_3)),
\end{align}
\]
respectively.  Note that these two transitions are not equivalent; but mapping
$N_2$ to any other fresh value would give a transition equivalent to the
second one. 
\end{example}

%%%%%

\begin{impNote}
|Combiner.remapToId| does the above to find the map.  Then
|ConsistentStateFinder.getMaps| creates the renamed state.

Then |ConsistentStateFinder.checkCompatible| checks that it's possible to
extend the remapping to the rest of $v_n$ while ensuring accordance --
\framebox{does this duplicate later work?}

And |ConsistentStateFinder.consistentStates| encapsulates this, for each
component of a particular $v_n$, and also finds the relevant post-states.

%% |TransitionTemplateExtender.instantiateTransitionTemplateBy| passes each such
%% to |TransitionTemplateExtender.extendTransitionTemplateBy|

%% |extendTransitionTemplate| calls |Extendable.isExtendable| which checks (1)
%% there is a view with the new state as principal and that is accordant with the
%% pre-state of the template; (2) if any component of the pre-state references
%% the new state, then there is a view corresponding to that.  I.e.~compatibility.

%% If that works, |extendTransitionTemplate| builds the new transitions.
\end{impNote}

%%%%%

The above process generates a collection of candidate component states~$c$ to
instantiate the hole in the transition template.  It is still necessary to
check compatibility of~$c$ with the pre-state~$v$ of the template.  We
consider the conditions of Definition~\ref{def:compatible} in turn.
%
\begin{enumerate}
\item We need to check that $\overline{V}$ contains a view $v_c$ such that
  $v_c.\princ = c$, and $v$ and~$v_c$ are accordant.  To do this, we first
  convert the partial view $(v.\fixed; c)$ into normal form $(v.\fixed; c_1)$
  (which might require remapping parameters of~$c$ to produce~$c_1$).  Let
  $\pi_1$ be the remapping that is the identity on the parameters of the fixed
  processes, and maps each parameter of~$c_1$ to the corresponding parameter
  of~$c$, so $\pi_1(c_1) = c$.  We then iterate through all normal-form
  views~$v_0$ in~$V$ that are extensions of $(v.\fixed; c_1)$.  For each, we
  seek to build an extension~$\pi_2$ of~$\pi_1$ such that $v_c \defs
  \pi_2(v_0)$ and $v$ are accordant: if $\pi_2$ maps the identity of a
  component from~$v_0$ to match the identity of a component from~$v$, then
  $\pi_2$ needs to unify those two components (so all parameters match).  Note
  that it is enough to consider remappings~$\pi_2$ that, for each
  parameter~$x$ of~$v_0$, maps~$x$ either to a parameter of~$v$ or to the next
  fresh identity.  If such a $\pi_2$ exists, then $v_c.\princ = \pi_2(c_1) =
  \pi_1(c_1) = c$, as required.

\item We consider each component~$c'$ of~$v$ that has a reference to~$c$.  For
  each, we need to test whether $\overline{V}$ contains a view $v'$ such that
  (a)~$v'.\fixed = v.\fixed$, (b)~$v'.\princ = c'$, and (c)~$v'$ contains~$c$.
  To do this, we convert the partial view $(v.\fixed; c')$ into normal form
  $(v.\fixed; c_1')$, say using remapping~$\pi_1$.  We then iterate over all
  normal-form views~$v_0$ in~$V$ that are extensions of this.  For each, we
  test whether the relevant component of~$v_0$ is a remapping of~$c$ under an
  extension~$\pi_2$ of~$\pi_1$.  Further, if $c'$ references some other
  component~$c''$ in~$v$, we test whether the relevant component of~$v_0$ is a
  remapping of~$c''$ under an extension $\pi_3$ of~$\pi_2$; we iterate this
  for each such~$c''$.  If this succeeds then $v' \defs \pi_3\inverse(v_0)$
  satisfies the required conditions.
\end{enumerate}

%%%%%

The following example illustrates this construction, following on from
Example~\ref{example:template-1}.
%
\begin{example}
For the candidate state $c = \Node_y(N_2,N_1)$, we proceed as follows. 
%
\begin{enumerate}
\item We convert the partial view $(fixed(T_0,N_0); \Node_y(N_2,N_1))$ into
  normal form $(fixed(T_0,N_0); \Node_y(N_1,N_2))$, corresponding to $\pi_1 =
  \set{T_0 \mapsto T_0, N_0 \mapsto N_0, {N_1 \mapsto N_2,}\linebreak[1] N_2
    \mapsto N_1}$.  We  then consider all extensions from~$V$.  In
  particular, to satisfy the conditions we would require a view $v_0 =
  (fixed(T_0,N_0); \Node_y(N_1,N_2), \Node_x(N_2,N_1))$, so $v_c =
  (fixed(T_0,N_0); \Node_y(N_2,N_1), \Node_x(N_1,N_2))$; then $v_c$ is
  accordant with the pre-state of the template, in particular agreeing on the
  component for~$N_1$.

\item The component $\Node_x(N_1,N_2)$ of the pre-state has a reference
  to~$c$.  We therefore construct the partial view $(fixed(T_0,N_0);
  \Node_x(N_1,N_2))$, which is already in normal form, and consider all
  extensions from~$V$.  In particular, to satisfy the conditions we would
  require a view $v_0 = (fixed(T_0,N_0); \Node_x(N_1,N_2),
  \Node_y(N_2,N_1))$.  Then $v' = v_0$ satisfies the required conditions.
\end{enumerate}
%
Note that each of the required views represents a loop in the linked list.  In
many linked-list algorithms, no such loop is ever constructed.  In such a
case, the check for compatibility would fail. 

For the candidate state $c = \Node_y(N_2,N_3)$, we would proceed similarly.
%
\begin{enumerate}
\item We would require a view of the form $(fixed(T_0,N_0); \Node_y(N_1,N_2),
  \Node_z(N_2,n)) \in V$ for some~$z$ and~$n$ ($n$~could match one of the
  other parameters in the view, or be a fresh identity or the distinguished
  value~$Null$).  The view $v_n$ from Example~\ref{example:template-1} is such
  a view.

\item We would require the view $(fixed(T_0,N_0); \Node_x(N_1,N_2),
  \Node_y(N_2,N_3))$.  Informally, this view shows that these two states of
  consecutive nodes are compatible.
\end{enumerate}
\end{example}

%%%%%

\begin{impNote}
|Extendability.isExtendable| implements this.  It calls
|Extendability.|\linebreak[1]|compatibleWith| to check item~1.  Then it checks part 2, calling
|findReferencingView| to  perform the check for a particular~$c'$. 
\end{impNote}

\begin{impNote}
|TransitionTemplateExtender.instantiateTransitionTemplateBy| calls
|ConsistentStateFinder.consistentStates| to find all candidate states, and
passes each such to |TransitionTemplateExtender.extendTransitionTemplateBy|.
Then
|extendTransitionTemplateBy| calls |Extendability.isExtendable| to check
compatibility; if that works, it builds the new transitions.
\end{impNote}

\begin{opt}
Both steps in the above process require iterating over all views in the
set~$V$ with a particular choice of fixed processes and principal.  The set is
implemented to make this iteration efficient: it uses a map from fixed
processes and principal to the set of relevant views.
\end{opt}

\begin{opt}
Consider the case when we create a new transition template, and aim to
instantiate it via all previous views.  Suppose the pre-state~$v$ of the
template contains a component $c_1$ that has a reference to the missing
component with identity~$id$.  We describe an optimisation that often reduces
the number of candidate component states considered.

We build the partial view $(v.\fixed; c_1)$, then convert this to normal
form, say giving $(v.\fixed; c_1')$ via remapping function~$\pi$.  We then
consider each extension~$v_1'$ of this in~$V$.  Each such~$v_1'$ will contain
a component~$c'$ with identity~$\pi(id)$, corresponding to the reference~$id$
in~$c_1$.  We can then take the candidate component to be~$c \defs
\pi\inverse(c')$.  The view $v' \defs \pi\inverse(v_1')$ then acts as the
required view for condition~\ref{item:compatible-2} of
Definition~\ref{def:compatible}.  
\end{opt}





%% Consider a remapping $\pi$ consistent with the above constraints, and suppose
%% $c = \pi(c_n) \trans[e] c'$.  Then we have an instantiation
%% \[
%% v \uplus \pi(c_n) \trans[e] v' \uplus c' .
%% \]
%% Let $id_1$ and $id_2$ be identities, neither of which is a parameter of~$v$
%% or~$id$; suppose $id_1$ is a parameter of~$\pi(c_n)$, but $id_2$ isn't.  Let
%% $\pi'$ be such that $\pi'(id_1) = id_2$,\, $\pi'(id_2) = id_1$, and is
%% otherwise the identity.  Consider $\pi'(\pi(c_n))$.  We have that
%% $\pi'(\pi(c_n)) \trans[\pi'(e)] \pi'(c')$.  But $\pi'(e) = e$ because ......  
%% Hence we also have the instantiation 
%% \[
%% v \uplus \pi'(\pi(c_n)) \trans[e] v' \uplus \pi'(c').
%% \]

%% Let $\pi'$ be a remapping that maps some parameter~$id_1$ of~$\pi(c_n)$ that
%% is not a 



%% , that are equal except they map some parameter $id$ of~$c_n$ to two
%% different values, say~$id_1$ and~$id_2$, neither of which is a parameter
%% of~$v$.  Then the two instantiations
%% \[
%% \begin{align}
%% v \uplus \pi(c_n) \trans[e] v' \uplus c' \\
%% v \uplus \p
%% \]

%% What about parameters of $e$? 

%% Require that if $\pi(c_n) \trans[e] c_n'$ then $\pi'(\pi(c_n)) \trans[e] c_n'$
%% for $\pi'$ identity on parameters of $v$ and~$id$. 

