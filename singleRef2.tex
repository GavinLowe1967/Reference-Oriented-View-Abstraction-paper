\subsection{Induced transitions}
\label{sec:singleRef-induced}

We now consider how to adapt the definition of induced transitions.  For the
moment, we do not consider the effects of symmetry reduction.  The definition
subsumes a fairly obvious adaptation of
Definition~\ref{def:induced-transition}.  But, in addition, the definition
induces transitions from a view that is missing a component that is referenced
by the principal.  For example, the transition~(\ref{trans:setNext-singleRef})
induces a transition on
\[\mit
v = (fixed; Th(t, n, n'), Nd_y(n', n''))
\]
to produce
\[\mit
v' = (fixed'; Th'(t, n, n'), Nd_y(n', n'')). 
\]

However, it turns out that also we need to:
%
\begin{enumerate}
\item consider another form of induced transitions, concerning secondary
  components from the extended transition; and

\item enforce extra conditions for induced transitions, in order to avoid our
  algorithm discovering certain false errors.
\end{enumerate}
%
We discuss point~1 next; we then give our revised definition of induced
transitions; and then justify the extra conditions mentioned in point~2.

%%%%%%%%%%%%%%%%%%%%%%%%%%%%%%%%%%%%%%%%%%%%%%%%%%%%%%%

\subsubsection{Secondary induced transitions}
\label{ssec:c2Refs}

%% However, we need an additional mechanism to capture some extra views that
%% would otherwise be missed, in particular  where a secondary
%% component gains a reference to another component.  

Consider again the transition (\ref{trans:setNext}).  When using full views,
this would induce a transition from 
\[\mit
v = (fixed,  Nd_x(n, null))\]
to produce
\[\mit
v' = (fixed';  Nd_x(n, n'), Nd_y(n', n'')).
\]
However, with restricted views and a direct adaptation of
Definition~\ref{def:induced-transition}, no corresponding transition
producing~$v'$ via transition~(\ref{trans:setNext-singleRef}) is induced,
because the component state $Nd_y(n', n'')$ is in neither~$v$ nor the
post-state of the view transition.

Instead, we consider the effect of the 
transition~(\ref{trans:setNext-singleRef}) on views such as
\[\mit
v = (fixed; Nd_y(n', n''), Nd_z(n'', n''')).
\]
We identify that the secondary component in the  transition gains a
reference to the principal node of~$v$, and use that to deduce the induced
transition. 

More generally, consider an extended transition $pre \trans[e] post$ and a
view $v$ such that $v$ and $pre$ are accordant, and such that a secondary
component~$sc$ of~$post$ has a reference to $v.\princ$.  Then (subject to some
extra conditions that we describe below) the transition induces a new
transition producing
\[\mit
v' = (post.\fixed; sc, v.\princ).
\]
We call this a \emph{secondary induced transition}. 

\begin{impNote}
At present this is done in EffectOn, via c2Refs. 
\end{impNote}

\begin{improve}
I don't think this is the
best way.  I think that we want to separately record that for every view
\[\mit 
(\fixed; Th(t, n, n'), Nd_y(n', n''))
\]
we generate the view~$v'$ (for all $y$ and~$n''$).  More generally, if a
secondary component changes state, and gains a reference to a missing
secondary component, then we create an appropriate record.
\end{improve}


%%%%%%%%%%%%%%%%%%%%%%%%%%%%%%%%%%%%%%%%%%%%%%%%%%%%%%%

\subsubsection{Definition of induced transitions}

We now adapt Definition~\ref{def:induced-transition} to describe how
transitions are induced for restricted views.  The first form of induced
transition is analogous to Definition~\ref{def:induced-transition} (the only
change is in clause~\ref{induce-clause-3}).  The second form is as discussed
in Section~\ref{ssec:c2Refs}.  We justify conditions~(b) and~(c) in
Sections~\ref{ssec:cross-refs} and~\ref{ssec:missing-common}, respectively:
the former says that if there is a missing reference from $pre$ to a component
of~$v$, or vice versa, then that cross reference must be consistent with the
set of views; the latter says that if both $pre$ and~$v$ have a missing
reference to the same component, then there must be a way of instantiating
that missing component consistent with the set of views.
%
\begin{definition}
\label{def:induced-transition-singleRef}
Consider an extended transition~$pre \trans[e] post$, and a view $v \in V$
such that:
%
\begin{enumerate}
\item[(a)] $v$ and $pre$ are accordant (recall that this implies that
  $v.\fixed = pre.\fixed$);

\item[(b)] If a component $c$ of $pre$ has a reference to a component~$c'$
  of~$v$, or vice versa, then $V$ also contains the view
\begin{eqnarray*}
v_\cross & = & (pre.\fixed, \set{c, c'})
\end{eqnarray*}
(i.e.~either $(pre.\fixed; c, c')$ or  $(pre.\fixed; c', c)$, respectively, in
the notation we have been using for views);

\item[(c)] If the principals of~$pre$ and~$v$ both have a reference to the
  same missing component, say with identity $id$, then there is some
  component~$c$ with identity~$id$ such that $V$ contains each of
\[
(pre.\fixed; pre.\princ, c) \quad\mbox{and}\quad (pre.\fixed; v.\princ, c) .
\]
In addition, if $c$ has a reference to a a secondary component $c_2$ of
$pre.\cpts$ or $v.\cpts$, then $V$ contains $(pre.\fixed; c, c_2)$.
\end{enumerate}
%
Then
%% \begin{description}
%% \item[Primary induced transitions.]
the transition induces a \emph{primary induced transition} $v \trans{} v'$ where
\begin{enumerate}
\item $v'.\fixed = post.\fixed$.

\item If $v.\princ$ is in $pre$ then $v'.\princ$ is the corresponding
  component in~$post$; otherwise $v'.\princ = v.\princ$.

%% \item\label{induce-clause-3} If $v'.\princ$ has no reference to another
%%   component, then $v'$ has no secondary component.  Otherwise the secondary
%%   component of~$v'$ is a component referenced by $v'.princ$, either from
%%   $post$ if the component is part of the transition, or otherwise from~$v$, if
%%   such a component exists.  If $v'.\princ$ references only components not in
%%   either $post$ or $v$, then no new transition is induced.

 % (Alternative version) 
\item\label{induce-clause-3} If $v'.\princ$ has no reference to another
  component, then $v'$ has no secondary component.  Otherwise the secondary
  component of~$v'$ is a component referenced by $v'.princ$, say with
  identity~$cId$, such that if $v.princ$ also referenced~$cId$, then a
  component for~$cId$ was included in~$v$.  The component for~$cId$ in~$v'$ is
  either from $post$ if the component is part of the transition, or otherwise
  from~$v$, if such a component exists.  If there was no component for~$cId$
  in either $post$ or~$v$, then no new transition is induced.
\end{enumerate}
%
%We call such a transition a \emph{primary induced transition}.

%\item[Secondary induced transitions.]
Further, suppose a secondary component acquires a reference to $v.\princ$ in
the transition (i.e.~it the component does not have a reference to $v.\princ$
before the transition, but does afterwards).  Let $sc$ be the state of that
secondary component in $post$.  Then the transition induces a \emph{secondary
  induced transition} producing
\[\mit
v' = (post.\fixed; sc, v.\princ).
\]
%We call such a transition a \emph{secondary induced transition}.
%\end{description}
\end{definition}


Clause~\ref{induce-clause-3} for primary induced transitions prevents the
secondary component of~$v'$ from being a missing component in~$v$.  For
example, consider the transition~(\ref{trans:setNext-singleRef}) together with
\begin{eqnarray*}
v & = & (fixed; Th''(t', n, n''), Nd_y(n'',null)) .
\end{eqnarray*}
%
We do not include an induced transition to produce
\begin{eqnarray*}
v' & = & (fixed'; Th''(t', n, n''),  Nd_x(n, n'))
\end{eqnarray*}
from~$v$, because $v$ has a missing component for~$n$.  However, we would
induce a transition to produce~$v'$ from
\begin{eqnarray*}
v & = & (fixed; Th''(t', n, n''),Nd_x(n, null) ).
\end{eqnarray*}

\framebox{Consider following}
%
\begin{opt}
\label{opt:induced-trans-singleRef}
We can reduce the number of cases to consider.
%
\begin{enumerate}
\item\label{case:opt-induced-trans-singleRef-1} If $pre.\princ = v.\princ$,
  and the secondary component of~$v$ is included in $pre$, then the induced
  transition is identical to the original transition.  There is no need to
  consider this case. 

\item\label{case:opt-induced-trans-singleRef-2} Suppose $pre.\fixed =
  post.\fixed$, and $v$ contains no component that changes state in the
  transition.  Then the primary induced transition creates a view identical
  to~$v$.  (Note, though, that a secondary induced transition might produce a
  new view.)  \framebox{IMPROVE:} if secondary induced transition possible, we
  consider both.

\item\label{case:opt-induced-trans-singleRef-3} Suppose $pre.\fixed =
  post.\fixed$, and $post$ contains no component with a reference to
  $v.\princ$; then no secondary induced transition is possible.  Similarly, if
  $pre.\fixed = post.\fixed$ and any component~$sc$ of $post$ that has a
  reference to~$v.\princ$ does not change state, then the secondary induced
  transition produces a view that is already in the set of views. 
  \framebox{IMPROVE} do this.

\item\label{case:opt-induced-trans-singleRef-4} Now suppose $pre.\fixed \ne
  post.\fixed$, but $v$ again contains no component that changes state in the
  transition, and all the conditions of
  Definition~\ref{def:induced-transition-singleRef} are satisfied. Then the
  new view~$v'$ produced by the primary induced transition will be different
  from~$v$.  However, suppose we subsequently consider another extended
  transition $pre' \trans{} post'$ with $post'.\fixed = post.\fixed$, and
  consider its effect on the same~$v$ (necessarily $pre'.\fixed = pre.\fixed =
  v.\fixed$), and $v$ again contains no component that changes state in the
  transition.  Then (assuming the conditions are satisfied) the primary
  induced transition will produce the same view~$v'$.  (Again, a secondary
  induced transition might produce a new view.)  
\end{enumerate}
We revisit these points in Optimisation~\ref{opt:effectOn-singleRef}.
\end{opt}

%% \begin{improve}
%% Can we reduce the number of combinations we have to consider?
%% At present we don't support Optimisation~\ref{opt:avoid-induced} with reduced
%% views: it's not sound for some reason.  Work out why.  It might be due to the
%% use of c2Refs.  \framebox{Is this still true?}
%% \end{improve}

%%%%%%%%%%%%%%%%%%%%%%%%%%%%%%%%%%%%%%%%%%%%%%%%%%%%%%%

\subsubsection{Compatible cross references}
\label{ssec:cross-refs}

We now justify condition~(b) from
Definition~\ref{def:induced-transition-singleRef}, that if a component~$c$
of~$pre$ has a reference to a component~$c'$ of~$v$, or vice versa, then $V$
must also contain the view
\begin{eqnarray*}
v_\cross & = & (pre.\fixed, \set{c, c'}).
\end{eqnarray*} 
In effect, this requires cross references between $v$ and~$pre$ to be
represented by other views.  If $v$ and $pre$ are views of the same system
state, then the view $v_X$ will also be a view of that state; hence requiring
it is sound.
%
We give two examples to show why not requiring the compatibility of cross
references can lead to the discovery of false errors. 

Consider the following transition for the lock-based queue.
\[
\begin{align}
(fixed(t, n_1, n_1); Enq_5^B(t, n_1, n_2), Nd_B(n_2, null)) 
  \trans(6)[setLast.t.n_2] \\
\qquad (fixed(t, n_1, n_2); Unlock(t), Nd_B(n_2, null)).
\end{align}
\]
The thread is enqueueing~$B$ in node~$n_2$.  It has previously set $n_1$ (the
last node in the list) to point to $n_2$, and now advances \SCALA{last}
to~$n_2$.  This implies that $n_1$ is of the form $Nd_x(n_1, n_2)$, for
some~$x$, in any corresponding concrete system state.  However, consider also
the view
%
\begin{eqnarray*}
v & = & (fixed(t, n_1, n_1) ; Nd_A(n_1, null)).
\end{eqnarray*}
This is accordant with the pre-state of the transition; so, without
condition~(b), a transition to
\begin{eqnarray*}
v' & = & (fixed(t, n_1, n_2) ; Nd_A(n_1, null))
\end{eqnarray*}
would be induced.  But now a subsequent dequeue would find the queue empty
(since $n_1$ is the dummy header node and has a null next reference), despite
the previous enqueue.  (In section~\ref{sec:???}, we will consider this
example together with a watchdog that expects a single~$B$ to be enqueued, and
signals an error if a thread finds the queue empty when it should contain~$B$;
thus the watchdog would falsely signal an error here.)

Condition~(b) of Definition~\ref{def:induced-transition-singleRef} would
require the existence of a view
\begin{eqnarray*}
v_\cross & = & (fixed(t, n_1, n_1); Enq_5^B(t, n_1, n_2), Nd_A(n_1, null)),
\end{eqnarray*}
in order to induce the transition that produces~$v'$, because the enqueueing
thread has a (missing) reference to~$n_1$.  However, no such view exists in
the model (since the thread is in a state where it has just updated~$n_1$ to
point to~$n_2$).  Hence condition~(b) prevents the false error.

%%%%%

The previous example showed that it is necessary to consider cross references
from a principal.  The following example shows that it is also necessary to
consider cross references from a secondary component.  

In Section~\ref{sec:lock-based-stack} we will consider a lock-based stack.
There, for the purposes of specification, we will ensure that if the value~$B$
is in the stack, all values below it in the stack equal~$A$, and all values
above it equal~$C$.  We will include a watchdog that, when $B$ is in the
stack, does not allow a pop of~$A$.  (We justify this watchdog in
Section~\ref{sec:lock-based-stack}.) 

Consider the transition
\[
\begin{align}
(Top(n_1), WD_B ; Pop(t, n_1, n_2), Nd_C(n_1, n_2)) \trans(6)[setTop.t.n_2] \\
\qquad  (Top(n_2), WD_B ; Pop'(t), Nd_C(n_1, n_2)) .
\end{align}
\]
The thread is popping $C$ from the node~$n_1$.  It advances the
\SCALA{top} variable to the next node~$n_2$.  Here $WD_B$ is the watchdog, in
a state where $B$ is in the stack.  

Consider also the view
\begin{eqnarray*}
v & = & (Top(n_1), WD_B ;  Nd_A(n_2, null)).
\end{eqnarray*}
This is accordant with the pre-state of the transition.  However, no concrete
system state is consistent with both~$v$ and the pre-state of the transition,
because the watchdog is consistent only with system states where the stack
contains a~$B$.  (The view $v$ on its own is consistent with a system state
where $n_2$ is at the bottom of the stack, below the $B$-node.)  Hence,
without condition~(b), a transition to
\begin{eqnarray*}
v' & = & (Top(n_2), WD_B ;  Nd_A(n_2, null)).
\end{eqnarray*}
would be induced.  But now a subsequent pop would obtain~$A$.  At this point
the watchdog would give an error: it does not allow $A$ to be popped when $B$
is in the stack.

With the addition of condition~(b), we would require the existence of the view
\begin{eqnarray*}
v_\cross & = &  (Top(n_1), WD_B ; Nd_C(n_1, n_2), Nd_A(n_2, null)).
\end{eqnarray*}
%
However, no such view exists in the model: it is an invariant of the model
that a $C$ node never points to an~$A$ node.  This prevents the false error.


\textbf{Note:} the case of considering cross references from a component
of~$v$ to a component of~$post$ is necessary to avoid false errors in the
Treiber Stack.

%% consistent with $pre$, so induces transition to
%% \[
%% (Top(n_2), WD_B ;  Nd_A(n_2, null)
%% \]
%% so subsequent pop gets A.


%%%%%%%%%%%%%%%%%%%%%%%%%%%%%%%%%%%%%%%%%%%%%%%%%%%%%%%

\subsubsection{Checking consistent common missing components}
\label{ssec:missing-common}

We now justify condition~(c) from
Definition~\ref{def:induced-transition-singleRef}, that if the principals
of~$pre$ and~$v$ both have a reference to the same missing component, say with
identity~$id$, then there must be some component~$c$ with identity~$id$ such
that $V$ contains each of
\[
(pre.\fixed; pre.\princ, c) \quad\mbox{and}\quad (pre.\fixed; v.\princ, c);
\]
and, in addition, if $c$ has a reference to a a secondary component $c_2$ of
$pre.\cpts$ or $v.\cpts$, then $V$ contains $(pre.\fixed; c, c_2)$.
%
In other words, there is some way of instantiating the missing component~$c$
that is common to both views, in a way that is consistent with other views.
When using full views, $c$ would have been included in both substates, and so
the check would have been included within
Definition~\ref{def:induced-transition}.

%% Consider transitions induced on view~$v$ by an extended transition $pre
%% \trans{e} post$.  In particular, consider the case where the principals in
%% $pre$ and $v$ both have a missing reference to the same component~$c$.  It
%% turns out that it is necessary to check that there is a state for the missing
%% component~$c$, consistent with both~$pre$ and~$v$.  (When using full views,
%% $c$ would have been included in both substates, and so the check would have
%% been included within Definition~\ref{def:induced-transition}.)


We explain the necessity using an example from the Treiber
Stack~\cite{treiber}, which we will discuss in Section~\ref{sec:treiber}.  In
that section, in order to capture the correctness condition of being a stack,
we arrange for the stack to hold at most a single node containing~$B$, and for
all nodes below it in the stack to contain~$A$; the correctness conditions
include that~$B$ is popped at most once.

%%%%%

\begin{figure}\small
% Add a label with text #2 just above #1
\def\addLabel#1#2{\draw #1++(0,0.5) node {#2};}
\begin{center}
\begin{tikzpicture}[>= angle 90]
\draw (0,0) node[draw] (n0) {$B$}; \addLabel{(n0)}{$n_0$}
\draw (n0) ++ (-2,0) node (top) {$Top$}; \draw[->] (top) -- (n0);
\draw (n0) ++ (2,0) node[draw,dotted] (n1) {$\,?\,$}; \addLabel{(n1)}{$n_1$}
\draw[->] (n0) -- (n1);
\draw (n1) ++ (2,0) node[draw] (n2) {$A$}; \addLabel{(n2)}{$n_2$}
\draw[->] (n1) -- (n2);
%%% threads
\draw (n0) ++ (1,-1.5) node[draw,circle] (t0) {$t_0$};
\draw[->] (t0) -- (n0); \draw[->] (t0) -- (n1);
\draw (n1) ++ (1,-1.5) node[draw,circle] (t1) {$t_1$};
\draw[->] (t1) -- (n1); \draw[->] (t1) -- (n2);
\end{tikzpicture}
\end{center}
\caption{Illustration of a scenario from the Treiber Stack.  Nodes are
  depicted in rectangular boxes, and threads in circles.}
\label{fig:missingCommon}
\end{figure}

%%%%%

Consider the following two views
%
\begin{eqnarray*}
pre & = &  (Top(n_0), WD_B; Th_{pop,B}(t_0, n_0, n_1), Nd_B(n_0, n_1)), \\
v & = & (Top(n_0), WD_B; Th_{pop,B}(t_1, n_1, n_2), Nd_A(n_2, null)).
\end{eqnarray*}
%
These are depicted in Figure~\ref{fig:missingCommon}.  Both principals are
threads that are attempting to perform a pop, of nodes~$n_0$ and~$n_1$,
respectively, each having read~$B$ from that node.  The fixed process $Top$
references the top node in the stack, here $n_0$.  
Both views have an omitted
reference to node~$n_1$, depicted as dotted in the figure.  Both are possible
views of the system, but they are not consistent, since no system state has
two $B$-nodes in the stack.

From the state $pre$, thread~$t_0$ can advance $Top$ to~$n_1$,
popping~$B$, in a transition that produces
\[\mit
(Top(n_1), WD_X; Th(t_0), Nd_B(n_0, n_1)).
\]
The watchdog has observed the pop of~$B$; it evolves to a state $WD_X$ where a
second pop of~$B$ should not happen.  
Without condition~(c), this transition  induces a transition from~$v$ to
\[\mit
(Top(n_1), WD_X; Th_{pop,B}(t_1, n_1, n_2), Nd_A(n_2, null)).
\]
But now $t_1$ can pop~$B$ (from~$n_1$).  The watchdog will then falsely signal
an error, since it has seen~$B$ popped twice. 

Including condition~(c) will prevent this false error, since there is no state
for~$n_1$ that is consistent with the states of both~$t_0$ and~$t_1$.
Thread~$t_0$ is in a state that is consistent only with~$n_1$ holding~$A$ (it
is an invariant of the model that a $B$-node can point only to an $A$-node);
and $t_1$ is in a state that is consistent only with~$n_1$ holding~$B$.

%% Without condition~(c), the analysis finds a false error: $t_0$ pops~$B$
%% (from~$n_0$), advancing $Top$ to~$n_1$; and then $t_1$ also pops~$B$
%% (from~$n_1$).

%% In order to avoid false positives like this, we identify if the principals of
%% the two views~$pre$ and~$v$ both have a reference to the same missing
%% component, say with identity $id$.  If so, we search the set of views~$V$ to
%% try to identify a component~$c$ with identity~$id$ such that each of
%% \[
%% (pre.\fixed, \set{pre.\princ, c}) \quad\mbox{and}\quad 
%% (v.\fixed, \set{v.\princ, c}
%% \]
%% are in $V$ (recall $pre.\fixed = v.\fixed$ here). 

\framebox{FIXME} Add explanation of final clause: ``If $c$ has a reference to
a a secondary component $c_2$ of $pre.\cpts$ or $v.\cpts$, then $V$ contains
$(pre.\fixed; c, c_2)$''.  The implementation checks this.  It is
necessary to avoid a false positive with the lock-free queue.
