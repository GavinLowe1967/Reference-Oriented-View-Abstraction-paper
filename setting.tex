\def\Node{Nd}
\def\Lock{Lk}
\def\Head{Hd}
\def\Last{Lt}
\def\Con{C}
\def\InitNode{InitNd}

\section{Setting}
\label{sec:setting}

In this section, we introduce more formally the class of systems that we
consider, and our framework.  Recall that we are interested in systems with an
unbounded number of component processes, perhaps from different families,
together with some fixed processes.

We give two small examples to introduce some of the ideas.  We start with a
toy mutual exclusion example.  A resource is shared by a number of threads; a
lock is used to ensure mutual exclusion.

Each thread, with identity $me$, is modelled by a state machine, illustrated
in the top-left of Figure~\ref{fig:mutual-exclusion}.  The thread repeatedly
locks the lock, starts using the resource, ends using the resource, and
unlocks the lock.  Likewise, the lock is modelled by the state machine in the
top-right of the figure.
%
Each state comprises a \emph{control state} and zero or
more \emph{parameters}; in the case of the lock, there are two control states,
which have zero and one parameters, respectively.  Each transition is labelled
by an \emph{event}. (We use CSP-style notation in the figure; the notation
$lock?t$ indicates that the process is willing to perform any event of the
form $lock.id$, and binds the name~$t$ to the corresponding value~$id$.)
Different processes synchronise on common events; for example, the thread with
identity~$me$ will synchronise with the lock on the events $lock.me$ and
$unlock.me$.

%%%%%%%%%%%%%%%%%%%%%%%%%%%%%%%%%%%%%%%%%%%%%%%%%%%%%%%

\begin{figure}
\small
\begin{center}
\begin{tikzpicture}[>= angle 90]
\draw (0,0) node (s0) {$Thread_0(me)$};
\path[draw, <-] (s0) -- ++(-1.7,0);
%% S1
\draw (s0) ++ (4,0) node (s1) {$Thread_1(me)$};
\path[draw, ->] (s0) -- node[above]{\scriptsize $lock.me$} (s1);
%% S2
\draw (s1) ++ (0,-1.5) node (s2) {$Thread_2(me)$};
\path[draw, ->] (s1) -- node[right]{\scriptsize $start.me$} (s2);
%% S3
\draw (s2) ++ (-4,0) node (s3) {$Thread_3(me)$};
\path[draw, ->] (s2) -- node[below]{\scriptsize $end.me$} (s3);
\path[draw, ->] (s3) -- node[left]{\scriptsize $unlock.me$} (s0);

%%%%%%%%%%%%%%%%%%%%%% Lock

\draw (s1)++(3.5,-0.75) node (lock) {$Lock$};
\path[draw, <-] (lock.west) -- ++(-0.5,0);
%% Lock'
\draw (lock)++(2.5,0) node (lock') {$Lock'(t)$};
\path[draw, ->] (lock) .. controls +(1.2,0.5) ..
  node[above]{\scriptsize $lock?t$} (lock');
\path[draw, ->] (lock') .. controls +(-1.3,-0.5) ..
  node[below]{\scriptsize $unlock.t$} (lock);

%%%%%%%%%%%%%%%%%%%%%% Watchdog

\draw (s3) ++ (2.0, -1.7) node (wd0) {$WD_0$};
\path[draw, <-] (wd0) -- ++ (-1.1,0);
%% WD1
\draw (wd0) ++ (2.5,0) node (wd1) {$WD_1(id)$};
\path[draw,->] (wd0) .. controls +(1.15,0.5).. 
  node[above] {\scriptsize $start?id$} (wd1);
\path[draw,->] (wd1) .. controls +(-1.35,-0.5).. 
  node[below] {\scriptsize $end.id$} (wd0);
%% WD2
\draw (wd1) ++ (2.6,0) node (wd2) {$WD_2$};
\path[draw, ->] (wd1) -- node[above] {\scriptsize $start?id'$} (wd2);
\path[draw,->] (wd2) .. controls +(1.4,1) and +(1.4,-1) .. 
  node[right] {\scriptsize $error$} (wd2);
\end{tikzpicture}
\end{center}
\caption{State machines for the toy mutual exclusion example.}
\label{fig:mutual-exclusion}
\end{figure}

%%%%%%%%%%%%%%%%%%%%%%%%%%%%%%%%%

The desired mutual exclusion property is captured by a \emph{watchdog},
modelled by the state machine at the bottom of
Figure~\ref{fig:mutual-exclusion}.  This observes threads starting and ending
using the resource, but performs the distinguished event $error$ if two
threads are using the resource simultaneously.  The correctness condition,
then, is that $error$ does not occur.

A system will consist of an arbitrary collection of threads, but a single lock
and a single watchdog.  Therefore we model the threads as components (each
with its own identity), but the lock and watchdog as fixed processes.
Formally, each system is parameterised by a set $ThreadID$ of thread
identities.  Each system state comprises a state for each thread $t \in
ThreadID$, and a state for each fixed process.  We want to verify all such
systems, i.e.~for an arbitrary collection of threads.

%%%%%%%%%%%%%%%%%%%%%%%%%%%%%%%%%%%%%%%%%%%%%%%%%%%%%%%

Our second example is for a lock-based queue that uses a linked list
internally.  The implementation uses two shared variables: |head| stores a
reference to a dummy header node; and |last| stores a reference to the last
node in the list (the same as |head| if the queue is empty).
Pseudo-code is in Figure~\ref{fig:lock-based-queue-pseudocode}.

%%%%%%%%%

\begin{figure}[htb]
\begin{scala}
class Node(x: D, var next: Node)
var head = new Node(_, null); var last = head
val Lock = new Lock
def dequeue: D = {
  Lock.lock; val h = head; val n = h.next
  if(n == null){ Lock.unlock; return Empty }
  else{ head = n; val x = n.x; Lock.unlock; return x }
}
def enqueue(x: D) = {
  Lock.lock; val l = last; val n = new Node(x, null) 
  l.next = n; last = n; Lock.unlock
}
\end{scala}
\caption{Pseudo-code for the lock-based queue example.}
\label{fig:lock-based-queue-pseudocode}
\end{figure}

%%%%%%%%%

Each node hold a piece of data~$x$, of some type~$D$, and a reference~$next$
to the next node in the linked list, which might be a special value~|null|.
In the state machine (top of Figure~\ref{fig:lock-based-queue}), each node has
an identity~$me$, and (once initialised) an additional parameter~$next$.  The
data value $x$~is treated as part of the control state: we use parameters only
for values from the type of a family of components.

%% corresponding to types and of the initialising channel, since it does not
%% correspond to the identity of a component.  We assume that the type~$D$ of
%% data is finite (in Section~\ref{sec:examples} we use techniques from data
%% independence to justify the use of finite types of data within our models).

%%%%%%%%%%%%%%%%%%%%%%%%%%%%%%%%%%%%%%%%%%%%%%%%%%%%%%%

\begin{figure}
\small
\begin{center}
\begin{tikzpicture}[>= angle 90]

%%%%%%%%%%%%%%%%%%%%%% Nodes

\draw (3,2.8) node (initNode) {$\InitNode(me)$};
\path[draw, <-] (initNode.west) -- ++(-0.6,0);
%% Node
\draw (initNode)++(5.4,0) node (node) {$\Node_x(me,next)$};
\path[draw, ->] (initNode) -- 
  node[above]{\scriptsize $initNode?t.me?x$} node[below]{\scriptsize $[next
  := Null]$} (node);
\path[draw,->] (node) .. controls +(-1.4,1) and +(1.4,1) .. 
% self-loops
  node[above] {\scriptsize
   $\begin{array}{c} getNext?t.me.next\\ getDatum?t.me.x \end{array}$} (node);
\path[draw,->] (node) .. controls +(-1.4,-1) and +(1.4,-1) .. 
  node[below] {\scriptsize
   $\begin{array}{c} setNext?t.me?next' \\ {[next := next']} \end{array}$}
 (node);

%%%%%%%%%%%%%%%%%%%%%% Threads

\draw (0,0) node (thread) {$Thread(me)$};
\path[draw, <-] (thread.north) -- ++(0,0.6);
%% Enq_1
\draw (thread)++(0.0,-5.1) node (enq1) {$Enq_1^x(me)$};
\path[draw, ->] (thread) .. controls +(-1.6,-1.0) and +(-1.6,+1.0) .. 
  node[left]{\scriptsize $\tau$} (enq1);
%% Enq_2
\draw (enq1)++(3.5,0) node (enq2) {$Enq_2^x(me)$};
\path[draw, ->] (enq1) -- node[above]{\scriptsize $lock.me$} (enq2);
%% Enq_3
\draw (enq2)++(4.5,0) node (enq3) {$Enq_3^x(me,l)$};
\path[draw, ->] (enq2) -- node[above]{\scriptsize $getLast.me?l$} (enq3);
%% Enq_4
\draw (enq3)++(0,1.7) node (enq4) {$Enq_4^x(me,l,n)$};
\path[draw, ->] (enq3) -- node[right]{\scriptsize $initNode.me?n.x$} (enq4);
%% Enq_5
\draw (thread)++(0,-3.4) node (enq5) {$Enq_5^x(me,n)$};
\path[draw, ->] (enq4) -- node[above]{\scriptsize $setNext.me.l.n$} (enq5);
%% Unlock
\draw (thread)++(0,-1.7) node (unlock) {$Unlock(me)$};
\path[draw, ->] (enq5) -- % .. controls +(-2,1.5) ..
   node[right]{\scriptsize $setLast.me.n$} (unlock);
\path[draw, ->] (unlock) -- node[right]{\scriptsize $unlock.me$} (thread);
%% Deq 1
\draw (thread)++(2.8,0) node (deq1) {$Deq_1(me)$};
\path[draw, ->] (thread) -- % .. controls +(2,-1.7) ..
  node[above] {\scriptsize $\tau$} (deq1);
%% Deq2
\draw (deq1)++(3.2,0) node (deq2) {$Deq_2(me)$};
\path[draw, ->] (deq1) -- node[above] {\scriptsize $lock.me$} (deq2);
%% Deq3
\draw (deq2) ++(4.5,0) node (deq3) {$Deq_3(me, h)$};
\path[draw, ->] (deq2) -- node[above] {\scriptsize $getHead.me?h$} (deq3);
\path[draw, ->] (deq3.south)++(-0.5,0) -- 
  node[above, sloped] {\scriptsize $getNext.me.h.Null$} (unlock.north east);
%% Deq4
\draw (deq3) ++ (0,-1.7) node (deq4) {$Deq_4(me, n)$};
\path[draw, ->] (deq3) -- node[right] {
  \scriptsize $\begin{align}getNext.me.h?n \\ {[n \ne Null]} \end{align}$
} (deq4); 
%% Deq5
\draw (deq4) ++ (-5.0,0) node (deq5) {$Deq5(me, n)$};
\path[draw, ->] (deq4) -- node[above] {\scriptsize $setHead.me.n$} (deq5);
\path[draw, ->] (deq5) -- % .. controls +(-5,1.7) ..  
  node[above, sloped] {\scriptsize $getDatum.me.n?d$} (unlock); 

%%%%%%%%%%%%%%%%%%%%%% Lock

\draw (enq1)++(0.0,-2.2) node (lock) {$\Lock$};
\path[draw, <-] (lock.west) -- ++(-0.6,0);
%% Lock'
\draw (lock)++(2.5,0) node (lock') {$\Lock'(t)$};
\path[draw, ->] (lock) .. controls +(1.2,0.5) ..
  node[above]{\scriptsize $lock?t$} (lock');
\path[draw, ->] (lock') .. controls +(-1.3,-0.5) ..
  node[below]{\scriptsize $unlock.t$} (lock);

%%%%%%%%%%%%%%%%%%%%%% Head

\draw (lock')++(4.5,0) node (head) {$\Head(head)$};
\path[draw, <-] (head.west) -- node[above] {\scriptsize $[head := Null]$}
   ++(-1.6,0);
\path[draw,->] (head) .. controls +(-1.4,1) and +(1.4,1) .. 
  node[above] {\scriptsize $getHead?t.head$} (head);
\path[draw,->] (head) .. controls +(-1.4,-1) and +(1.4,-1) .. 
  node[below] {\scriptsize 
    $\begin{array}{c} setHead?t?head' \\{} [head := head'] \end{array}$} 
  (head);

%%%%%%%%%%%%%%%%%%%%%% Last

\draw (head)++(4.5,0) node (last) {$\Last(last)$};
\path[draw, <-] (last.west) -- node[above] {\scriptsize $[last := Null]$}
   ++(-1.6,0);
\path[draw,->] (last) .. controls +(-1.4,1) and +(1.4,1) .. 
  node[above] {\scriptsize $getLast?t.last$} (last);
\path[draw,->] (last) .. controls +(-1.4,-1) and +(1.4,-1) .. 
  node[below] {\scriptsize 
    $\begin{array}{c} setLast?t?last' \\{} [last := last'] \end{array}$} 
  (last);

%%%%%%%%%%%%%%%%%%%%%% Constructor

\draw (lock)++(0.9,-2.5) node (const) {$\Con_0$};
\path[draw, <-] (const) -- ++(-1.1,0);
% Const'
\draw (const)++(3.2,0) node (const') {$\Con_1(n)$};
\path[draw,->] (const) -- node[above] {\scriptsize $initNode.C?n.v_0$}
  (const');
% Const''
\draw (const')++(3.1,0) node (const'') {$\Con_2(n)$};
\path[draw,->] (const') -- node[above] {\scriptsize $setHead.C.n$}
  (const'');
% Const'''
\draw (const'')++(3,0) node (const''') {$\Con_3$};
\path[draw,->] (const'') -- node[above] {\scriptsize $setLast.C.n$}
  (const''');
\end{tikzpicture}
\end{center}
%
\caption{Process state machines for the lock-based queue example.}
\label{fig:lock-based-queue}
\end{figure}

%%%%%%%%%%%%%%%%%%%%%%%%%%%%%%%%%%%%%%%%%%%%%%%%%%%%%%%

The middle of Figure~\ref{fig:lock-based-queue} gives the state machine of a
thread.  This is a fairly straightforward translation of the pseudocode.  We
model that, on each iteration, a thread chooses nondeterministically whether
to perform an enqueue or a dequeue; in the state machine, these transitions
are labelled with the internal event~$\tau$.

The lock, and shared |head| and |last| variables are each modelled as a fixed
process.  In addition, we model a \emph{constructor thread} (at the bottom of
the figure), which initialises the dummy header node and the two variables;
note that we model this as a fixed process, although it has an implicit
identity~$C$, of the same type as threads.

We want to verify systems using an arbitrary number of threads and an
arbitrary number of nodes.  We therefore consider systems with two families of
components (with distinct types of identities). 


%%%%%%%%%%%%%%%%%%%%%%%%%%%%%%%%%%%%%%%%%%%%%%%%%%%%%%%


\subsection{Processes}

We now formalise the representation of processes: they can be thought of as a
restricted form of CSP process.  In the next subsection, we lift this to
systems.

We assume disjoint types $\T_1, \ldots, \T_n$ that represent identities of
different families of components (e.g.~thread identities and node
identities), and let $\T = \T_1 \union \ldots \union \T_n$.  Each particular
system we consider will have components with identities from some subset
of~$\T$. 
%% The identities of components may be taken from disjoint types, such as thread
%% identities and node identities.  
Further, we allow these types to contain certain values that do not correspond
to a component; we call these values \emph{distinguished}.  In the queue
example we used two distinguished values: the identity $C$ of the constructor
(remember the constructor was a fixed process, not a component); and the
value~$Null$ representing the |null| node reference.  Given a set
$T \subseteq \T$, we write $\hat{T}$ for the non-distinguished elements
(i.e.~real component identities) in~$T$. 

Each process state is a pair $(cs, \vec p)$ where $cs$~is a control state and
$\vec p$ is a vector of identities, which we call \emph{parameters}. We will
often denote the pair $(cs, \seq{x_1,\ldots,x_k})$ as $cs(x_1,\ldots,x_k)$ (as
in the examples).  We write $q.\cs$ and $q.\params$ for the control state and
parameters of process state~$q$.  The first parameter of a component is always
its own identity.  Other parameters are references to other components.  For
the fixed processes, all parameters are references to components.  If $q$ is a
component state, we write $q.\id$ for its identity.
%% For a vector of parameters~$\vec p$, we write $\vec p(i)$ for the $i$th
%% entry (counting from~1) \framebox{needed?}, so for a component~$q$,\,
%% $q.\params(1) = q.\id$.

Similarly, an event is a pair $(ch, \vec p)$ where $ch$ is a channel and
$\vec{p}$ is a vector of identities.  We will often denote the pair
$(ch, \seq{x_1,x_2,\ldots,x_k})$ as $ch.x_1.x_2.\cdots.x_k$.  Formally, any
field of an event that does not correspond to a component identity, like a
data value in the queue example, is considered to be part of the channel name;
but in examples we use simpler notation.  The distinguished event~$\tau$
represents an internal event (with no parameters).

Each process is defined by a state machine.
%
\begin{definition}
\label{def:state-machine}
A \emph{state machine} is a tuple $(Q, \Sigma, q_0, \delta)$, where: $Q$ is a
set of \emph{states} (as above); $\Sigma$ is a set of \emph{visible events}
(again as above) with $\tau \nin \Sigma$; $q_0 \in Q$ is an initial state; and
\( \delta \subseteq Q \cross (\Sigma \union \set{\tau}) \cross Q \) is a
transition relation.
We write $q \trans(1)[a] q'$ if $(q,a,q') \in \delta$.
\end{definition}


%% Let $T$ be some potentially infinite set~$T$ of \emph{component identities}.
%% A \emph{parameterised state machine over~$T$} is a state machine
%% where:
%% \begin{itemize}
%% \item the states $Q$ are a subset of $CS \cross T^*$, for some finite
%% set~$CS$ of \emph{control states};

%% \item the events $\Sigma$ are a subset of $Chan \cross T^*$, for some
%% finite set~$Chan$ of \emph{channels}.
%% \end{itemize}

We assume a finite set of control states.  However, the parameters vary over a
potentially infinite set.  In examples like the lock-based
queue, where data values can be part of a control state, this requires the
type of data values to also be finite; in Section~\ref{sec:examples} we deduce
results about system that use an infinite data type from results about models
that uses a finite type, using techniques from data independence.  We
similarly assume a finite set of channels.

%Assume a state machine for each individual process.  
We assume that the states of a state machine are well typed in the sense that
two states with the same control state have the same number and types of
parameters: if $cs(x_1,\ldots,x_n), cs(x_1',\ldots,x_{n'}') \in Q$, then $n =
n'$, and $x_i$ and~$x_i'$ are from the same type, for each~$i$.  Likewise,
we assume that the set of visible events is well typed.

The processes in the above examples are symmetric in the references they hold.
We make this notion of symmetry formal now.
%
Recall that the type of identities is partitioned, $\T
= \T_1 \union \ldots \T_n$.  Let $\pi$ be a bijection on the non-distinguished
values of each type, i.e.~it is a function over $\hat\T$ such that if
$x \in \hat\T_i$ then $\pi(x) \in \hat\T_i$, and such that the restriction
of~$\pi$ to~$\hat\T_i$ is a bijection.  Note that $\pi$ is undefined over
distinguished values.  We call each such function~$\pi$ a \emph{remapping}.

We lift remapping~$\pi$ to vectors of identities by point-wise application
(note that this leaves distinguished values unchanged).  We then lift it to
states and events by $\pi(cs(\vec{x})) = cs(\pi(\vec{x}))$ and $\pi(c.\vec{x})
= c.\pi(\vec{x})$; we lift it to sets, etc., by point-wise application.
%% to process states and events by point-wise application,
%% i.e.~$\pi(cs(x_1,\ldots,x_k)) = cs(\pi(x_1),\ldots,\pi(x_k))$ and
%% $\pi(ch.x_1.\ldots.x_k) = ch.\pi(x_1).\ldots.\pi(x_k)$. 

We require each state $cs(\pi(\vec{x}))$ to be equivalent to $cs(\vec{x})$ but
with all events renamed by~$\pi$: formally the states are
$\pi$-bisimilar~\cite{moffat-symmetry}.
%
\begin{definition}
\label{def:pi-bisimilar}
Let $M = (Q, \Sigma, q_0, \delta)$ be a state machine, and let $\pi$ be a
remapping.  We say that $\mathord{\sim} \subseteq Q \times Q$ is
a \emph{$\pi$-bisimulation} iff $\pi(\Sigma) = \Sigma$, and whenever $(q_1,
q_2) \in \mathord{\sim}$ and $a \in \Sigma \union \set{\tau}$:
%
\begin{itemize}
% \item $\initst_1 \sim \initst_2$;
\item If $q_1\trans(1)[a] q_1'$ then
$\exists q_2' \in Q \cdot q_2 \trans(2)[\pi(a)] q_2' \land q_1' \sim q_2'$;

\item If $q_2 \trans(1)[a] q_2'$ then
$\exists q_1' \in Q \cdot q_1 \trans(3)[\pi^{-1}(a)] q_1' \land q_1' \sim q_2'$.
\end{itemize}
\end{definition}

\begin{definition}
\label{def:symmetric-state-machine}
A state machine $(Q, \Sigma, q_0, \delta)$ is \emph{symmetric} if for
every remapping~$\pi$,\, 
$\set{ (cs(\vec{x}), cs(\pi(\vec{x}))) \| cs(\vec{x}) \in Q}$ is a
$\pi$-bisimulation.
\end{definition}

The state machines in Figure~\ref{fig:lock-based-queue} are symmetric.  For
example, the states $\Node_A(N_0, N_1)$ and $\Node_A(N_1,N_4)$ are
$\pi$-bisimilar for each $\pi$ with $\pi(N_0) = N_1$ and $\pi(N_1) = N_4$.

This notion of symmetry is a natural condition.  In~\cite{gavin-tom-symmetry},
we proved that an \emph{arbitrary} process defined using machine-readable CSP
is symmetric under rather mild syntactic conditions, principally that the
definition of the process contains no non-distinguished constant from the
types of identities.

%%%%%



%%%%%%%%%%%%%%%%%%%%%%%%%%%%%%%%%%%%%%%%%%%%%%%%%%%%%%%

%%%%%%%%%%%%%%%%%%%%%%%%%%%%%%%%%%%%%%%%%%%%%%%%%%%%%%%

%% Figure~\ref{fig:lock-based-queue} gives a more interesting example,
%% representing a lock-based queue that uses a linked list internally;
%% pseudo-code is in Figure~\ref{fig:lock-based-queue-pseudocode}.  Here the
%% components come from two families: nodes that make up the linked list; and
%% threads that operate on the queue.  

%% Each node hold a piece of data~$x$, of some type~$D$, and a reference~$next$
%% to the next node in the linked list, which might be a special value~$null$.
%% In the state machine, each node has an identity~$me$ from the type of node
%% identities; $x$ is treated as part of the control state and of the
%% initialising channel, since it does not correspond to the identity of a
%% component.  We assume that the type~$D$ of data is finite (in
%% Section~\ref{sec:examples} we use techniques from data independence to justify
%% the use of finite types of data within our models).

%%%%%%%%%%%%%%%%%%%%%%%%%%%%%%%%%%%%%%%%%%%%%%%%%%%%%%%

%% \begin{figure}[htbp]\small
%% \begin{tikzpicture}[>= angle 90]
%% %%%%% Nodes
%% \draw (3,2.6) node (initNode) {$\InitNode(me)$};
%% \path[draw, <-] (initNode.west) -- ++(-0.6,0);
%% %% Node
%% \draw (initNode)++(5.4,0) node (node) {$\Node_x(me,next)$};
%% \path[draw, ->] (initNode) -- 
%%   node[above]{\scriptsize $initNode_{?x}?t.me?next$} (node);
%% \path[draw,->] (node) .. controls +(-1.4,1) and +(1.4,1) .. 
%%   node[above] {\scriptsize
%%    $\begin{align} getNext?t.me.next\\ getValue_x?t.me \end{align}$} (node);
%% \path[draw,->] (node) .. controls +(-1.4,-1) and +(1.4,-1) .. 
%%   node[below] {\scriptsize
%%    $\begin{array}{c} setNext?t.me.next'\\ {[next := next']} \end{array}$} (node);
%% %%%%%%%%%%%%%%%%%%%%%%%%%%%%%%%%%%%%%%%%%%%%%%%%%%%%%%% Threads
%% \draw (0,0) node (thread) {$Thread(me)$};
%% \path[draw, <-] (thread.north) -- ++(0,0.6);
%% %% Enqueue_1
%% \draw (thread)++(3.5,0) node (enq1) {$Enq_1^x(me)$};
%% \path[draw, ->] (thread) -- 
%%   node[above]{\scriptsize $\tau$} (enq1);
%% %% Enqueue_2^x
%% \draw (enq1)++(3.5,0) node (enq2) {$Enq_2^x(me)$};
%% \path[draw, ->] (enq1) -- node[above]{\scriptsize $lock.me$} (enq2);
%% %% Enqueue_3^x
%% \draw (enq2)++(4.0,0) node (enq3) {$Enq_3^x(me,l)$};
%% \path[draw, ->] (enq2) -- node[above]{\scriptsize $getLast.me?l$} (enq3);
%% %% Enqueue_4
%% \draw (enq3)++(0,-1.7) node (enq4) {$Enq_4(me,l,n)$};
%% \path[draw, ->] (enq3) -- 
%%   node[right]{\scriptsize $\begin{align}initNode_x.\\me?n.null\end{align}$}
%%  (enq4);
%% %% Enqueue_5
%% \draw (enq4)++(-4.0,0) node (enq5) {$Enq_5(me,n)$};
%% \path[draw, ->] (enq4) -- node[above]{\scriptsize $setNext.me.l.n$} (enq5);
%% %% Enqueue_6
%% \draw (enq5)++(-3.5,0) node (enq6) {$Enq_6(me)$};
%% \path[draw, ->] (enq5) -- node[above]{\scriptsize $setLast.me.n$} (enq6);
%% \path[draw, ->] (enq6) -- node[above,sloped]{\scriptsize $unlock.me$}
%% (thread);
%% %%%%%%%%%% Deq
%% \draw (thread)++(0.2,-4.8) node (deq1) {$Deq_1(me)$};
%% \path[draw, ->] (thread) -- 
%%   node[above,sloped]{\scriptsize $\tau$} (deq1);
%% %% Deq_2
%% \draw (deq1)++(2.8,0) node (deq2) {$Deq_2(me)$};
%% \path[draw, ->] (deq1) -- node[above]{\scriptsize $lock.me$} (deq2);
%% %% Deq_3
%% \draw (deq2)++(4.0,0) node (deq3) {$Deq_3(me,h)$};
%% \path[draw, ->] (deq2) -- node[above]{\scriptsize $getHead.me?h$} (deq3);
%% %% Deq_4
%% \draw (deq3)++(4.0,0) node (deq4) {$Deq_4(me,h)$};
%% \path[draw, ->] (deq3) -- node[above]{
%%   \scriptsize $\begin{array}{c} getLast.me?l \\ (h \ne l) \end{array}$} (deq4);
%% %% Deq_5
%% \draw (deq4)++(0,1.7) node (deq5) {$Deq_5(me,n)$};
%% \path[draw, ->] (deq4) -- node[right]{
%%   \scriptsize $\begin{align} getNext. \\ me.h?n \end{align}$} (deq5);
%% %% Deq_6
%% \draw (deq5)++(-4.0,0) node (deq6) {$Deq_6(me,n)$};
%% \path[draw, ->] (deq5) -- node[above]{\scriptsize $setHead.me.n$} (deq6);
%% %% Deq_7^x
%% \draw (deq6)++(-4.0,0) node (deq7) {$Deq_7^x(me)$};
%% \path[draw, ->] (deq6) -- node[above]{\scriptsize  $getValue_{?x}.me.n$} (deq7);
%% \path[draw, ->] (deq7) -- 
%%   node[below,sloped]{\scriptsize $unlock.me$} (thread);
%% %%%%% Deq, empty
%% %% Deq_3'
%% \draw (deq3)++(-7.0,-1.7) node (deq3') {$Deq_3'(me)$};
%% \path[draw, ->] (deq3) -- node[sloped,below]{
%%   \scriptsize $getLast.me.null$} (deq3');
%% %% %% Deq_4'
%% %% \draw (deq3')++(-3.5,0) node (deq4') {$Deq_4'(me)$};
%% %% \path[draw, ->] (deq3') -- node[above]{\scriptsize $popEmpty.me$} (deq4');
%% \path[draw,->] ([xshift = 10] deq3'.north west) .. controls +(-0.4,1.5)  .. 
%%   node[above,sloped,very near end] {\scriptsize $unlock.me$} (thread);
%% %%%%%%%%%%%%%%%%%%%%%%%%%%%%%%%%%%%%%%%%%%%%%%%%%%%%%%% Lock
%% \draw (deq3')++(0.0,-1.8) node (lock) {$\Lock$};
%% \path[draw, <-] (lock.west) -- ++(-0.5,0);
%% %% Lock'
%% \draw (lock)++(2.6,0) node (lock') {$\Lock'(t)$};
%% \path[draw, ->] (lock) .. controls +(1.3,0.5) ..
%%   node[above]{\scriptsize $lock?t$} (lock');
%% \path[draw, ->] (lock') .. controls +(-1.3,-0.5) ..
%%   node[below]{\scriptsize $unlock.t$} (lock);
%% %%%%%%%%%%%%%%%%%%%%%%%%%%%%%%%%%%%%%%%%%%%%%%%%%%%%%%% Head
%% \draw (lock')++(4.3,0) node (head) {$\Head(h)$};
%% \path[draw, <-] (head.west) -- node[above] {\scriptsize $[h := null]$}
%%    ++(-1.5,0);
%% \path[draw,->] (head) .. controls +(-1.4,1) and +(1.4,1) .. 
%%   node[above] {\scriptsize $getHead?t.h$} (head);
%% \path[draw,->] (head) .. controls +(-1.4,-1) and +(1.4,-1) .. 
%%   node[below] {\scriptsize 
%%     $\begin{array}{c} setHead?t?h' \; [h := h'] \end{array}$} 
%%   (head);
%% %%%%%%%%%%%%%%%%%%%%%%%%%%%%%%%%%%%%%%%%%%%%%%%%%%%%%%% Tail
%% \draw (head)++(4.5,0) node (last) {$\Last(l)$};
%% \path[draw, <-] (last.west) -- node[above] {\scriptsize $[l := null]$}
%%    ++(-1.5,0);
%% \path[draw,->] (last) .. controls +(-1.4,1) and +(1.4,1) .. 
%%   node[above] {\scriptsize $getLast?t.l$} (last);
%% \path[draw,->] (last) .. controls +(-1.4,-1) and +(1.4,-1) .. 
%%   node[below] {\scriptsize 
%%     $\begin{array}{c} setLast?t?l' \; [l := l'] \end{array}$} 
%%   (last);
%% %%%%%%%%%%%%%%%%%%%%%%%%%%%%%%%%%%%%%%%%%%%%%%%%%%%%%%% Constructor
%% \draw (lock)++(0.0,-2.3) node (const) {$\Con_0$};
%% % Const'
%% \draw (const)++(4.0,0) node (const') {$\Con_1(n)$};
%% \path[draw,->] (const) -- node[above] {\scriptsize $initNode_{v_0}.C?n.null$}
%%   (const');
%% % Const''
%% \draw (const')++(3.8,0) node (const'') {$\Con_2(n)$};
%% \path[draw,->] (const') -- node[above] {\scriptsize $setHead.C.n$}
%%   (const'');
%% % Const'''
%% \draw (const'')++(3.5,0) node (const''') {$\Con_3$};
%% \path[draw,->] (const'') -- node[above] {\scriptsize $setLast.C.n$}
%%   (const''');
%% \end{tikzpicture}
%% \caption{State machines for the lock-based queue example.}
%% \label{fig:lock-based-queue}
%% \end{figure}

%%%%%%%%%%%%%%%%%%%%%%%%%%%%%%%%%%%%%%%%%%%%%%%%%%%%%%%

%% \subsection*{Lock-based stack}

%% Example

%% \begin{figure}
%% \begin{scala}
%% class Node(x: D, next: Node)
%% var Top: Node = null; val Lock = new Lock
%% def Push(x: D) = {
%%   Lock.lock; val top = Top; val n = new Node(x, top)
%%   Top = n; Lock.unlock
%% }
%% def Pop: D = {
%%   Lock.lock; val top = Top
%%   if(top == null){ Lock.unlock; return Empty }
%%   else{ val n = top.next; Top = n; val x = top.x
%%         Lock.unlock; return x }
%% }
%% \end{scala}
%% \caption{Pseudo-code for the lock-based stack example.}
%% \label{fig:lock-based-stack-pseudocode}
%% \end{figure}

%% %%%%%%%%%%%%%%%%%%%%%%%%%%%%%%%%%%%%%%%%%%%%%%%%%%%%%%%

%% \begin{figure*}\small
%% \begin{tikzpicture}[>= angle 90]
%% %%%%% Nodes
%% \draw (3,1.9) node (initNode) {$InitNode(me)$};
%% \path[draw, <-] (initNode.west) -- ++(-0.6,0);
%% %% Node
%% \draw (initNode)++(5.4,0) node (node) {$Node_x(me,next)$};
%% \path[draw, ->] (initNode) -- 
%%   node[above]{\scriptsize $initNode_{?x}?t.me?next$} (node);
%% \path[draw,->] (node) .. controls +(-1.4,1) and +(1.4,1) .. 
%%   node[above] {\scriptsize
%%    $\begin{align} getNext?t.me.next\\ getValue_x?t.me \end{align}$} (node);
%% %%%%%%%%%%%%%%%%%%%%%%%%%%%%%%%%%%%%%%%%%%%%%%%%%%%%%%% Threads
%% \draw (0,0) node (thread) {$Thread(me)$};
%% \path[draw, <-] (thread.north) -- ++(0,0.6);
%% %% Push_1
%% \draw (thread)++(3.5,0) node (push1) {$Push_1^x(me)$};
%% \path[draw, ->] (thread) -- 
%%   node[above]{\scriptsize $\tau$} (push1);
%% %% Push_2^x
%% \draw (push1)++(3.5,0) node (push2) {$Push_2^x(me)$};
%% \path[draw, ->] (push1) -- node[above]{\scriptsize $lock.me$} (push2);
%% %% Push_3^x
%% \draw (push2)++(3.5,0) node (push3) {$Push_3^x(me)$};
%% \path[draw, ->] (push2) -- node[above]{\scriptsize $push_{?x}.me$} (push3);
%% %% Push_4
%% \draw (push3)++(0,-1.7) node (push4) {$Push_4(me,top)$};
%% \path[draw, ->] (push3) -- 
%%   node[right]{\scriptsize $getTop.me?top$}
%%  (push4);
%% %% Push_5
%% \draw (push4)++(-3.5,0) node (push5) {$Push_5(me,n)$};
%% \path[draw, ->] (push4) -- node[above]{\scriptsize $\begin{align}initNode_x.\\me?n.top\end{align}$} (push5);
%% %% Push_6
%% \draw (push5)++(-3.5,0) node (push6) {$Push_6(me)$};
%% \path[draw, ->] (push5) -- node[above]{\scriptsize $setTop.me.n$} (push6);
%% \path[draw, ->] (push6) -- node[above,sloped]{\scriptsize $unlock.me$}
%% (thread);
%% %%%%%%%%%% Pop
%% \draw (thread)++(0.2,-4.8) node (pop1) {$Pop_1(me)$};
%% \path[draw, ->] (thread) -- 
%%   node[above,sloped]{\scriptsize $\tau$} (pop1);
%% %% Pop_2
%% \draw (pop1)++(3.3,0) node (pop2) {$Pop_2(me)$};
%% \path[draw, ->] (pop1) -- node[above]{\scriptsize $lock.me$} (pop2);
%% %% Pop_3
%% \draw (pop2)++(3.5,0) node (pop3) {$Pop_3(me,top)$};
%% \path[draw, ->] (pop2) -- 
%%   node[above]{\scriptsize 
%%     $\begin{align}getTop.me?top\\(top \ne null)\end{align}$}
%%  (pop3);
%% %% Pop_4
%% \draw (pop3)++(4.0,0) node (pop4) {$Pop_4(me,n)$};
%% \path[draw, ->] (pop3) -- node[above]{\scriptsize $getNext.me.top?n$} (pop4);
%% %% Pop_5
%% \draw (pop4)++(0,1.7) node (pop5) {$Pop_5(me,n)$};
%% \path[draw, ->] (pop4) -- node[right]{\scriptsize $setTop.me.n$} (pop5);
%% %% Pop_6
%% \draw (pop5)++(-4.0,0) node (pop6) {$Pop_6^x(me,n)$};
%% \path[draw, ->] (pop5) -- node[above]{\scriptsize $getValue_{?x}.me.top$} (pop6);
%% %% Pop_7^x
%% \draw (pop6)++(-3.5,0) node (pop7) {$Pop_7^x(me)$};
%% \path[draw, ->] (pop6) -- node[above]{\scriptsize $pop_x.me$} (pop7);
%% \path[draw, ->] (pop7) -- 
%%   node[below,sloped]{\scriptsize $unlock.me$} (thread);
%% %%%%% Pop, empty
%% %% Pop_3'
%% \draw (pop2)++(0.0,-1.7) node (pop3') {$Pop_3'(me)$};
%% \path[draw, ->] (pop2) -- node[right]{\scriptsize $getTop.me.null$} (pop3');
%% %% Pop_4'
%% \draw (pop3')++(-3.5,0) node (pop4') {$Pop_4'(me)$};
%% \path[draw, ->] (pop3') -- node[above]{\scriptsize $popEmpty.me$} (pop4');
%% \path[draw,->] ([xshift = 10] pop4'.north west) .. controls +(-0.4,1.5)  .. 
%%   node[above,sloped,very near end] {\scriptsize $unlock.me$} (thread);
%% %%%%%%%%%%%%%%%%%%%%%%%%%%%%%%%%%%%%%%%%%%%%%%%%%%%%%%% Lock
%% \draw (pop4')++(0.9,-1.9) node (lock) {$Lock$};
%% \path[draw, <-] (lock.west) -- ++(-0.6,0);
%% %% Lock'
%% \draw (lock)++(3.0,0) node (lock') {$Lock'(t)$};
%% \path[draw, ->] (lock) .. controls +(1.5,0.5) ..
%%   node[above]{\scriptsize $lock?t$} (lock');
%% \path[draw, ->] (lock') .. controls +(-1.5,-0.5) ..
%%   node[below]{\scriptsize $unlock.t$} (lock);
%% %%%%%%%%%%%%%%%%%%%%%%%%%%%%%%%%%%%%%%%%%%%%%%%%%%%%%%% Top
%% \draw (lock')++(4.5,0) node (top) {$Top(top)$};
%% \path[draw, <-] (top.west) -- node[above] {\scriptsize $[top := null]$}
%%    ++(-1.6,0);
%% \path[draw,->] (top) .. controls +(-1.4,1) and +(1.4,1) .. 
%%   node[above] {\scriptsize $getTop?t.top$} (top);
%% \path[draw,->] (top.north east) .. controls +(1,1.4) and +(1,-1.4) .. 
%%   node[right] {\scriptsize 
%%     $\begin{align}setTop?t?top' \\{} [top := top'] \end{align}$} 
%%   (top.south east);
%% \end{tikzpicture}
%% \caption{State machines for the lock-based stack example.}
%% \label{fig:lock-based-stack}
%% \end{figure*}


%% The node can be initialised by a thread.  Subsequently, a thread
%% can get the value of $next$ or~$x$.  In other similar examples, there might be
%% additional transitions corresponding to a thread updating the $next$
%% reference.

%% The datatype uses a lock.  It uses a dummy header node (with initial
%% value~$\sm{v}_0$) for the linked list, referenced by a variable~\SCALA{head};
%% and it uses a variable \SCALA{last} to point to the last node in the list.
%% Further, it uses a \emph{constructor} which initialises the dummy header node,
%% and sets \SCALA{head} and \SCALA{tail} to point to it.

%% %The datatype uses a lock, and a variable $Top$ that points to the top node on
%% %the stack.  
%% Each thread performs enqueue and dequeue operations upon the queue: the
%% exact details aren't important here.  In the state machine, each thread
%% component has an identity~$me$ from a type of thread identities.  It
%% synchronises with node components to initialise them, or to get their $next$
%% or~$x$ fields.  It synchronises with the lock to lock and unlock the datatype,
%% and with the $Hd$ and $Lt$ processes  to get or set a reference to the current
%% head or last node.
%% %% It performs additional \emph{signal} events, $push_x.me$, $pop_x.me$ and
%% %% $popEmpty.me$ to indicate the operation it is performing and the result; we
%% %% later use these to capture the property that the system implements a stack.
%% In each state, each thread holds its own identity; in some states, it also
%% holds a reference to one or two nodes.

%% The fixed processes are: the lock process $\Lock$; $\Head$, $\Last$ modelling
%% the \SCALA{head} and \SCALA{tail} variables; and the constructor
%% process~$\Con$.
%% %% For the purposes of the formal development, it is simplest to assume a
%% %% single fixed process, which we can take as the product of the two parts.
%% The state machine for the lock allows the datatype to be locked and unlocked.
%% The state machines for $Head$ and $Last$ store a reference to the relevant
%% node, which threads may get or set.  The constructor initialises the dummy
%% header node, and sets $Head$ and $Tail$ to point to it; it is an active
%% process.  In Section~\ref{sec:examples} we describe how to extend the fixed
%% processes to include a watchdog part that checks that the system does indeed
%% implement a queue.

