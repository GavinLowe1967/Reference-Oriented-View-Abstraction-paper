\def\Node{Nd}
\def\Lock{Lk}
\def\Head{Hd}
\def\Last{Lt}
\def\Con{C}
\def\InitNode{InitNd}

\section{Setting}

Assume a state machine for each individual process. 

Each process state is a pair $(cs, ps)$ where $cs$~is a control state and $ps$
is a vector of parameters.  The first parameter of a component is always its
own identity.  Other parameters are references to other components.  For the
fixed process, all parameters are references to components.  Write $p.cs$ and
$p.\params$ for the control state and parameters of process state~$p$.  If $p$
is a component state, write $p.\id$ for its identity.  For a vector of
parameters~$ps$, we write $ps(i)$ for the $i$th entry (counting from~1), so
for a component~$p$,\, $p.\params(1) = p.\id$.
  
%%  We say that $s$ \emph{references} $id'$ if $id'$ is a parameter of~$s$
%% other than its own identity.

Write a system state as $s = (\fixed, \cpt)$ where $\fixed$ is the state of
the fixed process, and $\cpt$ is a set of states of components.  Write $\S$
for the set of all system states.  Write $s.\fixed$ and $s.\cpt$ for the fixed
processes or components, respectively.

If $s$ is a system state and $cpts$ is a set of component states with
identities disjoint from the identities of $s.\cpt$, we write $s \uplus cpts$
for $(s.\fixed, s.\cpt \union cpts)$.  Similarly, if $c$ is a component with
identities disjoint from those of $s.\cpts$, we write $s \uplus c$ for
$(s.\fixed, s.\cpt \union \set{c})$

%% If $cpt$ is a component state, we abuse notation and write $cpt \in
%% s$ for $cpt \in s.\cpt$.

Assume different components have different identities for now. 

%% If $id$ is an identity, we say that $id$ is in $s$ if there is a
%% component~$cpt \in s$ such that $s.\id = id$.  We say that $s$ references $id$
%% if a component of~$s$ references~$id$. 

Assume that components are partitioned into \emph{active} and \emph{passive}
components.  E.g.~threads and nodes.  Likewise, fixed processes are
partitioned into active and passive processes.  Each transition involves:
\begin{itemize}
\item precisely one active component or active fixed process;
\item at most one passive component;
\item possibly other passive fixed processes.
\end{itemize}

\improve{check this in implementation.}

For a transition with an active component~$princ$, assume that at most one
component~$c$ to which $princ$ does not initially have a reference, and such
that either $c$ synchronises on the transition or $princ$ acquires a
reference.
%
Also assume no other process acquires a reference to a component to which
$princ$ does not initially hold a reference.  

For a transition with an active server, ?? in view transitions, take the
principal to be the synchronising passive component if any, or arbitrary
otherwise.  Assume that process acquires at more one reference from the
transition ??? 


%%%%%%%%%%%%%%%%%%%%%%%%%%%%%%%%%%%%%%%%%%%%%%%%%%%%%%%

\begin{figure}
\begin{scala}
class Node(x: D, next: Node)
var head = new Node(v£$_0$£, null); var last = head
val Lock = new Lock
def enqueue(x: D) = {
  Lock.lock; val l = last; val n = new Node(x, null) 
  l.next = n; last = n; Lock.unlock
}
def dequeue: D = {
  Lock.lock; val h = head; val l = last
  if(h == l){ Lock.unlock; return Empty }
  else{ val n = h.next; head = n; val x = n.x
        Lock.unlock; return x }
}
\end{scala}
\caption{Pseudo-code for the lock-based queue example.}
\label{fig:lock-based-queue-pseudocode}
\end{figure}

%%%%%%%%%%%%%%%%%%%%%%%%%%%%%%%%%%%%%%%%%%%%%%%%%%%%%%%

Figure~\ref{fig:lock-based-queue} gives a more interesting example,
representing a lock-based queue that uses a linked list internally;
pseudo-code is in Figure~\ref{fig:lock-based-queue-pseudocode}.  Here the
components come from two families: nodes that make up the linked list; and
threads that operate on the queue.  

Each node hold a piece of data~$x$, of some type~$D$, and a reference~$next$
to the next node in the linked list, which might be a special value~$null$.
In the state machine, each node has an identity~$me$ from the type of node
identities; $x$ is treated as part of the control state and of the
initialising channel, since it does not correspond to the identity of a
component.  We assume that the type~$D$ of data is finite (in
Section~\ref{sec:examples} we use techniques from data independence to justify
the use of finite types of data within our models).

%%%%%%%%%%%%%%%%%%%%%%%%%%%%%%%%%%%%%%%%%%%%%%%%%%%%%%%

\begin{figure}[htbp]\small
\begin{tikzpicture}[>= angle 90]
%%%%% Nodes
\draw (3,2.6) node (initNode) {$\InitNode(me)$};
\path[draw, <-] (initNode.west) -- ++(-0.6,0);
%% Node
\draw (initNode)++(5.4,0) node (node) {$\Node_x(me,next)$};
\path[draw, ->] (initNode) -- 
  node[above]{\scriptsize $initNode_{?x}?t.me?next$} (node);
\path[draw,->] (node) .. controls +(-1.4,1) and +(1.4,1) .. 
  node[above] {\scriptsize
   $\begin{align} getNext?t.me.next\\ getValue_x?t.me \end{align}$} (node);
\path[draw,->] (node) .. controls +(-1.4,-1) and +(1.4,-1) .. 
  node[below] {\scriptsize
   $\begin{array}{c} setNext?t.me.next'\\ {[next := next']} \end{array}$} (node);
%%%%%%%%%%%%%%%%%%%%%%%%%%%%%%%%%%%%%%%%%%%%%%%%%%%%%%% Threads
\draw (0,0) node (thread) {$Thread(me)$};
\path[draw, <-] (thread.north) -- ++(0,0.6);
%% Enqueue_1
\draw (thread)++(3.5,0) node (enq1) {$Enq_1^x(me)$};
\path[draw, ->] (thread) -- 
  node[above]{\scriptsize $\tau$} (enq1);
%% Enqueue_2^x
\draw (enq1)++(3.5,0) node (enq2) {$Enq_2^x(me)$};
\path[draw, ->] (enq1) -- node[above]{\scriptsize $lock.me$} (enq2);
%% Enqueue_3^x
\draw (enq2)++(4.0,0) node (enq3) {$Enq_3^x(me,l)$};
\path[draw, ->] (enq2) -- node[above]{\scriptsize $getLast.me?l$} (enq3);
%% Enqueue_4
\draw (enq3)++(0,-1.7) node (enq4) {$Enq_4(me,l,n)$};
\path[draw, ->] (enq3) -- 
  node[right]{\scriptsize $\begin{align}initNode_x.\\me?n.null\end{align}$}
 (enq4);
%% Enqueue_5
\draw (enq4)++(-4.0,0) node (enq5) {$Enq_5(me,n)$};
\path[draw, ->] (enq4) -- node[above]{\scriptsize $setNext.me.l.n$} (enq5);
%% Enqueue_6
\draw (enq5)++(-3.5,0) node (enq6) {$Enq_6(me)$};
\path[draw, ->] (enq5) -- node[above]{\scriptsize $setLast.me.n$} (enq6);
\path[draw, ->] (enq6) -- node[above,sloped]{\scriptsize $unlock.me$}
(thread);
%%%%%%%%%% Deq
\draw (thread)++(0.2,-4.8) node (deq1) {$Deq_1(me)$};
\path[draw, ->] (thread) -- 
  node[above,sloped]{\scriptsize $\tau$} (deq1);
%% Deq_2
\draw (deq1)++(2.8,0) node (deq2) {$Deq_2(me)$};
\path[draw, ->] (deq1) -- node[above]{\scriptsize $lock.me$} (deq2);
%% Deq_3
\draw (deq2)++(4.0,0) node (deq3) {$Deq_3(me,h)$};
\path[draw, ->] (deq2) -- node[above]{\scriptsize $getHead.me?h$} (deq3);
%% Deq_4
\draw (deq3)++(4.0,0) node (deq4) {$Deq_4(me,h)$};
\path[draw, ->] (deq3) -- node[above]{
  \scriptsize $\begin{array}{c} getLast.me?l \\ (h \ne l) \end{array}$} (deq4);
%% Deq_5
\draw (deq4)++(0,1.7) node (deq5) {$Deq_5(me,n)$};
\path[draw, ->] (deq4) -- node[right]{
  \scriptsize $\begin{align} getNext. \\ me.h?n \end{align}$} (deq5);
%% Deq_6
\draw (deq5)++(-4.0,0) node (deq6) {$Deq_6(me,n)$};
\path[draw, ->] (deq5) -- node[above]{\scriptsize $setHead.me.n$} (deq6);
%% Deq_7^x
\draw (deq6)++(-4.0,0) node (deq7) {$Deq_7^x(me)$};
\path[draw, ->] (deq6) -- node[above]{\scriptsize  $getValue_{?x}.me.n$} (deq7);
\path[draw, ->] (deq7) -- 
  node[below,sloped]{\scriptsize $unlock.me$} (thread);
%%%%% Deq, empty
%% Deq_3'
\draw (deq3)++(-7.0,-1.7) node (deq3') {$Deq_3'(me)$};
\path[draw, ->] (deq3) -- node[sloped,below]{
  \scriptsize $getLast.me.null$} (deq3');
%% %% Deq_4'
%% \draw (deq3')++(-3.5,0) node (deq4') {$Deq_4'(me)$};
%% \path[draw, ->] (deq3') -- node[above]{\scriptsize $popEmpty.me$} (deq4');
\path[draw,->] ([xshift = 10] deq3'.north west) .. controls +(-0.4,1.5)  .. 
  node[above,sloped,very near end] {\scriptsize $unlock.me$} (thread);
%%%%%%%%%%%%%%%%%%%%%%%%%%%%%%%%%%%%%%%%%%%%%%%%%%%%%%% Lock
\draw (deq3')++(0.0,-1.8) node (lock) {$\Lock$};
\path[draw, <-] (lock.west) -- ++(-0.5,0);
%% Lock'
\draw (lock)++(2.6,0) node (lock') {$\Lock'(t)$};
\path[draw, ->] (lock) .. controls +(1.3,0.5) ..
  node[above]{\scriptsize $lock?t$} (lock');
\path[draw, ->] (lock') .. controls +(-1.3,-0.5) ..
  node[below]{\scriptsize $unlock.t$} (lock);
%%%%%%%%%%%%%%%%%%%%%%%%%%%%%%%%%%%%%%%%%%%%%%%%%%%%%%% Head
\draw (lock')++(4.3,0) node (head) {$\Head(h)$};
\path[draw, <-] (head.west) -- node[above] {\scriptsize $[h := null]$}
   ++(-1.5,0);
\path[draw,->] (head) .. controls +(-1.4,1) and +(1.4,1) .. 
  node[above] {\scriptsize $getHead?t.h$} (head);
\path[draw,->] (head) .. controls +(-1.4,-1) and +(1.4,-1) .. 
  node[below] {\scriptsize 
    $\begin{array}{c} setHead?t?h' \; [h := h'] \end{array}$} 
  (head);
%%%%%%%%%%%%%%%%%%%%%%%%%%%%%%%%%%%%%%%%%%%%%%%%%%%%%%% Tail
\draw (head)++(4.5,0) node (last) {$\Last(l)$};
\path[draw, <-] (last.west) -- node[above] {\scriptsize $[l := null]$}
   ++(-1.5,0);
\path[draw,->] (last) .. controls +(-1.4,1) and +(1.4,1) .. 
  node[above] {\scriptsize $getLast?t.l$} (last);
\path[draw,->] (last) .. controls +(-1.4,-1) and +(1.4,-1) .. 
  node[below] {\scriptsize 
    $\begin{array}{c} setLast?t?l' \; [l := l'] \end{array}$} 
  (last);
%%%%%%%%%%%%%%%%%%%%%%%%%%%%%%%%%%%%%%%%%%%%%%%%%%%%%%% Constructor
\draw (lock)++(0.0,-2.3) node (const) {$\Con_0$};
% Const'
\draw (const)++(4.0,0) node (const') {$\Con_1(n)$};
\path[draw,->] (const) -- node[above] {\scriptsize $initNode_{v_0}.C?n.null$}
  (const');
% Const''
\draw (const')++(3.8,0) node (const'') {$\Con_2(n)$};
\path[draw,->] (const') -- node[above] {\scriptsize $setHead.C.n$}
  (const'');
% Const'''
\draw (const'')++(3.5,0) node (const''') {$\Con_3$};
\path[draw,->] (const'') -- node[above] {\scriptsize $setLast.C.n$}
  (const''');
\end{tikzpicture}
\caption{State machines for the lock-based queue example.}
\label{fig:lock-based-queue}
\end{figure}




%%%%%%%%%%%%%%%%%%%%%%%%%%%%%%%%%%%%%%%%%%%%%%%%%%%%%%%

%% \subsection*{Lock-based stack}

%% Example

%% \begin{figure}
%% \begin{scala}
%% class Node(x: D, next: Node)
%% var Top: Node = null; val Lock = new Lock
%% def Push(x: D) = {
%%   Lock.lock; val top = Top; val n = new Node(x, top)
%%   Top = n; Lock.unlock
%% }
%% def Pop: D = {
%%   Lock.lock; val top = Top
%%   if(top == null){ Lock.unlock; return Empty }
%%   else{ val n = top.next; Top = n; val x = top.x
%%         Lock.unlock; return x }
%% }
%% \end{scala}
%% \caption{Pseudo-code for the lock-based stack example.}
%% \label{fig:lock-based-stack-pseudocode}
%% \end{figure}

%% %%%%%%%%%%%%%%%%%%%%%%%%%%%%%%%%%%%%%%%%%%%%%%%%%%%%%%%

%% \begin{figure*}\small
%% \begin{tikzpicture}[>= angle 90]
%% %%%%% Nodes
%% \draw (3,1.9) node (initNode) {$InitNode(me)$};
%% \path[draw, <-] (initNode.west) -- ++(-0.6,0);
%% %% Node
%% \draw (initNode)++(5.4,0) node (node) {$Node_x(me,next)$};
%% \path[draw, ->] (initNode) -- 
%%   node[above]{\scriptsize $initNode_{?x}?t.me?next$} (node);
%% \path[draw,->] (node) .. controls +(-1.4,1) and +(1.4,1) .. 
%%   node[above] {\scriptsize
%%    $\begin{align} getNext?t.me.next\\ getValue_x?t.me \end{align}$} (node);
%% %%%%%%%%%%%%%%%%%%%%%%%%%%%%%%%%%%%%%%%%%%%%%%%%%%%%%%% Threads
%% \draw (0,0) node (thread) {$Thread(me)$};
%% \path[draw, <-] (thread.north) -- ++(0,0.6);
%% %% Push_1
%% \draw (thread)++(3.5,0) node (push1) {$Push_1^x(me)$};
%% \path[draw, ->] (thread) -- 
%%   node[above]{\scriptsize $\tau$} (push1);
%% %% Push_2^x
%% \draw (push1)++(3.5,0) node (push2) {$Push_2^x(me)$};
%% \path[draw, ->] (push1) -- node[above]{\scriptsize $lock.me$} (push2);
%% %% Push_3^x
%% \draw (push2)++(3.5,0) node (push3) {$Push_3^x(me)$};
%% \path[draw, ->] (push2) -- node[above]{\scriptsize $push_{?x}.me$} (push3);
%% %% Push_4
%% \draw (push3)++(0,-1.7) node (push4) {$Push_4(me,top)$};
%% \path[draw, ->] (push3) -- 
%%   node[right]{\scriptsize $getTop.me?top$}
%%  (push4);
%% %% Push_5
%% \draw (push4)++(-3.5,0) node (push5) {$Push_5(me,n)$};
%% \path[draw, ->] (push4) -- node[above]{\scriptsize $\begin{align}initNode_x.\\me?n.top\end{align}$} (push5);
%% %% Push_6
%% \draw (push5)++(-3.5,0) node (push6) {$Push_6(me)$};
%% \path[draw, ->] (push5) -- node[above]{\scriptsize $setTop.me.n$} (push6);
%% \path[draw, ->] (push6) -- node[above,sloped]{\scriptsize $unlock.me$}
%% (thread);
%% %%%%%%%%%% Pop
%% \draw (thread)++(0.2,-4.8) node (pop1) {$Pop_1(me)$};
%% \path[draw, ->] (thread) -- 
%%   node[above,sloped]{\scriptsize $\tau$} (pop1);
%% %% Pop_2
%% \draw (pop1)++(3.3,0) node (pop2) {$Pop_2(me)$};
%% \path[draw, ->] (pop1) -- node[above]{\scriptsize $lock.me$} (pop2);
%% %% Pop_3
%% \draw (pop2)++(3.5,0) node (pop3) {$Pop_3(me,top)$};
%% \path[draw, ->] (pop2) -- 
%%   node[above]{\scriptsize 
%%     $\begin{align}getTop.me?top\\(top \ne null)\end{align}$}
%%  (pop3);
%% %% Pop_4
%% \draw (pop3)++(4.0,0) node (pop4) {$Pop_4(me,n)$};
%% \path[draw, ->] (pop3) -- node[above]{\scriptsize $getNext.me.top?n$} (pop4);
%% %% Pop_5
%% \draw (pop4)++(0,1.7) node (pop5) {$Pop_5(me,n)$};
%% \path[draw, ->] (pop4) -- node[right]{\scriptsize $setTop.me.n$} (pop5);
%% %% Pop_6
%% \draw (pop5)++(-4.0,0) node (pop6) {$Pop_6^x(me,n)$};
%% \path[draw, ->] (pop5) -- node[above]{\scriptsize $getValue_{?x}.me.top$} (pop6);
%% %% Pop_7^x
%% \draw (pop6)++(-3.5,0) node (pop7) {$Pop_7^x(me)$};
%% \path[draw, ->] (pop6) -- node[above]{\scriptsize $pop_x.me$} (pop7);
%% \path[draw, ->] (pop7) -- 
%%   node[below,sloped]{\scriptsize $unlock.me$} (thread);
%% %%%%% Pop, empty
%% %% Pop_3'
%% \draw (pop2)++(0.0,-1.7) node (pop3') {$Pop_3'(me)$};
%% \path[draw, ->] (pop2) -- node[right]{\scriptsize $getTop.me.null$} (pop3');
%% %% Pop_4'
%% \draw (pop3')++(-3.5,0) node (pop4') {$Pop_4'(me)$};
%% \path[draw, ->] (pop3') -- node[above]{\scriptsize $popEmpty.me$} (pop4');
%% \path[draw,->] ([xshift = 10] pop4'.north west) .. controls +(-0.4,1.5)  .. 
%%   node[above,sloped,very near end] {\scriptsize $unlock.me$} (thread);
%% %%%%%%%%%%%%%%%%%%%%%%%%%%%%%%%%%%%%%%%%%%%%%%%%%%%%%%% Lock
%% \draw (pop4')++(0.9,-1.9) node (lock) {$Lock$};
%% \path[draw, <-] (lock.west) -- ++(-0.6,0);
%% %% Lock'
%% \draw (lock)++(3.0,0) node (lock') {$Lock'(t)$};
%% \path[draw, ->] (lock) .. controls +(1.5,0.5) ..
%%   node[above]{\scriptsize $lock?t$} (lock');
%% \path[draw, ->] (lock') .. controls +(-1.5,-0.5) ..
%%   node[below]{\scriptsize $unlock.t$} (lock);
%% %%%%%%%%%%%%%%%%%%%%%%%%%%%%%%%%%%%%%%%%%%%%%%%%%%%%%%% Top
%% \draw (lock')++(4.5,0) node (top) {$Top(top)$};
%% \path[draw, <-] (top.west) -- node[above] {\scriptsize $[top := null]$}
%%    ++(-1.6,0);
%% \path[draw,->] (top) .. controls +(-1.4,1) and +(1.4,1) .. 
%%   node[above] {\scriptsize $getTop?t.top$} (top);
%% \path[draw,->] (top.north east) .. controls +(1,1.4) and +(1,-1.4) .. 
%%   node[right] {\scriptsize 
%%     $\begin{align}setTop?t?top' \\{} [top := top'] \end{align}$} 
%%   (top.south east);
%% \end{tikzpicture}
%% \caption{State machines for the lock-based stack example.}
%% \label{fig:lock-based-stack}
%% \end{figure*}



The node can be initialised by a thread.  Subsequently, a thread
can get the value of $next$ or~$x$.  In other similar examples, there might be
additional transitions corresponding to a thread updating the $next$
reference.

The datatype uses a lock.  It uses a dummy header node (with initial
value~$\sm{v}_0$) for the linked list, referenced by a variable~\SCALA{head};
and it uses a variable \SCALA{last} to point to the last node in the list.
Further, it uses a \emph{constructor} which initialises the dummy header node,
and sets \SCALA{head} and \SCALA{tail} to point to it.


%The datatype uses a lock, and a variable $Top$ that points to the top node on
%the stack.  
Each thread performs enqueue and dequeue operations upon the stack: the
exact details aren't important here.  In the state machine, each thread
component has an identity~$me$ from a type of thread identities.  It
synchronises with node components to initialise them, or to get their $next$
or~$x$ fields.  It synchronises with the lock to lock and unlock the datatype,
and with $Top$ to get or set a reference to the current top node in the stack.
%% It performs additional \emph{signal} events, $push_x.me$, $pop_x.me$ and
%% $popEmpty.me$ to indicate the operation it is performing and the result; we
%% later use these to capture the property that the system implements a stack.
In each state, each thread holds its own identity; in some states, it also
holds a reference to one or two nodes.

The fixed processes are: the lock process $\Lock$; $\Head$, $\Last$ modelling
the \SCALA{head} and \SCALA{tail} variables; and the constructor
process~$\Con$.
%% For the purposes of the formal development, it is simplest to assume a
%% single fixed process, which we can take as the product of the two parts.
The state machine for the lock allows the datatype to be locked and unlocked.
The state machines for $Head$ and $Last$ store a reference to the relevant
node, which threads may get or set.  The constructor initialises the dummy
header node, and sets $Head$ and $Tail$ to point to it; it is an active
process.  In Section~\ref{sec:examples} we describe how to extend the fixed
processes to include a watchdog part that checks that the system does indeed
implement a stack.

List of assumptions:
%
\begin{enumerate}
\item Each concrete transition involves exactly one active process (either a
component or fixed process).

\item Each transition involves at most one passive process.

\item Each transition can involve any number of passive fixed processes.

\item For each transition there is at most one component to which the
active component doesn't initially hold a reference, and that either
synchronises on the transition, or to which the active component gains a
reference.

\item\label{assump:secondary-cpts-new-refs}
In each transition, a passive component can gain a reference only to a
component to which the active process initially holds a reference.
\end{enumerate}
