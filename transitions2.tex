
\subsection{Induced transitions}

So far, we have considered only transitions that involved an active process
(either the principal or a fixed process). However, such a transition can
induce other abstract transitions.  For example the
transition~(\ref{trans:lock}) changes the state of a fixed process, and so
induces corresponding changes in other views; for instance, it induces abstract
transitions concerning another thread or a node in its initial state:
%
\[
\begin{align}
(fixed(\_, n_h, n_l); Thread(t'))  \trans(3)[lock.t] 
  (fixed(t, n_h, n_l); Thread(t')), \\
(fixed(\_, n_h, n_l); \InitNode(n))  \trans(3)[lock.t] 
  (fixed(t, n_h, n_l); \InitNode(n)),
\end{align}
\]
%
for $t' \in ThreadID$ with $t' \ne t$, and for $n \in NodeID$.

\begin{definition}
\label{def:induced-transition}
Consider an extended transition~$pre \trans[e] post$, and a view $v \in V$
such that $v$ and $pre$ are accordant.  Then the transition \emph{induces} a
new transition $v \trans v'$ where:
\begin{enumerate}
\item $v'.\fixed = post.\fixed$.

\item If $v.\princ$ is in $pre$ then $v'.\princ$ is the corresponding
  component in~$post$; otherwise $v'.princ = v.princ$.

\item The secondary components of~$v'$ are the components referenced by
  $v'.princ$, either from $post$ if the component is part of the transition,
  or otherwise from~$v$.  (Note that all such components are included in
  either $post$ or~$v$, by clause~\ref{assump:secondary-cpts-new-refs} of
  Assumption~\ref{assump:2}.)
\end{enumerate}
\end{definition}

%%%%%

For example, the  transition~(\ref{trans:initNode}) induces a transition
\[
\begin{align}
(fixed(t, n_h, n_l); \InitNode(n)) \trans[]
  (fixed(t, n_h, n_l); \Node_x(n, Null)),
\end{align}
\]
where $n$ evolves as in (\ref{trans:initNode}).
%
As another example, consider the extended transition where an enqueueing
thread sets the next reference of the last node:
\[
\begin{align}
(fixed(t, n_h, n_l);  Enq_4(t, n_l, n), \Node_y(n_l, Null), \Node_x(n, Null)) 
 \trans(6)[setNext.t.n_l.n] \\
\qquad (fixed(t, n_h, n_l);  Enq_5(t, n), \Node_y(n_l, n), \Node_x(n, Null)).
\end{align}
\]
This induces the following transition on the view of~$n_l$:
\[
\begin{align}
(fixed(t, n_h, n_l);  \Node_y(n_l, Null)) \trans{}  % setNext.t.n_l.n} \\
  (fixed(t, n_h, n_l);  \Node_y(n_l, n), \Node_x(n, Null)).
\end{align}
\]

%%%%%

We note some optimisations concerning the use of
Definition~\ref{def:induced-transition}.  
%
\begin{opt}
\label{opt:avoid-induced}
A straightforward use of Definition~\ref{def:induced-transition} would
consider many different extended transitions $pre \trans[e] post$ and
views~$v$ to create new post-views.  However, it turns out that many of these
induced transitions create identical post-views.
%
\begin{enumerate}
\item\label{case:avoid-induced-1} If $pre.\princ = v.\princ$ then the induced
  transition is identical to the original transition. 

\item\label{case:avoid-induced-2} Suppose $pre.\fixed = post.\fixed$ and $v$
  contains no component that changes state in the transition.  Then the new
  view~$v'$ is identical to~$v$.

\item\label{case:avoid-induced-3} Now suppose $pre.\fixed \ne post.\fixed$,
  but $v$ again contains no component that changes state in the transition.
  Then the new view~$v'$ will be different from~$v$.  However, suppose we
  subsequently consider another extended transition $pre' \trans{} post'$ with
  $post'.\fixed = post.\fixed$, and consider its effect on the same~$v$
  (necessarily $pre'.\fixed = pre.\fixed = v.\fixed$), and $v$ again contains
  no component that changes state in the transition.  Then this will induce a
  transition to the same post-view~$v'$.
\end{enumerate}
%
We identify such cases early, and avoid producing the duplicate post-views.
We discuss this further in Section~\ref{ssec:effectOn-full}. 
\end{opt}

\begin{impNote}
The above optimisation is implemented in
\texttt{Unification.isSufficientUnif}.  Not any more...
\end{impNote}

\begin{improve}
Can it be done earlier?  

Also, suppose a single component~$c$ of~$v$ changes state and either (1) is
the principal of~$V$, but doesn't gain any reference; or (2) is a secondary
component.  Then it is enough to consider one such case for each view~$v$ and
for each transition of the fixed processes and the component~$c$.  If there
are two such transitions $pre \trans{} post$ and $pre' \trans{} post'$ with
$post.\fixed = post'.\fixed$, and the same post-state~$c'$ for~$c$, then the
new view in each case will use $post.\fixed$ and with $c$ replaced by~$c'$
in~$v$, and maybe some components removed in case~(1).

This suggests storing tuples containing $v$, the index of~$c$ within~$v$, the
new state~$c'$, and $post.\fixed$ (at this level of abstraction).  When we
also consider symmetry reduction, I think we should be storing
$\pi\inverse(c')$. 

\framebox{Consider this}  
\end{improve}


%%%%%%%%%%%%%%%%%%%%%%%%%%%%%%%%%%%%%%%%%%%%%%%%%%%%%%%

\subsection{Correctness}
\label{sec:views-correctness}


The following definition summarises how we calculate the abstract transitions.

\begin{definition}
\label{def:abstract-transition}
Given a set $V$ of views, we build each active-process extended transition
$pre \trans[e] post$, and the corresponding abstract transitions $v \trans[e]
v'$, as in Definition~\ref{def:active-process-transition}.
%
We then also create every transition that $pre \trans[e] post$ induces on a
view in~$V$, as in Definition~\ref{def:induced-transition}. 
\end{definition}
%
%% Our goal in this section is to show that the views produced by such
%% transitions include all of~$aPost(V)$. 
%

The following proposition shows the correctness of this approach. 

%%%%%%%%%%%%%%%%%%%%%%%%%%%%%%%%%%%%%%%%%%%%%%%%%%%%%%%

\begin{prop}
\label{lem:abstract-transitions-sound}
Let $V$ be a set of views.  Then every abstract transition from~$V$ is
generated by the procedure in Definition~\ref{def:abstract-transition}.
\end{prop}

\begin{proof} 
Consider an abstract transition
\[
\gamma(V) \ni s \trans[e] s' \sqge_{\V} v'.
\]
We need to show that we generate $v'$ as part of the process described above.
%
Let $pId = v'.\princ.id$.
%%  be the identity of the principal of~$v'$, and suppose
%% that component has states $princ$ and $princ'$ in~$s$ and~$s'$, respectively
%% (so $v' = view(s', princ')$).
%
Lemma~\ref{lem:active-process-transitions} shows that, in several cases, an
active-process transition generates~$v'$. 
%
We perform a case analysis on the remaining cases.
%
\begin{enumerate}
%% \item
%% First suppose $pId$ is the active component in the concrete transition
%% \( {s \trans[e] s'} \).
%% Let $v = view(princ, s)$.  
%% %
%% Suppose another component is necessary for the transition, as in
%% Definition~\ref{def:active-process-transition}, i.e.~either another component
%% synchronises on the transition, or the principal gains a reference, and
%% suppose that component has state~$c$ in~$s$.  Then $c$ is consistent
%% with~$v$, since $v \uplus c \sqle s \in \gamma(V)$.  Hence $c$ is compatible
%% with~$v$, by Lemma~\ref{lem:consistent-implies-compatible}.
%% %
%% Let $pre = v \uplus c$, or let $pre = v$ if no additional component is
%% necessary; and let $post$ be the corresponding states in~$s'$.  Then the
%% technique of Definition~\ref{def:active-process-transition} builds the
%% transition \( pre \trans[e] post \).  Finally, the principal's view of $post$
%% is extracted; this equals~$v'$ by construction, as required. 

\item
Suppose the transition $s \trans[e] s'$ has an active component other
than~$pId$.  Let $pre \trans[e] post$ be the corresponding extended
transition, built as in Definition~\ref{def:active-process-transition}.  Let
$pId$ have state $princ$ in~$s$, and let
$v = view(princ, s)$.  Then necessarily $v$ and $pre$ are accordant, since
they are both substates of~$s$.  Then the transition $v \trans{} v'$ is
induced, as in Definition~\ref{def:induced-transition}.

\item\label{step:abs-trans-correct-3}
Finally suppose the transition $s \trans[e] s'$ has an active fixed process,
and involves a passive component other than~$pId$.  This case is identical to
the previous.
%% Consider an arbitrary corresponding view transition (i.e.~with an arbitrary
%% principal), and consider the corresponding extended transition $pre \trans[e]
%% post$.  This case is then identical to the previous case.
\end{enumerate}
\end{proof}

%%%%%

%% \begin{impNote}
%% The fact that point~3 covers all transitions with active fixed
%%   processes suggests that we could be more restrictive about generating such
%%   transitions in Definition~\ref{def:active-process-transition}.
%% \end{impNote}


\begin{opt}
Case~\ref{step:abs-trans-correct-3} above considered an \emph{arbitrary}
corresponding view transition.  All such view transitions will generate the
same abstract transition, so clearly, it would be enough to consider fewer, as
long as we include at least one.  In the case that the active fixed process
synchronises with a passive component~$c$, we consider just the view
transition with~$c$ as principal.  (In the case that the active fixed process
synchronises with no component, we consider all such view transitions; there
is potential for further optimisation here, but these cases seem rare.)
\end{opt}


%% If $s$ is a system state and $cpts$ is a set of component states with
%% identities disjoint from the identities of $s.\cpt$, we write $s \uplus cpts$
%% for $(s.\fixed, s.\cpt \union cpts)$.  Similarly, if $c$ is a component with
%% identities disjoint from those of $s.\cpts$, we write $s \uplus c$ for
%% $(s.\fixed, s.\cpt \union \set{c})$.
