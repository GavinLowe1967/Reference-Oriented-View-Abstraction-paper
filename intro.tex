\begin{abstract}
The \emph{parameterised verification problem} seeks to verify all members of
some collection of systems.  We consider the parameterised verification
problem applied to systems that are composed of an arbitrary number of
\emph{component processes}, together with some \emph{fixed processes}.
Processes synchronise together corresponding to steps of the system.  
Each component process has an \emph{identity} which may be passed to another
process and stored, and possibly subsequently passed to other processes.
Hence each process may store some references to other processes. 

We apply a variant of \emph{view abstraction} to this problem.  We define a
\emph{view} to record the state of the fixed processes, a particular component
called the principal, and all the components to which the principal holds a
reference.  Our approach calculates an upper bound on all views of all states
of systems with an arbitrary number of components.  Reasoning about these
views allows us to deduce correctness results about all such systems.  In
particular, we use techniques based on symmetry reduction to form this upper
bound.

%% We apply the techniques to a number of examples.  We can model both active
%% components, such as threads, and passive components, such as nodes in a linked
%% list: thus our approach allows the verification of unbounded concurrent
%% datatypes operated on by an unbounded number of threads.  We show how to
%% combine view abstraction with additional techniques in order to deal with
%% other potentially infinite aspects of the analysis: for example, we deal with
%% potentially infinite specifications, such as a datatype being a queue; and we
%% deal with unbounded types of data stored in a datatype.
\end{abstract}

%%%%%

\section{Introduction}

The \emph{parameterised verification problem} considers a collection of
systems $P(x)$ where the parameter~$x$ ranges over a potentially infinite set,
and asks whether such systems are correct for all values of~$x$.  In this
paper we consider the following instance of the parameterised verification
problem.  Each system is built from some number of \emph{component processes},
together with some \emph{fixed processes}.  The components may be taken from
one or more families; for instance, we will consider examples representing a
concurrent datatype based on a linked list: one family of components will
represent threads operating on the datatype; and another family will represent
nodes in the linked list.  All components from a particular family will be
symmetric to one another.  Thus these systems are parameterised by the number
of components from each family.  The fixed processes are independent
of the number of components; we use them to model shared resources such as
shared variables or locks; we also normally include a \emph{watchdog} as a
fixed process, which monitors the execution and signals an error if the
desired specification is not met.

The components and fixed processes synchronise together corresponding to a
system step.  For example, a system step where a thread reads or writes a
field of a node will correspond to a synchronisation between the thread and
the node; likewise, a system step where a thread reads or writes a shared
variable will correspond to a synchronisation between the thread and the fixed
process for the variable.  We call each synchronisation an \emph{event}: they
are analogous to CSP events~\cite{awr:UCS}.

Each component has an \emph{identity}, drawn from some potentially infinite
set.  These identities can be passed in events; thus a process can obtain the
identity of a component process, and possibly pass it on to a third process.
When process~$p$ holds the identity of a component~$c$, we say that $p$ has a
\emph{reference} to~$c$.  For example, a thread can hold references to nodes
that it is operating on; and a node can hold a reference to the next node in
the linked list.  This means that each process has a potentially infinite
state space (a finite control state combined with references from a
potentially infinite set).  We describe the setting for our work more formally
in the next section.

By varying the types of the identities, we vary the number of components in the
system.  Thus we are considering a family of systems, parameterised by the
types of identities.  The problem of showing that all such systems are
correct is undecidable in general~\cite{apt-kozen,tomasz-gavin-CA}; however,
verification techniques prove effective on a number of specific instances.

We apply the technique of \emph{view abstraction} to this problem; the general
technique was originally developed by Abdulla et al.~\cite{AHH}, and
subsequently adapted to this setting by us~\cite{gavin:view-abs}.  The idea of
view abstraction is that we abstract each system state by its \emph{views},
where each view records the states of some subset of the processes.

For each analysis, we fix a set of views of interest.  For example, Abdulla et
al.~\cite{AHH} took each view to contain some fixed number~$k$ of processes.
In our earlier work~\cite{gavin:view-abs}, we took each view to comprise the
fixed processes and some bounded number of components from each family (for
example, two threads and one node, or one thread and two nodes).  
We then calculate an upper bound on the set of such views of all
reachable states of all systems of arbitrary size.  By reasoning about this
set of views, we can show that all such systems are correct.

However, the approaches of the earlier papers suffer from a state-space
explosion.  Many views contain processes that are uncorrelated; these do not
help to capture useful properties of the system; but they do add a lot of
states to the analysis.

In this paper we propose a variant of view abstraction that aims to overcome
this state-space explosion, by restricting the set of views we consider.  The
aim is to pick views that contain processes whose states are more correlated
to one another.  More precisely, ech view is based on a particular component,
which we call the \emph{principal}.  The view contains the state of the
principal~$p$, the state of each component to which $p$ has a reference, and
the fixed processes.  For example, a view with a thread as principal would
record the state of that thread, the state of each node to which it has a
reference, and the fixed processes; a view with a node as principal would
record the state of that node, the state of the next node in the list, and the
fixed processes.  In each case, we can expect the states of those components
to be correlated in some way, i.e.~only certain combinations will be possible,
reflecting invariants of the setting under consideration. 



\framebox{More overview of rest of paper}

Invariants captured

Contributions

\framebox{Related work}

