\section{Reference-oriented views}
\label{sec:views}

We now define the set of views we use in our approach.  We say that a process
state $q$ \emph{references} a component state~$c$ if $c.\id \in q.\params$.
Note that $c.\id$ is not a distinguished value.

Consider a system state $s$, and a particular component $p \in s.\cpts$.  We
write $view(s, p)$ for the view of the system state from~$p$, i.e.~the
fixed processes of~$s$, and all the components of~$s$ to which $p$ has a
reference (including itself):
%
%% Let $\cpts' = \set{c \in \cpts | c.id \in cpt.params}$ be all
%% components in $\cpts$ that are referenced by $cpt$ (including $cpt$ itself).
%% We define the corresponding view to be $(\fixed, \cpt')$.  Write this as
%% $view(s, cpt)$.  
%
\begin{eqnarray*}
view(s, p) & = &
  (s.\fixed, \set{c \in s.\cpts \| c.\id \in p.\params}).
\end{eqnarray*}
%
We say that $p$ is the \emph{principal component} of this view, and the
other components are \emph{secondary components}.  Given a view~$v$, we write
$v.\princ$ for its principal component. 

We then define the set~$\V$ of views and the abstraction function~$\alpha$ in
terms of all such views.
%
\begin{eqnarray*}
\alpha(s) & = & \set{ view(s, p) \| p \in s.\cpts } \\
\V & = & \Union_{s \in \S} \alpha(s)
\end{eqnarray*}

In examples, we will write views by listing their fixed components in some
standard order, then listing the components with the principal first, and then
following the order of the references in the principal's parameters.  (Our
implementation uses the same order.)  For instance, in the example of
Figure~\ref{fig:lock-based-queue}, the views will include,
\[
\begin{align}
(\Lock'(t), \Head(n_h), \Last(n_l), \Con_3; 
  Enq_4(t, n_l, n), \Node_y(n_l, null),\Node_x(n, n_l)),
\end{align}
\]
for $t \in ThreadID$, $n_h, n_l, n \in NodeID$, and~$x, y \in D$.  Here the
principal~$t$ has references to~$n_l$ and~$n$, so the two corresponding nodes
are included as secondary components.
Likewise, the following would be a view of the node~$n_1$:
\[
(\Lock'(t), \Head(n_h), \Last(n_l), \Con_3; \Node_x(n_1, n_2), \Node(n_2, n_3)).
\]

The initial views $AInit$ are of the form
\[
\begin{array}{ll}
(\Lock, \Head(null), \Last(null), \Con_0; Thread(t)), &
   \mbox{for $t \in ThreadID$}, \\
(\Lock, \Head(null), \Last(null), \Con_0; \InitNode(n)), & 
   \mbox{for $n \in NodeID$}.
\end{array}
\]
This satisfies the condition $\alpha(Init) \subseteq AInit$ of
Proposition~\ref{prop:reachable}, where $Init$ is the set of initial system
states from Section~\ref{sec:setting}.

For examples based on Figure~\ref{fig:lock-based-queue}, in the interests of
conciseness, we define
\begin{eqnarray*}
fixed(t, n_h, n_l) & = & (\Lock'(t), \Head(n_h), \Last(n_l), \Con_3), \\
fixed(\_, n_h, n_l) & = & (\Lock, \Head(n_h), \Last(n_l), \Con_3).
\end{eqnarray*}
%
These are fixed processes where: the head and last nodes are $n_h$ and~$n_l$;
the lock is held by~$t$ or not held (respectively); and the constructor has
completed.

Note that if one of the principal's parameters is a distinguished value, then
there is no corresponding component in the view.  For example, the following
would be a valid view of node~$n$, whose |next| reference is the distinguished
value~|Null|: 
%% , where the thread's second reference is the distinguished value~$Null$ (in
%% fact, no such state is reachable in the lock-based queue example).
\[
(fixed(t, n_h, n_l);  \Node_x(n, Null)).
\]


%% We let $\V$ be the set of all views of system states:
%% %
%% \begin{eqnarray*}
%% \V & = & 
%%   \set{view(s, cpt) \| s \in \S, cpt \in s.\cpt}. %%  \union 
%%   %% \set{srvView(s) | s \in \S}.
%% \end{eqnarray*}
%% %
%% In the following, we use the word ``view'' for a member of~$\V$, and the word
%% ``substate'' (or just ``state'') for a general substate of an element of~$\S$.

%%%%%%%%%%%%%%%%%%%%%%%%%%%%%%%%%%%%%%%%%%%%%%%%%%%%%%%

% \subsection{Calculating abstract transitions}

{prop:reachable}

The critical thing. in order to apply Proposition~\ref{prop:reachable}, is
that given $V \subseteq \V$, we need to calculate a representation of
%
\begin{eqnarray*}
aPost(V) & = & \alpha(post(\gamma(V)))
\end{eqnarray*}
efficiently.  In fact, we slightly over-estimate $aPost(V)$.  To this end, we
need to deal with the following difficulties.
%
\begin{itemize}
\item Since we are dealing with an infinite number of systems, we need to deal
  with an infinite number of views.  Indeed, even the set of initial views,
  described above, is infinite, since the types of thread and node identities
  are unbounded.  We will deal with this using symmetry reduction.  If two
  views are symmetric to one another, i.e., one can be obtained from the other
  by a uniform remapping of parameters, then the subsequent behaviours of one
  can be deduced from the behaviours of the other: in particular, if one is
  error-free, then so is the other.  Thus it will be enough to consider just
  one of them, or, more generally, just one element of each symmetry
  equivalence class.  We will formalise these ideas in
  Section~\ref{sec:symmetry}.

\item Even if $V$ is finite, the set $\gamma(V)$ can become very large.  We
  want to calculate an upper bound to $aPost(V)$ without calculating all of
  $\gamma(V)$.  We describe how to do this in Section~\ref{sec:transitions}. 
\end{itemize}

%% We are seeking to identify views~$v$ such that
%% \[
%% \gamma(V) \ni s \trans{a} s' \sqge_\V v
%% \]
%% for some $s$, $s'$, $a$.



%%%%%%%%%%%%%%%%%%%%%%%%%%%%%%%%%%%%%%%%%%%%%%%%%%%%%%%
