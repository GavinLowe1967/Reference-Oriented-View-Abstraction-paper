\subsection{Restricted views}
\label{sec:effectOn-restricted}

We now consider how to adapt the approach to deal with restricted views.


\begin{definition} 
\label{def:representative-consistent-extension-singleRef}
Suppose we are performing an analysis using restricted views.  Consider an
extended transition~$pre \trans{e} post$ and a view~$v$.  Consider a renaming
function~$\pi$ that unifies some (maybe none) of the components of~$v$ with
components of~$pre$.  

We define a consistent extension~$\pi'$ of~$\pi$ to be a
\emph{representative primary consistent extension} if  $\pi'$ is as in
  Definition~\ref{def:representative-consistent-extension}, except with the
  addition of allowing $\pi'(x)$ to be
\begin{enumerate}
\item[5.] a parameter of a component of~$pre$.
\end{enumerate}

We define a consistent extension~$\pi'$ of~$\pi$ to be a \emph{representative
  secondary consistent extension} if a secondary component~$cId$ in the
transition acquires a parameter~$p$ of the same type as the identity $pId$ of
$v.\princ$, $\pi'(pId) = p$, and every other parameter of $v.\princ$ is
remapped either to (1)~a parameter of $post.\fixed$, (2)~a parameter of the
state of $cId$ in $post$, (3) a parameter of a component of~$pre$, or (4)~the
minimal fresh parameter.
\end{definition}

\framebox{FIX ME} I think the current implementation doesn't include~(3). 

*** Example here

\begin{prop}
Consider an extended transition $pre \trans{e} post$ and a view~$v$, both in
normal form, and a renaming function~$\pi$ unifying some components (maybe
none).  Let $\pi'$ be a consistent extension of~$\pi$, and let $v'$ be the
view produced by Definition~\ref{def:induced-transition-singleRef} considering
the effect of $pre \trans{e} post$ on~$\pi'(v)$.
%
Then there is an extension~$\pi''$ of~$\pi$ such that
%
\begin{itemize}
\item if the induced transition is a primary induced transition, then $\pi''$
  is a representative primary consistent extension; and

\item If the induced transition is a secondary induced transition, then $\pi''$
  is a representative secondary consistent extension; 
\end{itemize}
and in each case, letting $v''$ be the view produced by
Definition~\ref{def:induced-transition-singleRef} considering the effect of
$pre \trans{e} post$ on~$\pi''(v)$, we have $v'' \equiv v'$.
\end{prop}

%%%%%

\begin{proof}
We start by considering primary induced transitions.  We describe how to
construct the renaming~$\pi''$ and a renaming~$\hat\pi$ such that $\pi' =
\hat\pi \circ \pi''$.  The proof is similar to that of
Proposition~\ref{prop:unifying-renaming}.  Below, we concentrate on aspects of
the proof that are different.

As in the proof of Proposition~\ref{prop:unifying-renaming}, for each
parameter~$x$ of~$v$ such that $y = \pi'(x)$ is a parameter of either
$post.\fixed$ or a component of $v'$ taken from $post$, we define $\pi''(x) =
y$ and $\hat\pi(y) = y$.  The details are then as earlier.

In addition, for clause (b) of
Definition~\ref{def:induced-transition-singleRef}, suppose a component~$c$ of
$pre$ has a reference to a component~$\pi'(c')$ of~$\pi'(v)$, or vice versa.
Then, since clause~(b) is satisfied for~$\pi'(v)$, the current set of
representative views must contain a view equivalent to 
\begin{eqnarray*}
v_\cross & = & (pre.\fixed, \set{c,\pi'(c')}).
\end{eqnarray*}
%
We will ensure that $\pi''$ is such that
%
\begin{eqnarray*}
v_\cross' & = & (pre.\fixed, \set{c,\pi''(c')})
\end{eqnarray*}
%
is such that $v_\cross \equiv v_\cross'$, and hence that the current set of
representative views must contain a view equivalent to $v_\cross'$, as
required for clause~(b) for~$\pi''(v)$.  To this end, if $x$ is a parameter
of~$c'$ such that $y = \pi'(x)$ is a parameter of~$c$, then define $\pi''(x) =
y$ and $\hat\pi(y) = y$.  Note that $y$ is included under condition~5.  Also,
if $x$ was included under one of the cases considered earlier, then this
definition is consistent with that earlier case.  We return to clause~(b)
below, and justify that $v_\cross \equiv v_\cross'$.

In general, there might be several views required to satisfy clause (b),
corresponding to different cross references.  We perform the above procedure
for each such. 

As in the proof of Proposition~\ref{prop:unifying-renaming}, each other
parameter~$y$ in~$v'$ must necessarily be in a component taken from~$\pi'(v)$.
As earlier, if the parameter is $\pi'(x)$ with $x$ not in the domain of~$\pi$,
we define $\pi''(x)$ to be the minimal fresh parameter~$z$, and we let
$\hat\pi(z) = y$.  The details are as earlier.

We now return to clause~(b) and justify that $v_\cross \equiv v_\cross'$.
Note that we have defined $\pi''$ to agree with~$\pi'$, other than for the
case in the previous paragraph.  Hence $v_\cross$ and $v_\cross'$ differ only
on parameters~$x$ of~$c'$ such that $\pi''(x)$ is a fresh parameter.  Note
that each such fresh parameter is distinct from the parameters in~$pre.\fixed$
and~$c$ (by the definition of freshness).
%
Likewise, each corresponding parameter $\pi'(x)$ in $\pi'(c)$ is distinct from
the parameters of~$pre.\fixed$ (since each parameter in $pre.\fixed$ is in the
range of~$\pi$); and $\pi'(x)$ is distinct from any parameter of~$c$ (since
such~$x$ are not mapped to fresh parameters by~$\pi''$).
%
Thus, $v_\cross$ and $v_\cross'$ are related via the remapping that maps
each such $\pi''(x)$ to~$\pi'(x)$, and is otherwise the identity (this is the
same as the remapping that extends $\hat\pi$ with the identity function over
the parameters of~$c$).  Hence clause~(b) is satisfied for~$\pi''(v)$.

For clause (c) of Definition~\ref{def:induced-transition-singleRef}, suppose
the principals of $pre$ and $\pi'(v)$ both have a reference to the same
missing component, with identity~$id$.  Suppose the corresponding parameter in
the principal of~$v$ is~$x$, so $\pi'(x) = id$.  Then, since clause~(c) is
satisfied for~$\pi'(v)$, there is some component~$c$ with identity~$id$ such
that the current set of representative views must contain views equivalent to
each of
\[
(pre.\fixed, \set{pre.\princ, c}) \qquad\mbox{and}\qquad
(pre.\fixed, \set{\pi'(v).\princ, c}).
\]
We can assume, without loss of generality, that~$c$ contains none of the fresh
parameters chosen when defining~$\pi''$: if not, we can rename those
parameters of~$c$ to distinct fresh parameters.  Then
\begin{eqnarray*}
(pre.\fixed, \set{\pi'(v).\princ, c}) & \equiv & 
  (pre.\fixed, \set{\pi''(v).\princ, c}).
\end{eqnarray*}
They are related by~$\hat\pi$, extended by the identity function over any
additional parameters in~$c$.  Hence clause~(c) is satisfied for~$\pi''(v)$.

\paragraph{Secondary induced transitions}

Suppose a secondary component $sc$ of $post$ has a reference in parameter~$p$
to~$\pi'(v).\princ$, so a secondary induced transition from~$\pi'(v)$ produces
$(post.\fixed, \set{sc, \pi'(v).\princ})$.  Let $pId = v.\princ.\id$, so
$\pi'(pId) = p$.

We construct a corresponding representative secondary consistent
extension~$\pi''$ of~$\pi$ as follows.

We define $\pi''(pId) = p$ and $\hat\pi(p) = p$. 

For each other parameter~$x$ of $v.\princ$ such that $y = \pi'(x)$ is a
parameter of either $post.\fixed$ or~$sc$, we define $\pi''(x) = y$ and
$\hat\pi(y) = y$.  Each such value~$y$ is included under case~1 or~2 of
Definition~\ref{def:25}.   

We deal with clause~(b) of Definition~\ref{def:induced-transition-singleRef}
as earlier.  Relevant values are included under case~3 of
Definition~\ref{def:25}.

All other parameters map to minimal fresh param, as earlier ........

We deal with clause~(c) as earlier. 
\end{proof}

Note: above suggests we can do with fewer choices under 5 (primary case).  It
should only be invoked to map a parameter of~$c'$ to match a parameter of~$c$
if $c$ has a reference to the renaming of~$c'$, or vice versa.  Perhaps:
\begin{itemize}
\item for each $c \in pre$, for each $c' \in v$, for each parameter~$y$ of~$c$,
  map $c'.\id$ to~$y$ (if possible), and then continue, allowing remapping of
  other parameters of~$c'$ to other parameters of~$c$;

\item for each $c \in pre$, for each $c' \in v$, for each parameter~$y$
  of~$c'$, map $y$ to $c.\id$ (if possible), and then continue, allowing
  remapping of other parameters of~$c'$ to other parameters of~$c$.
\end{itemize}
%
But there might be multiple instances.  Perhaps start by considering all
combinations of cross references. 

%%%%%%%%%%%%%%%%%%%%%%%%%%%%%%%%%%%%%%%%%%%%%%%%%%%%%%%

\subsubsection{Previous version}

\framebox{Note, 2021/08/10.}  I think what follows isn't what we want: it is
proving the wrong result.  It considers a representative remapping to make $pre$
and $v$ accordant.  If two remappings would  produce the same
induced transition, only one is considered.  However, the side conditions of
Definition~\ref{def:induced-transition-singleRef} might be satisfied by one of
those remappings and not the other.  More concretely, consider the transition
\begin{eqnarray*}
(fixed(N_1); Th(T_1,N_1), Nd(N_2,N1)) & \trans{} &
 (fixed'(N_1); Th'(T_1), Nd(N_2,N_1))
\end{eqnarray*}
on
\begin{eqnarray*}
v & = & (fixed(N_1); Th''(T_1,N_2,N_1), Nd'(N_1,N_2)).
\end{eqnarray*}
%
At present, this will consider only the remapping $\set{T_1 \mapsto T_2, N_1
  \mapsto N_1,\linebreak[1] {N_2 \mapsto N_3}}$, which would require (for
condition~(b)) the view
\[
(fixed(N_1); Nd(N_2,N_1), Nd'(N_1,N_3))
\]
to induce a transition to
\[
\begin{align}
(fixed'(N_1); Th''(T_2,N_3,N_1), Nd'(N_1,N_3)) \equiv \\
\qquad (fixed'(N_1); Th''(T_1,N_2,N_1), Nd'(N_1,N_2)).
\end{align}
\]
We should also consider the remapping $\set{T_1 \mapsto T_2, N_1 \mapsto N_1,
  N_2 \mapsto N_2}$, which would require the view
\[
(fixed(N_1); Nd(N_2,N1), Nd'(N_1,N_2))
\]
to induce a transition to
\[
\begin{align}
(fixed'(N_1); Th''(T_2,N_2,N_1), Nd'(N_1,N_2)) \equiv \\
\qquad (fixed'(N_1); Th''(T_1,N_2,N_1), Nd'(N_1,N_2)).
\end{align}
\]
I think it would be sound to allow a parameter to be remapped to an arbitrary
parameter in~$pre$ (subject to the normal conditions).  It might be possible
to do with fewer cases. 


\begin{definition} 
%\label{def:representative-consistent-extension-singleRef}
Suppose we are performing an analysis using restricted views.  Consider an
extended transition~$pre \trans{e} post$ and a view~$v$.  Consider a renaming
function~$\pi$ that unifies some (maybe none) of the components of~$v$ with
components of~$pre$.  

We define a consistent extension~$\pi'$ of~$\pi$ to be a
\emph{representative primary consistent extension} if  $\pi'$ is as in
  Definition~\ref{def:representative-consistent-extension}.

We define a consistent extension~$\pi'$ of~$\pi$ to be a \emph{representative
  secondary consistent extension} if a secondary component~$cId$ in the
transition acquires a parameter~$p$ of the same type as the identity $pId$ of
$v.\princ$, $\pi'(pId) = p$, and every other parameter of $v.\princ$ is
remapped either to (1)~a parameter of $post.\fixed$, (2)~a parameter of the
state of $cId$ in $post$, or (3)~the minimal fresh parameter.
\end{definition}

%%%%%

\begin{example}
Consider the effect of the transition
\[
\begin{align}
(fixed; Th(T_0,N_0,N_1), Nd_x(N_0,N_2,null))   \trans{setNext.T_0.N_0.N_1} \\
\qquad (fixed'(N_3); Th'(T_0,N_0,N_1), Nd_x(N_0,N_2,N_1)) 
\end{align}
\]
on
\begin{eqnarray*}
v & = &  (fixed; Nd_y(N_0,N_1), Nd_z(N_1,null).
\end{eqnarray*}
%
This has representative secondary consistent extensions
\begin{eqnarray*}
\pi_X & = & \set{N_0 \mapsto N_1, N_1 \mapsto X},
  \quad \mbox{for $X = N_2, N_3, N_4$}.
\end{eqnarray*}
%
The secondary component of the transition acquires a reference to~$N_1$, so
the identity~$N_0$ of~$v.\princ$ is mapped to to~$N_1$.  The parameter~$N_1$
can be mapped to match the parameter~$N_3$ of the fixed processes, the
parameter~$N_2$ of the secondary component of the transition, or the minimal
fresh parameter~$N_4$.
%
This gives secondary transitions producing
\[{\it
(fixed'(N_3); Nd_x(N_0,N_2,N_1), Nd_y(N_1,X)),}
\] 
which are then reduced to distinct normal forms.
\end{example}

%% \framebox{The current implementation is incorrect.}  
%% I think we need all parameters of that secondary component in the
%% post-state.  Consider \[ \begin{align} (fixed; Th(t, n_1, n_2), Nd_A(n_1,
%% n_3, n_4)) \trans{setNext.t.n_1.n_2} \\ \qquad (fixed'; Th'(t), Nd_A(n_1,
%% n_2, n_4)) \end{align} \] on $(fixed; Nd_B(n_2, n_4))$.  This induces a
%% transition producing $(fixed'; Nd_A(n_1, n_2, n_4), Nd_B(n_2, n_4))$.  So
%% when producing a secondary induced transition, we need to include all
%% parameters of that secondary component in the post-state.  At present, the
%% implementation includes all parameters of all components in the \emph{pre}-
%% state (which is also inefficient).

%%%%%

\begin{prop}
Consider an extended transition $pre \trans{e} post$ and a view~$v$, both in
normal form, and a renaming function~$\pi$ unifying some components (maybe
none).  Let $\pi'$ be a consistent extension of~$\pi$, and let $v'$ be the
view produced by Definition~\ref{def:induced-transition-singleRef} considering
the effect of $pre \trans{e} post$ on~$\pi'(v)$.
%
Then there is an extension~$\pi''$ of~$\pi$ such that
%
\begin{itemize}
\item if the induced transition is a primary induced transition, then $\pi''$
  is a representative primary consistent extension; and

\item If the induced transition is a secondary induced transition, then $\pi''$
  is a representative secondary consistent extension; 
\end{itemize}
and in each case, letting $v''$ be the view produced by
Definition~\ref{def:induced-transition-singleRef} considering the effect of
$pre \trans{e} post$ on~$\pi''(v)$, we have $v'' \equiv v'$.
\end{prop}

%%%%%

\begin{proof}
For primary induced transitions, the proof is as for
Proposition~\ref{prop:unifying-renaming}.

Consider a secondary induced transition, producing~$v'$ as in the statement of
the proposition.  Then
%
\begin{eqnarray*}
v' & = & (post.\fixed, \set{sc, \pi'(v.\princ)}),
\end{eqnarray*}
%
where $sc$ is the secondary component that gains a reference to
$\pi'(v.\princ)$.  We describe how to construct~$\pi''$ and a
renaming~$\hat\pi$ such that $\pi' = \hat\pi \circ \pi''$.  This will imply
that $\hat\pi(v'') = v'$.

Let $pId = v.\princ.\id$ and $p = \pi'(pId)$; this equals the parameter
of~$sc$ that is a reference to $\pi'(v.\princ)$.  We define $\pi''(pId) = p$,
and $\hat\pi(p) = p$.  This is a consistent extension of~$\pi$, because~$\pi'$
is a consistent extension of~$\pi$.

For every other parameter~$x$ of~$v.\princ$ such that $y = \pi'(x)$ appears
in~$post.\fixed$ or~$sc$,  we define $\pi''(x) = y$, and $\hat\pi(y) = y$.
Note that each such value~$y$ is included under case~(1) or~(2) of
Definition~\ref{def:representative-consistent-extension-singleRef}.  Again,
this is a consistent extension of~$\pi$, because~$\pi'$ is a consistent
extension of~$\pi$.

For every other parameter~$x$ of $v.\princ$ not in the domain of~$\pi$, we
define $\pi''(x)$ to be the minimal fresh parameter~$z$, and define
$\hat\pi(z) = \pi'(x)$.  This is a consistent extension of~$\pi$, by
construction.
\end{proof}


% LocalWords:  unifications
