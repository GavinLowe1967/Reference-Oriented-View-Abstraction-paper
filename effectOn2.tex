\section{Symmetry reduction with reduced views}
\label{sec:effectOn-restricted}

\framebox{More here}

We now consider how to calculate induced transitions within the context of
symmetry reduction.  This adapts the technique with full views, described in
Section~\ref{sec:induced-symmetry}.

%%%%%%%%%%%%%%%%%%%%%%%%%%%%%%%%%%%%%%%%%%%%%%%%%%%%%%%

\subsubsection{Primary induced transitions}

The following definition is analogous to
Definition~\ref{def:representative-consistent-extension}.
%
\begin{definition} 
\label{def:representative-consistent-extension-singleRef-primary}
Suppose we are performing an analysis using restricted views.  Consider an
extended transition~$pre \trans{e} post$ and a view~$v$, both in normal form.
Consider a partial unification function~$\pi$.
%
We define a consistent extension~$\pi'$ of~$\pi$ to be a \emph{representative
  primary consistent extension} if $\pi'$ is as in
Definition~\ref{def:representative-consistent-extension}, except with the
addition of allowing $\pi'(x)$ to be
\begin{enumerate}
\item[5.] a parameter of a component $c$ of~$pre$ if $c$ has a reference to a
  component of~$\pi'(v)$ or vice versa.
\end{enumerate}
\end{definition}

%%%%%

We give an example to illustrate the necessity of the additional clause~5.
Consider the effect of the transition
\begin{eqnarray*}
(fixed(N_0); Th(T_0,N_1), Nd(N_1,N_0)) & \trans{} &
 (fixed'(N_0); Th'(T_0), Nd(N_1,N_0))
\end{eqnarray*}
on
\begin{eqnarray*}
v & = & (fixed(N_0); Th''(T_0,N_0), Nd'(N_0,N_1)).
\end{eqnarray*}
%
No unification of components is possible.  Without clause~5, we would consider
only the renaming $\pi' = \set{T_0 \mapsto T_1, N_0 \mapsto N_0,\linebreak[1]
  {N_1 \mapsto N_2}}$.  Condition~(b) of
Definition~\ref{def:induced-transition-singleRef}, applied to $\pi'(v)$, would
then require the current set of representative views to contain
%
\begin{eqnarray*}
v_\cross & = & (fixed(N_0); Nd(N_1,N_0), Nd'(N_0,N_2))
\end{eqnarray*}
in order to induce a transition to
\[
\begin{align}
(fixed'(N_0); Th''(T_1,N_0),  Nd'(N_0,N_2)) \equiv \\
\qquad  (fixed'(N_0); Th''(T_0,N_0), Nd'(N_0,N_1)).
\end{align}
\]
However, suppose the current representative views do not contain~$v_\cross$,
but do contain the two views
\begin{eqnarray*}
v_\cross' & = & (fixed(N_0); Nd(N_1,N_0), Nd'(N_0,N_1)) \\
v_\cross'' & = & (fixed(N_0); Nd'(N_0,N_1), Nd(N_1,N_0)).
\end{eqnarray*}
%
With the additional clause~5, we also consider the renaming $\pi'' =
\set{{T_0 \mapsto T_1}, \linebreak[1] N_0 \mapsto N_0,\linebreak[1] {N_1
    \mapsto N_1}}$, because the component $c = Nd(N_1,N_0)$ contains a
reference to the component $Nd'(N_0,N_1)$ of~$\pi''(v)$.  Condition~(b) is now
satisfied for $\pi''(v)$, because of the presence of~$v_\cross'$
and~$v_\cross''$.  This then induces a transition to
\[
\begin{align}
(fixed'(N_0); Th''(T_1,N_0), Nd'(N_0,N_1)) \equiv \\
\qquad (fixed'(N_0); Th''(T_0,N_0), Nd'(N_0,N_1)),
\end{align}
\]
as one would expect.

\framebox{IMPROVE:} improve example, so as to force agreement on some third
parameter rather than the ugly cycle of references. 

%%%%%

The following proposition is analagous to 
Proposition~\ref{prop:unifying-renaming}.
%
\begin{prop}
\label{prop:unifying-renaming-singleRef-primary}
Consider an extended transition $pre \trans{e} post$ and a view~$v$, both in
normal form, and  a minimal renaming function~$\pi$ that is the identity on
parameters of $v.\fixed$, and that unifies some components (maybe none).
%
Let $\pi'$ be a consistent extension of~$\pi$, and let $v'$ be the view
produced by a primary induced transition considering the effect of $pre
\trans{e} post$ on~$\pi'(v)$.
%
Then there is a representative primary consistent extension~$\pi''$ of~$\pi$
such that, letting $v''$ be the view produced by a primary induced transition
considering the effect of $pre \trans{e} post$ on~$\pi''(v)$, we have $v''
\equiv v'$.
\end{prop}

%%%%%

\begin{proof}
We describe how to construct the renaming~$\pi''$ and a renaming~$\hat\pi$
such that $\pi' = \hat\pi \circ \pi''$.  The proof is similar to that of
Proposition~\ref{prop:unifying-renaming}.  Below, we concentrate on aspects of
the proof that are different.

As in the proof of Proposition~\ref{prop:unifying-renaming}, for each
parameter~$x$ of~$v$ such that $y = \pi'(x)$ is a parameter of either
$post.\fixed$ or a component of $v'$ taken from $post$, we define $\pi''(x) =
y$ and $\hat\pi(y) = y$.  The details are then as earlier.

In addition, for clause (b) of
Definition~\ref{def:induced-transition-singleRef}, suppose a component~$c$ of
$pre$ has a reference to a component~$\pi'(c')$ of~$\pi'(v)$, or vice versa.
Then, since clause~(b) is satisfied for~$\pi'(v)$, the current set of
representative views must contain a view equivalent to 
\begin{eqnarray*}
v_\cross & = & (pre.\fixed, \set{c,\pi'(c')}).
\end{eqnarray*}
%
We will ensure that $\pi''$ is such that
%
\begin{eqnarray*}
v_\cross' & = & (pre.\fixed, \set{c,\pi''(c')})
\end{eqnarray*}
%
is such that $v_\cross \equiv v_\cross'$, and hence that the current set of
representative views must contain a view equivalent to $v_\cross'$, as
required for clause~(b) for~$\pi''(v)$.  To this end, if $x$ is a parameter
of~$c'$ such that $y = \pi'(x)$ is a parameter of~$c$, then define $\pi''(x) =
y$ and $\hat\pi(y) = y$.  Note that $y$ is included under condition~5.  Also,
if $x$ was included under one of the cases considered earlier, then this
definition is consistent with that earlier case.  We return to clause~(b)
below, and justify that $v_\cross \equiv v_\cross'$.

In general, there might be several views required to satisfy clause (b),
corresponding to different cross references.  We perform the above procedure
for each such. 

As in the proof of Proposition~\ref{prop:unifying-renaming}, each other
parameter~$y$ in~$v'$ must necessarily be in a component taken from~$\pi'(v)$.
If the parameter is $\pi'(x)$ with $x$ not in the domain of~$\pi$,
we define $\pi''(x)$ to be the minimal fresh parameter~$z$, and we let
$\hat\pi(z) = y$.  The details are as earlier.

We now return to clause~(b) and justify that $v_\cross \equiv v_\cross'$.
Note that we have defined $\pi''$ to agree with~$\pi'$, other than for the
case in the previous paragraph.  Hence $v_\cross$ and $v_\cross'$ differ only
on parameters~$x$ of~$c'$ such that $z = \pi''(x)$ is a fresh parameter.  Note
that each such fresh parameter~$z$ is distinct from the parameters
in~$pre.\fixed$ and~$c$ (by the definition of freshness).
%
Likewise, each corresponding parameter $y = \pi'(x)$ in $\pi'(c)$ is distinct
from the parameters of~$pre.\fixed$ (since each parameter in $pre.\fixed$ is
in the range of~$\pi$); and $y$ is distinct from any parameter of~$c$
(since if $y = \pi'(x)$ equals a parameter of~$c$, then $\pi''$ maps~$x$
to~$y$, not to a fresh parameter).
%
Thus, $v_\cross$ and $v_\cross'$ are related via the renaming that maps each
such $z = \pi''(x)$ to~$y = \pi'(x)$, and is otherwise the identity (this is
the same as the renaming that extends $\hat\pi$ with the identity function
over the parameters of~$c$).  Hence clause~(b) is satisfied for~$\pi''(v)$.

For clause (c) of Definition~\ref{def:induced-transition-singleRef}, suppose
the principals of $pre$ and $\pi'(v)$ both have a reference to the same
missing component, with identity~$id$.  Suppose the corresponding parameter in
the principal of~$v$ is~$x$, so $\pi'(x) = id$.  Then, since clause~(c) is
satisfied for~$\pi'(v)$, there is some component~$c$ with identity~$id$ such
that the current set of representative views must contain views equivalent to
each of
\[
(pre.\fixed, \set{pre.\princ, c}) \qquad\mbox{and}\qquad
(pre.\fixed, \set{\pi'(v).\princ, c}).
\]
We can assume, without loss of generality, that~$c$ contains none of the fresh
parameters chosen when defining~$\pi''$: if not, we can rename those
parameters of~$c$ to distinct fresh parameters.  We need to show that the
current set of representative views also contains a view equivalent to
$(pre.\fixed, \set{\pi''(v).\princ, c})$.  But
\begin{eqnarray*}
(pre.\fixed, \set{\pi''(v).\princ, c}) & \equiv & 
  (pre.\fixed, \set{\pi'(v).\princ, c}).
\end{eqnarray*}
They are related by~$\hat\pi$, extended by the identity function over any
additional parameters in~$c$.  Hence clause~(c) is satisfied for~$\pi''(v)$.
\end{proof}

Note: above suggests we can do with fewer choices under 5 (primary case).  It
should only be invoked to map a parameter of~$c'$ to match a parameter of~$c$
if $c$ has a reference to the renaming of~$c'$, or vice versa.  Perhaps:
\begin{itemize}
\item for each $c \in pre$, for each $c' \in v$, for each parameter~$y$ of~$c$,
  map $c'.\id$ to~$y$ (if possible), and then continue, allowing renaming of
  other parameters of~$c'$ to other parameters of~$c$;

\item for each $c \in pre$, for each $c' \in v$, for each parameter~$y$
  of~$c'$, map $y$ to $c.\id$ (if possible), and then continue, allowing
  renaming of other parameters of~$c'$ to other parameters of~$c$.
\end{itemize}
%
But there might be multiple instances.  Perhaps start by considering all
combinations of cross references. 

%%%%%%%%%%%%%%%%%%%%%%%%%%%%%%%%%%%%%%%%%%%%%%%%%%%%%%%

\subsubsection{Secondary induced transitions}


\begin{definition} 
\label{def:representative-consistent-extension-singleRef-secondary}
Suppose we are performing an analysis using restricted views.  Consider an
extended transition~$pre \trans{e} post$ and a view~$v$, both in normal form.
Consider a partial unification function~$\pi$.
%
%% We define a consistent extension~$\pi'$ of~$\pi$ to be a
%% \emph{representative primary consistent extension} if  $\pi'$ is as in
%%   Definition~\ref{def:representative-consistent-extension}, except with the
%%   addition of allowing $\pi'(x)$ to be
%% \begin{enumerate}
%% \item[5.] a parameter of a component of~$pre$.
%% \end{enumerate}
%
We define a consistent extension~$\pi'$ of~$\pi$ to be a \emph{representative
  secondary consistent extension} if a secondary component~$cId$ in the
transition acquires a parameter~$p$ of the same type as the identity $pId$ of
$v.\princ$, $\pi'(pId) = p$, and every other parameter of $v.\princ$ is
remapped either to:
\begin{enumerate}
\item \label{item:effectOn-secondary-1} a parameter of $post.\fixed$,

\item\label{item:effectOn-secondary-2} a parameter of the
state of $cId$ in $post$, 

\item\label{item:effectOn-secondary-3} a parameter of a component of~$pre$, or

\item\label{item:effectOn-secondary-4} the minimal fresh parameter.
\end{enumerate}
\end{definition}

\framebox{FIX ME} I think the current implementation doesn't include~(3). 

%%%%%

\begin{example}
Consider the effect of the transition
\[
\begin{align}
(fixed; Th(T_0,N_0,N_1), Nd_x(N_0,N_2,null))   \trans{setNext.T_0.N_0.N_1} \\
\qquad (fixed'(N_3); Th'(T_0,N_0,N_1), Nd_x(N_0,N_2,N_1)) 
\end{align}
\]
on
\begin{eqnarray*}
v & = &  (fixed; Nd_y(N_0,N_1), Nd_z(N_1,null).
\end{eqnarray*}
%
No unification of components is possible.
This has representative secondary consistent extensions
\begin{eqnarray*}
\pi_X & = & \set{N_0 \mapsto N_1, N_1 \mapsto X},
  \quad \mbox{for $X = N_2, N_3, N_4$}.
\end{eqnarray*}
%
The secondary component of the transition acquires a reference to~$N_1$, so
the identity~$N_0$ of~$v.\princ$ is mapped to to~$N_1$.  The parameter~$N_1$
can be mapped to match: the parameter~$N_3$ of the fixed processes; the
parameter~$N_2$ of the secondary component $Nd_x(N_0,N_2,N_1)$ in the post
state (mapping to~$N_0$ would require unification, which is impossible; and
mapping to~$N_1$ would make the mapping non-injective); or the minimal fresh
parameter~$N_4$.
%
This gives secondary transitions producing
\[{\it
(fixed'(N_3); Nd_x(N_0,N_2,N_1), Nd_y(N_1,X)),}
\] 
which are then reduced to distinct normal forms.
\end{example}

\framebox{***} Above example doesn't include case~3.


The following proposition is analagous to
Propositions~\ref{prop:unifying-renaming}
and~\ref{prop:unifying-renaming-singleRef-primary}.
%
\begin{prop}
\label{prop:unifying-renaming-singleRef-secondary}
Consider an extended transition $pre \trans{e} post$ and a view~$v$, both in
normal form.  Consider also a partial unification function~$\pi$.
%
Let $\pi'$ be a consistent extension of~$\pi$, and let $v'$ be the view
produced by a secondary induced transition considering the effect of $pre
\trans{e} post$ on~$\pi'(v)$.
%
Then there is a representative secondary consistent extension~$\pi''$ of~$\pi$
such that, letting $v''$ be the view produced by a secondary induced
transition considering the effect of $pre \trans{e} post$ on~$\pi''(v)$, we
have $v'' \equiv v'$.
\end{prop}

%%%%%

\begin{proof}
Suppose a secondary component $sc$ of $post$ has a reference in parameter~$p$
to~$\pi'(v).\princ$, so a secondary induced transition from~$\pi'(v)$ produces
$(post.\fixed, \set{sc, \pi'(v).\princ})$.  Let $pId = v.\princ.\id$, so
$\pi'(pId) = p$.

We construct a corresponding representative secondary consistent
extension~$\pi''$ of~$\pi$, and $\hat\pi$ such that $\pi' = \hat\pi \circ
\pi''$, as follows.  The proof is similar to that of earlier propositions, so
we concentrate on the differences.  As required, we define $\pi''(pId) = p$
and $\hat\pi(p) = p$.

For each other parameter~$x$ of $v.\princ$ such that $y = \pi'(x)$ is a
parameter of either $post.\fixed$ or~$sc$, we define $\pi''(x) = y$ and
$\hat\pi(y) = y$.  Each such value~$y$ is included under
case~\ref{item:effectOn-secondary-1} or~\ref{item:effectOn-secondary-2} of
Definition~\ref{def:representative-consistent-extension-singleRef-secondary}.

We deal with clause~(b) of Definition~\ref{def:induced-transition-singleRef}
as in Proposition~\ref{prop:unifying-renaming-singleRef-primary}.  If a
component~$c$ of~$pre$ has a reference to a component~$\pi'(c')$ of~$\pi'(v)$,
or vice versa, then the current set of representative views must contain a
view equivalent to $v_\cross = (pre.\fixed, \set{c,\pi'(c')})$.  If $x$ is a
parameter of~$c'$ such that $y = \pi'(x)$ is a parameter of~$c$, then define
$\pi''(x) = y$ and $\hat\pi(y) = y$.  Such values~$y$ are included under
case~\ref{item:effectOn-secondary-3} of
Definition~\ref{def:representative-consistent-extension-singleRef-secondary}.
This ensures that $(pre.\fixed, \set{c,\pi''(c')}) \equiv v_\cross$, as
required for clause~(b) for~$\pi''(v)$.  

As in the proofs of earlier propositions, for each other parameter $y =
\pi'(x)$ in~$\pi'(c)$, we define~$\pi''(x)$ to be the minimal fresh
parameter~$z$, and let $\hat\pi(z) = y$.

We deal with clause~(c) as in
Proposition~\ref{prop:unifying-renaming-singleRef-primary}.
\end{proof}

%%%%%

\begin{opt}
\label{opt:effectOn-singleRef}
Recall Optimisation~\ref{opt:induced-trans-singleRef}.
\begin{enumerate}
\item For case~\ref{case:opt-induced-trans-singleRef-1}, we avoid unifying the
  principal of~$v$ with the principal pf~$pre$ and also unifying the secondary
  component of~$v$ with a secondary component of~$pre$. 

\item For case~\ref{case:opt-induced-trans-singleRef-2}, for primary induced
  transitions, if $pre.\fixed = post.\fixed$ we require that at least one
  component of~$v$ unifies with a component that changes state.

\item For case~\ref{case:opt-induced-trans-singleRef-3}, for secondary induced
  transitions, if $pre.\fixed = post.\fixed$ we require that $v.\princ$ is
  mapped to a reference of a secondary component of $post$ that changes state
  in the transition.  \framebox{DO THIS} -- maybe, there aren't many secondary
  transitions.
\end{enumerate}
\end{opt}

%%%%%%%%%%

The following two optimisations are more involved.
%
Let $\pi \rangeRes fixed$ represent $\pi$ range-restricted to the parameters
of~$fixed$: 
%
\begin{eqnarray*}
\pi \rangeRes fixed & = & \set{ x \mapsto y \| (x \mapsto y) \in \pi 
  \land \mbox{$y$ is a parameter of $fixed$} }. 
\end{eqnarray*}
%
% Note: domain restricting to the parameters of v.cpts gives no advantage:
% outside these parameters, \pi will be the identity on other params of
% v.\fixed. 
%

The following lemma will justify
Optimisation~\ref{opt:changing-servers-singleRef}. 
%
\begin{lemma}
\label{lem:changing-servers-singleRef}
Consider two transitions $pre \trans{} post$ and $pre' \trans{} post'$ such
that $pre.\fixed = pre'.\fixed$ and $post.\fixed = post'.\fixed$.  Consider
primary induced transitions based on a view~$v$.  Suppose we use remapping
functions~$\pi$ and~$\pi'$, respectively, such that in each case no component
is unified.  Suppose further that $\pi' \rangeRes post.\fixed = \pi \rangeRes
post.\fixed$.  Then the two induced transitions produce equivalent new views.
\end{lemma}

%%%%%

\begin{proof}
The fact that there are no unifications means that all components in the new
view are taken from $\pi(v)$ or $\pi'(v)$, respectively.  Thus the new views
are
\[
(post.\fixed, \pi(v.\cpts)) \qquad\mbox{and}\qquad 
 (post.\fixed, \pi'(v.\cpts)).
\]
These are equivalent: parameters of $post.\fixed$ appear in the same positions
in the two views (because of the condition relating $\pi$ and~$\pi'$); for
other parameters, the values $\pi(x)$ and $\pi'(x)$ appear in corresponding
positions of the two views.
\end{proof}

%%%%%

The following optimisation exploits the above lemma.  Note that we apply it
only when $pre.\fixed \ne post.\fixed$, or else this is subsumed within an
earlier optimisation.
%
\begin{opt}
\label{opt:changing-servers-singleRef}
We store a set of tuples $(post.\fixed, v, \pi \rangeRes post.\fixed)$ such that
we have produced a primary induced transition based on a transition $pre
\trans{} post$ acting on~$v$ using unification function~$\pi$, and such that
$pre.\fixed \ne post.\fixed$ and $v$ unifies with no component. 

Suppose subsequently we consider a similar case: that is, the same
$post.\fixed$ and~$v$, and a renaming $\pi'$ such that $\pi' \rangeRes
post.\fixed = \pi \rangeRes post.\fixed$, and no unifications.  Then, by
Lemma~\ref{lem:changing-servers-singleRef}, the
primary induced transition (if the side conditions are satisfied) would
produce the same view as the previous case. 
%
We identify this redundancy early and so avoid much of the construction (in
particular, we avoid evaluating the side conditions).
\end{opt}

%%%%%%%%%%

The next optimisation is an adaptation of the previous, to deal with the case
of two primary induced transitions that are blocked because condition~(b) is
not satisfied, but where one subsumes the other. 
%% %
%% Let $c \domRes \pi$ represent $\pi$ domain-restricted to the parameters
%% of~$c$:
%% %
%% \begin{eqnarray*}
%% c \domRes \pi & = & \set{  x \mapsto y \| (x \mapsto y) \in \pi 
%%   \land \mbox{$x$ is a parameter of $c$} }.
%% \end{eqnarray*}

%%%%%

\begin{definition}
Consider a transition $pre \trans{} post$, a view~$v$, and a unification
function~$\pi$ that unifies no component.  Let $crossRefs(pre, v, \pi)$ denote
the set of normal forms of all views $(pre.\fixed, \set{c, c'})$ such that
there is a cross reference from component~$c$ of~$pre$ to component $c'$ of
$\pi(v.\cpts)$, or vice versa.  Note that for condition~(b) we require the set
of views to contain all elements of $crossRefs(pre, v, \pi)$.
\end{definition}

%%%%%

\begin{lemma}
\label{lem:singleRef-condition-b-opt}
Consider two transitions $pre \trans{} post$ and $pre' \trans{} post'$ such
that $pre.\fixed = pre'.\fixed$.  
%% Consider a view~$v$ such that $v.\fixed = pre.\fixed$.
Consider the effects of the two transitions on a view~$v$ via unification
functions~$\pi$ and~$\pi'$, respectively.
% , such that in each case no component is unified.  
Suppose:
%
\begin{itemize}
\item $crossRefs(pre, v, \pi) \subseteq crossRefs(pre', v, \pi')$;

\item condition~(b) is satisfied for the transition induced by $pre' \trans{}
  post'$.
\end{itemize}
%
Then condition~(b) is also satisfied by the transition induced by $pre
\trans{} post$.
\end{lemma}
%
\begin{proof}
Consider the transition induced by $pre \trans{} post$.  Condition~(b)
requires each view $v_\cross \in crossRefs(pre, v, \pi)$.
%% For each $(c, c') \in crossRefs(pre, v, \pi)$,\, condition~(b) requires the
%% view $v_\cross = (pre.\fixed,\linebreak[1] \set{c,c'})$.
By the first assumption, $crossRefs(pre', v, \pi')$
contains each such~$v_\cross$;
so by the second assumption, the current set of views each such~$v_\cross$.
%% contains the corresponding view $v_\cross' = (pre'.\fixed,\linebreak[1]
%% \set{c, c'})$.  But $v_\cross = v_\cross'$, so the requirement is
%% satisfied.
\end{proof}

%%%%%


\begin{opt}
Consider a primary induced transition corresponding to a transition $pre
\trans{} post$ acting on~$v$ using unification function~$\pi$, such that
$pre.\fixed \ne post.\fixed$, and $v$ unifies with no component.  For each
such case, if condition~(b) is not satisfied (but the other conditions are
satisfied), we store a tuple $(post.\fixed, v, \pi \rangeRes post.\fixed,
crossRefs(pre, v, \pi))$.

Suppose subsequently we consider a similar case, that is a primary induced
transition corresponding to a transition $pre' \trans{} post'$ acting on the
same view~$v$, such that $post'.\fixed = post.\fixed$, and using unification
function~$\pi'$ such that $\pi' \rangeRes post.\fixed = \pi \rangeRes
post.\fixed$, and no unifications.  Suppose further that $crossRefs(pre', v,
\pi') \supseteq crossRefs(pre, v, \pi)$.  Then by
Lemma~\ref{lem:singleRef-condition-b-opt}, if condition~(b) is subsequently
satisfied for this induced transition, then the same will be true of the
former induced transition.  Further, by
Lemma~\ref{lem:changing-servers-singleRef}, the two induced transitions
produce equivalent new views.

In this case, we identify that the latter transition will not induce any
additional views, and so avoid much of the calculation. 

Note that Optimisation~\ref{opt:changing-servers-singleRef} can be seen as a
special case of this where there are no cross references.
\end{opt}

Implementation note: for $crossRefs$, it's enough to store the normalised
components. 

%% Suppose we are considering a primary induced transition based on a transition
%% $pre \trans{} post$ acting on~$v$ using unification function~$\pi$, and such
%% that $pre.\fixed \ne post.\fixed$, and $v$ unifies with no component; but
%% suppose we are unable to create the induced transition because condition~(b)
%% is not satisfied (but the other conditions are satisfied).  Then we store a
%% tuple $(post.\fixed, v, \pi \rangeRes post.\fixed, crossRefs)$ where
%% $crossRefs$ represents the required cross references, as follows.  Each
%% element of $crossRefs$ is a tuple $(c, i, v.\cpts(i) \domRes \pi)$,
%% representing a cross reference from component~$c$ of~$pre$ to
%% component~$v.\cpts(i)$, or vice versa, so that we require the view $v_\cross =
%% (pre.\fixed, \set{c, \pi(v.\cpts(i))})$.  If all such $v_\cross$ are
%% subsequently added, we can fire the induced transition; as in the previous
%% optimisation, this produces a view equivalent to $(post.\fixed, \pi(v.\cpts))$

%% Suppose subsequently we consider a similar case ---that is, the same
%% $post.\fixed$ and~$v$, and a renaming~$\pi'$ such that $\pi' \rangeRes
%% post.\fixed = \pi \rangeRes post.\fixed$, and no unifications, and with cross
%% references represented by $crossRefs'$ where $crossRefs' \supseteq
%% crossRefs$--- then we proceed no further.  If subsequently all the cross
%% references represented by $crossRefs'$ are satisfied, then the same would be
%% true of $crossRefs$: in particular, if $(c, i, v.\cpts(i) \domRes \pi) \in
%% crossRefs'$ and $(c, i, v.\cpts(i) \domRes \pi') \in crossRefs$ with
%% $v.\cpts(i) \domRes \pi = v.\cpts(i) \domRes \pi'$, then these represent the
%% same requirement for the view $v_\cross = (pre.\fixed, \set{c,
%%   \pi(v.\cpts(i))}) = (pre.\fixed, \set{c, \pi'(v.\cpts(i))})$.  And as in the
%% previous item, the two induced transitions would produce the same resulting
%% view.


 %% For case~\ref{case:opt-induced-trans-singleRef-4}, we store a list of
 %%  pairs $(post.\fixed, v)$ for which we have produced a primary induced
 %%  transition based on a transition $pre \trans{} post$ acting on~$v$ such that
 %%  $pre.\fixed \ne post.\fixed$ and $v$ unifies with no component.  If
 %%  subsequently we consider a similar case ---that is, the same $post.\fixed$
 %%  and~$v$, and no unifications--- we identify that it will not produce any new
 %%  views and so avoid the construction.  \framebox{DO THIS}

 %%  \framebox{***} This doesn't seem to work.  With the first transition, one
 %%  renaming of $v$ might succeed, and a second fail clause~(b) or~(c).  But
 %%  with the second transition, the second renaming might succeed, because
 %%  $pre.\cpts$ is different.  I think it's sound when clauses~(b) and~(c) are
 %%  vacuously satisfied, i.e.~there are no cross references or common
 %%  references.  This could be considered in EffectOn.apply, although that's
 %%  rather late.  Alternatively, it could be considered by first performing all
 %%  renamings corresponding to clauses 2, 3, 5, then checking the condition,
 %%  and then (if appropriate) performing renamings corresponding to clauses 1
 %%  and~4. 

 %%  Or to only consider it when no unifications and no cross references.  Then
 %%  clauses 2, 3, 5 give no values. 

 %%  How about if we've previously done such a transition that is equivalent when
 %%  range-restricted to the parameters of $post.\fixed$?  \framebox{New version}


Note: can bail out early in case 1, or case 2 if there are no possible
secondary transitions.  In case 4, need to store the relevant information only
when we produce the transition (when the side conditions are satisfied); and
can  bail out slightly early.

%%%%%%%%%%%%%%%%%%%%%%%%%%%%%%%%%%%%%%%%%%%%%%%%%%%%%%%
