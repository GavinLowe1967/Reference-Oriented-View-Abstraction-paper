\subsection{Restricted views}
\label{sec:effectOn-restricted}

We now consider how to adapt the approach to deal with restricted views.  For
primary induced transitions, the approach will be as in the previous
subsection.  For secondary induced transitions, the approach will be slightly
different. 

\framebox{Note, 2021/08/10.}  I think what follows isn't what we want: it is
proving the wrong result.  It considers a representative remapping to make $pre$
and $v$ accordant.  If two remappings would  produce the same
induced transition, only one is considered.  However, the side conditions of
Definition~\ref{def:induced-transition-singleRef} might be satisfied by one of
those remappings and not the other.  More concretely, consider the transition
\begin{eqnarray*}
(fixed(N_1); Th(T_1,N_1), Nd(N_2,N1)) & \trans{} &
 (fixed'(N_1); Th'(T_1), Nd(N_2,N_1))
\end{eqnarray*}
on
\begin{eqnarray*}
v & = & (fixed(N_1); Th''(T_1,N_2,N_1), Nd'(N_1,N_2)).
\end{eqnarray*}
%
At present, this will consider only the remapping $\set{T_1 \mapsto T_2, N_1
  \mapsto N_1,\linebreak[1] {N_2 \mapsto N_3}}$, which would require (for
condition~(b)) the view
\[
(fixed(N_1); Nd(N_2,N_1), Nd'(N_1,N_3))
\]
to induce a transition to
\[
\begin{align}
(fixed'(N_1); Th''(T_2,N_3,N_1), Nd'(N_1,N_3)) \equiv \\
\qquad (fixed'(N_1); Th''(T_1,N_2,N_1), Nd'(N_1,N_2)).
\end{align}
\]
We should also consider the remapping $\set{T_1 \mapsto T_2, N_1 \mapsto N_1,
  N_2 \mapsto N_2}$, which would require the view
\[
(fixed(N_1); Nd(N_2,N1), Nd'(N_1,N_2))
\]
to induce a transition to
\[
\begin{align}
(fixed'(N_1); Th''(T_2,N_2,N_1), Nd'(N_1,N_2)) \equiv \\
\qquad (fixed'(N_1); Th''(T_1,N_2,N_1), Nd'(N_1,N_2)).
\end{align}
\]
I think it would be sound to allow a parameter to be remapped to an arbitrary
parameter in~$pre$ (subject to the normal conditions).  It might be possible
to do with fewer cases. 


\begin{definition} 
\label{def:representative-consistent-extension-singleRef}
Suppose we are performing an analysis using restricted views.  Consider an
extended transition~$pre \trans{e} post$ and a view~$v$.  Consider a renaming
function~$\pi$ that unifies some (maybe none) of the components of~$v$ with
components of~$pre$.  

We define a consistent extension~$\pi'$ of~$\pi$ to be a
\emph{representative primary consistent extension} if  $\pi'$ is as in
  Definition~\ref{def:representative-consistent-extension}.

We define a consistent extension~$\pi'$ of~$\pi$ to be a \emph{representative
  secondary consistent extension} if a secondary component~$cId$ in the
transition acquires a parameter~$p$ of the same type as the identity $pId$ of
$v.\princ$, $\pi'(pId) = p$, and every other parameter of $v.\princ$ is
remapped either to (1)~a parameter of $post.\fixed$, (2)~a parameter of the
state of $cId$ in $post$, or (3)~the minimal fresh parameter.
\end{definition}

%%%%%

\begin{example}
Consider the effect of the transition
\[
\begin{align}
(fixed; Th(T_0,N_0,N_1), Nd_x(N_0,N_2,null))   \trans{setNext.T_0.N_0.N_1} \\
\qquad (fixed'(N_3); Th'(T_0,N_0,N_1), Nd_x(N_0,N_2,N_1)) 
\end{align}
\]
on
\begin{eqnarray*}
v & = &  (fixed; Nd_y(N_0,N_1), Nd_z(N_1,null).
\end{eqnarray*}
%
This has representative secondary consistent extensions
\begin{eqnarray*}
\pi_X & = & \set{N_0 \mapsto N_1, N_1 \mapsto X},
  \quad \mbox{for $X = N_2, N_3, N_4$}.
\end{eqnarray*}
%
The secondary component of the transition acquires a reference to~$N_1$, so
the identity~$N_0$ of~$v.\princ$ is mapped to to~$N_1$.  The parameter~$N_1$
can be mapped to match the parameter~$N_3$ of the fixed processes, the
parameter~$N_2$ of the secondary component of the transition, or the minimal
fresh parameter~$N_4$.
%
This gives secondary transitions producing
\[{\it
(fixed'(N_3); Nd_x(N_0,N_2,N_1), Nd_y(N_1,X)),}
\] 
which are then reduced to distinct normal forms.
\end{example}

%% \framebox{The current implementation is incorrect.}  
%% I think we need all parameters of that secondary component in the
%% post-state.  Consider \[ \begin{align} (fixed; Th(t, n_1, n_2), Nd_A(n_1,
%% n_3, n_4)) \trans{setNext.t.n_1.n_2} \\ \qquad (fixed'; Th'(t), Nd_A(n_1,
%% n_2, n_4)) \end{align} \] on $(fixed; Nd_B(n_2, n_4))$.  This induces a
%% transition producing $(fixed'; Nd_A(n_1, n_2, n_4), Nd_B(n_2, n_4))$.  So
%% when producing a secondary induced transition, we need to include all
%% parameters of that secondary component in the post-state.  At present, the
%% implementation includes all parameters of all components in the \emph{pre}-
%% state (which is also inefficient).

%%%%%

\begin{prop}
Consider an extended transition $pre \trans{e} post$ and a view~$v$, both in
normal form, and a renaming function~$\pi$ unifying some components (maybe
none).  Let $\pi'$ be a consistent extension of~$\pi$, and let $v'$ be the
view produced by Definition~\ref{def:induced-transition-singleRef} considering
the effect of $pre \trans{e} post$ on~$\pi'(v)$.
%
Then there is an extension~$\pi''$ of~$\pi$ such that
%
\begin{itemize}
\item if the induced transition is a primary induced transition, then $\pi''$
  is a representative primary consistent extension; and

\item If the induced transition is a secondary induced transition, then $\pi''$
  is a representative secondary consistent extension; 
\end{itemize}
and in each case, letting $v''$ be the view produced by
Definition~\ref{def:induced-transition-singleRef} considering the effect of
$pre \trans{e} post$ on~$\pi''(v)$, we have $v'' \equiv v'$.
\end{prop}

%%%%%

\begin{proof}
For primary induced transitions, the proof is as for
Proposition~\ref{prop:unfiying-renaming}.

Consider a secondary induced transition, producing~$v'$ as in the statement of
the proposition.  Then
%
\begin{eqnarray*}
v' & = & (post.\fixed, \set{sc, \pi'(v.\princ)}),
\end{eqnarray*}
%
where $sc$ is the secondary component that gains a reference to
$\pi'(v.\princ)$.  We describe how to construct~$\pi''$ and a
renaming~$\hat\pi$ such that $\pi' = \hat\pi \circ \pi''$.  This will imply
that $\hat\pi(v'') = v'$.

Let $pId = v.\princ.\id$ and $p = \pi'(pId)$; this equals the parameter
of~$sc$ that is a reference to $\pi'(v.\princ)$.  We define $\pi''(pId) = p$,
and $\hat\pi(p) = p$.  This is a consistent extension of~$\pi$, because~$\pi'$
is a consistent extension of~$\pi$.

For every other parameter~$x$ of~$v.\princ$ such that $y = \pi'(x)$ appears
in~$post.\fixed$ or~$sc$,  we define $\pi''(x) = y$, and $\hat\pi(y) = y$.
Note that each such value~$y$ is included under case~(1) or~(2) of
Definition~\ref{def:representative-consistent-extension-singleRef}.  Again,
this is a consistent extension of~$\pi$, because~$\pi'$ is a consistent
extension of~$\pi$.

For every other parameter~$x$ of $v.\princ$ not in the domain of~$\pi$, we
define $\pi''(x)$ to be the minimal fresh parameter~$z$, and define
$\hat\pi(z) = \pi'(x)$.  This is a consistent extension of~$\pi$, by
construction.
\end{proof}


% LocalWords:  unifications
