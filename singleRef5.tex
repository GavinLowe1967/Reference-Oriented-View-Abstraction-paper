\subsubsection{Sufficient remappings for condition~(b)}
\label{sec:necessary-b}

Out approach will be as follows: (1) consider all
unification functions~$\pi_u$; (2) for each such unification function, produce
all minimal consistent (primary- or secondary-)result-defining
extensions~$\pi_{rd}$; (3) extend each such result-defining extension (in a
result-consistent way, so as to maintain the property of being
result-defining).  However, it turns out that we might have to consider more
than one extension of each~$\pi_{rd}$: it might be that only certain
extensions satisfy conditions~(b) and~(c) of
Definition~\ref{def:induced-transition-singleRef}.  We consider a set $\Pi$ of
extensions, such that if any extension satisfies the conditions then a member
of~$\Pi$ does.

In this section we consider specifically condition~(b), and prove results
concerning sets of extensions that are sufficient to consider.  In particular,
if~$\pi_{rd}$ does not itself create any linkage for condition~(b), it is not
necessary to consider extensions that create linkages.  In
Section~\ref{sec:necessary-c}, we do similarly for condition~(c).  In
Section~\ref{sec:necessary-algorithm} we present our algorithm for producing
these extensions.


\framebox{Note} if linkage involves unified component, then this is implied by
the views.

%%%%%%%%%%%%%%%%%%%%%%%%%%%%%%%%%%%%%%%%%%%%%%%%%%%%%%%%%%%%


Given a result-defining remapping~$\pi_{rd}$, we consider which extensions it
is sufficient to consider in order to test whether one of them satisfies
condition~(b).  We show that if $\pi_{rd}$ does not itself create any linkages
for condition~(b), there is no need to consider extensions that create such
linkages \framebox{ref}.  However, if a linkage exists in $\pi_{rd}$, then it might be
necessary to consider extensions that produce additional linkages.
%
The following example illustrates the former of these points (the idea is
formalised in Lemma~\ref{lem:necessary-renamings}).
%% Note that the above process creates linkages only where they are, in a sense,
%% required.  We show below (Lemma~\ref{lem:necessary-renamings}) that any other
%% remapping is subsumed by one of the remappings we do consider.  The following
%% example illustrates the idea.
%%%%%%
\begin{example}
\label{example:singleRef:unnecessary-linkage}
Consider again (from Example~\ref{example:29}) the normal-form transition
\begin{eqnarray*}
(fixed(N_0); Th(T_0,N_1), Nd_x(N_1,N_2)) & \trans{} &
  (fixed'(N_0); Th'(T_0,N_1), Nd_x(N_1,N_2))
\end{eqnarray*}
acting on 
\[
(fixed(N_0); Nd_y(N_1,N_2), Nd_z(N_2,N_3)).
\]
This has result-defining mapping~$\pi_{rd} = \set{N_0 \mapsto N_0}$.  Consider
the extension $\pi = \set{{N_0 \mapsto N_0},\linebreak[1] N_1 \mapsto N_3,
N_2 \mapsto N_4,\linebreak[1] {N_3 \mapsto N_1}}$.  This creates a linkage
on~$N_1$.  However, this linkage is unnecessary: replacing $N_1$ in the range
of~$\pi$ with a fresh parameter $N_5$ creates the remapping $\pi'
= \set{N_0 \mapsto N_0,\linebreak[1] {N_1 \mapsto N_3,} N_2 \mapsto
N_4,\linebreak[1] {N_3 \mapsto N_5}}$, which would lead to the same final
induced view (up to equivalence), but requires a subset of the linking views.
\end{example}

In cases such as in the above example, we consider only the extension using
fresh parameters, since it subsumes the other (and is simpler to handle).  We
seek sufficient conditions for one extension to subsume another. 

%%%%%

By contrast, the following example illustrates that if the result-defining
mapping creates a linkage, it can be necessary to consider more than one
extension: it might be the case that one such extension satisfies
condition~(b) but the other doesn't.
%
\begin{example}
Consider the effect of the transition
\begin{eqnarray*}
(fixed(N_0); Th(T_0,N_1), Nd(N_1,N_0)) & \trans{} &
 (fixed'(N_0); Th'(T_0), Nd(N_1,N_0))
\end{eqnarray*}
on
\begin{eqnarray*}
v & = & (fixed(N_0); Th''(T_0,N_0,N_1), Nd'(N_0,N_1)).
\end{eqnarray*}
%
No unification of components is possible.  The only consistent result-defining
extension is $\pi_{rd} = \set{{N_0 \mapsto N_0}}$.  This necessarily creates a
linkage via~$N_0$.

For an extension~$\pi$ of~$\pi_{rd}$, condition~(b) of
Definition~\ref{def:induced-transition-singleRef}, would require the current
set of views to contain a view equivalent to
%
\begin{eqnarray*}
v_\cross & = & (fixed(N_0); Nd(N_1,N_0), Nd'(N_0,\pi(N_1)))
\end{eqnarray*}
in order to induce a transition to
\[
\begin{align}
(fixed'(N_0); Th''(T_1,N_0,N_1),  Nd'(N_0,N_2)) \equiv 
 (fixed'(N_0); Th''(T_0,N_0,N_1), Nd'(N_0,N_1)).
\end{align}
\]
Depending upon the value of $\pi(N_1)$, values for~$v_\cross$ fall into two
equivalence classes, with representative members
\[
\begin{align}
(fixed(N_0); Nd(N_1,N_0), Nd'(N_0,N_1)) \quad\mbox{and}\quad
(fixed(N_0); Nd(N_1,N_0), Nd'(N_0,N_2)).
\end{align}
\]
It is necessary to consider both these possibilities: it is possible that the
current set of representative views contains only one of them.

Note that the former case (with $\pi(N_1) = N_1$) creates two additional
linkages via~$N_1$, which for condition~(b) would require views equivalent to
\[
\begin{array}{c}
(fixed(N_0); Nd'(N_0,N_1), Nd(N_1,N_0)) \quad\mbox{and}\quad
(fixed(N_0);Th''(T_0,N_0,N_1),Nd(N_1,N_0)).
\end{array}
\]
\end{example}

%%%%%

Given a result-defining remapping~$\pi_{rd}$, we identify the linkages
created by~$\pi_{rd}$.  
%
\begin{definition}
We say that $\pi_{rd}$ \emph{creates a cross reference} between component~$c$
of~$pre$ and component~$c'$ of~$v$ if:
%
\begin{itemize}
\item For some parameter $x$ of~$c$ (other than its identity), $(c'.id \mapsto
x) \in \pi_{rd}$; or

%% \item $c'.\id \in \dom \pi_{rd}$ and $\pi_{rd}(c'.id)$ is a parameter of~$c$
%% (other than its identity); or

\item For some parameter $x$ of~$c'$ (other than its identity), $(x \mapsto
c.\id) \in \pi_{rd}$.
\end{itemize}
\end{definition}
%
The two clauses correspond to a cross reference from $c$ to~$c'$ (remapped by
each extension of~$\pi_{rd}$), and vice versa, respectively.

%%%%%

%% Consider a cross reference created by~$\pi_{rd}$ involving component~$c$
%% of~$pre$ and component~$c'$ of~$v$.  We extend~$\pi_{rd}$ to the remaining
%% parameters of~$c'$, including the possibility of mapping each either to a
%% parameter of~$c$, or leaving is unmapped (for the moment).

The following lemma gives conditions for two different extensions to require
the same cross-reference view under condition~(b).
%
\begin{lemma}
\label{lem:singleRef:cond-b-extensions}
Consider a (primary- or secondary-)result-defining mapping~$\pi_{rd}$ (not
necessarily minimal).  Suppose $\pi_{rd}$ creates a cross reference involving
component~$c$ of~$pre$ and component~$c'$ of~$v$.  Let $\pi$ and~$\pi'$ be two
result-consistent extensions, so condition~(b) requires views equivalent to
\[
v_\cross  =  (pre.\fixed, \set{c, \pi(c')}) \qquad\mbox{and}\qquad
v_\cross'  =  (pre.\fixed, \set{c, \pi'(c')}),
\]
respectively.  Suppose $\pi$ and~$\pi'$ restricted to the parameters of~$c'$
range-agree on all parameters of~$c$.  Then $v_\cross \equiv v_\cross'$.
\end{lemma}
%
\begin{proof}
$v_\cross$ and $v_\cross'$ are related by the mapping that is the identity on
  all parameters of $pre.\fixed$ and~$c$, and otherwise relates $\pi(x)$
  to~$\pi'(x)$ for parameters~$x$ of~$c'$.
\end{proof}

The above lemma shows that, as far as this linkage is concerned, it is enough
to consider just one of~$\pi$ and~$\pi'$: they require equivalent
cross-reference views.  This suggests the following definition.

%%%%%

\begin{definition}
\label{def:cross-reference-view-defining}
Let $\pi_{rd}$ be a result-defining mapping that creates a cross reference
between component~$c$ of~$pre$ and component~$c'$ of~$v$.
%
Let $\pi$ be a result-consistent extension of~$\pi_{rd}$.  We say that $\pi$
is a \emph{cross-reference-view-defining mapping} for~$c$ and~$c'$ if for each
parameter~$x$ of~$c'$ with $x \nin \dom \pi_{rd}$,\, $\pi$ either maps $x$ to
a parameter of $c$ or leaving $x$ unmapped (and adds no mappings of other
parameters).
%
We define the \emph{completing extensions} of~$\pi$ to be the total (over the
parameters of~$v$) result-consistent extensions of~$\pi$ that do not map any
additional parameter of~$c'$ to a parameter of~$c$.
\end{definition}

%%%%%

The following lemma follows directly from the definition: the
cross-reference-view-defining extensions partition the result-consistent
extensions. 
%
\begin{lemma}
\label{lem:cross-reference-view-defining-partition}
Consider a result-defining mapping~$\pi_{rd}$ that creates a cross reference
between components $c$ of $pre$ and $c'$ of~$v$.  Then every (complete over
the parameters of~$v$) result-consistent extension of $\pi_{rd}$ is a
completing extension of some cross-reference-view-defining extension for~$c$
and~$c'$.
\end{lemma}

%%%%%

The following lemma shows that each cross-reference-view-defining remapping
defines an equivalence class for condition~(b): either all completing
extensions satisfy the condition, or none does.
%
\begin{lemma}
\label{lem:cross-reference-view-defining}
Consider a result-defining mapping~$\pi_{rd}$ that creates a cross reference
between components $c$ of $pre$ and $c'$ of~$v$.  Let $\pi$ be a
cross-reference-view-defining extension for~$c$ and~$c'$.  Suppose that some
completing extension of~$\pi$ satisfies this instance of condition~(b).
Then \emph{every} completing extension of~$\pi$ satisfies this instance of
condition~(b).
\end{lemma}

%%%%%%

\begin{proof}
%
%Suppose $\pi$ is a cross-reference-view-defining extension for~$c$ and~$c'$.
Let $\pi_1$ and~$\pi_2$ be two completing extensions of~$\pi$.  From
Lemma~\ref{lem:singleRef:cond-b-extensions}, the required views for this
instance of condition~(b) for~$\pi_1$ and~$\pi_2$ are equivalent; hence either
both or neither is satisfied.
\end{proof}

%%%%%

Collectively the above two lemmas show that, for each cross reference created
by~$\pi_{rd}$, it is sufficient to consider the cross-reference-view-defining
extension , and to consider one completing extension of each.  However, we
need to consider all cross references.  We sketch our treatment; we make it
formal in Section~\ref{sec:necessary-algorithm}.  We start with a minimal
result-defining mapping (either primary or secondary), and consider each cross
reference in turn.  We calculate the cross-reference-view-defining extensions.
We then iteratively extend those remappings corresponding to other cross
references (but consistently with being a completing extension of those cross
references already considered).  An extension that does map a particular
parameter has the possibility of creating another cross reference involving
different components; we repeat the process with those.  If a parameter is
left unmapped, consideration of another cross reference might subsequently
cause it to become mapped (again, consistently with cross references already
considered).  Once all such cross references have been considered, we map the
remaining parameters to distinct fresh parameters.  In particular, we consider
only cross references created by the original result-defining mapping, or that
are created by our treatment of other cross references; we do not create new
cross references unless necessary.

