
\subsubsection{Optimisations}

Recall that, at least in principle, we need to consider transitions induced by
a transition on a view: (1) for each view~$v$ found on the previous ply, and
\emph{all} transitions found on earlier plies; and (2) for each transition
found on the current ply and \emph{all} views found on earlier plies.  This
constitutes a lot of cases.  We explain how we avoid considering most cases,
restricting only to those that might actually produce a new view.  We make use
of the following lemma.

%%%%%

\begin{lemma}\label{lem:mightGiveSufficientUnifs}
Consider a transition $pre \trans[e] post$ and a view~$v$ such that
$pre.\fixed = v.\fixed$.   Suppose:
%
\begin{enumerate}
\item\label{cond:mightGiveSufficientUnifs-1}
  Every parameter of $post.\fixed$ is also a parameter of $pre.\fixed$;

\item\label{cond:mightGiveSufficientUnifs-2}
  No component of $v$ is unified with 
%% % is in the same control state as
  a component of~$pre$ that changes state in the transition; and

\item\label{cond:mightGiveSufficientUnifs-3}
  Either $post.\fixed = pre.\fixed$, or we previously used a transition
  $pre' \trans[e'] post'$ with $post'.\fixed = post.\fixed$, acting on
  this~$v$, to induce a transition that created a new view, and such that no
  component of~$v$ was unified with a component of~$pre'$ that changed state.
\end{enumerate}
%
Then any view induced by the transition on~$v$ is a view that had already been
found. 
\end{lemma}

%%%%%

\begin{proof}
Condition~\ref{cond:mightGiveSufficientUnifs-2}  means that
%%  no component of~$v$ changes state in the induced transition.  This means
%% that
if we use unification function~$\pi'$, then the resulting view is $v' =
(post.\fixed, \pi'(v.\cpts))$.
%
However, $\pi'$ is the identity over parameters of $pre.\fixed$; so by
condition~\ref{cond:mightGiveSufficientUnifs-1}, $\pi'$ is also the identity
over parameters of $post.\fixed$.  Hence $v' = \pi'(post.\fixed, v.\cpts)$.
%
\begin{itemize}
\item If $post.\fixed = pre.\fixed$, then $v'$ is equivalent to~$v$, so no new
  view is produced.

\item Otherwise, by condition~\ref{cond:mightGiveSufficientUnifs-3}, we
  previously used a transition $pre' \trans[e'] post'$ with $post'.\fixed =
  post.\fixed$, acting on this~$v$, to induce a transition that created a new
  view, and such that no component of~$v$ was unified with a component
  of~$pre'$ that changed state.  But this earlier case would have produced a
  view equivalent to~$v'$, by the same argument as above.  Hence no new view
  is created in this case.
\end{itemize}
\end{proof}

In order to use this result, we record information so that
condition~\ref{cond:mightGiveSufficientUnifs-3} can be decided: whenever we
induce a transition where the fixed processes change state and no unified
component changes state, we record this fact (specifically, the relevant~$v$
and $post.\fixed$).
%%% IMPLEMENTATION: ComponentView0.{addDoneInduced,containsDoneInduced}

For a given view~$v$ being processed on the current ply, we want to iterate
over all transitions $pre \trans[e] post$ from previous plies that might
induce a transition to a new view~$v'$.  We store the previous transitions in
a way that makes this iteration efficient.  For
condition~\ref{cond:mightGiveSufficientUnifs-2}, we use the fact that if two
states are in different control states, they cannot be unified: this allows us
to avoid building the unification function in most cases.  The transitions are
partitioned according to the states of the fixed processes of the pre-state;
this allows just those transitions with $pre.\fixed = v.\fixed$ to be
selected.
%%% IMPLEMENTATION: ServersTransitionSet
Further, each such subset of the transitions is indexed in various ways:
%
\begin{enumerate}
\item\label{item:mightGiveSufficientUnifs-1} Those transitions such that
  $post.\fixed$ contains a parameter not in $pre.\fixed$ are stored
  separately;
%%% IMPLEMENTATION: ServersTransitionSet.acquiringTrans

\item\label{item:mightGiveSufficientUnifs-2} Those transitions not included
  under the previous item are stored indexed against the states of the
  fixed processes of the post-state;
%%% IMPLEMENTATION: ServersServersTransitionSet

\item\label{item:mightGiveSufficientUnifs-3} Those transitions not included
  under the first item are also stored indexed against each control
  states~$cs$ such that the transition has a component that changes state from
  control state~$cs$.
\end{enumerate}

For a given view~$v$, we consider induced transitions caused by previous
transitions (with matching initial states for the fixed processes) as follows:
%
\begin{itemize}
\item \emph{all} transitions under item~\ref{item:mightGiveSufficientUnifs-1}
  (cf.~condition \ref{cond:mightGiveSufficientUnifs-1} of
  Lemma~\ref{lem:mightGiveSufficientUnifs}); 

\item for each possible $post.\fixed$ not under
  item~\ref{item:mightGiveSufficientUnifs-1}: 
  %
  \begin{itemize}
  \item if we have not previously induced a new transition from such a
    combination, all the corresponding transitions under
    item~\ref{item:mightGiveSufficientUnifs-2}
    (cf.~condition~\ref{cond:mightGiveSufficientUnifs-3} of
    Lemma~\ref{lem:mightGiveSufficientUnifs}); and

  \item otherwise, for each component control state~$cs$ in~$v$, the
    transitions stored against~$cs$ under
    item~\ref{item:mightGiveSufficientUnifs-3}
    (cf.~condition~\ref{cond:mightGiveSufficientUnifs-2} of
    Lemma~\ref{lem:mightGiveSufficientUnifs}).
  \end{itemize}
\end{itemize}
%
Lemma~\ref{lem:mightGiveSufficientUnifs} then shows that these transitions are
sufficient. 

Similarly, for a given transition of the current ply, we want to iterate over
all views~$v$ from previous plies that might induce a transition to a new
view~$v'$.  We likewise store the previous views in a way that makes this
iteration efficient.  We partition the views based on the states of the fixed
processes, to allow the selection of just those that match a particular
transition.
%%% IMPLEMENTATION: ServerBasedViewSet
Each such subset is indexed in various ways.
%
\begin{enumerate}
\item\label{item:mightGiveSufficientUnifs-views1} For each possible state
  $fixed$ of the fixed processes, an over-approximation of the views~$v$ such
  that we have not previously induced a transition using~$v$, where the fixed
  processes change state to~$fixed$, and no unified component changes state.
  (This over-approximation is updated during the iteration below).
%%% IMPLEMENTATION: IndexSet within PrincTypesViewSet 

\item\label{item:mightGiveSufficientUnifs-views2} For each control state~$cs$,
  those views with a component in control state~$cs$.
\end{enumerate}

For a given transition $pre \trans[a] post$, we consider induced transitions
on previous views (with fixed processes matching $pre.\fixed$) as follows:
%
\begin{itemize}
\item If $post.\fixed$ contains a parameter not in $pre.\fixed$, then all such
  views (cf.~condition~\ref{cond:mightGiveSufficientUnifs-1} of
  Lemma~\ref{lem:mightGiveSufficientUnifs});

\item Otherwise:
  %
  \begin{itemize}
  \item If $post.\fixed \ne pre.\fixed$, then those views~$v$ such that we
    have not previously induced a transition using~$v$ where the fixed
    processes change state to~$post.\fixed$ and no unified component changes
    state (cf.~condition~\ref{cond:mightGiveSufficientUnifs-3} of
    Lemma~\ref{lem:mightGiveSufficientUnifs}).  These views are obtained from
    the over-approximation from
    item~\ref{item:mightGiveSufficientUnifs-views1}, and the
    over-approximation is updated at this point.

  \item For each control state~$cs$ of a component of~$pre$, those views
    stored against~$cs$ under item~\ref{item:mightGiveSufficientUnifs-views2}
    (cf.~condition~\ref{cond:mightGiveSufficientUnifs-2} of
    Lemma~\ref{lem:mightGiveSufficientUnifs}).
  \end{itemize}
\end{itemize}
%
Lemma~\ref{lem:mightGiveSufficientUnifs} then shows that these views are
sufficient.

In addition, when either of the above iterations produces a view and
transition that might be unified, we form the unification(s), and subsequently
check whether the conditions of Lemma~\ref{lem:mightGiveSufficientUnifs} are
satisfied for that unification function.  This further reduces the number of
cases that needs to be reduced; however, the above efficient iterations are
the more effective optimisation. 



%%%%%%%%%%%%%%%%%%%%%%%%%%%%%%%%%%%%%%%%%%%%%%%%%%%%%%%%%%%%

%% \begin{opt}
%% Recall Optimisation~\ref{opt:avoid-induced}.  
%% \begin{itemize}
%% \item For case~\ref{case:avoid-induced-1}, we avoid unifying the principal
%%   of~$v$ with the principal of~$pre$.

%% \item For case~\ref{case:avoid-induced-2}, if $pre.\fixed = post.\fixed$ we
%%   require that at least one component of~$v$ unifies with a component that
%%   changes state.

%% \item For case~\ref{case:avoid-induced-3}, we store a record of pairs
%%   $(v, post.\fixed)$ for which we have considered the effect of a transition
%%   $pre \trans[e] post$ on~$v$ for which $pre.\fixed \ne post.\fixed$,\,
%%   %% $unifs$ describes which components of~$v$ unify with which components
%%   %% of~$pre$ (
%%   and $v$
%%   unifies with no component.
%% %
%%   If subsequently we consider a similar case ---that is, the same
%%   $post.\fixed$ and~$v$, and no unifications--- we identify that it will not
%%   produce any new views, and so avoid the construction.
%% \end{itemize}
%% \end{opt}



%% \begin{impNote}
%% Done in isSufficientUnif. 

%% \framebox{Not any more}
%% \end{impNote}

%% \begin{improve}
%% The third case is not quite the same as in
%% Optimisation~\ref{opt:avoid-induced}.  There we included cases where $pre$
%% and~$v$ shared a component that did not change state.  An obvious attempt to
%% implement this goes wrong: if there are two cases with the same $post.\fixed$
%% and~$v$, that include unifications of \emph{different} components that do not
%% change state, they might give different new views (with the unified components
%% having different parameters in $post.\fixed$).
%% %
%% For example
%% \begin{eqnarray*}
%% pre & = & (fixed; Th_1(T_0,N_0), Nd_A(N_0,N_1)) \\
%% post & = & (fixed'(N_1); Th_1'(T_0), Nd_A(N_0,N_1)) \\
%% v & = & (fixed; Th_v(T_0,N_0), Nd_A(N_0,N_1))
%% \end{eqnarray*}
%% unifying the two $Nd_A$ processes, with remapping $\set{T_0 \mapsto T_1, N_0
%%   \mapsto N_0,\linebreak[1] {N_1 \mapsto N_1}}$, induces
%% \[
%% (fixed'(N_1); Th_v(T_1,N_0), Nd_A(N_0,N_1) \equiv
%% (fixed'(N_0); Th_v(T_0,N_1), Nd_A(N_1,N_0)).
%% \]
%% But with 
%% \begin{eqnarray*}
%% pre' & = & (fixed; Th_2(T_0,N_0,N_1), Nd_A(N_0,N_2), Nd_N(N_1,Null)) \\
%% post' & = & (fixed'(N_1); Th_2'(T_0), Nd_A(N_0,N_2), Nd_N(N_1,Null))
%% \end{eqnarray*}
%% unifying the two $Nd_A$ processes, with remapping $\set{T_0 \mapsto T_1, N_0
%%   \mapsto N_0,\linebreak[1] {N_1 \mapsto N_2}}$, induces a transition to
%% \[\mit
%% (fixed'(N_1); Th_v(T_1,N_0), Nd_A(N_0,N_2)) \equiv
%%   (fixed'(N_0); Th_v(T_0,N_1), Nd_A(N_1, N_2)).
%% \]


%% If we have two cases concerning the same $post.\fixed$, $v$, and component~$c$
%% of~$v$ that unifies and doesn't change state, and $c$ unifies respectively
%% with~$c_1$ and~$c_2$ which differ only on parameters not in $post.\fixed$,
%% then it is enough to consider just one such case.  In the above example, we
%% had $c_1 = Nd_A(N_0,N_1)$,\, $c_2 = Nd_A(N_0,N_2)$ which differ on~$N_1$. 

%% \framebox{Do this.}  For the components, it's enough to store the list of
%% parameters shared with $post.\fixed$ and their indices. 
%% \end{improve}

 
%%%%%%%%%%%%%%%%%%%%

\begin{improve}
If the principal of~$v$ is unified with a component of~$pre$, and loses a
reference to a component~$c$, then all remappings of~$c$ produce the same final
view.  Thus it is enough to consider a single remapping, say the remapping that
agrees with the remapping within other components, but otherwise maps to fresh
parameters.  There is no need to explicitly construct the remapping of~$c$.  
%
\framebox{Consider this.}

This won't work with singleRef, however, because of cross references or
missing references. 
\end{improve}

%%%%%

\begin{improve} 
If the principal and another component~$c$ of~$v$ are both unified with
components of $pre$, and the principal of~$v$ loses the reference to~$c$ in
the transition, then for case~\ref{clause:remap-2} of
Definition~\ref{def:representative-consistent-extension}, we can ignore
parameters of~$c$.  This is only relevant for remapping of some third
component.
%
This might be useful when a node has a reference to the thread, but loses
that reference.
\end{improve}
