\section{Symmetry reduction}
\label{sec:symmetry}

We now present our use of symmetry reduction.  The technique of the previous
section produces a potentially infinite set of views, because the types of
component identities are potentially infinite.  However, many such views are
symmetric: one can be obtained from another by uniformly remapping the
parameters.  Our approach is to store a single view from each symmetry
equivalence class.   

Recall that we define a \emph{remapping} to be a partial injective
function~$\pi$, defined over non-distinguished identities, and that preserves
types (for example, maps each node identity to a node identity, and each
thread identity to a thread identity).
%
If $\pi$ is a remapping function defined over the parameters of a process
state~$q$, then we write $\pi(q)$ for the effect of replacing each
parameter~$x$ of~$q$ by~$\pi(x)$.  We extend this to sets of states, views,
etc., by pointwise application.  For example, given
%
\begin{eqnarray*}
v & = & 
  \begin{align}
  (\Lock'(T_1), \Head(N_0), \Last(N_1), \Con_3; 
    Enq_4(T_1, N_1, N_3), \Node_y(N_1, Null),\Node_x(N_3, N_1))
  \end{align} \\
\pi & = & 
  \set{ T_1 \mapsto T_0, N_0 \mapsto N_1, N_1 \mapsto N_2, N_3 \mapsto N_0 },
\end{eqnarray*}
%
we have 
\begin{eqnarray*}
\pi(v) & = & 
  \begin{align}
  (\Lock'(T_0), \Head(N_1), \Last(N_2), \Con_3; 
   Enq_4(T_0, N_2, N_0), \Node_y(N_2, Null),\Node_x(N_0, N_2)) .
  \end{align}
\end{eqnarray*}

%%%%%

Recall that in Definition~\ref{def:system} we assumed that the fixed process
and components are symmetric.  In~\cite{gavin:view-abs} we proved the
following result, which shows that the system as a whole is symmetric.
%
\begin{lemma}
\label{lem:system-pi-bisimilar}
The state machine defined by a system is symmetric: if $s \in \S$ and
$\pi$ is a remapping then $s \sim_\pi \pi(s)$.
\end{lemma}
%
Thus the transitions of a state can be deduced from the transitions of a
symmetric state. 

%%%%%

\begin{definition}
We say that views~$v$ and $v'$ are \emph{equivalent}, written $v \equiv v'$,
if there is a remapping~$\pi$ such that $\pi(v) = v'$. 

We say that transitions $v_1 \trans[e_1] v_1'$ and $v_2 \trans[e_2] v_2'$ are
\emph{equivalent} if there is a remapping~$\pi$ such that $\pi(v_1) = v_2$,\,
$\pi(e_1) = e_2$ and $\pi(v_1') = v_2'$.

We say that transition templates $v_1 \uplus \hole{id_1} \trans[e_1] v_1'
\uplus \hole{id_1}$ and $v_2 \uplus \hole{id_2} \trans[e_2] v_2' \uplus
\hole{id_2}$ are \emph{equivalent} if there is a remapping~$\pi$ such that
$\pi(v_1) = v_2$,\, $\pi(e_1) = e_2$,\, $\pi(v_1') = v_2'$ and $\pi(id_1) = id_2$.

Note that each of these is an equivalence relation.

Given a set~$V$ of views, we write $\overline{V}$ to represent its closure
under~$\equiv$:
%
\begin{eqnarray*}
\overline{V} & \defs & \set{v \in \V \| \exists v' \in V \spot v \equiv v'}.
\end{eqnarray*}
%
We use similar notation for sets of transitions and transition templates.
\end{definition}

%%%%%


Our approach, then, is to store a single representative of each view
encountered so far: we store a set~$V$ that represents its
closure~$\overline{V}$.  We treat transitions and transition templates
similarly, storing sets $trans$ and $transTemplates$ of representatives, which
represent $\overline{trans}$ and $\overline{transTemplates}$, respectively.
%
Below we explain how to simulate the various steps in
Figures~\ref{fig:algorithm-2} and~\ref{fig:algorithm-3} using this
representation. 
%
But first we describe more precisely how we represent the sets. 

%% \framebox{Explain} why it's enough to store views, up to equivalence.  As in
%% previous paper.  I think we want roughly Lemma 20--Lemma 23.


%%%%%%%%%%%%%%%%%%%%%%%%%%%%%%%%%%%%%%%%%%%%%%%%%%%%%%%

\subsection{Normal form for views}

We represent each equivalence class of views by a single representative.  More
precisely, we define a normal form of views, and take the representatives to
be the views in normal form: there will be one per equivalence class.  We
describe the normal form below. 

For each type of identities, we assume a linear order over the type.  In
examples below, for node identities we will assume an order $N_0 < N_1 <
N_2 < \ldots$; and similarly for thread identities.  (In the implementation,
identities are represented by non-negative integers: we use the integer
order.)

The following definition captures our normal form.
%
\begin{definition}
Within views, extended views, etc., we use a standard order for the fixed
processes (in the implementation, this order is defined in the input script).
We order components in the order we have been using in examples: principal
first, followed by secondary components in the order corresponding to the
references from the principal (without repetitions).  For extended views, any
additional component is added at the end.  

Given this ordering of processes, we say that a view or extended view is in
\emph{normal form}, if for each type, the first occurrences of each parameter,
when read from left to right, form an initial segment of that type (according
to the linear order described above).   
\end{definition}

For example, the following view is in normal form; we have underlined the
first occurrence of each parameter.
\[
\begin{align}
(\Lock'(\underline{T_0}), \Head(\underline{N_0}), 
    \Last(\underline{N_1}), \Con_3; 
  Enq_4(T_0, N_1, \underline{N_2}), \Node_y(N_1, Null),\Node_x(N_2, N_1)) .
\end{align}
\]

Note that for each equivalence class, there is a unique view in normal form.
We take this to be the representative of that equivalence class.  We write
$rep(v)$ for the representative of view~$v$. 

A view can be reduced to normal form by a simple left-to-right traversal,
keeping track of the remapping built so far, and extending it to map the first
occurrence of each parameter to the first unused parameter of that type.

%%%%%%%%%%

\begin{definition}
An active-process or extended transition $pre \trans{} post$ is in normal form
if $pre$ is in normal form, and the components of~$post$ are in the same order
as for~$pre$.  (Note that every parameter in $post$ is also in~$pre$, by
Assumption~\ref{assump:2}, so this defines the order of parameters in $post$.
Note also that $post$ itself might not be in normal form.)

 %% and the first occurrences of each parameter within the concatenation of $pre$
 %% and $post$ form an initial segment of that type.

A transition template $pre \uplus \hole{id} \trans[e] post \uplus \hole{id}$
is in normal form if $pre$ is in normal form, the components of~$post$ are in
the same order as for~$pre$, and $id$ is either a parameter of~$pre$ or is the
next parameter of its type.
\end{definition}

For example, consider transition templates of the form
\begin{equation}
\begin{alignc}
(fixed(T_0, N_0, N_1);   Enq_3^x(T_0, N_1), \Node_y(N_1, Null)) \uplus \hole{n}
    \trans(7)[initNode_x.T_0.n.Null] \\
\qquad (fixed(T_0, N_0, N_1);
   Enq_4^x(T_0, N_1, n), \Node_y(N_1, Null)) \uplus \hole{n}.
\end{alignc}
\label{trans-template:initNode}
\end{equation}
Necessarily $n \ne N_1$ (since the missing component's identity cannot match
that of a component in the view).  This would be in normal form for $n = N_0$
or~$N_2$ (it turns out that the former would have no instantiation in the
running example of a queue: the $\Head$ process can never reference a node in
the $\InitNode$ state).

%\framebox{Example} of transition in normal form. 

%% \begin{lemma}
%% \label{lem:normal-form-fixed}
%% If $s$ and~$s'$ are both in normal form, and $\pi(s)$ and~$\pi'(s')$ are
%% substates of the same system state, then $s.\fixed = s'.\fixed$.
%% \end{lemma}

%%%%%%%%%%%%%%%%%%%%%%%%%%%%%%%%%%%%%%%%%%%%%%%%%%%%%%%

\subsection{Producing active-process transitions and transition templates}

Recall that we store a set~$V$ of representative views found so far, which
represents its closure~$\overline{V}$.  Given a view~$v \in V$, we can
directly calculate its active-process transitions and transition templates, by
considering the transitions of the individual processes: if there is a missing
component that has the relevant event in its alphabet, then a transition
template is formed, otherwise, a transition.

By Lemma~\ref{lem:system-pi-bisimilar}, if $v \trans[e] v'$ then $\pi(v)
\trans[\pi(e)] \pi(v')$, and vice versa.  Adding $rep(viewOf(v'))$ to the
set~$V$ of representative views also adds $viewOf(\pi(v')) = \pi(viewOf(v'))$
to its closure~$\overline{V}$, so captures the active-process transitions
from~$\pi(v)$.  Thus it is enough to explicitly consider transitions from just
the representatives.


Note that for an active-process transition, every parameter in the post-state
is also in the pre-state.  Hence every active-process transition from a view
in normal form will itself be in normal form.  Also, two transitions $v
\trans[e] v'$ and $\pi(v) \trans[\pi(e)] \pi(v')$ have the same normal form.
This means that adding the active-process transitions from a representative
view~$v$ to the set $trans$ of representative transitions also adds the
active-process transitions from $\pi(v)$ to the closure~$\overline{trans}$.

We can treat transition templates in a similar way, calculating them from the
view transitions of a view~$v \in V$.  We calculate all view transitions,
which generates all transition templates with all possible values for the
identity of the missing component.  (In the implementation, each type of
identities is finite, but needs to be large enough to be able to produce all
representative transitions and transition templates, in a sense that we make
precise \framebox{later}).  For example, this procedure would generate
transition templates of the form of~(\ref{trans-template:initNode}) for every
$n \ne N_1$.  But for $n > N_2$, the template is equivalent to that for $n =
N_2$.  Filtering out the templates not in normal form leaves suitable
representatives.


%% Consider, first, the case where the
%% identity of the extra component is in the pre-state of the transition, as a
%% parameter of either a fixed process or a secondary component (e.g., as in
%% transition~(\ref{trans:getNext})).  Such transition templates can be
%% calculated directly, as for active-process transitions.  And, as for
%% active-process transitions, transition templates from views $v$ and~$\pi(v)$
%% are themselves equivalent, so it is enough to consider just one.

%% Now suppose the identity of the extra component is not in the pre-state of the
%% transition (as in transition~(\ref{trans:initNode})).  From $v \in V$, we
%% calculate normal-form transition templates $v \uplus \hole{id} \trans[e] v'
%% \uplus \hole{id}$.  There are two possibilities.
%% %
%% \begin{itemize}
%% \item The identity $id$ may match a parameter in~$v$, but not the identity of
%%   a component; this case is very similar to as in the previous paragraph.

%% \item The identity $id$ might be the next unused parameter of that type.
%%   Consider some other identity~$id'$ distinct from all parameters in the
%%   transition (so $id' > id$).  Let $\pi$ be a remapping such that $\pi(id) =
%%   id'$ and otherwise $\pi$~is the identity function over parameters of the
%%   transition.  Then the corresponding transition template using~$id'$ would be
%%   $v \uplus \hole{id'} \trans[\pi(e)] \pi(v') \uplus \hole{id'}$.  This is an
%%   equivalent transition template; hence there is no need to add it to the set
%%   of representative templates. 
%% \end{itemize}


