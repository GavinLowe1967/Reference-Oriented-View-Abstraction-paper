\section{Symmetry reduction}

Note: some of this lives before the definition of the abstract algorithm.
Some is more related to the BFS implementation. 

Idea of symmetry reduction.

Define renaming functions: partial injective function over non-distinguished
identities, preserving types.

For each type of identities, assume a linear order over the type.  In examples
below, for node identities we will assume a total order $N_0 < N_1 < N_2 <
\ldots$; and similarly for thread identities.  (In the implementation,
identities are represented by non-negative integers: we use the integer
order.)

\begin{definition}
Define normal form.  Fixed processes in some fixed order (defined in the input
script).  For a view, list components, with principal first, followed by
secondary components in the order corresponding to the references from the
principal (without repetitions).  For extended views necessary for extended
transitions, append any additional component at the end.

For each type, the first occurrences of each value must form an
initial segment of the type.
\end{definition}


Example.

\begin{lemma}
If $s$ and~$s'$ are both in normal form, and $\pi(s)$ and~$\pi'(s')$ are
substates of the same system state, then $s.\fixed = s'.\fixed$.
\end{lemma}
