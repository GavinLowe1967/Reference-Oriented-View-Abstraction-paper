\section{Calculating induced transitions with symmetry reduction}
\label{sec:induced-symmetry}

In this section, we consider consider how to calculate the transition induced
by a transition $pre \trans{e} post$ on a view~$v$, given that both are in
normal form and represent all members of their equivalence classes.  
%
Following Definition~\ref{def:induced-transition}, in principle, we need to
find all ways of renaming parameters of~$v$ and~$pre$ to produce views that
are accordant.  However, if two different renamings would produce equivalent
post-views, it is enough to consider just one of them.  It is therefore enough
to keep the parameters of $pre$ fixed, and to rename the parameters of~$v$.
Each renaming can be defined by an renaming~$\pi$ over the parameters of~$v$.
That is, we consider renamings~$\pi$ such that $\pi(v)$ and~$pre$ are
accordant, and consider the effect of the transition on~$\pi(v)$.
%
Below, we restrict the range of such renamings further, while ensuring we
produce a representative of each equivalence class of resulting post-views.


%% We start by considering full views; in Section~\ref{sec:effectOn-restricted}
%% we consider restricted views.

%%%%%%%%%%%%%%%%%%%%%%%%%%%%%%%%%%%%%%%%%%%%%%%%%%%%%%%

%\subsection{Full views} 
%\label{ssec:effectOn-full}

Recall that in order to be accordant, if $\pi(v)$ and $pre$ each have a
component with the same identity, then those components must be equal.  We say
that the renaming has \emph{unified} these components.
%
\begin{definition}
Let $v$ and $pre$ be in normal form.  Then renaming function~$\pi$ over the
parameters of~$v$ is a \emph{unification function} if $\pi(v)$ and $pre$ are
accordant.
%
If $c$ is a component of~$v$, $c'$ is a component of~$pre$, and $\pi(c) = c'$,
we say that $c$ and~$c'$ are \emph{unified}.
\end{definition}
%
Of course, it is possible to unify two components only if they have the same
control state. 

Recall that if $\pi(v)$ and~$pre$ are accordant, then their fixed processes
are equal.  The following lemma follows from the way we have defined normal
forms. 
%% But $v$ and $pre$ are stored in normal form, so this will be the
%% case only if the fixed processes of $v$ and~$pre$ are equal, and
%% $\pi$ is be the identity function on all parameters of those fixed
%% processes.  
%
\begin{lemma}
If $v$ and~$pre$ are both in normal form, and $\pi(v)$ and~$pre$ are
accordant, then $v.\fixed = pre.\fixed$ and $\pi$ is the identity over the
parameters of~$v.\fixed$.
\end{lemma}

%%%%%

\begin{example}
We use a running example to illustrate the technique.  Consider
\begin{eqnarray*}
pre & = &
   (fixed(N_0); Th(T_0, N_1, N_2), Nd_A(N_1, N_3), Nd_B(N_2, N_0, N_4)), 
\\
post & = & 
  (fixed'(N_5); Th'(T_0, N_1, N_2), Nd_C(N_1, N_2, N_3), Nd_B(N_2, N_0, N_4)) ,
\\
v & = & 
  (fixed(N_0); Nd_A(N_1, N_2), Nd_B(N_2, N_0, N_3)).
\end{eqnarray*}
%
$pre$ and~$v$ have fixed processes in the same states, so there is the
potential to produce a renaming~$\pi(v)$ accordant with~$pre$.  We start with
the partial renaming function $\pi = \set{N_0 \mapsto N_0}$, the identity over
the parameters of the fixed processes.

It is worth noting that the identifiers $N_1$, $N_2$ and~$N_3$ that appear in
both~$pre$ and~$v$ might represent different parameters, since each substate
is just a representative of its equivalence class.  The fact that the same
identifiers appear in each substate is just an artefact of the way we produce
normal forms.
\end{example}

%%%%%

In general, any subset of the components of~$v$ might be unified with
components of~$pre$ (including no unifications).  For each potential choice of
which components to unify, we construct the minimal renaming function that
extends the identity function over the parameters of the fixed processes so as
to rename the parameters of each unified component of~$v$ to the parameters of
the corresponding component of~$pre$.  We call the resulting function a
\emph{partial unification function}.  However, some such choices of
unifications might prove impossible, if no such function exists.
%
\begin{definition}
Let $v$ and $pre$ be in normal form with $v.\fixed = pre.\fixed$.  Then a
partial renaming function~$\pi$ is a \emph{partial unification function} if
\begin{itemize}
\item $\pi$ is the identity function over the parameters of $v.\fixed$;

\item $\pi$ is defined over the parameters of a subset of the components
  of~$v$, mapping each such component to unify it with a component of $pre$;

\item $\pi$ is undefined over all parameters not included under the previous
  two points.
\end{itemize}
\end{definition}


%%%%%

\begin{example}
We continue with the running example.  Unifying the two $Nd_A$ processes
gives the partial unification function
%
\begin{eqnarray*}
\pi_1 & = & \set{N_0 \mapsto N_0, N_1 \mapsto N_1, N_2 \mapsto N_3}.
\end{eqnarray*}
%
Alternatively, unifying the two $Nd_B$ processes gives
\begin{eqnarray*}
\pi_2 & = & \set{N_0 \mapsto N_0, N_2 \mapsto N_2, N_3 \mapsto N_4}.
\end{eqnarray*}
It is not possible to unify both pairs simultaneously, since $N_2$ cannot be
mapped consistently.  In addition, the renaming~$\pi$ corresponds to unifying
no components.
\end{example}

We now need to extend each such renaming function in a consistent way.  The
following definition describes what that means.
%
\begin{definition}
\label{def:consistent-extension}
Let $pre \trans{e} post$ be an extended transition, and $v$ be a view, both in
normal form, with $v.\fixed = pre.\fixed$.
%
Consider a partial unification function~$\pi$.  We define an extension~$\pi'$
of~$\pi$ to be a \emph{consistent extension} if (a)~the renaming~$\pi'$ is
injective and defined over all parameters of~$v$; and (b)~whenever a
component~$c$ in~$v$ is not unified with any component of~$pre$, the identity
of~$c$ is not mapped to an identity of a component~$c'$ in~$pre$ (since that
would require unifying~$c$ and~$c'$).
\end{definition}

%%%%%

The following lemma follows directly from the definitions.
%
\begin{lemma}
Let $pre \trans{e} post$ be an extended transition, and $v$ be a view, both in
normal form, with $v.\fixed = pre.\fixed$.  Then every unification
function~$\pi$ is a consistent extension of a partial unification function,
and vice versa.
\end{lemma}

In principal, then, we need to consider all consistent extensions of all
partial unification functions.  However, many different consistent extensions
will end up producing the same new view, up to equivalence, so it will be
enough to build only representative renamings.  Recall
(Definition~\ref{def:induced-transition}) that when we build an induced
transition, the state of each component in the new view is: (1)~the state
from~$post$ for components in the transition; or (2)~the state from~$v$ for
other components.  The unifications already tell us the states in case~(1).
In case~(2), each parameter in the component might or might not be the same as
a parameter taken from $post$.  We need to find all ways of renaming
parameters in such components to give distinct new views, up to equivalence.
The following definition captures the required consistent extensions; we
verify this fact as Proposition~\ref{prop:unifying-renaming}.

%%%%%

\begin{definition}
\label{def:representative-consistent-extension}
Consider an extended transition~$pre \trans{e} post$, a view~$v$, and a
partial unification function~$\pi$.  We define a consistent extension~$\pi'$
of~$\pi$ to be a \emph{representative consistent extension} if for every
parameter~$x$ in~$v$ but not in the domain of~$\pi$,\, $\pi'(x)$ is one of the
following:
%
\begin{enumerate}
\item\label{clause:remap-1} a parameter of $post.\fixed$; 

\item\label{clause:remap-2} a parameter of the state in~$post$ of a unified
  component;

\item\label{clause:remap-3} if the principal of~$v$ is unified with a
  component~$c$, and $c$~gains a reference to a component with identity~$id$,
  then a parameter of the state in~$post$ of~$id$ (note that $pre$ and $post$
  must contain a component with identity~$id$, by
  clause~\ref{assump:secondary-cpts-new-refs} of Assumption~\ref{assump}); or

\item\label{clause:remap-4} a minimal fresh parameter, i.e.~a parameter
  different from those in~$pre$ and~$post$, and in each case choosing (in the
  order in which those parameters~$x$ appear in~$v$) the minimal such value
  that hasn't been used previously. 
\end{enumerate}
%
%% Further, under case~\ref{clause:remap-4}, \emph{minimal} fresh parameters are
%% chosen for each such~$x$, in the order in which those parameters~$x$ appear
%% in~$v$ (i.e.~in each case choosing the smallest fresh parameter not already
%% used).
%
Note that these are subject to the conditions~(a) and~(b) of
Definition~\ref{def:consistent-extension}. 
%% These are all subject to two provisos: (a)~that the renaming remains
%% injective; and (b)~a parameter that is the identity of a component~$c$ in~$v$
%% that is not unified with any component of~$pre$ cannot be mapped to an
%% identity of a component in~$pre$ (since that would require unifying~$c$).
\end{definition}
%
%% When a parameter is mapped to a fresh parameter, all choices of the fresh
%% parameter give equivalent post-states.  Hence it is enough to make the choice
%% in one way: we pick the minimal fresh parameter each time. 


%%%%%

\begin{example}
We continue the running example, considering just the unification
corresponding to~$\pi_1$.  Consider what values~$N_3$ can be mapped to.  By
clause~\ref{clause:remap-1}, it can be mapped to~$N_5$.  By
clause~\ref{clause:remap-2}, it can be mapped to~$N_2$.  By
clause~\ref{clause:remap-3}, it can be mapped to~$N_4$ (from the post-state of
the $Nd_B$ process).  Finally, by clause~\ref{clause:remap-4}, it can be
mapped to the minimal fresh value, namely~$N_6$.
%
This then produces four renamings:
\[
\pi_X = \set{N_0 \mapsto N_0, N_1 \mapsto N_1, N_2 \mapsto N_3, N_3 \mapsto X},
\quad \mbox{for $X = N_2, N_4, N_5, N_6$}.
\]
For each such renaming~$\pi_X$,\, $v$ gets remapped to a view of the form
\[
(fixed(N_0); Nd_A(N_1, N_3), Nd_B(N_3, N_0, X)).
\]
The $fixed$ and $Nd_A$ processes evolve under the transition \( pre
\trans{} post \), and the latter gains a reference to~$N_2$ in
$post$.  This produces the four views
\[
(fixed'(N_5); Nd_C(N_1, N_2, N_3), Nd_B(N_2, N_0, N_4), Nd_B(N_3, N_0, X)).
\]
Finally, these views need to be reduced to normal form.  Note that the four
normal forms are distinct.
\end{example}

%%%%%

The following proposition shows that considering representative consistent
extensions is enough to produce all induced transitions, up to equivalence. 
%
\begin{prop}
\label{prop:unifying-renaming}
Consider an extended transition $pre \trans{e} post$ and a view~$v$, both in
normal form, with $v.\fixed = pre.\fixed$.  Consider also a partial
unification function~$\pi$.
%
Let $\pi'$ be a consistent extension of~$\pi$, and
let $v'$ be the view produced by Definition~\ref{def:induced-transition}
considering the effect of $pre \trans{e} post$ on~$\pi'(v)$.  
%
Then there is a representative consistent extension~$\pi''$ of~$\pi$, such
that, letting $v''$ be the view produced by
Definition~\ref{def:induced-transition} considering the effect of $pre
\trans{e} post$ on~$\pi''(v)$, we have $v'' \equiv v'$.
\end{prop}

%%%%%%

\begin{proof}
Let $pre$, $post$, $v$, $\pi$, $\pi'$ and~$v'$ be as in the statement of the
lemma.  We describe how to construct the renaming~$\pi''$, and a
renaming~$\hat\pi$ such that $\pi' = \hat\pi \circ \pi''$.  This will imply
that $\hat\pi(v'') = v'$.  The following figure illustrates (and acts as an
aide-m\'emoire).
%
\begin{center}
\begin{tikzpicture}[>= angle 90]
\draw (0,0) node (v) {$v$};
%
\draw (2,1) node (pi'v) {$\pi'(v)$};
\path[draw, ->] (v) -- node[above]{\scriptsize $\pi'$} (pi'v);
%
\draw (2,-1) node (pi''v) {$\pi''(v)$};
\path[draw, ->] (v) -- node[above]{\scriptsize $\pi''$} (pi''v);
\path[draw, ->] (pi''v) -- node[right]{\scriptsize $\hat\pi$} (pi'v);
%
\draw (pi'v)++(1.5,0) node (induces1) {induces};
\draw (induces1)++(1.5,0) node (v') {$v'$};
%
\draw (pi''v)++(1.5,0) node (induces2) {induces};
\draw (induces2)++(1.5,0) node (v'') {$v''$};
\path[draw, ->] (v'') -- node[right]{\scriptsize $\hat\pi$} (v');
\end{tikzpicture}
\end{center}
%
In fact, $\hat\pi$ will be the identity over all parameters except for fresh
parameters chosen under clause~\ref{clause:remap-4} of
Definition~\ref{def:representative-consistent-extension}.

We define $\pi''$ as an extension of~$\pi$.  This implies that $\pi''$
and~$\pi'$ agree on values in the domain of~$\pi$, and unify the same
components.  For all values~$y$ in the range of~$\pi$, we define $\hat\pi(y) =
y$. 

Recall that $v'.\fixed = v''.\fixed = post.\fixed$; each component of~$v'$ is
taken from~$post$ if the corresponding component of~$\pi'(v)$ is in~$pre$, and
otherwise is taken from~$\pi'(v)$; and each component of~$v''$ is taken
from~$post$ if the corresponding component of~$\pi''(v)$ is in~$pre$, and
otherwise is taken from~$\pi''(v)$.  We arrange for~$v'$ and~$v''$ to take the
same components from~$post$.

%% For each parameter~$x$ of~$v.\fixed$, we define $\pi''(x) = x$.  This means
%% $\pi''(v).\fixed = pre.fixed$.  And we define $\hat\pi(x) = x$. 

For each parameter~$x$ of~$v$ such that $y = \pi'(x)$ is a parameter of
$post.\fixed$, we define $\pi''(x) = y$, and $\hat\pi(y) = y$.  Note that each
such value~$y$ is included under case~\ref{clause:remap-1} of
Definition~\ref{def:representative-consistent-extension}.  This ensures
that~$\pi''(v)$ and~$\pi'(v)$ agree on all such~$y$; and hence~$v'$ and~$v''$
agree on all such~$y$.  Note that this is a consistent extension of~$\pi$,
because~$\pi'$ is a consistent extension of~$\pi$.  

For each parameter~$x$ such that $y = \pi'(x)$ is a parameter of a component
of~$v'$ taken from~$post$, we again define $\pi''(x) = y$, and $\hat\pi(y) =
y$.  Note that each such value $y$ is included under
case~\ref{clause:remap-2} or \ref{clause:remap-3} of
Definition~\ref{def:representative-consistent-extension}.  This ensures
that~$\pi''(v)$ and~$\pi'(v)$ agree on all such~$y$; and hence~$v'$ and~$v''$
agree on all such~$y$; in particular, they include the same components taken
from~$post$.  Again note that this is a consistent extension of~$\pi$,
because~$\pi'$ is a consistent extension of~$\pi$. 

Finally, each other parameter~$y$ in~$v'$ must necessarily be in a component
taken from~$\pi'(v)$.  Suppose the parameter is $\pi'(x)$, and $x$ is not in
the domain of~$\pi$.  We define $\pi''(x)$ to be the minimal fresh
parameter~$z$, chosen as in
Definition~\ref{def:representative-consistent-extension}; and we let
$\hat\pi(z) = y$.  This ensures that $v''$ uses~$z$ wherever $v'$ uses~$y$.
Also note that this is a consistent extension of~$\pi$, by construction.
\end{proof}

%%%%%

\begin{impNote}
\texttt{Unification.combine} produces the remapped version of~$v$, together
with information about the unifications.  If uses \texttt{allUnifs} to find
all choices of unification and corresponding renaming function~$\pi$.  Then
\texttt{extendUnif} extends this, and produces the remapped state.

\texttt{EffectOn.apply} produces the corresponding post-views.
\end{impNote}

%%%%%

\subsection{Optimisations} 


%%%%%

\begin{opt}
Recall Optimisation~\ref{opt:avoid-induced}.  
\begin{itemize}
\item For case~\ref{case:avoid-induced-1}, we avoid unifying the principal
  of~$v$ with the principal of~$pre$.

\item For case~\ref{case:avoid-induced-2}, if $pre.\fixed = post.\fixed$ we
  require that at least one component of~$v$ unifies with a component that
  changes state.

\item For case~\ref{case:avoid-induced-3}, we store a record of pairs
  $(v, post.\fixed)$ for which we have considered the effect of a transition
  $pre \trans{e} post$ on~$v$ for which $pre.\fixed \ne post.\fixed$,\,
  %% $unifs$ describes which components of~$v$ unify with which components
  %% of~$pre$ (
  and $v$
  unifies with no component.
%
  If subsequently we consider a similar case ---that is, the same
  $post.\fixed$ and~$v$, and no unifications--- we identify that it will not
  produce any new views, and so avoid the construction.
\end{itemize}
\end{opt}


\begin{impNote}
Done in isSufficientUnif. 
\end{impNote}

\begin{improve}
The third case is not quite the same as in
Optimisation~\ref{opt:avoid-induced}.  There we included cases where $pre$
and~$v$ shared a component that did not change state.  An obvious attempt to
implement this goes wrong: if there are two cases with the same $post.\fixed$
and~$v$, that include unifications of \emph{different} components that do not
change state, they might give different new views (with the unified components
having different parameters in $post.\fixed$).
%
For example
\begin{eqnarray*}
pre & = & (fixed; Th_1(T_0,N_0), Nd_A(N_0,N_1)) \\
post & = & (fixed'(N_1); Th_1'(T_0), Nd_A(N_0,N_1)) \\
v & = & (fixed; Th_v(T_0,N_0), Nd_A(N_0,N_1))
\end{eqnarray*}
unifying the two $Nd_A$ processes, with remapping $\set{T_0 \mapsto T_1, N_0
  \mapsto N_0,\linebreak[1] {N_1 \mapsto N_1}}$, induces
\[
(fixed'(N_1); Th_v(T_1,N_0), Nd_A(N_0,N_1) \equiv
(fixed'(N_0); Th_v(T_0,N_1), Nd_A(N_1,N_0)).
\]
But with 
\begin{eqnarray*}
pre' & = & (fixed; Th_2(T_0,N_0,N_1), Nd_A(N_0,N_2), Nd_N(N_1,Null)) \\
post' & = & (fixed'(N_1); Th_2'(T_0), Nd_A(N_0,N_2), Nd_N(N_1,Null))
\end{eqnarray*}
unifying the two $Nd_A$ processes, with remapping $\set{T_0 \mapsto T_1, N_0
  \mapsto N_0,\linebreak[1] {N_1 \mapsto N_2}}$, induces a transition to
\[\mit
(fixed'(N_1); Th_v(T_1,N_0), Nd_A(N_0,N_2)) \equiv
  (fixed'(N_0); Th_v(T_0,N_1), Nd_A(N_1, N_2)).
\]


If we have two cases concerning the same $post.\fixed$, $v$, and component~$c$
of~$v$ that unifies and doesn't change state, and $c$ unifies respectively
with~$c_1$ and~$c_2$ which differ only on parameters not in $post.\fixed$,
then it is enough to consider just one such case.  In the above example, we
had $c_1 = Nd_A(N_0,N_1)$,\, $c_2 = Nd_A(N_0,N_2)$ which differ on~$N_1$. 

\framebox{Do this.}  For the components, it's enough to store the list of
parameters shared with $post.\fixed$ and their indices. 
\end{improve}

 
%%%%%%%%%%%%%%%%%%%%

\begin{opt}
If the principal of~$v$ is unified with a component of~$pre$, and loses a
reference to a component~$c$, then all renamings of~$c$ produce the same final
view.  Thus it is enough to consider a single renaming, say the renaming that
agrees with the renaming within other components, but otherwise maps to fresh
parameters.  There is no need to explicitly construct the renaming of~$c$.  
%
\framebox{Consider this.}

This won't work with singleRef, however, because of cross references or
missing references. 
\end{opt}

%%%%%

\begin{improve} 
If the principal and another component~$c$ of~$v$ are both unified with
components of $pre$, and the principal of~$v$ loses the reference to~$c$ in
the transition, then for case~\ref{clause:remap-2} of
Definition~\ref{def:representative-consistent-extension}, we can ignore
parameters of~$c$.  This is only relevant for renaming of some third
component.
%
This might be useful when a node has a reference to the thread, but loses
that reference.
\end{improve}
