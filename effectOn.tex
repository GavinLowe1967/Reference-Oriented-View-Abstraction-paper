\subsection{Calculating induced transitions}
\label{ssec:induced-symmetry}

We now consider how to calculate the transitions induced
by a transition $pre \trans[e] post$ on a view~$v$, given that both are in
normal form and represent all members of their equivalence classes.  
%
Following Definition~\ref{def:induced-transition}, in principle we need to
find all ways of remapping parameters of~$v$ and~$pre$ to produce views that
are accordant.  However, if two different remappings would produce equivalent
post-views, it is enough to consider just one of them.  It is therefore enough
to keep the parameters of $pre$ fixed, and to rename the parameters of~$v$.
That is, we consider remappings~$\pi$ such that $\pi(v)$ and~$pre$ are
accordant, and consider the effect of the transition on~$\pi(v)$.
%
\begin{definition}
Let $v$ and $pre$ be in normal form.  Then remapping function~$\pi$ over the
parameters of~$v$ is a \emph{unification function} if $\pi(v)$ and $pre$ are
accordant.
\end{definition}
%
Below, we restrict the range of unification functions considered, while
ensuring we produce a representative of each equivalence class of resulting
post-views.

Recall that in order to be accordant, the fixed processes of $\pi(v)$
and~$pre$ must be equal.  Lemma~\ref{lem:remapping-id-on-fixed-params} then
implies that $v.\fixed = pre.\fixed$ and $\pi$ is the identity over the
parameters of~$v.\fixed$.  Our approach therefore starts with a partial
remapping representing this partial identity.

%%%%%

\begin{example}
We use a running example to illustrate the technique.  Consider
\begin{eqnarray*}
pre & = &
   (fixed(N_0); Th(T_0, N_1, N_2), Nd_A(N_1, N_3), Nd_B(N_2, N_0, N_4)), 
\\
post & = & 
  (fixed'(N_5); Th'(T_0, N_1, N_2), Nd_C(N_1, N_2, N_3), Nd_B(N_2, N_0, N_4)) ,
\\
v & = & 
  (fixed(N_0); Nd_A(N_1, N_2), Nd_B(N_2, N_0, N_3)).
\end{eqnarray*}
%
$pre$ and~$v$ have fixed processes in the same states, so there is the
potential to produce a remapping~$\pi(v)$ accordant with~$pre$.  We start with
the partial remapping function $\pi = \set{N_0 \mapsto N_0}$, the identity over
the parameters of the fixed processes.
\end{example}

%%%%%

In general, any subset of the components of~$v$ might be unified with
components of~$pre$ (including no unifications).  For each potential choice of
which components to unify, we construct the minimal remapping function that
extends the identity function over the parameters of the fixed processes so as
to rename the parameters of each unified component of~$v$ to the parameters of
the corresponding component of~$pre$.  We call the resulting function a
\emph{partial unification function}.  However, some such choices of
unifications might prove impossible, if no such function exists.
%
\begin{definition}
Let $v$ and $pre$ be in normal form with $v.\fixed = pre.\fixed$.  Then a
partial remapping function~$\pi$ is a \emph{partial unification function} if
\begin{itemize}
\item $\pi$ is the identity function over the parameters of $v.\fixed$;

\item $\pi$ is defined over the parameters of a subset of the components
  of~$v$, mapping each such component to unify it with a component of $pre$;

\item $\pi$ is undefined over all parameters not included under the previous
  two points.
\end{itemize}
\end{definition}

%%%%%

\begin{example}
\label{example:effectOn-running-2}
We continue with the running example.  Unifying the two $Nd_A$ processes
gives the partial unification function
%
\begin{eqnarray*}
\pi_1 & = & \set{N_0 \mapsto N_0, N_1 \mapsto N_1, N_2 \mapsto N_3}.
\end{eqnarray*}
%
Alternatively, unifying the two $Nd_B$ processes gives
\begin{eqnarray*}
\pi_2 & = & \set{N_0 \mapsto N_0, N_2 \mapsto N_2, N_3 \mapsto N_4}.
\end{eqnarray*}
It is not possible to unify both pairs simultaneously, since $N_2$ cannot be
mapped consistently.  In addition, the remapping~$\pi$ corresponds to unifying
no components.
\end{example}

We now need to extend each such remapping function in a consistent way.  The
following definition describes what that means.
%
\begin{definition}
\label{def:consistent-extension}
Let $pre \trans[e] post$ be a transition, and $v$ be a view, both in normal
form, with $v.\fixed = pre.\fixed$.
%
Consider a partial unification function~$\pi$.  We define an extension~$\pi'$
of~$\pi$ to be a \emph{consistent extension} if (a)~the remapping~$\pi'$ is
injective and defined over all parameters of~$v$; and (b)~whenever a
component~$c$ in~$v$ is not unified with any component of~$pre$, the identity
of~$c$ is not mapped to an identity of a component~$c'$ in~$pre$ (since that
would require unifying~$c$ and~$c'$).
\end{definition}

%%%%%

The following lemma follows directly from the definitions.
%
\begin{lemma}
Let $pre \trans[e] post$ be a transition, and $v$ be a view, both in normal
form, with $v.\fixed = pre.\fixed$.  Then every unification function~$\pi$ is
a consistent extension of a partial unification function, and vice versa.
\end{lemma}

In principal, then, we need to consider all consistent extensions of all
partial unification functions.  However, many different consistent extensions
will end up producing the same new view, up to equivalence, so it will be
enough to build only representative remappings.  Recall
(Definition~\ref{def:induced-transition}) that when we build an induced
transition, the state of each component in the new view is: (1)~the state
from~$post$ for components in the transition; or (2)~the state from~$v$ for
other components.  The choice of which components were unified already tell us
the states in case~(1).  In case~(2), each parameter in the component might or
might not be the same as a parameter taken from $post$.  We need to find all
ways of remapping parameters in such components to give distinct new views, up
to equivalence.  It turns out that it is enough to map to only certain
parameters of~$post$, which we call \emph{result-relevant}.
%
\begin{definition}
\label{def:result-relevant}
Consider a transition $pre \trans[e] post$ and a view~$v$, both in normal form.
Consider a partial unification function~$\pi_u$.  We define the
\emph{result-relevant parameters} to be:
%
\begin{enumerate}
\item\label{clause:result-relevant-1} All parameters of $post.\fixed$;

\item\label{clause:result-relevant-2} For each component of~$v$ that is
  unified with a component of~$pre$, all parameters of that component
  in~$post$;

\item\label{clause:result-relevant-3} If the principal of~$v$ is unified with
  a component of~$pre$, and acquires a reference to a component~$c'$
  of~$post$, then all parameters of~$c'$.
\end{enumerate}
\end{definition}

%%%%%

The following lemma shows that the result of an induced transition is defined
(up to equivalence) by the result-relevant parameters in the unifying
function.
We say that two remapping functions~$\pi_1$ and~$\pi_2$ \emph{range-agree}
on~$y$ if they map the same variables to~$y$:
\[
\forall x \spot \pi_1(x) = y \iff \pi_2(x) = y.
\]
Necessarily there is at most one~$x$ that the remapping functions map to~$y$.
%
\begin{lemma}
\label{lem:remappings-equivalent}
Consider a transition $pre \trans[e] post$ and a view~$v$, both in normal form.
Consider a partial unification function~$\pi_u$, and two consistent
extensions~$\pi_1$ and~$\pi_2$, and suppose that $\pi_1$ and $\pi_2$
range-agree on all result-relevant parameters.  Let $v_1 = \pi_1(v)$ and $v_2
= \pi_2(v)$; and let $v_1'$ and~$v_2'$ be the views produced by
Definition~\ref{def:induced-transition} considering the effect of $pre
\trans[e] post$ on~$v_1$ and~$v_2$, respectively.  Then $v_1' \equiv v_2'$. 
\end{lemma}

%%%%%

\begin{proof}
Note that $v_1'$ and~$v_2'$ contain the same processes that are taken
from~$post$, namely: (a) the fixed processes; (b) if the principal is unified
with a component of~$pre$, then the corresponding component of~$post$, and all
components of~$post$ to which it has a reference; (c) if the principal is not
unified, then any secondary components that are unified.  In each case, all
parameters of these processes are result-relevant.  All other processes are
taken from~$v_1$ and~$v_2$, respectively; by assumption, these agree on
result-relevant parameters.  Hence $v_1'$ is related to~$v_2'$ by the mapping
$\hat\pi$ that maps $\pi_1(x)$ to~$\pi_2(x)$ for each parameter~$x$ (so is the
identity over the result-relevant parameters).  Hence $v_1' \equiv v_2'$.
\end{proof}

%%%%%

It is therefore enough to consider the remappings defined as follows.
%
\begin{definition}
\label{def:representative-consistent-extension}
Consider a transition~$pre \trans[e] post$, a view~$v$, and a
partial unification function~$\pi$.  We define a consistent extension~$\pi'$
of~$\pi$ to be a \emph{representative consistent extension} if for every
parameter~$x$ in~$v$ but not in the domain of~$\pi$,\, $\pi'(x)$ is either:
\begin{itemize}
\item a result-relevant parameter;

\item a minimal fresh parameter, i.e.~a parameter different from those
  in~$pre$ and~$post$, and in each case choosing (in the order in which those
  parameters~$x$ appear in~$v$) the minimal such value that hasn't been used
  previously.
\end{itemize}
\end{definition}

%%%%%

The following proposition shows that considering representative consistent
extensions is enough to produce all induced transitions, up to equivalence.
It follows immediately from Lemma~\ref{lem:remappings-equivalent}.
%
\begin{prop}
\label{prop:unifying-remapping}
Consider a transition $pre \trans[e] post$ and a view~$v$, both in normal
form, with $v.\fixed = pre.\fixed$.  Consider also a partial unification
function~$\pi$.
%
Let $\pi'$ be a consistent extension of~$\pi$, and
let $v'$ be the view produced by Definition~\ref{def:induced-transition}
considering the effect of $pre \trans[e] post$ on~$\pi'(v)$.  
%
Then there is a representative consistent extension~$\pi''$ of~$\pi$, such
that, letting $v''$ be the view produced by
Definition~\ref{def:induced-transition} considering the effect of $pre
\trans[e] post$ on~$\pi''(v)$, we have $v'' \equiv v'$.
\end{prop}

%%%%%

For a given transition $pre \trans[e] post$ and a view~$v$ such that
$pre.\fixed = v.fixed$, we therefore consider all partial unification
functions~$\pi$, all consistent extensions~$\pi'$ of~$\pi$, calculate the
view~$v'$ produced by $pre \trans[e] post$ acting on $\pi'(v)$, and then
calculate the representative of~$v'$.


%% \begin{definition}
%% \label{def:representative-consistent-extension}
%% Consider a transition~$pre \trans[e] post$, a view~$v$, and a
%% partial unification function~$\pi$.  We define a consistent extension~$\pi'$
%% of~$\pi$ to be a \emph{representative consistent extension} if for every
%% parameter~$x$ in~$v$ but not in the domain of~$\pi$,\, $\pi'(x)$ is one of the
%% following:
%% %
%% \begin{enumerate}
%% \item\label{clause:remap-1} a parameter of $post.\fixed$ (necessarily distinct
%%   from the parameters of $pre.\fixed$); 

%% \item\label{clause:remap-2} a parameter of the state in~$post$ of a unified
%%   component;

%% \item\label{clause:remap-3} if the principal of~$v$ is unified with a
%%   component~$c$, and $c$~gains a reference to a component with identity~$id$,
%%   then a parameter of the state in~$post$ of~$id$ (note that $pre$ and $post$
%%   must contain a component with identity~$id$, by
%%   clause~\ref{assump:secondary-cpts-new-refs} of Assumption~\ref{assump:2}); or

%% \item\label{clause:remap-4} a minimal fresh parameter, i.e.~a parameter
%%   different from those in~$pre$ and~$post$, and in each case choosing (in the
%%   order in which those parameters~$x$ appear in~$v$) the minimal such value
%%   that hasn't been used previously. 
%% \end{enumerate}
%% %
%% Note that these are subject to the conditions~(a) and~(b) of
%% Definition~\ref{def:consistent-extension}. 
%% \end{definition}

%%%%%

\begin{example}
We continue the running example, considering just the unification
corresponding to~$\pi_1$ in Example~\ref{example:effectOn-running-2}.
Consider what values~$N_3$ can be mapped to.  Considering
Definition~\ref{def:result-relevant}: by
clause~\ref{clause:result-relevant-1}, it can be mapped to~$N_5$; by
clause~\ref{clause:result-relevant-2}, it can be mapped to~$N_2$; by
clause~\ref{clause:result-relevant-3}, it can be mapped to~$N_4$ (from the
post-state of the $Nd_B$ process).  Finally, it can be mapped to the minimal
fresh value, namely~$N_6$.
%
This then produces four remappings:
\[
\pi_X = \set{N_0 \mapsto N_0, N_1 \mapsto N_1, N_2 \mapsto N_3, N_3 \mapsto X},
\quad \mbox{for $X = N_2, N_4, N_5, N_6$}.
\]
For each such remapping~$\pi_X$,\, $v$ gets remapped to a view of the form
\[
(fixed(N_0); Nd_A(N_1, N_3), Nd_B(N_3, N_0, X)).
\]
The $fixed$ and $Nd_A$ processes evolve under the transition \( pre
\trans{} post \), and the latter gains a reference to~$N_2$ in
$post$.  This produces the four views
\[
(fixed'(N_5); Nd_C(N_1, N_2, N_3), Nd_B(N_2, N_0, N_4), Nd_B(N_3, N_0, X)).
\]
Finally, these views are reduced to normal form.  Note that the four
normal forms are distinct.
\end{example}

%%%%%

%%%%%%

%% \begin{proof}
%% Let $pre$, $post$, $v$, $\pi$, $\pi'$ and~$v'$ be as in the statement of the
%% proposition.  We describe how to construct the remapping~$\pi''$, and a
%% remapping~$\hat\pi$ such that $\pi' = \hat\pi \circ \pi''$.  This will imply
%% that $\hat\pi(v'') = v'$.  The following figure illustrates (and acts as an
%% aide-m\'emoire).
%% %
%% \begin{center}
%% \begin{tikzpicture}[>= angle 90]
%% \draw (0,0) node (v) {$v$};
%% %
%% \draw (2,1) node (pi'v) {$\pi'(v)$};
%% \path[draw, ->] (v) -- node[above]{\scriptsize $\pi'$} (pi'v);
%% %
%% \draw (2,-1) node (pi''v) {$\pi''(v)$};
%% \path[draw, ->] (v) -- node[above]{\scriptsize $\pi''$} (pi''v);
%% \path[draw, ->] (pi''v) -- node[right]{\scriptsize $\hat\pi$} (pi'v);
%% %
%% \draw (pi'v)++(1.5,0) node (induces1) {induces};
%% \draw (induces1)++(1.5,0) node (v') {$v'$};
%% %
%% \draw (pi''v)++(1.5,0) node (induces2) {induces};
%% \draw (induces2)++(1.5,0) node (v'') {$v''$};
%% \path[draw, ->] (v'') -- node[right]{\scriptsize $\hat\pi$} (v');
%% \end{tikzpicture}
%% \end{center}
%% %
%% In fact, $\hat\pi$ will be the identity over all parameters except for fresh
%% parameters chosen under clause~\ref{clause:remap-4} of
%% Definition~\ref{def:representative-consistent-extension}.

%% We define $\pi''$ as an extension of~$\pi$.  This implies that $\pi''$
%% and~$\pi'$ agree on values in the domain of~$\pi$, and unify the same
%% components.  For all values~$y$ in the range of~$\pi$, we define $\hat\pi(y) =
%% y$. 

%% Recall that $v'.\fixed = v''.\fixed = post.\fixed$; each component of~$v'$ is
%% taken from~$post$ if the corresponding component of~$\pi'(v)$ is in~$pre$, and
%% otherwise is taken from~$\pi'(v)$; and each component of~$v''$ is taken
%% from~$post$ if the corresponding component of~$\pi''(v)$ is in~$pre$, and
%% otherwise is taken from~$\pi''(v)$.  We arrange for~$v'$ and~$v''$ to take the
%% same components from~$post$.

%% For each parameter~$x$ of~$v$ such that $y = \pi'(x)$ is a parameter of
%% $post.\fixed$, we define $\pi''(x) = y$, and $\hat\pi(y) = y$.  Note that each
%% such value~$y$ is included under case~\ref{clause:remap-1} of
%% Definition~\ref{def:representative-consistent-extension}.  This ensures
%% that~$\pi''(v)$ and~$\pi'(v)$ agree on all such~$y$; and hence~$v'$ and~$v''$
%% agree on all such~$y$.  Note that this is a consistent extension of~$\pi$,
%% because~$\pi'$ is a consistent extension of~$\pi$.  

%% For each parameter~$x$ such that $y = \pi'(x)$ is a parameter of a component
%% of~$v'$ taken from~$post$, we again define $\pi''(x) = y$, and $\hat\pi(y) =
%% y$.  Note that each such value $y$ is included under
%% case~\ref{clause:remap-2} or \ref{clause:remap-3} of
%% Definition~\ref{def:representative-consistent-extension}.  This ensures
%% that~$\pi''(v)$ and~$\pi'(v)$ agree on all such~$y$; and hence~$v'$ and~$v''$
%% agree on all such~$y$; in particular, they include the same components taken
%% from~$post$.  Again note that this is a consistent extension of~$\pi$,
%% because~$\pi'$ is a consistent extension of~$\pi$. 

%% Finally, each other parameter~$y$ in~$v'$ must necessarily be in a component
%% taken from~$\pi'(v)$.  Suppose the parameter is $\pi'(x)$, and $x$ is not in
%% the domain of~$\pi$.  We define $\pi''(x)$ to be the minimal fresh
%% parameter~$z$, chosen as in
%% Definition~\ref{def:representative-consistent-extension}; and we let
%% $\hat\pi(z) = y$.  This ensures that $v''$ uses~$z$ wherever $v'$ uses~$y$.
%% Also note that this is a consistent extension of~$\pi$, by construction.
%% \end{proof}

%%%%%

\begin{impNote}
\texttt{Unification.combine} produces the remapped version of~$v$, together
with information about the unifications.  If uses \texttt{allUnifs} to find
all choices of unification and corresponding remapping function~$\pi$.  Then
\texttt{extendUnif} extends this, and produces the remapped state.

\texttt{EffectOn.apply} produces the corresponding post-views.
\end{impNote}

%%%%%%%%%%%%%%%%%%%%%%%%%%%%%%%%%%%%%%%%%%%%%%%%%%%%%%%
