

%%%%%%%%%%%%%%%%%%%%%%%%%%%%%%%%%%%%%%%%%%%%%%%%%%%%%%%

\subsection{Correctness}

Formalise procedure.

\begin{lemma}
Let $V$ be a set of views.  Every abstract transition from~$V$ is generated by
the procedure in Definition~\ref{???}.
\end{lemma}

\begin{proof}
Consider an abstract transition
\[
\gamma(V) \ni s \trans{e} s' \sqge_{\V} v'.
\]
We need to show that we generate $v'$ as part of the process described above.

Let $fixed = s.\fixed$ and $fixed' = s'.\fixed$.  

Consider a corresponding extended transition $pre \trans{e} post$, i.e.~such
that: $pre.\fixed = fixed$,\, $post.\fixed = fixed'$; the principal of $pre$
and $post$ is the active component of the concrete transition; if another
component to which the principal has a reference synchronises on the
transition, then that is included in~$pre$ and $post$ (otherwise choose an
arbitrary view of the principal); and $pre$ and~$post$ also include any other
component relevant to the transition (i.e.~that synchronises on the transition
or to which the principal gains a reference).

\framebox{active fixed component}

Let $pId = v'.\princ.id$ be the identity of the principal of~$v'$, and suppose
that component has states $princ$ and $princ'$ in~$s$ and~$s'$, respectively
(so $princ \trans{e} princ'$).

\begin{enumerate}
\item First suppose that $pId$ is the active component in the concrete
  transition $s \trans{e} s'$, and so in the extended transition $pre
  \trans{e} post$.  And suppose that either (a)~no other component
  synchronises on the transition, or (b)~that another component does
  synchronise and is included in~$v'$, or (c)~that the principal gains a
  reference to another component in the transition, and that component is
  included in~$v'$.  In each case, the extended transition captures the
  relevant information, and so $v'$ can be extracted from~$post$.

\item Now suppose again that $pId$ is the active component in the concrete
  transition $s \trans{e} s'$, but that the secondary component~$c$ in~$v'$ is
  not in the corresponding extended transition (as in the previous case).
  Then necessarily $pId$ had a reference to~$c$ in~$s$, and $c$ did not change
  state in the transition.  Let $v = (fixed, \set{princ, c})$.  Then the
  transition $v \trans{} v'$ is induced, as in Definition\ref{def:induced-transition}

??????????

\end{enumerate}

\end{proof}
