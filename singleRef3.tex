\subsection{Correctness}

The following definition summarises how we calculate the abstract transitions
when using restricted views.
%
\begin{definition}
\label{def:transitions-singleRef}
Given a set $V$ of views, we build each active-process extended transition
$pre \trans{e} post$, and the corresponding abstract transitions $v \trans{e}
v'$, as in Definition~\ref{def:active-process-transition-singleRef}.
%
We then also create every transition that $pre \trans{e} post$ induces on a
view in~$V$, as in Definition~\ref{def:induced-transition-singleRef}.
\end{definition}

The following lemma will prove useful.
%
\begin{lemma}
\label{lem:induced-conditions}
Let $V$ be a set of views, and suppose $s \in \gamma(V)$.  Let $v$ and~$pre$
both be substates of~$s$.  Then conditions~a--c of
Definition~\ref{def:induced-transition-singleRef} are satisfied.
\end{lemma}
%
\begin{proof}
It is easy to check that each of the required views is also a substate of~$s$,
and so a member of~$V$.
\end{proof}

%%%%%

The following proposition proves the correctness of the way we calculate
abstract transitions.
%
\begin{prop}
Let $V$ be a set of views.  Every abstract transition from~$V$ is generated by
the procedure in Definition~\ref{def:transitions-singleRef}.
\end{prop}

\begin{proof}
Consider an abstract transition
\[
\gamma(V) \ni s \trans{e} s' \sqge_{\V} v'.
\]
We need to show that we generate $v'$ as part of the process described above.

Let $pId = v'.\princ.id$ be the identity of the principal of~$v'$.  If $v'$
has a secondary component, let $cId$ be its identity (so $pId$ has a
reference~$cId$ in~$s'$); if $v'$ has no secondary component, then $cId$ is
undefined.  Given an identity~$id$, we write $s(id)$ and~$s'(id)$ for the
states of the corresponding components in~$s$ and~$s'$.

We perform a case analysis.  At various points we will induce a transition
from an extended transition $pre \trans{e} post$ and some view~$v$.  In each
case, $v$ will be a view of~$s$, and so a member of~$V$; and $v$ and~$pre$
will both be substates of~$s$, so conditions~a--c of
Definition~\ref{def:induced-transition-singleRef} will be satisfied, by
Lemma~\ref{lem:induced-conditions}.
%
\begin{enumerate}
\item
First suppose $pId$ is the active component in the concrete transition \( {s
  \trans{e} s'} \), and suppose that either: 
\begin{enumerate}
\item[(a)] $v'$ has no secondary component; 

\item[(b)] $v'$ has a secondary component~$cId$,\, $s(pId)$ holds a reference
  to~$cId$, and $pId$ synchronises either with~$cId$ or no other component;
  
\item[(c)] $v'$ has a secondary component~$cId$,\, $s(pId)$ does not hold a
  reference to $pId$, and $pId$ synchronises either with~$cId$ or no other
  component; or

\item[(d)] $v'$ has a secondary component~$cId$,\, $s(pId)$ does not
  hold a reference to $pId$, and $pId$ synchronises with some other
  component~$cId' \ne cId$.
\end{enumerate}

We build a view~$v$ of~$s$ for $pId$ in each of the four cases as follows:
\begin{enumerate}
\item[(a)] Let $v$ be an arbitrary view of~$s$ for $pId$;

\item[(b)] Let $v = (s.\fixed, \set{s(pId), s(cId)})$;

\item[(c)] Let $v$ be an arbitrary view of~$s$ for $pId$;

\item[(d)] Let $v = (s.\fixed, \set{s(pId), s(cId')})$
%%  where $c_1$ is the state of~$cId'$ in~$s$
  (note that $s(pId)$ must hold a reference to~$cId'$, by
  clause~\ref{assump:max-one-extra-component} of Assumption~\ref{assump},
  since $pId$ gains a reference to~$cId$ in the transition).
\end{enumerate}

As in the proof of Lemma~\ref{lem:abstract-transitions-sound}, let $pre$ be
the extension of~$v$ to include any other component that is necessary for the
transition (this will be $s(cId)$ in cases~c and~d); or let $pre = v$ if no
other component is necessary.  Let $post$ be the corresponding states in~$s'$.
Then in each case the technique of
Definition~\ref{def:active-process-transition-singleRef} builds the transition
\( pre \trans{e} post \), and extracts the view $v'$ from~$post$.


\item %%% 2
Now suppose $pId$ is the active component in the concrete transition \( {s
  \trans{e} s'} \), but the previous case does not apply; i.e.~$v'$ has a
secondary component with identity~$cId$,\, $s(pId)$ has a reference to~$cId$,
and $pId$ synchronises with some component~$cId'$ other than~$cId$.

Then, as in the previous case, the technique of
Definition~\ref{def:active-process-transition-singleRef} builds a transition
\( pre \trans{e} post \) that includes the other component~$cId'$, but
not~$cId$.
%
Let $v = (s.\fixed, \set{s(pId), s(cId)})$.
%; this is a view of~$s$, so $v \in V$.
Then the transition \( pre \trans{e} post \) induces a transition $v \trans{}
v'$.
%% : in particular, $v$ and~$pre$ are both substates of~$s$, so
%% conditions~a--c of Definition~\ref{def:induced-transition-singleRef} are
%% satisfied, by Lemma~\ref{lem:induced-conditions}.


\item %%% 3
\label{case:singleRef-correct-3}
Now suppose the transition $s \trans{e} s'$ has an active component other
than~$pId$.  The component for~$pId$ might be a secondary component in the
transition, or it might not be involved in the transition.  Then as in
previous cases, the technique of
Definition~\ref{def:active-process-transition-singleRef} builds an extended
transition \( pre \trans{e} post \) that involves all the relevant
components.  We perform a case analysis.
%
\begin{enumerate}
\item First, suppose that $v'$ has a secondary component~$cId$ which
  is included in the extended transition, and $s(pId)$ has a reference
  to~$cId$; $pId$ might or might not be in the extended transition.
%  
  Let $v = (s.\fixed, \set{s(pId), s(cId)})$.  This produces a primary induced
  transition $v \trans{} v'$.  Note that the state for~$cId$ in~$v'$ is taken
  from~$s'$ in this case.

\item Next suppose that either $v'$ has no secondary component (so $s'(pId)$
  has no reference to another component), or $v'$ has a secondary
  component~$cId$ which is included in the extended transition, but $s(pId)$
  does \emph{not} have a reference to~$cId$.  In the former case, $pId$ might
  or might not be in the extended transition; but in the latter case, $pId$
  changes state so must be in the transition.  Let $v$ be an arbitrary view
  of~$s$ for~$pId$.  This produces a primary induced transition $v \trans{}
  v'$.  The state for~$cId$ in~$v'$ is again taken from~$s'$. 

%% \item First suppose that either $v'$ has no secondary component (so $s'(pId)$
%%   has no reference to another component), or the secondary component~$cId$
%%   of~$v'$ is included in the extended transition; $pId$ might or might not be
%%   in the extended transition.  If $s(pId)$ has a reference to~$cId$ then let
%%   $v = (s.\fixed, \set{s(pId), s(cId)})$; otherwise (so either $v'$ has no
%%   secondary component or $pId$ is in the transition and gains a reference
%%   to~$cId$), let $v$ be an arbitrary view of~$s$ for~$pId$.  Then the extended
%%   transition induces a transition $v \trans{} v'$ (again using
%%   Lemma~\ref{lem:induced-conditions}).

\item Next suppose that $v'$ has a secondary component with identity~$cId$,
  and neither $pId$ nor~$cId$ is included in the extended transition; so
  neither changes state, and $pId$ has a reference to~$cId$ in both~$s$
  and~$v'$.  Let $v = (pre.\fixed, \set{s(pId), s(cId)})$, which is a view
  of~$s$.  This again produces a primary induced transition $v \trans{} v'$ (at
  most the fixed processes change state).

\item Finally suppose that again $v'$ has a secondary component with
  identity~$cId$ that is not in the extended transition (so $s'(cId) =
  s(cId)$), but that $pId$ \emph{is} in this transition.
  \begin{itemize}
  \item Suppose $s(pId)$ has a reference to~$cId$.  Let $v =
    (pre.\fixed,\linebreak[1] \set{s(pId), s(cId)})$.  Then, as in earlier
    cases, this gives a primary induced transition $v \trans{} v'$.

  \item Suppose $s(pId)$ does not have a reference to~$cId$ (so $pId$ acquires
    the reference in the transition).  Let $v$ be an arbitrary view of~$s$
    for~$cId$.  Then this gives a secondary induced transition producing~$v' =
    (post.\fixed, \set{s'(pId), s'(cId)})$ (so $sc = s'(pId)$ in the notation
    of Definition~\ref{def:induced-transition-singleRef}).
  \end{itemize}
\end{enumerate}

\item
Finally suppose the transition $s \trans{e} s'$ has an active fixed process.
Consider an arbitrary corresponding view transition (i.e.~with an arbitrary
principal), and consider the corresponding extended transition $pre \trans{e}
post$.  Then this extended transition induces a transition that produces~$v'$,
exactly as in case~\ref{case:singleRef-correct-3}.
\end{enumerate}

\end{proof}

%%%%%

\begin{improve}
The uses of secondary induced transitions seem rather restricted: only when
$pId$ acquires the reference in the transition. 
\end{improve}

%%%%%

\begin{impNote}
When considering transitions induced by a transition $pre \trans{e} post$ upon
an accordant view~$v$, it will often be the case that one or both of the
conditions~(b) or~(c) does not hold with the current set of views.  However,
it is possible that the additional views required by those conditions are
encountered later in the search, at which point the new transition can be
induced.  To this end, we store information about such potential induced
transitions, so that if the required views are subsequently encountered, they
can then be fired.
\end{impNote}
