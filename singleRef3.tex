\subsection{Correctness}

The following definition summarises how we calculate the abstract transitions
when using restricted views.
%
\begin{definition}
\label{def:transitions-singleRef}
Given a set $V$ of views, we build each active-process extended transition
$pre \trans{e} post$, and the corresponding abstract transitions $v \trans{e}
v'$, as in Definition~\ref{def:active-process-transition-singleRef}.
%
We then also create every transition that $pre \trans{e} post$ induces on a
view in~$V$, as in Definition~\ref{def:induced-transition-singleRef}.
\end{definition}

The following lemma will prove useful.
%
\begin{lemma}
\label{lem:induced-conditions}
Let $V$ be a set of views, and suppose $s \in \gamma(V)$.  Let $v$ and~$pre$
both be substates of~$s$.  Then conditions~a--c of
Definition~\ref{def:induced-transition-singleRef} are satisfied.
\end{lemma}
%
\begin{proof}
It is easy to check that each of the required views is also a substate of~$s$,
and so a member of~$V$.
\end{proof}

%%%%%

The following proposition proves the correctness of the way we calculate
abstract transitions.
%
\begin{prop}
Let $V$ be a set of views.  Every abstract transition from~$V$ is generated by
the procedure in Definition~\ref{def:transitions-singleRef}.
\end{prop}

\begin{proof}
Consider an abstract transition
\[
\gamma(V) \ni s \trans{e} s' \sqge_{\V} v'.
\]
We need to show that we generate $v'$ as part of the process described above.

Let $pId = v'.\princ.id$ be the identity of the principal of~$v'$, and suppose
that component has states $princ$ and $princ'$ in~$s$ and~$s'$, respectively
(so $v'$ is a view of~$s'$ for $princ'$).  If $v'$ has a secondary component,
let $cId$ be its identity, and let $c$ and $c'$ be the states of that
component in~$s$ and~$s'$, respectively; if $v'$ has no secondary component,
then $cId$, $c$ and~$c'$ are undefined.  We perform a case analysis.

\begin{enumerate}
\item
First suppose $pId$ is the active component in the concrete transition \( {s
  \trans{e} s'} \), and suppose that either: (a)~$v'$ has no secondary
component; (b)~$princ$ synchronises with the secondary component~$cId$ on the
transition; (c)~$princ$ synchronises with no other component; or (d)~$princ$
synchronises with some other component and gains a reference to~$cId$ from the
transition.

If $princ$ has a reference to~$cId$ and either synchronises on the transition
with it (case~b), or synchronises with no other component~(case~c), then let
$v$ be the corresponding view of~$s$, i.e.~$(s.\fixed, \set{princ, c})$.
Otherwise (case~a or where $princ$ gains a reference to $cId$ from the
transition in either case~b or case~d), if $princ$ synchronises with some
other component~$c_1$ to which it has a reference, let $v$ be the
view~$(s.\fixed, \set{princ, c_1})$; otherwise, let $v$ be an arbitrary view
of~$s$ for~$princ$.

As in the proof of Lemma~\ref{lem:abstract-transitions-sound}, let $pre$ be
the extension of~$v$ to include any other component, maybe~$c$, that is
necessary for the transition; or let $pre = v$ if no other component is
necessary.  Let $post$ be the corresponding states in~$s'$.  Then the
technique of Definition~\ref{def:active-process-transition-singleRef} builds
the transition \( pre \trans{e} post \), and extracts the view $v'$
from~$post$.


\item 
Now suppose $pId$ is the active component in the concrete transition \( {s
  \trans{e} s'} \), but the previous case does not apply; i.e.~$v'$ has a
secondary component with identity~$cId$,\, $princ$ has a reference to~$cId$,
but $princ$ synchronises with some other component than~$cId$.  

Then, as in the previous case, the technique of
Definition~\ref{def:active-process-transition-singleRef} builds a transition
\( pre \trans{e} post \) that includes the other component, but not~$cId$.
%
Let $v = (s.\fixed, \set{princ, c})$; this is a view of~$s$, so $v \in V$.
Then the transition \( pre \trans{e} post \) induces a transition $v \trans{}
v'$: in particular, $v$ and~$pre$ are both substates of~$s$, so
conditions~a--c of Definition~\ref{def:induced-transition-singleRef} are
satisfied, by Lemma~\ref{lem:induced-conditions}.



%% \item 
%% Now suppose that the principal~$pId$ of~$v'$ is a passive component that
%% synchronises on the concrete transition.  Then as in previous cases, the
%% technique of Definition~\ref{def:active-process-transition-singleRef} builds a
%% transition \( pre \trans{e} post \) that involves all the relevant components,
%% including $pId$.  

%% Suppose, first, that either $v'$ has no secondary component (so $princ'$ has
%% no reference to another component), or the secondary component of~$v'$ is
%% included in the transition \( pre \trans{e} post \).  Let $v$ be a view of~$s$
%% for~$princ$.  Then the transition induces a transition $v \trans{} v'$ (again
%% using Lemma~\ref{lem:induced-conditions}).

%% Alternatively, suppose that $v'$ has a secondary component with
%% state~$c$ that is not included in the transition $pre \trans{e} post$.
%% Let $v$ be a view of~$s$ for~$c$.  Then the  transition induces a
%% transition producing~$v'$ (in particular using the final clause of
%% Definition~\ref{def:induced-transition-singleRef} with $sc = princ'$). 

%% \item
%% Finally suppose that the principal~$pId$ of~$v'$ does not synchronise on the
%% concrete transition.  Again, the technique of
%% Definition~\ref{def:active-process-transition-singleRef} builds a transition
%% \( pre \trans{e} post \) that involves all the relevant components.

%% Suppose that $v'$ has no secondary component, or the secondary component is
%% included in the transition \( pre \trans{e} post \).  Let $v$ be a view of~$s$
%% for~$princ$.  Then the transition induces a transition $v \trans{} v'$ (again
%% using Lemma~\ref{lem:induced-conditions}).

%% Alternatively, suppose that $v'$ has a secondary component with state~$c$ that
%% is not included in the transition $pre \trans{e} post$.  Let $v = (pre.\fixed,
%% \set{princ, c})$, which is a view of~$s$.  Then the transition again induces a
%% transition $v \trans{} v'$.


\item % Combining two previous cases. 
Now suppose the transition $s \trans{e} s'$ has an active component other
than~$pId$.  The component for~$pId$ might be a secondary component, or it
might not be involved in the transition.  Then as in previous cases, the
technique of Definition~\ref{def:active-process-transition-singleRef} builds
an extended transition \( pre \trans{e} post \) that involves all the relevant
components.

First suppose that either $v'$ has no secondary component (so $princ'$ has no
reference to another component), or the secondary component of~$v'$ is
included in the extended transition; $pId$ might or might not be in the
extended transition.  Let $v$ be a view of~$s$ for~$princ$.  Then the extended
transition induces a transition $v \trans{} v'$ (again using
Lemma~\ref{lem:induced-conditions}).

Next suppose that $v'$ has a secondary component with state~$c$, and neither
$pId$ nor the secondary component is included in the extended transition; so
neither changes state, and $pId$ has a reference to the secondary component in
both~$s$ and~$v'$.  Let $v = (pre.\fixed, \set{princ, c})$, which is a view
of~$s$.  Then the transition again induces a transition $v \trans{} v'$.

Finally suppose that again $v'$ has a secondary component with state~$c$ that
is not in the extended transition, but that $pId$ is in this transition.  Let
$v$ be a view of~$s$ for~$c$.  Then the transition induces a transition
producing~$v'$, using the final clause of
Definition~\ref{def:induced-transition-singleRef} (with $sc = princ'$).


\item
Finally suppose the transition $s \trans{e} s'$ has an active fixed process.
Consider an arbitrary corresponding view transition (i.e.~with an arbitrary
principal), and consider the corresponding extended transition $pre \trans{e}
post$.  This case is then identical to the previous case.
\end{enumerate}

\end{proof}

%%%%%

\begin{improve}
The uses of induced transitions seem rather restricted.  Does that allow us to
consider fewer induced transitions?  

The final case could be re-written so that the final subcase applies only when
$pId$ acquires the reference to~$c$ in the transition
\end{improve}



%% Let $fixed = s.\fixed$ and $fixed' = s'.\fixed$.  

%% Consider a corresponding extended transition $pre \trans{e} post$, i.e.~such
%% that: $pre.\fixed = fixed$,\, $post.\fixed = fixed'$; the principal of $pre$
%% and $post$ is the active component of the concrete transition; if another
%% component to which the principal has a reference synchronises on the
%% transition, then that is included in~$pre$ and $post$ (otherwise choose an
%% arbitrary view of the principal); and $pre$ and~$post$ also include any other
%% component relevant to the transition (i.e.~that synchronises on the transition
%% or to which the principal gains a reference).

%% \framebox{active fixed component}

%% Let $pId = v'.\princ.id$ be the identity of the principal of~$v'$, and suppose
%% that component has states $princ$ and $princ'$ in~$s$ and~$s'$, respectively
%% (so $princ \trans{e} princ'$).

%% \begin{enumerate}
%% \item First suppose that $pId$ is the active component in the concrete
%%   transition $s \trans{e} s'$, and so in the extended transition $pre
%%   \trans{e} post$.  And suppose that either (a)~no other component
%%   synchronises on the transition, or (b)~that another component does
%%   synchronise and is included in~$v'$, or (c)~that the principal gains a
%%   reference to another component in the transition, and that component is
%%   included in~$v'$.  In each case, the extended transition captures the
%%   relevant information, and so $v'$ can be extracted from~$post$.

%% \item Now suppose again that $pId$ is the active component in the concrete
%%   transition $s \trans{e} s'$, but that the secondary component~$c$ in~$v'$ is
%%   not in the corresponding extended transition (as in the previous case).
%%   Then necessarily $pId$ had a reference to~$c$ in~$s$, and $c$ did not change
%%   state in the transition.  Let $v = (fixed, \set{princ, c})$.  Then the
%%   transition $v \trans{} v'$ is induced, as in Definition\ref{def:induced-transition}

