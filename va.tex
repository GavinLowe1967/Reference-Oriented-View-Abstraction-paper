\section{View abstraction}
\label{sec:view-abstraction}

We now outline the general theory of view abstraction, following~\cite{AHH}.
Fix a collection of systems. 

We define a substate relation over system states.
% Define views as substates.
\begin{eqnarray*}
(\fixed, \cpt ) \sqle (\fixed', \cpt') & iff & 
  \fixed = \fixed' \land \cpt \subseteq \cpt' .
\end{eqnarray*}

%% Assume $\S$ is downwards-closed under taking substates. 

View abstraction is performed with respect to a particular set of subsets of
system states, known as \emph{views}.  Fix a set $\V \subseteq \S$ of views
for the remainder of this section.  In~\cite{AHH} (with a slightly different
system model) views were taken to be all substates containing $k$~processes,
for a particular choice of~$k$.  As described in the Introduction, in this
paper we will take the views to be all substates containing the fixed process,
a principal component, and all components to which the principal has a
reference; we define this formally in the next section.

We define an \emph{abstraction function}~$\alpha$.  Given a system state $s$,
let $\alpha(s)$ be all views consistent with~$s$:
%
\begin{eqnarray*}
\alpha(s) & = &
  \set{ v \| v \in \V \land v \sqle s}.
\end{eqnarray*}
%
We lift~$\alpha$ to sets of states by pointwise application. 
%
We define a corresponding \emph{concretisation function}.
Given a set $V$ of views, we define
%
\begin{eqnarray*}
\gamma(V) & = &   \set{ s \| s \in \S \land \alpha(s) \subseteq V }.
\end{eqnarray*}
%
Informally, $\gamma(V)$ gives all system states consistent with~$V$. 

The following lemmas follows directly from the definitions, as in~\cite{AHH}.
% 
\begin{lemma}
If $S, S' \subseteq \S$ and $V, V' \subseteq \V$, then
\begin{enumerate}
\item $S \subseteq S' \implies \alpha(S) \subseteq \alpha(S')$;

\item $V \subseteq V' \implies \gamma(V) \subseteq \gamma(V')$;

\item $\alpha(\gamma(V)) \subseteq V$;

\item $S \subseteq \gamma(\alpha(S))$.
\end{enumerate}
\end{lemma}
%
%% \begin{proof}
%% First two immediate.

%% Suppose $v \in \alpha(\gamma(V))$.  Then $v \in \alpha(s)$ for some $s \in
%% \gamma(V)$.  But then $\alpha(s) \subseteq V$, so $v \in V$.

%% Suppose $s \in S$.  Then $\alpha(s) \subseteq \alpha(S)$ so $s \in
%% \gamma(\alpha(S))$. 
%% \end{proof}

\begin{lemma}
\label{lem:galois}
$(\alpha, \gamma)$ forms a Galois connection: if $S \subseteq \S$ and $V
  \subseteq \V$ then 
\[
\alpha(S) \subseteq V \iff S \subseteq \gamma(V).
\]
\end{lemma}
%
%% \begin{proof}
%% Suppose  $\alpha(S) \subseteq V$.  Then $S \subseteq \gamma(\alpha(S))
%% \subseteq \gamma(V)$.

%% Suppose $S \subseteq \gamma(V)$.  Then $\alpha(S) \subseteq \alpha(\gamma(V))
%% \subseteq V$.
%% \end{proof}

Recall that $\R$ denotes all reachable system states.  The basic idea of view
abstraction is to construct an upper bound~$V$ of $\alpha(\R)$.  The above
lemma then tells us that $\R \subseteq \gamma(V)$.  By reasoning about~$V$, we
can deduce facts about all reachable states. 

We define an abstract post-image function over sets of views.
%
\begin{definition}
If $V \subseteq \V$ then 
\begin{eqnarray*}
aPost(V) & \defs & \alpha(post(\gamma(V)).
\end{eqnarray*}
\end{definition}
%
The following lemma shows that the abstract post-image function simulates the
concrete one.  The lemma follows easily from  Lemma~\ref{lem:galois}.
%
\begin{lemma}
If $V \subseteq \V$ and $S \subseteq \gamma(V)$ then 
$post(S) \subseteq \gamma(aPost(V))$.
\end{lemma}
%
The following proposition shows how we can approximate the set of reachable
system states by iterating the abstract post-image from some set of initial
views.  The result is proved via an easy induction, using the previous lemma.
%
\begin{prop}
\label{prop:reachable}
Suppose $AInit \subseteq \V$ is such that $\alpha(Init) \subseteq AInit$.  Then
\begin{eqnarray*}
\R & \subseteq & \gamma (aPost^*(AInit)).
\end{eqnarray*}
\end{prop}

The above proposition does not immediately give us an algorithm, since the
right-hand side is, in general, an infinte set.  For any particular definition
of the set~$\V$ of views, we need to find a way of calculating $aPost$
finitely.  We return to this question once we have formally defined the
set~$\V$ we employ. 

%% Other things go through as before.  
