\section{The algorithm}
\label{sec:algorithm}

We now present our algorithm, and prove its correctness.  We start with an
abstract algorithm, in Figure~\ref{fig:algorithm}.  We give a more
implementation-oriented version in Section~\ref{ssec:algorithm-2}.

The abstract algorithm takes as inputs the definition of a collection of
systems, as in Definition~\ref{def:system} (which defines the set~$\V$ of
views, and the abstraction and concretisation functions, $\alpha$
and~$\gamma$), and a set $AInit$ of initial views such that $\alpha(Init)
\subseteq {AInit}$.  It maintains a set $V \subseteq \V$ of views encountered
so far, and performs a breadth-first search.  On each iteration it calculates
all transitions from the current views~$V$
(Definition~\ref{def:abstract-transition}), and adds the new views to~$V$.
This continues until either a transition on $error$ is found, or a fixed point
is reached.

\begin{figure}[t]
\[
\begin{align}
V := AInit \\
while(true)\{ \\
\quad \mbox{if there is a transition $pre \trans(2)[error] post$
  from~${V}$ then return $failure$} \\
% \quad \Let  X = aPost({V}) \\
% \quad \mbox{for $(v' \in aPost(V))$
\quad \mbox{for each $v'$ produced by a transition from $V$, 
  if $v' \nin {V}$ then 
  $V := V \union \set{v'}$} \\
\quad \mbox{if no new view was added then return $success$} \\ 
\}
\end{align}
\]
\caption{The initial algorithm}
\label{fig:algorithm}
\end{figure}

%%%%%

In Section~\ref{sec:symmetry}, we use symmetry reduction to represent the
set~$V$.  This will allow us to prove that the algorithm reaches a fixed point
after a finite number of iterations.  We prove here that this fixed point will
be enough to deduce correctness of the systems under consideration. 

%%%%%

\begin{lemma}
\label{lem:alg-fixed-point}
If the algorithm does not return $failure$, and $V$ reaches a fixed
point~$V_{fix}$, then $\R \subseteq \gamma( {V_{fix}} )$.
\end{lemma}
%
\begin{proof}
We show that after $n$ iterations,
\begin{eqnarray*}
{V} & \supseteq & \textstyle\Union_{i = 0}^n aPost^i({AInit}),
\end{eqnarray*}
%
by induction on~$n$.  The base case is trivial.  Suppose the result holds
for~$V$ at the start of an iteration.  Let $V'$ be the next value of~$V$.
Proposition~\ref{lem:abstract-transitions-sound} shows that the generated
transitions include all abstract transitions, so
\[
\begin{array}{rcl}
{V'} % & = & {V} \union {X} \\ 
& \supseteq & {V} \union aPost({V}) \\
& \supseteq & \Union_{i = 0}^n aPost^i({AInit}) \union
      aPost(\Union_{i = 0}^n aPost^i({AInit})) \\
& \supseteq & \Union_{i = 0}^n aPost^i({AInit}) \union 
  aPost(aPost^n({AInit})) \\
& = & \Union_{i = 0}^{n+1} aPost^i({AInit})
\end{array}
\] 
using the inductive hypothesis and monotonicity of $aPost$. 

Hence the presumed fixed point must satisfy
\[
{V_{fix}} \supseteq  \textstyle\Union_{i = 0}^\infty aPost^i({AInit}) = 
  aPost^*({AInit}).
\]
The result then follows from Proposition~\ref{prop:reachable}.
\end{proof}

%%%%%

\begin{prop}
If the algorithm returns $success$, then the system is error-free, for all
systems starting in a state in $Init$.
\end{prop}
%
\begin{proof}
If the algorithm returns $success$ then necessarily it has reached a fixed
point~$V_{fix}$.  Suppose, for a contradiction, that some reachable state $s
\in \R$ has a transition on $error$.  Then Lemma~\ref{lem:alg-fixed-point}
implies that $s \in \gamma({V_{fix}})$, so $s \in \gamma({V})$ at some
iteration of the algorithm.  Hence there is an extended transition on $error$
from~${V}$ (as in Lemma~\ref{lem:active-process-transitions}), so the
algorithm returns $failure$, giving a contradiction.
\end{proof}

%%%%%%%%%%%%%%%%%%%%%%%%%%%%%%%%%%%%%%%%%%%%%%%%%%%%%%%

\subsection{An implementation-oriented algorithm}
\label{ssec:algorithm-2}

We now give a more implementation-oriented algorithm.  The algorithm in
Figure~\ref{fig:algorithm} implicitly considered \emph{all} transitions
from~$V$ on each ply of the search.  Instead, on each ply we want to expand
only those views found on the previous ply, as is standard for a breadth-first
search.

This is complicated by the fact that a new view may be the result of
\emph{several} previous views, not just one.  For example, a transition from
an earlier ply might induce a transition on a newly found view; or a
transition template from an earlier ply might now produce a new concrete
transition based on a newly found view.


%%%%%%%%%%

\begin{figure}
\[
\begin{align}
V := AInit; thisPly := AInit \\
while(thisPly \ne \set{})\{ \\
\quad newTrans := \set{}; newTransTemplates := \set{}; nextPly := \set{} \\
\quad \For(v \in thisPly)\ process(v) \\
\quad \comment{update for next round} \\
\quad V := V \union nextPly; thisPly := nextPly; trans := trans \union newTrans \\
\quad  transTemplates := transTemplates \union newTransTemplates \\
\quad \comment{extract views from new transitions} \\
\quad \For ((pre \trans[e] post) \in newTrans)\{ \\
\qquad \Let v = viewOf(post); 
  \If(v \nin V)\{ V := V \union \set{v}; thisPly := thisPly \union \set{v} \}\\
\quad \}\\
\} \\
\mbox{return $success$}
\\[2ex]
%%%%%%
\comment{Produce abstract transitions from $v$.} \\
\Def process(v) = \{ \\ % process(v) 
%% Transitions
\quad \comment{closed view transitions from~$v$} \\
\quad \For ((v \trans[e] v') \in transitionsOf(v)) \
   addTransition(v \trans[e] v') \\
%% Transition templates
\quad \comment{transition templates from $v$} \\
\quad \For ((v \uplus \hole{id}  \trans[e] v' \uplus \hole{id}) 
  \in transTemplatesOf(v) \{ \\
\qquad  newTransTemplates := newTransTemplates \union
   \set{v \uplus \hole{id}  \trans[e] v' \uplus \hole{id}} \\
\qquad \Foreach \mbox{compatible component state~$c$ and post-state $c'$} \\
\qquad\quad           addTransition(v \uplus c \trans[e] v' \uplus c') \\
\quad \} \\
% effectOfPreviousTransitions(v) 
\quad \comment{induced transitions on~$v$} \\
\quad \For ((pre \trans[e] post) \in trans)\ 
  makeInduced(pre \trans[e] post,v) \\
% effectOfPreviousTransitionTemplates
\quad \comment{instantiate transition templates using $v$} \\
\quad \For ((pre \uplus \hole{id} \trans[e] post \uplus \hole{id})
   \in transTemplates) \\
\qquad  \Foreach \mbox{compatible component $c$ of $v$ and post-state~$c'$} \\
\qquad\quad addTransition(pre \uplus c \trans[e] post \uplus c') \\
\} % end of process
\end{align}
\]
\caption{The implementation-oriented algorithm.}
\label{fig:algorithm-2}
\end{figure}

%%%%%%%%%%

\begin{figure}
\[
\begin{align}
\comment{Store and process the transition $pre \trans[e] post$.} \\
\Def addTransition(pre \trans[e] post) = \{ \\
\quad  \If (e = error)\ \mbox{return $failure$} \\
\quad \If ((pre \trans[e] post) \nin trans)\ 
        newTrans := newTrans \union \set{pre \trans[e] post} \\
     % effectOnOthers
\quad \For (v_X \in V)\ makeInduced(pre \trans[e] post, v_X) \\
\} 
\\ [2ex]
%%%%%
\comment{Produce transitions induced by $pre \trans[e] post$ on~$v$.} \\
\Def makeInduced(pre \trans[e] post, v) = \{ \\
\quad \If(\mbox{$pre \trans[e] post$ induces a transition on $v$})\{ \\
\qquad \Let v' \mbox{ be the view induced};\
       \If (v' \nin V)\ nextPly := nextPly \union \set{v'} \\
\quad \} \\
\} 
\end{align}
\]
\caption{The implementation-oriented algorithm (continued).}
\label{fig:algorithm-3}
\end{figure}

%%%%%%%%%%

The algorithm is in Figures~\ref{fig:algorithm-2} and~\ref{fig:algorithm-3}.
It maintains several sets:
%
\begin{itemize}
\item $V$, which contains all views found on previous plies;

\item $trans$, which contains all active-process transitions found on previous
  plies;

\item $transTemplates$, which contains all transition templates found on
  previous plies.
\end{itemize}
%
It also maintains several sets corresponding to the current ply of the search:
%
\begin{itemize}
\item $thisPly$, which contains views found on the previous ply, and that are
  expanded on the current ply;

\item $nextPly$, which contains views found on the current ply, and that will
  be expanded on the following ply;

\item $newTrans$, which contains new active-process transitions found on this
  ply;

\item $newTransTemplates$, which contains new transition templates found on
  this ply.
\end{itemize}

The main loop calls $process(v)$ for each view~$v$ of the current ply.  At the
end of each ply, it updates the sets for the following ply; this includes
extracting the principal's view of the post-state of each new transition. 

The function $process(v)$ considers all transitions that require~$v$,
possibly combined with other views from~$V$.  There are several aspects of
this. 

It builds each closed view transition $v \trans[e] v'$ from~$v$, as in
Definition~\ref{def:active-process-transition}; we assume a function
$transitionsOf$ that does this.  For each such transition, it calls
$addTransition$ (Figure~\ref{fig:algorithm-3}).  If the transition involves
the distinguished event $error$, the main algorithm returns $failure$.
Otherwise the transition is stored.  Further, the function $makeInduced$ is
called for each view~$v_X$ found on earlier plies, to produce transitions
induced on~$v_X$, as in Definition~\ref{def:induced-transition}.

The function $process(v)$ also builds each transition template $v \uplus
\hole{id} \trans[e] v' \uplus \hole{id}$ from~$v$, as in
Definition~\ref{def:active-process-transition}; we assume a function
$transitionTemplatesOf$ that does this.  It stores each transition template.
It also tries to instantiate the template.  If finds each component state~$c$
such that $c$ has identity~$id$ and is compatible with $v$ (with respect
to~$V$), and corresponding post-state~$c'$, as in
Definition~\ref{def:active-process-transition}, and builds
the transition $v \uplus c \trans[e] v' \uplus c'$.  It then calls
$addTransition$ on each such transition.

Further, $process(v)$ considers each transition $pre \trans[e] post$ found on
a previous ply, and builds all transitions that this induces on~$v$.

Finally, $process(v)$ tries to use~$v$ to instantiate each transition template
$pre \uplus \hole{id} \trans[e] post \uplus \hole{id}$ found on a previous
ply.  It considers a compatible component state~$c$ of~$v$ with identity~$id$
(if any), and corresponding post-state~$c'$, as in
Definition~\ref{def:active-process-transition}, and builds the transition $pre
\uplus c \trans[e] pre' \uplus c'$.  It then calls $addTransition$ on each
such transition.


\begin{lemma}
On each iteration, the procedure in Figures~\ref{fig:algorithm-2}
and~\ref{fig:algorithm-3} produces every view in $aPost(V)$ that is not
already in~$V$.
\end{lemma}

%%%%%

\begin{proof}
We consider the different forms of abstract transition from
Definition~\ref{def:abstract-transition}, and use
Proposition~\ref{prop:abstract-transitions-sound}. 

First consider a closed view transition $v \trans[e] v'$ that requires no
additional component; and let $v'' = viewOf(v')$.  Then this transition will
be considered within $process(v)$ on the ply after $v$ is first encountered.

Now consider a view built from a transition template $v_1 \uplus \hole{id}
\trans[e] v_1' \uplus \hole{id}$, instantiating the hole with a compatible
component state~$c$ with identity~$id$.  Suppose the view $v_1$ is produced
on ply~$n-1$, so the transition template is built on ply~$n$.  Consider the
views required for compatibility of~$c$ and~$v_1$
(Definition~\ref{def:compatible}); suppose the last of these to be produced
is~$v_2$, on ply~$k$.
%% Consider a transition template $v \uplus \hole{id} \trans[e] v' \uplus
%% \hole{id}$ produced on ply~$n$.  Consider a compatible component state~$c$
%% with identity~$id$, and consider the views required for compatibility of~$c$
%% and~$v$ (Definition~\ref{def:compatible}); suppose the last of these to be
%% produced is~$v_1$, on ply~$k$. 
If $n \le k$, then the template is instantiated with~$c$ within $process(v_2)$
on ply~$k+1$.  If $k < n$ then the template is instantiated within
$process(v_1)$ on ply~$n$.  In each case, the new view is generated on the
first ply that it appears in $aPost(V)$.

Finally consider a transition $pre \trans[e] post$ produced on ply~$n$, a
view~$v$ produced on ply~$k$, and an induced transition producing view~$v'$.
If $n \le k$, the induced transition will be produced within $process(v)$ on
ply~$k+1$.  If $k < n$ then the induced transition will be produced within
$addTransition(pre \trans[e] post)$ on ply~$n$.  In each case, $v'$~is
generated on the first ply that it appears in $aPost(V)$.
\end{proof}

Note that when instantiating a transition template, we need to
consider \emph{every} component~$c$ of the relevant view, not just the
principal.  It might be that the relevant $c$ is not the principal of the
\emph{last} relevant view to be found.
\begin{improve}
However, in the case of instantiating a new template with prior views, I think
it would be enough to consider just the principal.  Try this.
\end{improve}
